\documentclass[11pt]{article}
%\usepackage{a4,psfig,color,xcolor}
\usepackage[utf8]{inputenc}
\usepackage[fleqn]{amsmath}
\usepackage{amssymb,latexsym,wasysym,dsfont,stmaryrd}
\usepackage{xspace}
\usepackage{graphicx}
\usepackage[top=2cm,left=2cm,right=2cm,bottom=2cm]{geometry}
\usepackage{framed}
\usepackage{xcolor}
%\usepackage{mymacros}
%\usepackage[usenames]{color}
\usepackage{natbib}
\usepackage{setspace}
% Pas de numeros de pages
\pagestyle{empty}




\begin{document}
\begin{spacing}{1.25}
\begin{center}
  {\large \bf ICTCS 2017 Call For Papers}
\end{center}

%\vspace{1cm}

\begin{center}
  {\bf 18th Italian Conference on Theoretical Computer Science}\\
  {\bf Naples, Italy, September 26-28, 2017}\\
  {\bf http://www.??????}
\end{center}
\end{spacing}

\begin{spacing}{1}
  
ICTCS 2017, the 2017 Italian Conference on Theoretical Computer
Science is the 18th conference of the Italian Chapter of EATCS. It
will be held in Naples, September 26-28 2016 at University of Naples.

~\\
The scope of the meeting is fostering the cross-fertilization of ideas
stemming from different areas of Theoretical Computer Science. Hence,
the Italian Conference on Theoretical Computer Science represents an
occasion for meeting and exchanging ideas and for sharing experiences
between researchers. It also provides the ideal environment where
junior researchers and PhD students can meet senior researchers.

~\\
Contributions in any area of theoretical computer science are warmly
solicited. The event is open to both Italian and foreign researchers,
which are welcome to submit papers and attend the Conference.

~\\
Typical, but not exclusive, topics of interest include:

~\\
\indent\indent agents, algorithms, argumentation, automata theory, \\
\indent\indent automated theorem proving, complexity theory,\\
\indent\indent computational logic, computational social choice, \\
\indent\indent concurrency, cryptography, discrete mathematics, \\
\indent\indent distributed computing, dynamical systems, formal methods, \\
\indent\indent  game theory, graph theory, knowledge representation languages,\\
\indent\indent model checking, process algebras, quantum computing,\\
\indent\indent rewriting systems, security and trust, semantics,\\
\indent\indent specification and verification, systems biology, types


\paragraph{Program co-chairs:}
~\\

 Dario Della Monica (University of Naples)%(Università degli Studi di Napoli ``Federico II")

Bastien Maubert (University of Naples)%(Università degli Studi di Napoli ``Federico II")

\paragraph{Program committee:}
~\\

????????????????
\paragraph{Important dates:}
~\\

\begin{tabular}{ll}
Abstract and Papers Submission:&  \textcolor{red}{1 June 2017}\\
Notification of acceptance:&    6 July 2017\\
Final version:&               31 August 2017\\
Conference:&              26-28 September 2017  
\end{tabular}

\paragraph{Submissions:}
~\\

Two types of contributions are solicited.

\begin{itemize}
\item Regular papers: up to 12 pages in LNCS style. Full original papers,
presenting novel results, not appeared or submitted elsewhere.

\item Communications: up to 5 pages in LNCS style. Suitable for extended
abstracts of papers already appeared, or submitted, or to be
submitted, elsewhere; papers reporting on ongoing researches on which
the authors wish to get feedback at ICTCS and possibly intended to be
included in future publications; overviews of PhD-theses, research
projects, etc...
\end{itemize}

In case of need, to ease the reviewing process, the authors of regular
papers may add an appendix containing further material (or indicate a
web site containing longer version of the paper). In any case the
reviewers are not required to consider such further material in their
evaluation. All contributions must be written in English.

For each accepted contribution, at least one of the authors is
required to attend at the conference and present the paper. All
accepted contributions (communications and regular papers) will appear
in a number of CEUR Workshop Proceedings.



\paragraph{Post-conference issue on Theoretical Computer Science:}
~\\

Selected papers from ICTCS-2017 will be invited to a special issue
of the journal Theoretical Computer Science.
The authors of a selection of the regular papers presented at the
conference will be asked to submit an improved version of their papers.
The selection will be determined by considering the outcome of the
conference reviewing phase.
The papers submitted for the journal issue should have been significantly
revised and extended with respect to the conference versions.
A second reviewing process, meeting the high standard of quality of the
international journal will select the papers to be accepted for the
special issue, among the invited ones.

\paragraph{Submission page:}
~\\

Authors are invited to submit their manuscripts in PDF via the
EasyChair system at the link: 

??????

\end{spacing}


\end{document}

%%% Local Variables: 
%%% mode: latex
%%% TeX-master: t
%%% End: 
