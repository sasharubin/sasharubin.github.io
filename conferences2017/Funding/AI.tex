\documentclass[12pt]{article}
%\usepackage{a4,psfig,color,xcolor}
\usepackage[utf8]{inputenc}
\usepackage[fleqn]{amsmath}
\usepackage{amssymb,latexsym,wasysym,dsfont,stmaryrd}
\usepackage{xspace}
\usepackage{graphicx}
\usepackage[top=2cm,left=4cm,right=4cm,bottom=2cm]{geometry}
\usepackage{framed}
\usepackage{xcolor}
%\usepackage{mymacros}
%\usepackage[usenames]{color}
\usepackage{natbib}
\usepackage{setspace}
\usepackage{paralist}
% Pas de numeros de pages
\pagestyle{empty}




\begin{document}
\begin{spacing}{1}
\begin{center}
  {\Large \bf ICTCS'17 and CILC'17 (co-located)}
\end{center}

%\vspace{1cm}

\begin{center}
  {Application to ``Funding Opportunities for Promoting AI Research''}\\
  ~\\
  {Dario Della Monica, Bastien Maubert, Aniello Murano and Sasha Rubin}\\
  {Universit\`a degli Studi di Napoli ``Federico II''}
\end{center}

\subsection*{Recurrent event}

We hereby request funding to help organise the 18th
edition of the Italian Conference on Theoretical Computer Science
(ICTCS) and the 32nd edition of the
Italian Conference on Computational Logic (CILC), which will be
co-located at the University of Naples (September 26-29th 2017).

ICTCS is the conference of the Italian Chapter of the European Association for
Theoretical Computer Science (IC-EATCS). As such it encompasses all
aspects of theoretical computer science, including
AI and formal methods in AI: typical topics of interest include
agents, argumentation, computational social choice, game theory and
knowledge representation languages. The conference has formal
proceedings, and around this event revolve
established members of the Theoretical Computer Science and AI communities.

CILC is the annual conference of the Group of researchers and Users of Logic
Programming (GULP). Since 1986, it is
in Italy the most important event around computational logic, where users,
researchers and developers come to   meet and exchange ideas.
During its 31 years of recurrence, the event has continually widened its
horizons from the field of traditional logic programming to the more general
areas of declarative programming and its applications in various neighboring
fields, such as Artificial Intelligence or Deductive Databases.

\subsection*{What?}

The funding we request is meant to cover the following expenses:

\begin{compactenum}[$(i)$]
\item travel and accommodation for three invited speakers, one for
  ICTCS and two common invited speakers for ICTCS/CILC,
\item financial support for PhD students or young postdoc researches
  (registration fees/travel expenses).
\end{compactenum}

\subsection*{How much?}
We request the following amount for supporting the aforementioned expenses:
\begin{compactenum}[$(i)$]
\item travel and accommodation for three invited speakers: \textbf{2100 euros}
  (700 euros per person),
\item support for 10 young researchers: \textbf{3500 euros} (350 euros per
  person).
\end{compactenum}
The total amount requested is \textbf{5600 euros}.

\subsection*{Why?}

CILC and ICTCS are two well-established Italian conferences, which will be
co-located for the first time.
This will result in a very exciting meeting, rich of fruitful discussions.
Philosophy of both conferences is to bring researchers together to discuss
innovative ideas, including ongoing investigation which can benefit from
constructive feedback before turning into solid contribution.
According to their aims, a low-fee policy has always been applied by these
conferences to attract a large audience, with a special attention to
young researchers, which often cannot dispose of adequate financial resources.
Given that we do not have at the moment any external financial support, the
requested funds are essential for us to organize a successful event while
keeping attendance fees low.

\subsection*{Who?}

Direct beneficiaries of the support will be mostly PhD students and young
post-doc researchers.
We also plan to cover travel costs to three high-profile invited speakers (two
of which will be common to both conferences), which means that all participants
will indirectly benefit from the funds as they will be offered high-quality
(plenary) lectures.

\subsection*{When?}

ICTCS'17 will be held from September 26 to September 28, 2017, and
CILC'17 from September 26 to September 29, 2017.

\subsection*{Where?}

ICTCS'17 and CILC'17 will take place at the modern and newly built campus of the
University of Naples, in the suburban area of San Giovanni a Teduccio, Naples.

\subsection*{Contact details}

Dario Della Monica\\
\texttt{\indent dario.dellamonica@unina.it \\ \indent
  http://wpage.unina.it/dario.dellamonica/}

\medskip

\noindent
Bastien Maubert\\
\texttt{\indent bastien.maubert@gmail.com \\ \indent http://bastien-maubert.fr}

\medskip

\noindent
Aniello Murano\\
\texttt{\indent murano@unina.it \\ \indent http://people.na.infn.it/~murano/}

\medskip

\noindent
Sasha Rubin\\
\texttt{\indent sasha.rubin@unina.it \\
\indent http://forsyte.at/alumni/rubin/}

\end{spacing}
\end{document}

%%% Local Variables: 
%%% mode: latex
%%% TeX-master: t
%%% End: 
