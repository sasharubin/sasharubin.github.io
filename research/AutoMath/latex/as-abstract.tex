%%%%%%%%%%%%%%%%%%%%%%%%%%%%%%%%%%%%%%%%%%%%%%%%%%%%%%%%%
%
% automatic structures - abstract
% 
%%%%%%%%%%%%%%%%%%%%%%%%%%%%%%%%%%%%%%%%%%%%%%%%%%%%%%%%
%=====================================================
% abstract
%===========
\begin{abstract} 
In order to compute with infinite mathematical structures one needs a finitary description of the structure. 
Automata are a natural choice for describing structures because of their close relationship with 
mathematical logic. A structure is called automatic if its domain and atomic predicates are recognised by automata. 
The striking feature of automatic structures is that every logically definable relation is recognised by an automaton. 
The first part of this chapter develops and surveys the logical and model-theoretic properties of automatic structures.
The second part focuses on foundational questions such as: How does one tell if a given structure is automatic? 
What does it mean for descriptions of the same structure to be different? 
As prerequisite to this chapter one should be familiar with elementary notions from logic, as well as automata on infinite objects such as strings and trees.
%
%A mathematical structure (such as an order, group, algebra, etc.) whose de�nable re- 
%6 
%lations can be computed by automata is called automatic. The standard models of automata are 
%7 
%those operating synchronously on �nite or in�nite strings or trees. Each gives rise to a class of 
%8 
%automatic structures. This chapter discusses their logical, computational and algebraic properties. 
%9 
%Automatic structures are de�ned in terms of logical interpretations. Equivalent de�nitions in terms 
%10 
%of presentations by automata are given. The focus of this chapter is on foundational questions. After 
%11 
%introducing the logical and automata-theoretic background, we focus on structures presentable by 
%12 
%in�nite strings/trees, and then on structures presentable by �nite strings/trees. 

\end{abstract}
