%%%%%%%%%%%%%%%%%%%%%%%%%%%%%%%%%%%%%%%%%%%%%%%%%%%%%%%%%
%
% automatic structures - abstract
% 
%%%%%%%%%%%%%%%%%%%%%%%%%%%%%%%%%%%%%%%%%%%%%%%%%%%%%%%%
%=====================================================
% abstract
%===========
\begin{abstract} 
In order to compute with infinite mathematical structures one needs, typically at least, a finitary
description/presentation of the structure. Automata and their equivalent logical formalisms are natural choices.
These give rise to the automatic structures whose salient feature is that
every definable relation is recognised by an automaton. This chapter focuses on foundational questions:
What are equivalent ways of defining automatic structures? Under what operations are automatic structures closed? How does one
tell if a given structure is automatic? What does it mean for two presentations of the same structure to be different?
%
%A mathematical structure (such as an order, group, algebra, etc.) whose de�nable re- 
%6 
%lations can be computed by automata is called automatic. The standard models of automata are 
%7 
%those operating synchronously on �nite or in�nite strings or trees. Each gives rise to a class of 
%8 
%automatic structures. This chapter discusses their logical, computational and algebraic properties. 
%9 
%Automatic structures are de�ned in terms of logical interpretations. Equivalent de�nitions in terms 
%10 
%of presentations by automata are given. The focus of this chapter is on foundational questions. After 
%11 
%introducing the logical and automata-theoretic background, we focus on structures presentable by 
%12 
%in�nite strings/trees, and then on structures presentable by �nite strings/trees. 

\end{abstract}
