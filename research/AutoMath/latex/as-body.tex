%======================================================
\section{Introduction} \label{AS:sec:introduction}
%======================================================

This chapter is about using automata to answer algorithmic and model-theoretic questions about general mathematical structures.
We give an extended introduction that traces one of the paths that led to the development of automatic structures. We start with
the problem of providing algorithms for answering logical queries about mathematical structures of interest.

\subsection{The Problem of decidability}
Much elementary algebra and geometry can be
expressed in the first-order (FO) language of the structure of {\em real arithmetic}
\[(\R,+,\times,+1,<,=,0,1).\]
Variables in this FO language vary over the real numbers, and formulas allow quantification over variables 
($\forall x \in \R$ and $\exists x \in \R$), and may make use of the arithmetic operations of addition ($+$), multiplication ($\times$) and successor ($+1$), the relations less-than ($<$) and equality ($=$), the constants zero ($0$) and one ($1$), and the Boolean connectives (negation, conjunction, etc).
Tarski proved that there is an algorithm that decides the truth or falsity of
every FO-sentence in this language (a sentence is a formula with no free variables). One says that the FO-theory (the set of true FO-sentences) of real arithmetic is decidable. Tarski's proof uses a technique called effective quantifier elimination. The approach looks familiar:
the statement that a quadratic has two real roots
\[
\exists x,y \in \R.\,  [x \neq y \wedge ax^2 + bx + c = 0 \wedge ay^2 + by + c =0]
\]
is equivalent to, and thus can be replaced, by the simpler quantifier-free condition
$a \neq 0 \wedge 4ac < b^2$ (in these formulas, $a,b,c$ represent terms built from the constants and the arithmetic operations, and 
we use the usual abbreviations and write, for example, $x^2$ for $x \times x$). The truth or falsity of the quantifier-free sentence
can be simply checked by evaluating the sentence on the given terms $a,b,c$.


A good technique will do more than simply prove decidability. Tarski's procedure gives
some insight into the nature of the sets and relations definable by
formulas of the language.  Indeed, Tarski's procedure transforms a given
formula into an equivalent formula with no quantifiers and the same
free variables (and it also provides a proof from the axioms for a real closed field that the two formulas are equivalent). 
A model-theoretic consequence is that every FO-definable set in real arithmetic is a finite union of intervals with algebraic endpoints. An algorithmic consequence
is that one can effectively determine the number of real roots of a given polynomial.
Tarski's paper \cite{Tars51} contains a nice discussion of his result
and its uses. For an elementary proof see \cite{MiOz02}.

Effective quantifier elimination is just one technique for proving theories decidable and understanding the definable relations. Other notable approaches
are providing a finite or computably enumerable axiomatisation of a theory\footnote{Here a theory is any set of FO-sentences closed under logical deduction. If, in addition to being finitely axiomatisable, the theory is also complete (every sentence or its negation is in the set), then the theory is decidable. Showing completeness is the bread and butter of model theory.}, and  the composition method
(associated with the names Feferman, Vaught and Shelah) that leverages a decomposition of a structure into pieces whose theories determine, 
and can be used to decide, the theory of the original structure.

Of course, not all theories are decidable. If a theory can express some undecidable problem, then that theory is undecidable. For instance, the FO-theory of the structure of integer arithmetic $(\N,+,\times,<,=,0,1)$ is very expressive (it can express much of number theory) and undecidable.\footnote{An outline of the proof of the undecidability of integer arithmetic is as follows: one can define a pairing function, which gives one the ability to define finite sequences of integers, and so encode FO-formulas by terms, and also whether or not a given FO-sentence of arithmetic is provable from a given computable set of FO-axioms; in particular, whether or not a given Turing machine halts on a given input.} 

% Further, if a theory can express a long unsolved problem then proving decidability will be difficult. For instance, 

Fragments and variations of the structure of integer arithmetic provide a search-space for interesting decidable theories.
A good starting point for decidability is to remove either multiplication or addition. For instance, the FO-theory of $(\N,+)$, known as {\em Presburger arithmetic},  is decidable. Indeed, the structure $(\N,+,<,(\equiv_m)_{m \geq 1})$, where $x \equiv_m y$ if and only if $x$ is congruent to $y$ modulo $m$, admits effective quantifier elimination and is thus decidable~\cite{Pres29}.\todo{is $0$ needed?}
 One may then try to slowly introduce more expressive predicates. The FO-theory of the structure $(\N,+,\text{Pr})$, where $\text{Pr}(x)$ if and only if $x$ is prime, is known to be undecidable only under some number theoretic assumptions related to the twin-primes conjecture~\cite{BJW93}. In contrast, the FO-theory of $(\N,+,|)$ where $x | y$ if and only if $x$ divides $y$, is undecidable since multiplication is definable in terms of addition and divisibility~\cite{Tars49}.

% For instance, the FO-theory of $(\N,\times)$, known as {\em Skolem arithmetic}, is decidable~\cite{??}. On the other hand, 

A tangential approach is to look at simpler arithmetic structures but more powerful logics. For instance, the FO-theory of $(\N,+1,0)$ is extremely simple: it admits effective quantifier elimination which means that the FO-definable sets are finite or co-finite. On the other hand, suppose we also allow quantification over relations $R \subseteq \N^d$ and include the membership relation $\tup{x} \in R$.  This is called second-order (SO) language. Unsurprisingly, the SO-theory of $\one$ is undecidable since one can define $0$ as the unit of addition, and simulate the recursive definition of addition in terms of successor, and of multiplication in terms of addition. For instance, $x + y = z $ if and only if $\forall M \subseteq \N \times \N. [(0,x) \in M \wedge \text{closed}(M) \to (y,z) \in M]$ where $\text{closed}(M)$ is the formula $\forall u,v. (u,v) \in M \to (u+1,v+1) \in M$. With this as background Tarski asked (in lectures, as reported in \cite{Robi58}): Is the monadic second-order theory of $\one$ decidable? In the monadic second-order (MSO) language, variables range over
{\em subsets} of the domain $\N$. This is where automata enter the story.

\todo{check everywhere that $\Phi$ is used that it should not be $\phi$}
% Instead of while, under the same assumptions, the MSO-theory of $(\N,+1,\text{Pr})$ 
% is decidable

% (the MSO-theory is like the FO-theory except that one can quantify over subsets as well as elements of $\N$).
% \todo{say that highly undecidable means sigma one one complete}


\subsection{Decidability via automata}

To answer Tarski's question, it is natural to view arbitrary subsets of $\N$ as infinite binary strings: for a set $A \subseteq \N$ write $\chi(A)$ for its characteristic $\{0,1\}$-labelled $\omega$-string. It is also natural to view tuples of subsets of $\N$ as infinite binary strings: for a tuple of strings $(\alpha_1,\dots,\alpha_r)$ write $\conv(\tup{\alpha})$ for the convoluted $\{0,1\}^r$-labelled $\omega$-string (cf. Definition \ref{AS:dfn:convtrees}), and thus $\conv(\chi(A_1), \dots, \chi(A_r))$ is an $\omega$-string representing the tuple $(A_1,\dots,A_r)$ of sets of integers. With this direct and simple coding in mind, B\"uchi \cite{Buch62} defined appropriate automata that operate on infinite strings, $\omega$-string automata, and proved that these are effectively closed under the usual operations, namely, intersection, (notably) complementation, projection, and that one can decide if the language of a given $\omega$-string automaton is empty or not. This means that every MSO-formula of $\one$ can be compiled into automata, and thus that the MSO-theory of $\one$ is decidable. Indeed, a sentence of the form $\exists X \Phi(X)$ holds in $\one$ if and only if the $\omega$-string automaton for $\Phi$ has non-empty language.


\begin{theorem}[\cite{Buch62}]
A set of tuples $(A_1,\dots,A_r)$ of integers is MSO definable in $\one$ if and only if the set of $\{0,1\}^r$-labelled strings
$\conv(\chi(A_1),\dots,\chi(A_r))$ is recognised by an $\omega$-string automaton. Moreover, the translation between formulas and automata is effective. In particular, the MSO-theory of $\one$ is decidable.
\end{theorem}

Historically, a simpler variation of Tarski's question was solved first. Namely, is the weak-monadic second-order (WMSO) theory of the structure $\one$ is decidable? In the WMSO language variables range over {\em finite subsets} of $\N$. By viewing tuples of finite subsets of $\N$ as finite binary strings B\"uchi \cite{Buch60}, Elgot \cite{Elgo61} and Trahtenbrot \cite{trah62} independently showed that WMSO-formulas of $\one$ define regular languages. 

% The reason this is possible is that logical operations (such as conjunction and quantification) correspond to effective operations on automata (such as intersection and projection). In slogan form: formulas can be compiled into automata. This establishes that the WMSO-theory of $\one$ (called WS1S) is decidable. Indeed, a sentence of the form $\exists X \Phi(X)$ holds in $\one$ if and only if the automaton for $\Phi$ has non-empty language (itself a decidable property). 
% The structure  $\frakT_r$ has domain $\{0,1, \dots,r-1\}^\ast$ and binary relations $\suc_i$ ($i \in \{0,1, \dots, r-1\}$) 
% consisting of the pairs $(w,wi)$, where $w \in \{0,1,\dots,r-1\}^\ast$. The (W)MSO theory of $\frakT_r$ is called $(W)SrS$. 
% Note that $\frakT_1$ is isomorphic to $\one$ (if we view $+1$ as a binary relation rather than a unary function).

There is a natural next step. Let $\frakT_2$ denote the structure $(\{0,1\}^\ast,\suc_0, \suc_1)$ where $\suc_i$ consists of the pairs $(w,wi)$, where $w \in \{0,1\}^\ast$.  The (W)MSO theory of $\frakT_2$ is called (W)S2S. 
B\"uchi \cite{Buch62} asked whether S2S, the MSO-theory of $\frakT_2$, is decidable. Subsets of $\twodom$ are naturally viewed as $\{0,1\}$-labeled trees.
Doner \cite{Done70} and Thatcher and Wright \cite{ThWr68} used tree automata to show that WS2S is decidable. In a milestone paper Rabin \cite{Rabi69} introduced automata operating on $\omega$-trees to prove that S2S is decidable. 
%The case of $SrS$ is no harder.
%\footnote{Shelah \cite{Shel75} uses his composition to prove many decidability results (including S1S). 
%However, the only known proofs that S2S is decidable use automata.}

As expected, more than just decidability was established. We have already indicated that definable relations of $\frakT_2$ are, modulo a coding of sets as trees, recognised by automata. The converse also holds. For a set $A \subseteq \twodom$ write $\chi(A)$ for its characteristic $\{0,1\}$-labelled tree (Definition \ref{AS:dfn:chartree}). For a tuple of trees $(t_1,\dots,t_n)$ write $\conv(\tup{t})$
for the convoluted $\{0,1\}^n$-labelled tree (Definition \ref{AS:dfn:convtrees}).  

\begin{theorem}[\cite{Rabi69}] \label{AS:thm:RABIN}
A set of tuples $(A_1,\dots,A_r)$ of sets of strings is MSO definable in $\two$ if and only if the set of $\{0,1\}^r$-labelled trees 
$\conv(\chi(A_1),\dots,\chi(A_r))$ is recognised by an $\omega$-tree automaton. Moreover, the translation between formulas and automata is effective.
\end{theorem}


An accessible proof of Theorem~\ref{AS:thm:RABIN} can be found in \cite{Thom90}. Decidability of S2S follows using the fact that it is decidable whether or not a given $\omega$-tree automaton has a non-empty language. 

This approach to decidability and definability can be applied to other structures. This can be done in two equivalent ways: via automatic presentations and via logical interpretations in $\frakT_2$.

% This central theorem can be leveraged to deduce many other decidability and definability results. 

\subsection{Decidability via automatic presentations}

The reason that the MSO-theory of $\frakT_2$ is decidable is that there is a coding of the {subsets of its domain} by trees such that the coded domain and the coded relations are recognised by $\omega$-tree automata. One might take this as a definition, that is, a structure $\frakA$ is ``automatically presentable'' if there is a coding of subsets of its domain by trees such that the domain and atomic relations of $\frakA$ are recognised by $\omega$-tree automata. This would imply that the MSO-theory of $\frakA$ is decidable. Historically, however, there was more interest in the FO-theory of structures. Thus, the following definition was given: a structure $\frakA$ is \emph{$\omega$-tree automatically presentable} if there is a coding of the {elements of its domain} by trees such that its domain and atomic relations are, modulo the coding, recognised by $\omega$-tree automata~\cite{BLGr00} and cf.~\cite{KhNe95,Hodg76}.\todo{check what Hodgson actually defined} Consequently, FO-definable relations of $\frakA$ are represented, modulo a coding, by automata, and the FO-theory of $\frakA$ is decidable. 

We illustrate this technique on some integer arithmetics. 
We have already seen that Presburger arithmetic, the FO-theory of $(\N,+)$, is decidable via the technique of effective quantifier elimination. 
We now show that $(\N,+)$ has an automatic presentation. Actually, we show to code natural numbers by finite strings (which are special cases of $\omega$-trees) 
so that the $\N$ and $+$ are, modulo the coding, recognised by finite-string automata (instead of $\omega$-tree automata).   
We code $n \in \N$ by the shortest binary string $bin(n) \in \twodom$ representing it, i.e., if $u = bin(n)$ then $n = \sum_{u_i = 1} 2^i$. \todo{check notation $u_i$} Note that the set of codes of all natural numbers is $\{\epsilon\} \cup \{0,1\}^\ast1$, which is regular. Moreover, the relation 
$\{\conv((bin(n),bin(m),bin(n+m))) \st n,m \in \N\}$ is regular since the usual bit-carry procedure for addition can be implemented by an automaton. 

Actually, under this encoding, the relation $n |_2 n$ iff $n$ is a power of $2$ and $n$ divides $m$ is also, modulo the coding, regular. 
That is, an automaton takes $\conv(bin(n),bin(m))$ as input and checks that $bin(n) \in 0^\ast 1$ and that the $|bin(n)|$th bit of $bin(m)$ is equal to 1. Thus, $(\N,+,|_2)$ is automatically presentable. Using base $k$ codings instead one gets that $(\N,+,|_k)$ is automatically presentable~\ref{buchi?elgot?}. On the other hand, $(\N,+,|)$ has no automatic presentation since its FO-theory is undecidable~\ref{}.

Interestingly, $(\N,\times)$ has an automatic presentation. To see this, code $n$ by a finite-tree $tr(n)$ representing its 
prime-decomposition, e.g., $2^{d_2} 3^{d_3} 5^{d_5}$ is coded by a tree whose shape is a comb with three teeth and $d_i$ is written on the $i$th tooth in binary. 
Thus, a finite-tree automaton can recognised tuples $\conv(tr(n),tr(m),tr(nm))$ by simply verifying the addition on each tooth. However, there is no automatic presentation of $(\N,\times)$ using finite strings~\cite{}. 

Finally, $(\N,+,\times)$ has no automatic presentation since its FO-theory is undecidable~\ref{}. Other ways of proving that a structure has no automatic presentation are given in Section~\ref{}.
% This allows one to translate every FO-formula $\Phi(y_1,\cdots,y_r)$ of $(\N,+)$ into a WMSO-formula $\Phi(Y_1,\cdots,Y_r)$ of $\one$ such that
% $\Phi(\tup{y})$ holds in $(\N,+)$ iff $\Phi(X(y_1), \cdots, X(y_r))$ holds in $\one$. Indeed, the translation simply replaces variable names (i.e., $y$ becomes $Y$), and the relation $+$ is replaced by the relation $add$.
% In other words, the two formulas define the same relation modulo the coding. This relation, viewed as one on strings,
% is recognised by a finite-string automaton (cf. Theorem \ref{AS:thm:RABIN}). Finally, every FO-formula over $(\N,+)$ defines a relation that, modulo the coding, is recognised by a finite-automaton. Decidability follows from the fact that the problem of whether or not the language of 
% a given finite-string automaton is empty.

% Using $\omega$-strings instead of finite-strings one can get that $(\R,+)$ has an automatic presentation, and thus its FO-theory is decidable. 


\subsection{Decidability via interpretations}
Interpretations are a familiar idea in arithmetic, e.g., rational numbers can be coded as pairs of integers $(a,b) \in \Z \times (\Z \setminus \{0\})$, and relations and operations on rationals can be defined in terms of these pairs. For instance, two codes $(a,b), (c,d)$ represent equal rationals if and only if $a\times d = b \times c$. Since the definitions can be written in first-order logic, we say that the structure $(\Q,+,\times,=)$ is FO-interpretable in $(\Z,+,\times,=)$. 
The importance of interpretations is that they allow one to reduce questions (about definability and decidability) of one structure to another. For instance, if $\frakC$ is FO-interpretable in $\frakB$ then the FO-theory (resp. MSO-theory) of $\frakC$ is computable from the FO-theory (resp. MSO-theory) of $\frakB$. 

We will require a more general notion, \emph{set interpretations}, that translate elements in $\frakC$ to subsets of the domain of $\frakB$. This reduces the FO-theory of $\frakC$ to the MSO-theory of $\frakB$. Here are two illustrative examples.
%Interpretations are systematically studied in model theory. %(see \cite{Hodg93}).

\todo{define language of automaton somewhere early}

\begin{example}
We show how to interpret the structure $(\N,+)$ in $\one$ by giving a \emph{finite-set interpretation}~\cite{Robi58,Buch60,Elgo61}. That is, 
we code $n \in \N$ by the finite set $X(n) \subset \N$ where $n = \sum_{x \in X(n)} 2^x$. E.g., $X(13) = \{0,2,3\}$. To finish defining the interpretation we should find a WMSO formula $add(X_1,X_2,X_3)$, in the signature of $\one$, such that 
\[
a+b = c \mbox{ iff } add(X(a),X(b),X(c)).
\]
The formula implements the usual bit-carry procedure for addition:
it guesses the existence of the carry set and uses the successor $+1$ to scan the
sets one place at a time verifying the addition. We can thus deduce from Theorem~\ref{AS:thm:RABIN} that the FO-theory of $(\N,+)$ is decidable.
% This allows one to translate every FO-formula $\Phi(y_1,\cdots,y_r)$ of $(\N,+)$ into a WMSO-formula $\Phi(Y_1,\cdots,Y_r)$ of $\one$ such that
% $\Phi(\tup{y})$ holds in $(\N,+)$ iff $\Phi(X(y_1), \cdots, X(y_r))$ holds in $\one$. Indeed, the translation simply replaces variable names (i.e., $y$ becomes $Y$), and the relation $+$ is replaced by the relation $add$.
% In other words, the two formulas define the same relation modulo the coding. This relation, viewed as one on strings,
% is recognised by a finite-string automaton (cf. Theorem \ref{AS:thm:RABIN}). Finally, every FO-formula over $(\N,+)$ defines a relation that, modulo the coding, is recognised by a finite-automaton. Decidability follows from the fact that the problem of whether or not the language of 
% a given finite-string automaton is empty.
\end{example}

% \begin{corollary} \label{AS:cor:pres}
% Modulo a certain coding of natural numbers into binary strings,  every FO-definable relation of $(\N,+)$ is recognised by an automaton. 
% The translation from formulas to automata is effective. Consequently, the FO-theory of $(\N,+)$ is decidable.
% \end{corollary}

% The following argument establishing this corollary can be found in \cite{Robi58,Buch60,Elgo61}. There is a {\em finite-set interpretation} of
% $(\N,+)$ in $\one$.
% The interpretation codes $n \in \N$ as a {\em finite subset} of $\N$ (e.g., thirteen is coded first in binary as $1101$ and then as the set $\{0,2,3\}$). This coding has the property that there is a WMSO formula $\Phi_+(X_1,X_2,X_3)$ (written in the signature of $\one$) expressing that 
% $X'_1 + X'_2 = X'_3$, where $X'_i$ is the unique number coded by the set $X_i$. The formula implements the usual bit-carry procedure for addition:
% it guesses the existence of the carry set and uses the successor $+1$ to scan the
% sets one place at a time verifying the addition. 
% This allows one to translate every FO-formula $\Phi(y_1,\cdots,y_n)$ of $(\N,+)$ into a WMSO-formula
% $\Phi(Y_1,\cdots,Y_n)$ of $\one$  with the property that
% the two formulas define the same relation modulo the coding. This relation, viewed as one on strings,
% is recognised by a finite-string automaton (cf. Theorem \ref{AS:thm:RABIN}). Finally, every FO-formula over $(\N,+)$ defines a relation that, modulo the coding, is recognised by a finite-automaton. Decidability follows from the fact that the problem of whether or not the language of 
% a given finite-string automaton is empty is decidable.

\begin{example} \todo{find a better example?}
Consider the rational ordering $(\Q,<)$. We provide a {\em FO-interpretation} of $(\Q,<)$ in $\frakT_2$ in which 
every rational is coded by an element of $\twodom$~\cite{Rabi69}. Let $x \sqcap y$ denote the longest common prefix of $x$ and $y$, and define the 
binary relation  $\prec$ on $\frakT_2$ where $x \prec y$ if $(x \sqcap y)0$ is a prefix of $x$ or 
$(x \sqcap y)1$ is a prefix of $y$. %%% CHECK
Note that $(\twodom,\prec)$ is a countable dense linear order with no endpoints;  so it is isomorphic, say via $\mu$, to $(\Q,<)$. We can thus deduce from Theorem~\ref{} that the MSO-theory of $(\Q,<)$ is decidable.

Actually, the same coding $\mu:\Q \to \twodom$ gives a set-interpretation of $(\Power(\Q),\subset,<_{sing})$ into $\frakT_2$. Here 
$X <_{sing} Y$ iff $X$ and $Y$ are singletons, say $X = \{x\}$ and $Y = \{y\}$, and $x < y$. Indeed, every set $X \subseteq \Q$ is coded by the set $\mu(X) \subset \twodom$; containment $x \in X$ corresponds to containment $\mu(x) \in \mu(X)$, and the ordering $q < r$ corresponds to the ordering $\mu(q) <_{sing} \mu(r)$. 
We can thus deduce from Theorem~\ref{} that the FO-theory of  $(\Power(\Q),\subset,<_{sing})$ is decidable. \todo{change $<$ to $<_{sing}$ in dfn later}
\todo{fix this... ``corresponds'' is too vague}
% Since the relation $\prec$ is MSO-definable in $\frakT_2$ one can translate every MSO-formula $\Phi(X_1,\cdots,X_n)$ of $(\Q,<)$ into an MSO-formula
% $\Phi'(X_1,\cdots,X_n)$ of $\frakT_2$  with the property that
% the two formulas define the same relation modulo the isomorphism. Theorem \ref{AS:thm:RABIN}  ensures that every MSO-definable relation of $(\Q,<)$, viewed as a relation on $\omega$-trees, is recognised by an $\omega$-tree automaton. Decidability follows from the fact that the problem of whether or not the language of a given $\omega$-tree automaton is empty is decidable.
\end{example}


% We provide The second example illustrates that an interpretation in which elements are coded by elements (instead of coding elements by sets, as in the first example) results in more expressive theories being decidable (namely, MSO instead of FO).

% \begin{corollary} \label{AS:cor:rationals}
% Modulo a certain coding of rational numbers into $\omega$-trees, every MSO-definable relation of $(\Q,<)$ is recognised by an automaton. The translation from formulas to automata is effective. Consequently, the MSO-theory of $(\Q,<)$ is decidable.
% \end{corollary}


% In summary, suppose $\A$ has a decidable MSO-theory. If $\B$ is MSO-interpretable in $\A$ then $\B$ has a decidable MSO-theory; and if $\B$ is set-interpretable in $\A$ then $\B$ has decidable FO-theory. 

This suggests an alternative (and equivalent) definition of a structure $\frakB$ being ``automatic'', i.e., that $\frakB$ be set-interpretable in $\frakT_2$.
To summarise, MSO formulas in $\frakT_2$ can be compiled into $\omega$-tree 
automata, and if $\frakB$ is set-interpretable in $\frakT_2$ (which means that 
$\frakB$ can be defined in $\frakT_2$ via MSO formulas whose free variables are first-order variables),
then FO formulas of $\frakB$ can also be compiled into $\omega$-tree automata. This shows that automata may be used
to reason about the FO-theories of structures that are set-interpretable in $\frakT_2$.


\subsection{Outline and scope of this chapter}

% The fact that formulas can be compiled into automata gives the {\em fundamental theorem of automatic structures}, which is 
% exploited to prove many results about automatic structures:
% 
% 
% \begin{theorem} \label{AS:thm:fundthm}
% Suppose $\frakA$ is Rabin-automatic. For every FO-formula $\varphi(x_1,\cdots,x_r)$, the set of trees $\conv(\chi(x_1), \cdots, \chi(x_r))$ such that 
% $\varphi(\tup{x})$ holds in $\frakA$ is recognised by an $\omega$-tree automaton. Moreover, the translation from formulas to automata is effective. 
% Thus, the FO-theory of $\frakA$ is decidable.
% \end{theorem}

% \begin{enumerate}
% \item Modulo coding, every FO-definable relation in a Rabin-automatic structure
% is recognised by an $\omega$-tree automaton.
% \item The FO-theory of every Rabin-automatic structure is decidable. 
% \end{enumerate}



%  \todo{define relational structure. finite number of relations.}

% \begin{definition}
%  A relation structure $\frakA$ is {\em Rabin-automatic} if the domain $\A$ is a regular set of $\{0,1\}$-labelled trees, and for each atomic relation $R_i$,
%  there is an $\omega$-tree automaton that accepts the language $\{\conv(\chi(x_1),\cdots,\chi(x_n)) : (x_1,\cdots,x_n) \in R_i\}$.
% \end{definition}

% 


% We give some examples. Since definability is preserved by isomorphism, to show that a structure has decidable FO-theory it is enough to show that 
% it is isomorphic to a Rabin-automatic structure. \todo{define isomorphic}
% 
% 
% \begin{example}
% We have already seen that Presburger arithmetic, the FO-theory of $(\N,+)$, is decidable via the technique of effective quantifier elimination. 
% 
% We now show that $(\N,+)$ is isomorphic to a Rabin-automatic structure. In fact, we code integers by finite strings (instead of the more general 
% $\omega$-trees)~\cite{Robi58,Buch60,Elgo61}. That is, Every finite set of natural numbers $X = \{a_0, a_1, \dots, a_n\}$ determines a number $N(X) := \sum_{i \leq n} 2^i a_i$. E.g., $N(\{0,2,3\}) = 2^0 + 2^2 + 2^3 = 13$ and $N(\emptyset) = 0$.
% 
% \end{example}
% 
% \begin{corollary} \label{AS:cor:pres}
% Modulo a certain coding of natural numbers into binary strings,  every FO-definable relation of $(\N,+)$ is recognised by an automaton. 
% The translation from formulas to automata is effective. Consequently, the FO-theory of $(\N,+)$ is decidable.
% \end{corollary}
% 
% 

\todo{clean up ``automatic'' vs ``automatic presentable'' structures}

Here is an outline of the problems discussed in this chapter. \todo{give section refs}

\begin{enumerate}
\item To what extent can we increase the expressive power of the logical language (FO) and still retain decidability of automatic structures? 
%This is the problem of extending the fundamental theorem of automatic structures.
\item Which operations on structures preserve being automatic?
\item What do automatic structures look like? Meaningful answers can be used to show that a given structure is not automatic.
\item An automatic structure may be set-interpretable in a number of ways. How are these ways different?
This is the problem of studying non-equivalent automatic presentations.
\item What about presenting structures using other mechanical formalisms?
\end{enumerate}

{\bf Other accounts and omissions.} 
The definition of automatic presentations directly in terms of automata follows~\cite{KhNe95}, while the logical-interpretation angle follows~\cite{Blum99,BlGr00,CoLo07}.

Automatic structures (over finite and infinite strings) were introduced in \cite{Hodg76,Hodg83} for proving decidability. They were independently introduced in \cite{KhNe95} where the motivation stresses that automatic structures are part of computable model theory, a subject that looks at the effective content of mathematical statements. Here one asks for the automatic content of standard theorems from mathematics such as K\H{o}nig's tree lemma, Ramsey's partition results and Cantor's results on countable linear orders, see \cite{Rubi08,Kusk03,Kusk10,KuLo10,HuLi13}. Besides many foundational contributions, the thesis \cite{Blum99} began the study of automatic structures with automata operating over finite and infinite trees. 



The complexity of automatic structures can be measured in a number of ways: the computational complexity of their theories (including query evaluation and model checking problems) can be found in \todo{you might add a reference to
the work by Durand-Gasslin and Habermehl on Presburger arithmetic and on
automatic structures of bounded degree (using
Ehrenfeucht-Fraissee-relations)} \cite{BlGr00,Kusk09,KuLo11,BaGrRu11,KuWe11}; complexity measures such as Scott rank, Cantor-Bendixson rank and ordinal height of automatic structures can be found in \cite{KRS05,KhMi09,Husc13,HKLL13}; the complexity of the isomorphism problem for classes of automatic structures can be found in \cite{Rubi08,KLL10LICS,KLL13,KLM13,Kuske14}. For model theoretic considerations (quantifier-elimination, VC-dimension, axiomatisations, non-standard models) see \cite{Blum99,BLSS03,BaGrRu11}. Useful resources regarding the relationship between automatic structures and other finitary formalism (automatic groups, numeration systems, pushdown graphs, etc.) can be found in \cite{BlGr00,BaGrRu11,Kart13}. The study of whether one can decide if a tree-automatic structure is string-automatic began in \cite{Husc12}. 
% Some results about different presentations of an automatic structure can be found in \cite{KuWe11, Husc12}.
%FIX SENTENCE
% incorporate:

%LUKASZ
%I thought already a while ago that this could be a good idea, but I'm
%afraid that there might be a problem regarding our old STACS paper -
%and it is also my fault. You see - when working with Faried on the
%non-automaticity of reals we basically repeated our old proof. But now
%(as is suggested already in the paragraph just before section 4.1 of
%our paper) we did this in parallel with a linear odering relation.
%This gives a strenghtening of a previous result of Kuske in addition
%to re-proving our stacs result and the main goal of showing that reals
%are not automatic. That was a stacs'12 paper and it was selected for a
%post-conference journal issue. On the way, Faried with Wied managed to
%find a much more elegant proof of the other part of the paper - but
%the non-injective part stayed there and appeared in Theory Comput.
%Syst. last year. It is in Section 4 (pages 11-19) below.
%  http://www.logic.rwth-aachen.de/pub/abuzaid/modelprop.pdf
%I'm not sure if that really disqualifies the idea of making a journal
%version of our paper, but it surely makes it look less reasonable, I'm
%afraid. I'm very sorry that it turned out this way, I hope you don't
%have any hard feelings about that!

%FARIED
%Anyway, we have already done a few things. First of all we could apply our techniques for omega-automatic structures to show that several (countable) structures are not automatic with any advice. It seems to be the case that most structures that are known to be not automatic are also not automatic with advice.
%
%Further we have a student, Frederic Reinhardt, that found some other examples of structures that are not automatic in the usual sense but automatic with advice in his diploma thesis. 
%He showed that the torsion-free abelian groups of rank 1 have presentations with parameters (which is mostly due to fact that all of them are substructures of (Q, +) ) and some trees where the degree of the vertices grows with the depth in the tree. He also has some of the non-automaticity results in his thesis. If you are interrested I could send it to you.
%He is also applying for the algosyn research training group. So maybe he will join our group in the near future. 
%
%We are now focusing on decidability questions for the advice. We think we can show that all advices that are used to represent the torsion-free abelian groups of rank 1 have decidable MSO theories. Also we would like to collect the model theoretic results that are known for automatic structures and see what changes if we allow advices. One such question would be for example has every structure that is omega-automatic with a decidable advice also an automatic presentation with a  decidable advice (without the decidability of the advice one already get this from your paper with Lukasz and Vince


See  \cite{Gure85} for a treatment of decision problems  for MSO via automata as well as via composition. An equivalent way of identifying MSO and automata is that strings or trees are themselves considered structures (over a certain natural signature) and the regular languages correspond to sets of structures satisfying MSO sentences. This is usually discussed in the context of finite model theory, see for instance \cite{EbFl95}. See \cite{Rabi77} for FO decision problems via standard techniques  (quantifier elimination and completeness considerations). %, and the comprehensive book \cite{BGG97} for the classical decision problem. 

%A note on citations. By default I have cited what I think is the best reference, though not necessarily the first, for a given result. %The interested reader may refer to \cite{Rubi08} for some historical comments.


{\bf Acknowledgements.} I am grateful to Vince B{\'a}r{\'a}ny, Achim Blumensath, Nathana\"el Fijalkow, Olivier Finkel and Dietrich Kuske for their comments and corrections; Martin Huschenbett for discussing Christian Delhomm{\'e}'s work \cite{Delh04} (Theorem \ref{AS:thm:treeautdecomp}); Faried Abu Zaid for discussing his work and for bringing my attention to Frederic Reinhardt's Diploma Thesis \cite{Reinhardt13}; and Wolfgang Thomas and Anil Nerode for their historical insight.

%=====================================================
\section{Logic, automata and interpretations} \label{AS:sec:back}
%=====================================================


 
\subsection{Logical languages} \label{AS:subsec:logic}

A \emph{structure}  $\frakA = (\A, R_1, \dots, R_N)$ consists of a set $\A$ called the {\em domain}, and relations\footnote{We deal with relational structures. This is no real handicap since we can replace an operation by its graph. For instance, addition is taken as those triples $(x,y,z)$ such that $x+y = z$.} $R_i \subseteq \A^{r_i}$. The names of the $R_i$ together with their arities $r_i \in \N$ form the \emph{signature} of the structure. If $\frakB$ is the restriction of $\frakA$ to some set $\B \subseteq \A$ then write $\frakB \subseteq \frakA$ and say that $\frakB$ is a \emph{substructure} of $\frakA$. We write $(\frakA,P)$ for the structure $\frakA$ expanded by the predicate $P$. Two structures $\frakA,\frakB$ over the same signature are \emph{isomorphic} if there exists a bijection $f:\A \to \B$ such that for every relation $R$ of arity $r$ in the signature, and all $x_1, \cdots, x_r \in \A$, we have that 
$R(x_1, \dots, x_r)$ holds in $\frakA$ iff $R(f(x_1), \dots, f(x_r))$ holds in $\frakB$.

\begin{example}
The structure  $\frakT_r$ has domain $\{0,1, \dots,r-1\}^\star$ and for each  $i$ from $\{0,1, \dots, r-1\}$ a binary relation $\suc_i$ consisting of pairs $(w,wi)$. 
Note that $\frakT_1$ is isomorphic to $\one$ via the mapping $f:0^n \mapsto n$.
\end{example}

To express properties of structures we need a logical language. Formulas of monadic
second-order logic (MSO) are constructed using logical connectives (`and',
`or', `not'), individual variables $x,y,z$ (that are intended to range over
elements of the domain), set variables $X,Y,Z,\dots$ (that are intended to
range over subsets of the domain), quantification over these variables, the
subset $X \subseteq Y$ relation, the equality relations $X = Y$ and $x=y$, the membership relation $x \in Y$, 
and finally they may
use the names of the relations such as $R(x,y)$ from a fixed signature. \todo{remove subset relation?}
Formulas of weak monadic second-order logic (WMSO) are defined as for MSO
except that set variables are intended to range over finite subsets of the
domain. Thus WMSO formulas have the same syntax as MSO formulas, but the semantics is different.
Formulas of first-order logic (FO) are defined as for MSO but without 
set variables. 

A formula written $\Phi(X_1,\dots,X_n,x_1,\dots,x_m)$ means that $\Phi$'s
free variables are included in the set $\{X_1,\dots,X_n,x_1,\dots,x_m\}$.
A {\em sentence} is a formula without free variables. A sentence is {\em valid} if it is true in all structures.
Here is a formula of two free individual variables $x,y$ in the signature of the structure $(\N,+1)$:
\[
(\forall Z) [x \in Z \wedge (\forall z)(z \in Z \implies z+1 \in Z) \implies y \in Z].
\]
We can see that it is satisfied by those pairs of natural numbers $(x,y)$ such that $x \leq y$. 
One can similarly define the prefix relation $\pref$ in $\frakT_2$. We often write (W)MSO formulas as $\Phi(\tup{X})$ even if some of the free
variables are individual variables.\todo{do we do this?}

Note that we are appealing to our natural sense of what it means for a sentence to be true of a structure. 
For a rigourous definition of truth and satisfaction in
mathematical logic see, for instance, \cite{Bool07}. We will use the shorthand $\frakA \models \Phi$
(read \emph{$\frakA$ models $\Phi$}) to mean that the sentence $\Phi$ is true in
$\frakA$. Two formulas $\Phi(\tup{X})$ and $\psi(\tup{Y})$ are {\em equivalent over $\frakA$}
if \[\frakA \models \forall \tup{X} \forall \tup{Y} [\Phi(\tup{X}) \iff \psi(\tup{Y})].\]

An MSO-formula $\Phi(\tup{X},\tup{x})$, in the signature of $\frakA$,
{\em defines} the relation 
\[
\Phi^\frakA := \{(A_1,\dots,A_k,a_1,\dots,a_n) \st \frakA \models \Phi(\tup{A},\tup{a}), A_i \subseteq \A, a_j \in \A \}.
\]

Similarly, a WMSO-formula $\Phi(\tup{X},\tup{x})$ {\em defines} the relation 
\[
\Phi^\frakA := \{(A_1,\dots,A_k,a_1,\dots,a_n) \st \frakA \models \Phi(\tup{A},\tup{a}), A_i \textrm{ is finite}, A_i \subseteq \A, a_j \in \A \}.
\]

A central problem in mathematical logic has been establishing  the (non-)decidability
of theories.
Let $\eL$ be one of MSO, WMSO, or FO. The {\em $\eL$-theory} of a structure is the set
of $\eL$-sentences true in that structure.  A set $X$ of sentences is {\em
decidable} if there is an algorithm that correctly decides, given a sentence
$\Phi$ in the language of $\eL$, whether or not $\Phi \in X$.  We say that a
structure has {\em decidable $\eL$-theory} if its $\eL$-theory is decidable.  

\subsection{Rabin's theorem}

The B\"uchi/Elgot/Trahtenbrot revolution led to increasingly complex
structures in which definable relations correspond with some type of automata. 
The cornerstone is Rabin's theorem: MSO definability in $\two$ coincides with recognisability by $\omega$-tree automata.
Tree automata operate on {\em (binary) $A$-labeled trees} $t:\twodom  \to A$. 
For a definition of Rabin tree-automaton see \cite{Thom90}. 
We code tuples of sets as $\omega$-trees in two steps. 
First, code a set as a tree, and second, code a tuple of trees as a single tree by laying the tuples alongside each other.


\begin{definition}[characteristic $\omega$-tree] \label{AS:dfn:chartree}
For a set $Y \subseteq \twodom$ define
its {\em characteristic tree} $\chi(Y)$ as the $\{0,1\}$-labeled $\omega$-tree with a $1$ in position $w \in \twodom$ if and only if $w \in Y$.
\end{definition}


\begin{definition}[convoluting $\omega$-trees] \label{AS:dfn:convtrees}
Let $\tup{t} = (t_1,\dots,t_k)$
be a $k$-tuple of $\{0,1\}$-labelled $\omega$-trees. 
The {\em convolution} $\conv(\tup{t})$ is the $\{0,1\}^k$-labelled $\omega$-tree such that for all positions $w \in \twodom$
the $i$th component of $\conv(\tup{t})(w)$ is equal to $t_i(w)$.
\end{definition} 

\begin{theorem}[Rabin's theorem \cite{Thom90}] \label{AS:thm:Rabin}
For each MSO-formula $\Phi(\tup{X})$ in the signature of $\frakT_2$ there is an $\omega$-tree automaton (and vice-versa) such that the language recognised by the automaton is
\[
\{\conv(\chi({X_1}),\dots,\chi({X_k})) \st \frakT_2 \models \Phi(\tup{X})\}.
\]
The translations are effective.
\end{theorem}


\begin{proposition} \cite{Thom90} \label{AS:thm:Rabin:dec}
The emptiness problem for Rabin-automata is decidable and consequently the MSO-theory of $\frakT_2$ is decidable.
Moreover, there is an effective procedure that given an automaton $M$ with non-empty language produces (a finite presentation of) a 
regular $\omega$-tree accepted by $M$ (this fact is called Rabin's basis theorem).
\end{proposition}

Similar results hold with $\frakT_1$ replacing $\frakT_2$ and are known as B\"uchi's theorem(s). 
These can be proven directly or as corollaries by coding $X \subseteq \N$ by the tree $T$ with $T^{-1}(1) = \{0^n \st n \in X\}$.
Similar results also hold for WMSO. The standard reference is \cite{Thom90}.

%%%%%%%%%%%%%%%%%%%%%%%%%%%%%%%%%%%%%%%%%%%%%%%%%%%%%%%%%%%%%%%%%%%%%%%%%%%%%%%%%%%%%%%%%%%%5
\subsubsection*{Rabin's theorem with additional set quantifiers} \label{AS:sec:beyond}
%%%%%%%%%%%%%%%%%%%%%%%%%%%%%%%%%%%%%%%%%%%%%%%%%%%%%%%%%%%%%%%%%%%%%%%%%%%%%%%%%%%%%%%%%%%%%%%%%%%%%%%%%%%%%%

%%%%%%%%%
%=====================================================
%\subsection{Additional quantifiers} \label{AS:sec:addquan}
%=====================================================

We show that we can enrich MSO by certain set quantifiers (such as `there are finitely many sets $X$ such that \dots')
and still get decidability  for $\frakT_2$. We do this by showing that formulas with the additional quantifiers are actually equivalent to vanilla MSO formulas. 

\begin{lemma} \label{AS:lem:msofindef}
The property ``$X$ is finite'' is $\mso$-definable in $\frakT_2$.
\end{lemma}

\begin{proof}
The following simple argument is from \cite{Rabi69}.
The lexicographic (total) ordering $\lex$ on $\twodom$ is $\mso$-definable in $\frakT_2$.
Thus ``$X \subseteq \twodom$ is finite'' is definable by the formula that says that every $B \subseteq X$ has
both a maximum and minimum element with respect to $\lex$.
\end{proof}

So, every WMSO-definable relation $R$ of $\frakT_2$ is MSO-definable in $\frakT_2$ 
(simply relativise the set quantifiers). In particular the following results also hold with WMSO replacing MSO.
For a cardinal $\kappa$ let $\exists^{\geq \kappa}$ denote the quantifier ``there exists at least $\kappa$ many sets $X$ such that \dots''.
Write $\msokappa$ for $\mso$ enriched by the quantifier $\exists^{\geq \kappa}$. 

\begin{proposition} \cite{BKRa}
For every $\msocount$ formula $\Phi(\tup{X})$ there is an $\mso$ formula 
$\Phi'(\tup{X})$ equivalent to $\Phi(\tup{X})$ over $\frakT_2$.
\end{proposition}

\begin{proof}
The following are equivalent:
\begin{enumerate}
\item There are only finitely many $X$ satisfying $\Phi(X,\tup{Y})$. 
\item There is a finite set $Z$ such that every pair of different sets $X_1,X_2$ which both satisfy $\Phi(X_i,\tup{Y})$ differ on $Z$.
\end{enumerate}
The second condition can be expressed in $\mso$ using Lemma \ref{AS:lem:msofindef}.
\end{proof}

\begin{theorem} \cite{BKRa}
For every $\msounc$ formula $\Phi(\tup{X})$ there is an $\mso$ formula
$\Phi'(\tup{X})$ equivalent to $\Phi(\tup{X})$ over $\frakT_2$.
\end{theorem}

\begin{proof}
The proof uses the composition method and basic ideas from descriptive set theory. We sketch a proof of the simpler case of $\frakT_1$ following \cite{KuLo08JSL}. Say that two subsets of $\N$
have the {\em same end} if their symmetric difference is finite.  There is a constant $K$ (that depends only on $\Phi$) such that the following are equivalent for all $\tup{Y}$:

\begin{enumerate}
\item There are uncountably many $X$ satisfying $\Phi(X,\tup{Y})$.
\item There are $K$ many sets $X$, each satisfying $\Phi(X,\tup{Y})$, and that pairwise have different ends.
\end{enumerate}
The second condition can be expressed in MSO. We argue correctness.
The forward direction follows since each end class is countable. For the reverse let $K$ be larger than the number of states of an $\omega$-automaton for $\Phi$. The idea is that if there are too many sets with different ends we can find two that behave the same and so shuffle these to get uncountably many.  For ease of writing assume that $\tup{Y}$ is a singleton and so write $Y$ instead. For a set $X_i$ satisfying $\Phi(X_i,Y)$ write
$\rho_i$ for some accepting run of the automaton on 
$\conv(\chi_{X_i},\chi_Y)$. 
\todo{ $\rho_1[n]$ is presumably the n'th state in the run $\rho_1$, on
page 28, $\alpha[i]$ seems to be the prefix of $\alpha$ of length i -
this should be made consistent and defined.} There are two sets, say $X_1$ and $X_2$, and an infinite set $H \subset \N$ such that $\rho_1[n] = \rho_2[n]$ for all $n \in H$ (otherwise from some point on all pairs of runs disagree contradicting that the automaton has $<K$ states). Without loss we can assume, by passing to an infinite subset if required, that for all $n < m \in H$ with no element of $H$ between them, 
both $\rho_1[n,m]$ and $\rho_2[n,m]$ mention final states and $\chi_{X_1}[n,m-1] \neq \chi_{X_2}[n,m-1]$. List $H$ as $h_1 < h_2 < h_3 < \cdots$.
By knitting segments of the runs we see that the automaton accepts every string of the form $\conv(\chi_{X_1}[0,h_1-1]z_1z_2z_3\cdots,\chi_Y)$ where $z_n \in \{\chi_{X_1}[h_n,h_{n+1}-1] , \chi_{X_2}[h_n,h_{n+1}-1]\}$.
This gives uncountably many distinct sets $X$ satisfying $\Phi(X,Y)$.
\end{proof}

Write $\msomod$ for $\mso$ enriched by all quantifiers parameterised by $k,m \in \N$ of the form ``exists a set $X$, whose cardinality is congruent to $k$ modulo $m$, such that \dots''. The proof of the following theorem can be adapted from \cite{KRS04} or \cite{KuLo08JSL}.
\begin{theorem} 
For every $\msomod$ formula $\Phi(\tup{X})$ there is an $\mso$ formula
$\Phi'(\tup{X})$ equivalent to $\Phi(\tup{X})$ over $\frakT_2$.
\end{theorem}

\begin{remark} \todo{update remark... no longer have $T_r$}
Since $\frakT_r$ (for $r < \omega$) is MSO-interpretable in $\frakT_2$, the results above hold, for instance, with $\frakT_1$ replacing $\frakT_2$.
\end{remark}

\subsection{Interpretations}

\todo{you one write about 1-dimensional interpretations. Is this
sufficient?}
\todo{?A good reference for interpretations is \cite{Hodg93}.} \todo{mention interps are often between theories}

Intuitively, an \emph{interpretation} is a tuple $\I = (\delta, \Phi_1, \dots, \Phi_N)$ of formulas that induces a function 
$\frakB \mapsto \I(\frakB) := (\delta^\frakB, \Phi_1^\frakB, \dots, \Phi_N^\frakB)$. Thus, interpretations define one structure $\I(\frakB)$ 
inside another $\frakB$, and allow one to transfer decidability and definability properties from $\frakB$ to $\I(\frakB)$. 
Depending on the types of the free variables we get different types of interpretations.

\subsubsection{Set Interpretations}

\begin{definition}
A \emph{set interpretation} is a tuple  $\I = (\delta, \Phi_1, \dots, \Phi_N)$ of MSO-formulas from a fixed signature such that a) all the free variables occuring in the formulas of $\I$ are set variables, b) $\delta$ has one free variable, and c) the sentences $\forall \tup{X}. \Phi_i(X_1, \dots, X_{r_i}) \to \bigwedge_{j \leq r_i} \delta(X_j)$ are valid. A \emph{finite set interpretation} is the same but with ``WMSO'' replacing ``MSO'' and ``finite set'' replacing ``set''.
\end{definition}

If all formulas of $\I$ are in the signature of $\frakB$ then by condition b) $\delta^\frakB$ consists of subsets of $\B$, that is, $\delta^\frakB \subseteq 2^\B$ 
(recall from Section~\ref{AS:subsec:logic} that if $\delta$ is a WMSO-formula then $\delta^\B$ consists of finite subsets of $\B$). Moreover, by condition c) $\Phi_i^\frakB$ are relations over $\delta^\frakB$.\footnote{Condition c) means that if $\frakB \models \Phi(b_1, \dots, b_{r_i})$ then $\frakB \models \bigwedge_j \delta(b_j)$. This condition is simply a convenience: given a tuple $(\delta, \Phi_1, \dots, \Phi_N)$ satisfying a) and b), the tuple $(\delta, \Psi_1, \dots, \Psi_N)$ where $\Psi_i(x_1, \dots, x_{r_1}) := \Phi_i(\tup{x}) \wedge \bigwedge_{j \leq r_i} \delta(x_j)$ is a set interpretation.}   Thus $\I(\frakB) := (\delta^\frakB, \Phi_1^\frakB, \dots, \Phi_N^\frakB)$ is a structure. 
If $\I(\frakB)$ is isomorphic to the structure $\frakC$ via the isomorphism $\mu:\delta^\frakB \to \C$, then we say that $\I$ \emph{interprets} $\frakC$ in $\frakB$ via co-ordinate map $\mu$. Note that the signature of $\frakC$ consists of $N$ relations, the ith having the same arity as $\Phi_i$. In particular, $\I$ interprets $\I(\frakB)$ in $\frakB$ (via the identity co-ordinate map). Set-interpretations allow one to interpret FO-formulas of $\frakC$ by MSO-formulas of $\frakB$:

\todo{perhaps use f inverse instead of f?}
\begin{lemma} \label{AS:lem:translation:set-interpretation} 
Suppose that $\I$ is a (finite) set-interpretation that interprets the structure $\frakC$ in $\frakB$ via co-ordinate map $\mu$.
Let $\Phi(x_1,\dots,x_k)$ be a FO-formula in the signature of $\frakC$. 
There is a (W)MSO formula $\Phi_\I(X_1,\dots,X_k)$ in the signature of $\frakB$ such that for all (finite) sets $B_i \in \delta^\frakB \subseteq 2^\B$,
\[
\frakC \models \Phi(\mu(B_1),\dots,\mu(B_k)) \mbox{ if and only if } \frakB \models \Phi_\I(B_1,\dots,B_k).
\]
Moreover, the translation $\Phi \mapsto \Phi_\I$ is effective.
\end{lemma}

\begin{proof}
The idea is to replace FO-variables $x$ in $\Phi$ by corresponding set-variables $X$, relativise all quantifiers to $\delta$, and replace 
occurences of the atomic formula $R_i$ (from the signature of $\frakC$) by $\Phi_i$. Formally, define $\Phi_\I$ inductively by 
$(R_i(\tup{x}))_\I  := \Phi_i(\tup{X})$, $(x = y)_\I := X = Y$,
$(\Psi \wedge \Xi)_\I  := \Psi_\I \wedge \Xi_\I$,
$(\neg \Psi)_\I  := \neg \Psi_\I$,
$(\exists x_i \Psi)_\I  := \exists X_i [\delta(X_i) \wedge \Psi_\I]$.~\footnote{Disjunction and universal quantification are definable in terms of the other operations, i.e., negation, conjunction and existential quantification.}\todo{if this footnoted property is used elsewhere, then pull it out near dfn of logic}
\end{proof}



% \begin{definition}
% Let $\I$ be an $\eL$-definition. If $\frakA$ is isomorphic to $\I(\frakB)$, say via $f$, then say that {\em $\frakA$ is  $\eL$-interpretable in $\frakB$ via co-ordinate map $f$}.
% \end{definition}


\begin{lemma} \label{AS:lem:set-interpretation:decidability}
Suppose that $\frakB$ has decidable (W)MSO-theory. If $\frakC$ is (finite) set-interpretable 
in $\frakB$ then $\frakC$ has decidable FO-theory. 
\end{lemma}
\begin{proof}
For a FO-sentence $\Phi$ of $\frakC$, Lemma \ref{AS:lem:translation:set-interpretation} produces a sentence $\Phi_\I$ in the signature of $\frakB$ preserving truth. 
Apply the given decision procedure to $\Phi_\I$. 
\end{proof}

\subsubsection{Point Interpretations}

We define point interpretations which map elements in the interpreted structure to elements (not sets) in the interpreting structure.

\begin{definition}
A \emph{point interpretation} is a tuple  $\I = (\delta, \Phi_1, \dots, \Phi_N)$ of MSO-formulas from a fixed signature such that a) all the free variables occuring in the formulas of $\I$ are first-order variables, b) $\delta$ has one free variable, and c) the sentences $\forall \tup{x}. \Phi_i(x_1, \dots, x_{r_i}) \to \bigwedge_{j \leq r_i} \delta(x_j)$ are valid. Point-interpretations in which all formulas are first-order formulas are called \emph{FO-interpretations}.
\end{definition}

If all formulas of $\I$ are in the signature of $\frakB$ then $\delta^\frakB$ is a subset of $\B$. As for set-interpretations, define 
$\I(\frakB) := (\delta^\frakB, \Phi_1^\frakB, \dots, \Phi_N^\frakB)$, and if $\I(\frakB)$ is isomorphic to the structure $\frakC$ via the isomorphism 
$\mu:\delta^\frakB \to \C$, then we say that $\I$ \emph{interprets} $\frakC$ in $\frakB$ via co-ordinate map $\mu$. 


\begin{lemma} \label{AS:lem:translation:point-interpretation} 
Suppose that $\I$ is a point interpretation that interprets the structure $\frakC$ in $\frakB$ via co-ordinate map $\mu$.
Let $\Phi(x_1,\dots,x_k)$ be a FO-formula (resp. WMSO-formula, MSO-formula) in the signature of $\frakC$. 
There is a FO-formula (resp. WMSO-formula, MSO-formula) $\Phi_\I(X_1,\dots,X_k)$ in the signature of $\frakB$ such that for elements
$b_i \in \delta^\frakB \subseteq \B$,
\[
\frakC \models \Phi(\mu(b_1),\dots,\mu(b_k)) \mbox{ if and only if } \frakB \models \Phi_\I(b_1,\dots,b_k).
\]
Moreover, the translation $\Phi \mapsto \Phi_\I$ is effective.
\end{lemma}

\begin{proof}
The proof is similar to that of Lemma~\ref{AS:lem:translation:set-interpretation} except that we don't replace variables. The only additional cases are $X = Y, X \subseteq Y$ and $x \in Y$. But these are straightforward, e.g., $(x \in X)_\I := x \in X$.
\end{proof}



% \begin{definition}
% Let $\I$ be an $\eL$-definition. If $\frakA$ is isomorphic to $\I(\frakB)$, say via $f$, then say that {\em $\frakA$ is  $\eL$-interpretable in $\frakB$ via co-ordinate map $f$}.
% \end{definition}
The following is immediate:

\begin{proposition}
Let $\eL$ be one of FO, WMSO, or MSO. Suppose that $\frakB$ has decidable $\eL$-theory. 
If $\frakC$ is point-interpretable in $\frakB$ then $\frakC$ has decidable $\eL$-theory.
\end{proposition}

% For example  for $r < \omega$ the structure $\frakT_r$ is point interpretable in $\frakT_2$ and consequently the MSO-theory of $\frakT_r$ is decidable.


% The other types of interpretations that are central to this story are (finite)-set interpretations in which elements of $\frakA$ are coded by (finite) 
% subsets of the domain of $\frakB$~\cite{ElRa66,CoLo07}.
% \begin{definition}
% Let $\frakA,\frakB$ be relational structures (possibly over different signatures). Suppose  $\frakA = (\A,R_1, \dots, R_N)$ and $R_i$ has arity $r_i$. 
% Say that $\frakA$ is \emph{set-interpretable} in $\frakB$ via $\left<f,\I\right>$ if
% \begin{enumerate}
% \item $f:\A \to 2^\B$ is a function mapping elements of $\frakA$ to subsets of $\frakB$, 
% \item $\I = (\delta, \Phi_1, \dots, \Phi_N)$ is a tuple of MSO-formulas in the signature of $\frakB$ in which the free variables are set variables, 
% $\delta$ has $1$ free variable, and $\Phi_i$ has $r_i$ free variables,
% %and $\Phi_i$ has, say, $r_i$ free variables.
% \item the set $\delta^\frakB$ (of subsets of $\frakB$ on which $\delta$ holds) is equal to $f(A)$, and
% \item the set $\Phi_i^\frakB$ (of tuples of subsets of $\frakB$ on which $\Phi_i$ holds) is equal to the set of tuples $(f(a_1),\dots,f(a_{r_i}))$ such that
% $\Phi(a_1,\cdots,a_{r_i})$ holds.
% \end{enumerate}
% Define the structure 
% $\I(\frakB) := (\delta^\frakB, \Phi_1^\frakB, \dots, \Phi_N^\frakB).$ 
% This structure is said to be {\em $\eL$-definable in $\frakB$}. The tuple $\I$ is called an {\em $\eL$-definition}.
% 
% A {\em (finite)-set interpretation} of $\frakA$ in $\frakB$ to be like a (W)MSO-interpretation except
% that the free variables are (finite) set
% variables. 
% 
% \end{definition}

\iffalse
?? ALT DFN ??

\begin{definition}
\begin{enumerate}
\item A structure FO-interpretable in $\Power_f(\frakT_1)$ is called {\em finite-string automatic}.
\item A structure FO-interpretable in $\Power(\frakT_1)$ is called {\em $\omega$-string automatic}.
\item A structure FO-interpretable in $\Power_f(\frakT_2)$ is called {\em finite-tree automatic}.
\item A structure FO-interpretable in $\Power(\frakT_2)$ is called {\em $\omega$-tree automatic}.
\end{enumerate}
\end{definition}
\fi
%=====================================================
\section{Basics of automatic structures} \label{AS:sec:autstr}
%=====================================================

\subsection{Rabin-automatic structures}

% Since formulas are seen as automata, we may view a Rabin-automatic structure $\frakA$ as being presented by automata $(M,M_1,\dots,M_N)$ where 
% $M$ is an automaton of $\delta^\frakA$, and $M_i$ is an automaton for $\Phi_i^\frakA$. This is sometimes taken as the definition. 
% Write $\conv^{-1} L(M_i)$ for the set of tuples $(t_1, \dots, t_{r_i})$ of trees such that $\conv(t_1, \dots, t_{i_r}) \in L(M_i)$.


\begin{definition}[$\omega$-tree automatic presentation \cite{Blum99,BlGr00}] \label{AS:dfn:rap} % \begin{definition} \label{AS:dfn:raut} \cite{Blum99} 
An {\em $\omega$-tree automatic presentation} is a tuple $\J = (M, M_1, \dots, M_N)$ of $\omega$-tree automata such that 
\[
\frakJ = (L(M), \conv^{-1} L(M_1), \dots, \conv^{-1} L(M_N)) 
\]
is a structure, namely, $\conv(\tup{t}) \in L(M_i)$ implies $t_j \in L(M)$ for each $j \leq r_i$.
If $\frakJ$ is isomorphic to a structure $\frakC$ via an isomorphism $\mu:L(M) \to \C$, then we say that $\J$ is an 
 {\em $\omega$-tree automatic presentation of $\frakC$ via co-ordinate map $\mu$}. In short, we say that $\frakC$ is \emph{Rabin-automatic} or \emph{$\omega$-tree automatic} (via co-ordinate map $\mu$). The collection of Rabin-automatic structures is written $\raut$.
\end{definition}

A set interpretation of $\frakA$ in $\frakT_2 = \two$ gives a way to reduce FO-formulas of $\frakA$ into MSO-formulas of $\frakT_2$, which in turn can be represented by  $\omega$-tree automata. By transitivity, FO-formulas of $\frakA$ can be represented by $\omega$-tree automata. This suggests defining the Rabin-automatic structures as those that are set-interpretable in $\frakT_2$. Naturally, the two definitions are equivalent:

\begin{proposition}\cite{BlGr00}
A structure is set-interpretable in $\frakT_2$ if and only if it has an $\omega$-tree automatic presentation. 
Moreover, translations between set interpretations and automatic presentations are effective.
\end{proposition}

\begin{proof}
\todo{check!}
Suppose the set interpretation $\I = (\delta, \Phi_1, \dots, \Phi_N)$ interprets $\frakC$ in $\frakT_2$ via co-ordinate mapping $\mu$. Let $M$ be an $\omega$-tree automaton for the language $\delta^{\frakT_2}$, and $M_i$ an $\omega$-tree automaton for the languages $\conv \Phi_i^{\frakT_2} = \{\conv(\tup{t}) \st \frakT_2 \models \Phi_i(\tup{t})\}$. These automata exist and are computable by Theorem~\ref{AS:thm:RABIN} (Rabin's Theorem). So, $(M,M_1, \dots, M_N)$ is an $\omega$-tree automatic presentation of $\frakC$ via co-ordinate map that sends a tree $t \in L(M)$ to the element $\mu^{-1}(\chi^{-1}(t))$.

Conversely, suppose $(M,M_1, \dots, M_N)$ is an $\omega$-tree automatic presentation of $\frakC$ via co-ordinate mapping $\mu$. By Theorem~\ref{AS:thm:Rabin} (Rabin's Theorem) there exist MSO-formula $\delta$, $\Phi_1, \dots, \Phi_N$ such that the language of $M$ is $\{\chi(X) \st \frakT_2 \models \delta(X)\}$ and the language of $M_i$ is $\{\conv(\chi({X_1}),\dots,\chi({X_k})) \st \frakT_2 \models \Phi_i(\tup{X})\}$. So, $(\delta,\Phi_1, \dots, \Phi_N)$ is a set-interpretation of $\frakC$ in $\frakT_2$ via co-ordinate map that sends a set $X \in \delta^{\frakT_2}$ to the element $\mu^{-1}(\chi(X))$.
\end{proof}

% The collection of these structures is written $\raut$.

\todo{ do one-dimensional interpretations suffice?}
% \end{definition}

% The elements of the Rabin-automatic structure are naturally viewed as $\omega$-trees. 




\begin{example}
 The following structures are Rabin-automatic. \todo{ you should, at least, provide proof ideas or references for
these claims.}
\begin{enumerate}
 \item The power structure $\Power[\frakT_r]$ ($r \geq 1$) as well as its substructure $\Power_f[\frakT_r]$ whose domain consists of the finite subsets of $\frakT_r$.
 \item The power structure of the ordering of the rationals, namely $\Power[(\mathbb{Q},<)]$.
 \item Presburger arithmetic $(\N,+)$.
 \item Skolem arithmetic $(\N,\times)$.
 \item The structure $(\Power(\N),\subseteq, =^*)$ where $X =^* Y$ means that $X$ and $Y$ have finite symmetric difference.
 \item Every ordinal $(\alpha,<)$ where $\alpha < \omega^{\omega^\omega}$.
\end{enumerate}
\end{example}


\subsection{The fundamental theorem}

The {\em fundamental theorem of automatic structures} says that FO-definable relations in Rabin-automatic structures are, modulo coding, regular. 

\begin{definition}
Let $\frakA$ be Rabin-automatic via co-ordinate map $\mu$. For every relation $R \subseteq \A^k$, denote by
$\code{R}$ the set of $\omega$-trees 
\[\{\conv(\chi({\mu^{-1}(a_1))},\dots,\chi({\mu^{-1}(a_k)})) \st  (a_1,\dots,a_k) \in R\}.\]
\end{definition}


\begin{theorem}[Fundamental theorem, cf. \cite{KhNe95,BlGr00}] \label{AS:thm:fundthm} 
Let $\frakA$ be Rabin-automatic via co-ordinate map $\mu$.
\begin{enumerate}
\item For every first-order definable relation $R$ in $\frakA$ the set of trees $\code{R}$ is recognised by an $\omega$-tree automaton.
\item The first-order theory of a Rabin-automatic structure is decidable.
\end{enumerate}
\end{theorem}

\begin{proof}
The first item follows from Lemma~\ref{AS:lem:translation:set-interpretation} and Theorem~\ref{AS:thm:Rabin} (Rabin's Theorem). The second item follows from Lemma~\ref{AS:lem:set-interpretation:decidability} and Theorem~\ref{AS:thm:Rabin:dec}.
\end{proof}

We now rephrase the results on additional quantifiers on MSO into FO. We overload notation so that, for example, $\exists^\kappa$ denotes the quantifier `there exists at least $\kappa$ many individual elements $x$ such that'. The following theorem is now immediate. Instances of it can be found in \cite{BlGr00,KRS04,KuLo08JSL}.

\begin{theorem}[Extension of fundamental theorem] \label{AS:thm:FOext}
Let $\frakA$ be Rabin-automatic.
\begin{enumerate}
\item For every $\FOext$-definable relation $R$ in $\frakA$ the set $\code{R}$ is recognised by an $\omega$-tree  automaton.
\item The $\FOext$ theory of $\frakA$ is decidable.
\end{enumerate}
\end{theorem}

How far can we push this? First we need a rigourous definition of quantifier.
This is neatly provided by Lindstr\"om's definition of `generalised quantifier',
see \cite{Lind66}. We don't have a clear picture of those generalised quantifiers that can
be added to FO and still get the properties as in Theorem \ref{AS:thm:FOext}.
However here is a special case. Define the
{\em cardinality quantifier parameterised by $C$}, for $C$  a class of
cardinals, as `there exists exactly $\alpha$ many elements $x$ such that \ldots,
where $\alpha \in C$'. Examples include $\exists, \existsmod,\exists^{\geq \aleph_0}$, and
`there exist a prime number of elements such that \dots'.

It turns out that the only cardinality quantifiers we can add to FO and still get the fundamental theorem are, essentially, 
the ones mentioned in Theorem \ref{AS:thm:FOext}.

\begin{theorem} [cf. \cite{Rubi08}]
Let $Q_C$ be a cardinality quantifier parameterised by $C$.  Suppose for every
$\frakA \in \raut$ and every $\FO(Q)$-definable relation $R$ in $\frakA$,
the set $\code{R}$ is recognised by a Rabin-automaton.  Then every $\FO(Q)$-definable relation 
is already $\FOext$-definable in $\frakA$. 
\end{theorem}


\begin{proof} 
We illustrate the proof for a set $C \subset \N$ of finite cardinals.
Consider the Rabin-automatic presentation of  $\frakA := (\N,\leq)$ in which $n \in \N$ is coded
by the set $\{0^n\} \subset \{0,1\}^\ast$. Since the set $C \subset \N$  is $\FO(Q_C)$-definable in $\frakA$, $\code{C}$ 
is recognised by a tree-automaton. But the trees in $\code{C}$ are essentially unary strings and so the language $\code{C}$ is ultimately periodic. So $C$ is already $\FO(\existsmod)$-definable in $\frakA$. 
\end{proof}

What about extensions of FO by set quantification?  Unfortunately $\wmso$ is
too much to hope for.  Since the configuration graph $\frakG$ of a Turing machine (with the
single-transition edge relation) is automatic, and reachability
is expressible in $\wmso$, the halting problem reduces to the $\wmso$-theory of a certain $\frakG$. It is a research programme to understand which quantifiers can be added to automatic structures and retain decidability, see \cite{Rubi04,KuLo08JSL}. 

%=====================================================
%=====================================================
%=====================================================
%Part 2

%%%%%%%%%%%%%%%%%%%%%%%%%%%%%%%%%%%%%%%%%%%%%%%%%%%%%%%%%%%%%%%%%%%%%%%%%%%%%%%%%%%%%%%%%%%%%%%%%%%%%%%%%%%%%%%%%%%%%%%%%5
%%%%%%%%%%%%%%%%%%%%%%%%%%%%%%%%%%%%%%%%%%%%%%%%%%%%%%%%%%%%%%%%%%%%%%%%%%%%%%%%%%%%%%%%%%%%%%%%%%%%%%%%%%%%%%%%%%%%%%%%%5
%%%%%%%%%%%%%%%%%%%%%%%%%%%%%%%%%%%%%%%%%%%%%%%%%%%%%%%%%%%%%%%%%%%%%%%%%%%%%%%%%%%%%%%%%%%%%%%%%%%%%%%%%%%%%%%%%%%%%%%%%5
\subsection{Other classes of automatic structures}
%%%%%%%%%%%%%%%%%%%%%%%%%%%%%%%%%%%%%%%%%%%%%%%%%%%%%%%%%%%%%%%%%%%%%%%%%%%%%%%%%%%%%%%%%%%%%%%%%%%%%%%%%%%%%%%%%%%%%%%%%5
%%%%%%%%%%%%%%%%%%%%%%%%%%%%%%%%%%%%%%%%%%%%%%%%%%%%%%%%%%%%%%%%%%%%%%%%%%%%%%%%%%%%%%%%%%%%%%%%%%%%%%%%%%%%%%%%%%%%%%%%%5
%%%%%%%%%%%%%%%%%%%%%%%%%%%%%%%%%%%%%%%%%%%%%%%%%%%%%%%%%%%%%%%%%%%%%%%%%%%%%%%%%%%%%%%%%%%%%%%%%%%%%%%%%%%%%%%%%%%%%%%%%5

We introduce subclasses of $\raut$ related to automata on finite strings/trees and infinite strings. 
Each has a machine theoretic characterisation as in Definition \ref{AS:dfn:rap}.
A member of any of the four classes is said to be {\em automatic}. 

\subsubsection*{Finite-string automatic structures}

Recall  $\Power_f(\frakA)$ is the substructure of $\Power(\frakA)$ restricted to the finite subsets of $\A$.

\begin{definition} 
A structure is called {\em finite-string automatic} if it is FO-interpretable in $\Power_f[\frakT_1]$. This collection of structures is written $\waut$.
\end{definition}

We have seen in Corollary \ref{AS:cor:pres} that $(\N,+)$ is {\em finite-string automatic}. 
Note that such structures have countable domain. 

\begin{example} \todo{mention that "bi-interpretable" refers to FO}
The following structures are bi-interpretable with $\Power_f[\frakT_1]$:
\begin{enumerate} 
\item For $|\Sigma| \geq 2$, $(\Sigma^\ast,\{\sigma_a\}_{a \in \Sigma},\pref,\el)$ 
where $\sigma_a$ holds on pairs $(w,wa)$, $\el$ holds on pairs $(u,v)$ such that $|u|=|v|$, and $\pref$ is the prefix relation.
\item For $k \geq 2$, $(\N,+,|_k)$
where $|_k$ is the binary relation on $\N$ with $x |_k y$ if $x$ is a power of $k$ and $x$ divides $y$.
\end{enumerate}
\end{example}

The following definition turns a finite set $Y$ into a finite string $\chi_Y$ as in $\chi_{\{1,3\}} = 0101$.

\begin{definition}[characteristic finite-string]
For a finite set $Y \subset \N$ define
its {\em characteristic string} $\chi_Y$ as the $\{0,1\}$-labeled string of length $\max_{y \in Y} y + 1$ with a $1$ in position $n$ if and only if $n \in Y$.
\end{definition}


\begin{definition}[convoluting finite strings]
Let $\tup{w} = (w_1,\dots,w_k)$
be a $k$-tuple of $\{0,1\}$-labelled finite strings. Let $l:= \max_i |w_i|$.
The {\em convolution} $\conv(\tup{w})$ is the $\{0,1,\blank\}^k$-labelled string of length $l$ such that for all positions $n \leq l$
the $i$th component of $\conv(\tup{w})[n]$ is equal to $w_i[n]$ if $n \leq |w_i|$ and the blank symbol $\blank$ otherwise.
\end{definition} 


\begin{definition}
Suppose $f$ is an isomorphism witnessing $\frakA \in \waut$. For a relation $R \subseteq \A^k$ denote by
$\code{R}$ the set of finite-strings  $\{\conv(\chi_{f(a_1)},\dots,\chi_{f(a_k)}) \st  \tup{a} \in R\}$. 
\end{definition}

Just as for Rabin-automatic structures, there is a fundamental theorem for finite-string automatic structures. We do not state it in full;
simply replace Rabin-automatic by finite-string automatic and $\omega$-tree automata by finite-string automata.
However, we do slightly generalise the analogous definition of automatic presentation to cover an arbitrary alphabet $\Sigma$.

\begin{definition}[finite-string automatic presentation \cite{KhNe95}] \label{AS:dfn:fsap}
Fix a finite alphabet $\Sigma$. Suppose that $f: \frakA \simeq  (\B,S_1,\dots,S_N)$ and
\begin{enumerate}
\item the elements of $\B$ are finite strings from $\Sigma^\ast$;
\item the set $\B$ is recognised by a finite-string automaton, say $M_\B$; 
\item the set $\{\conv(\tup{t}) \st \tup{t} \in S_i\}$ is recognised by a finite-string automaton, say $M_i$, for $i \leq N$.
\end{enumerate}
The data $\left<(M_\B,M_1,\dots,M_N), f \right>$ is called a {\em finite-string automatic presentation} of $\frakA$.
\end{definition}

\begin{proposition}[Machine theoretic characterisation \cite{BlGr00}] \label{AS:prop:MTC}
A structure is FO-interpretable in $\Power_f[\frakT_1]$ if and only if it 
has a finite-string automatic presentation over an alphabet with $|\Sigma| \geq 2$.
\end{proposition}

\begin{proof}
If $|\Sigma| > 2$ recode a string $w$ by replacing individual symbols with binary blocks of size $\log_2 |\Sigma|$.
\end{proof}

\subsubsection*{Relationships amongst the classes of automatic structures}

There are two more standard classes of automatic structures, see \cite{Blum99}.
A {\em B\"uchi-automatic structure} is one FO-interpretable in $\Power[\frakT_1]$. Collectively these are denoted $\baut$. An {\em $\omega$-string} is a function from $\N$ to $\{0,1\}$ and automata operating on (convolutions of tuples of) these are called {\em $\omega$-string automaton}. One can similarly characterise $\omega$-string automatic structures as those with {\em $\omega$-string automatic presentations}. For example, $([0,1),+,<) \in \baut$ where $+$ is taken modulo $1$ (the usual binary coding works).
A {\em finite-tree automatic structure} is one FO-interpretable in $\Power_f[\frakT_2]$. Collectively these are denoted $\taut$. A {\em finite tree} is a function from a finite prefix-closed subset of $\twodom$ to $\{0,1\}$ and automata operating on (convolutions of tuples of) these are called {\em finite-tree automata}. These automatic structures are those with {\em finite-tree automatic presentations} (see \cite{Blum99,BeLiNe07}). For example  $(\N,\times) \in \taut$ (decompose $n$ into
$\prod_{i} p_i^{e_i}$ where $p_i$ is the $i$th prime and so code $n$ as a tree with $e_i$ written in binary on
branch $0^i1^\ast$).  
%For the case of unranked finite-tree automatic structures see \cite{BeLiNe07}.


\begin{proposition} \label{AS:prop:relations}
$\waut$ is a proper subset of $\baut$ and of $\taut$, each of which is a proper subset of $\raut$.
\end{proposition} 

\begin{proof} The inclusions follow since finite-strings are special cases of finite-trees, etc.
The structure of Skolem arithmetic $(\N,\times)$ separates $\taut$ from $\waut$ (see \cite{Blum99}).
We will see (Theorem \ref{AS:thm:sep}) that a structure separating $\raut$ from $\baut$ is $(\Power(\{l,r\}^\star),\subset,V)$ where $V$ is the unary relation consisting of those sets $X$ such that the characteristic tree of $X$ (a function $\{l,r\}^\star \to \{0,1\}$) has the property that every infinite path is labelled with only finitely many $1$s.
\end{proof}

\begin{proposition} \todo{missing citation}
If $\frakA \in \baut$ is countable then $\frakA \in \waut$.
If $\frakA \in \raut$ is countable then $\frakA \in \taut$.
\end{proposition}

\begin{proof}
The reason is that every $\omega$-string automaton whose language is countable accepts only ultimately periodic strings with a uniform bound on the length of the periods. A similar conditions holds for $\omega$-tree automata that accept countable languages.
\end{proof}
%
%=====================================================
%=====================================================
%=====================================================
\subsection{Operations on automatic structures}
%=====================================================
%=====================================================
%=====================================================

This section gives some basic answers to the question:
\begin{quote}
Which operations on automatic structures preserve automaticity?
\end{quote}
 See \cite{BCL07} for a survey of operations that preserve decidability.\todo{explan citation better... in what context?}
 
\subsubsection*{Closure under interpretations}
Let $\av$ stand for `finite-string', `$\omega$-string', `finite-tree', or `$\omega$-tree'.
Since FO-definitions compose we have that:
\begin{proposition} \cite{Blum99}
The $\av$-automatic structures are closed under FO-interpretations.
\end{proposition}

\begin{example}
$\av$-automatic structures are closed under FO-definable expansions: if $\frakA$ is $\av$-automatic and $\Phi$ is a FO-formula over the signature of $\frakA$ then $(\frakA,\Phi^\frakA)$ is $\av$-automatic.
\end{example}

There is a more general notion of interpretation called a {\em FO-interpretation of dimension $d$}.
Here $\delta$ has $d \in \N$ free variables and each $\Phi_i$ has $d\times r_i$ free variables.

\begin{proposition} \cite{BlGr00}
The $\av$-automatic structures are closed under  FO-interpretations of dimension $d$.
\end{proposition}

\begin{proof}
We illustrate the idea for $\waut$. Suppose $\frakB \in \waut$. It is enough to show that if $\frakA = \I(\frakB)$ then we can find a finite-string automatic presentation of $\frakA$. 
An element $a$ of $\frakA$ is a $d$-tuple of elements $(b_1,\dots,b_d)$ of $\frakB$ each of which is coded by a finite string 
$\code{b_i}$. Coding the element $a$ by the string $\conv(\code{b_1},\dots,\code{b_d})$ we get a finite-string automatic presentation of $\frakA$ over alphabet 
$\{0,1,\blank\}^d$. By Proposition~\ref{AS:prop:MTC} $\frakA$ is in $\waut$.
\end{proof}


Say $\frakA$ and $\frakB$ are FO-interpretable in $\frakU$. Then the disjoint union of $\frakA$ and $\frakB$ is $2$-dimensionally interpretable in $\frakU$. Similarly for their direct product.
Thus $\av$-automatic structures are closed under disjoint union and direct product.

The {\em (weak) direct power} of $\frakA$ is a structure with the same signature as $\frakA$, its domain consists 
of (finite) sequences of $\A$, and the 
interpretation of a relation symbol $R$ is the set of sequences $\sigma$ such that $R^\frakA(\sigma(n))$ holds for all $n$.
For example the weak direct power of $(\N,+)$ is isomorphic to $(\N,\times)$; the isomorphism sends $n$ to  the finite sequence $(e_i)_i$ where
$ \prod p_i^{e_i}$ is the prime power decomposition of $n$. Since $(\N,\times)$ is neither in $\waut$ nor $\baut$, these string classes are not closed under weak direct power (or direct power).
 
\begin{proposition} \label{AS:prop:powerclosure} \cite{Blum99}
Each of  $\taut$ and $\raut$ is closed under weak direct power. The class $\raut$ is closed under direct power.
\end{proposition}

\begin{proof}
We illustrate the second statement.
Let $\frakA$ be a Rabin-automatic structure with relation symbol $R$.
Let $\sigma = (a_n)_n$ be an element of the direct power of $\frakA$.
Code the sequence $\sigma$ by the tree $t_\sigma$ whose subtree at $0^n1$ is the tree $\code{a_n}$.
The interpretation of $R$ in the direct power is recognised by a tree automaton: it  processes $t_\sigma$ by checking
that the convolution of the subtrees rooted at $0^n1$ is recognised by the automaton for $R^\frakA$.
\end{proof}

\subsubsection*{Closure under quotients} \label{AS:subsub:quotient}
Let $\frakA = ({\A},R_1,\dots,R_N)$ be a structure.
An equivalence relation $\epsilon$ on the domain ${\A}$ is called a {\em congruence for $\frakA$} if each relation $R_i$ satisfies the following property:
for every pair of $r_i$-tuples $\tup{a},\tup{b}$ of elements of ${\A}$, if $(a_j,b_j) \in  \epsilon$ for $1 \leq j \leq r_i$ then $R_i(\tup{a})$ if and only if $R_i(\tup{b})$.
The {\em quotient of $\frakA$ by $\epsilon$}, written $\frakA/\epsilon$ is the structure whose domain is the set of equivalence classes of $\epsilon$ and whose $i$th relation
is the image of $R_i$ by the map sending $u \in {\A}$ to the equivalence class of $u$. Consider the question:
If $(\frakA,\epsilon)$ is $\av$-automatic, is $\frakA/\epsilon$ $\av$-automatic?\footnote{This question is sometimes phrased as asking if every structure with a non-injective automatic presentation has an injective automatic presentation.}

\subsubsection*{$\waut$:} Yes \cite{BlGr00}.
There is a regular well-ordering of the set of finite strings, for instance the length-lexicographic
ordering $\llex$. Use this order to define a regular set $D$ of unique $\epsilon$-representatives.
Then restrict the presentation of $\frakA$ to $D$ to get a presentation of $\frakA / \epsilon$.

\subsubsection*{$\taut$:} Yes \cite{CoLo07}.
As opposed to the finite string case, there is no regular well ordering of the set of
all finite trees. \todo{where is the non-existence of a regular well-ordering shown?
did you define what a regular relation is (I don't think you did)?}
However one can still convert a finite-tree automatic presentation of $(\frakA,\epsilon)$ into
one for $\frakA/\epsilon$. The idea is to associate with each tree $t$ a
new tree $\hat{t}$ of the following form: the domain is the intersection 
of the prefix-closures of the domains of all trees that are $\epsilon$-equivalent to $t$; 
a node is labelled $\sigma$ if $t$ had label $\sigma$ in that position; 
a leaf $x$ is additionally labelled by those states $q$ from which the 
automaton for ${\epsilon}$ accepts the pair consisting of the subtree of $t$ 
rooted at $x$ and the tree with empty domain.
Using transitivity and symmetry of $\epsilon$, if $\hat{t} = \hat{s}$ 
then $t$ is $\epsilon$-equivalent to $s$. 
Moreover each equivalence class is associated with finitely many new trees, 
and so a representative may be chosen using any fixed regular linear ordering 
of the set of all finite trees.\footnote{The construction 
given in \cite{CoLo07} is slightly more general and allows one to effectively 
factor finite set interpretations in any tree.}
\todo{changed ``finite subset ints'' to ``finite set ints''}

\subsubsection*{$\baut$:} It depends.
Kuske and Lohrey \cite{KuLo08JSL} observed that there is no set of unique representatives of the equal almost-everywhere relation $\sim_{\textrm{ae}}$ that is regular.
Thus we can't quotient using the trick that worked for $\waut$. In fact, there is a structure in $\baut$ whose quotient is not $\omega$-string 
automatic \cite{HKMN08}. The proof actually shows that the structure has no Borel presentation, see Theorem \ref{AS:thm:borel}.
However, every regular $\omega$-string equivalence relation with countable index has an $\omega$-regular set of unique representatives.
This follows from the following more general result  \cite{BKRu08}:
If $\epsilon$ has countable index on $A$ then there exist finitely many $\omega$-strings $x_1,\dots,x_c$ 
so that every $x$  in $A$ is $\epsilon$-equivalent to some $y$ which is $\sim_{\textrm{ae}}$-equivalent to some  $x_i$.  Thus if $\frakA /\epsilon$ is 
countable and $(\frakA,\epsilon) \in \baut$ then $\frakA/\epsilon \in \baut$.

\subsubsection*{$\raut$:} A finer analysis shows that the quotient of the Rabin-automatic structure from the previous paragraph is not in $\raut$ \cite{HKMN08}.
Thus $\raut$ is not closed under quotients. However it is not yet known what happens in the case that congruence-relation has countable index.
~\\

Nonetheless, quotients still have decidable theory since to decide the truth in the quotient structure, for a given sentence replace $=$ by $\epsilon$ and apply the decision procedure for $(\frakA,\epsilon)$.

\begin{proposition} \cite{CoLo07}
If $(\frakA,\epsilon)$ is $\av$-automatic then the quotient $\frakA/\epsilon$ has decidable FO-theory.
\end{proposition}

\subsubsection*{Countable elementary substructures}

In this section we show a way of producing, from an uncountable B\"uchi- or Rabin-automatic structure $\frakB$, a countable substructure $\frakA$
with the same theory. Thus although $\frakA$ may not itself be automatic, it has decidable theory.


\begin{example}
This example is taken from \cite{Dauc93} (pg. $106$). Code reals in base $2$ and so get an $\omega$-string automatic presentation of  $\frakA := (\R,+,<)$. The substructure of $\frakA$ consisting of those reals coded by ultimately periodic $\omega$-strings is isomorphic to $\frakB := (\Q,+,<)$. We will see that the $\FO$-theory of these two structures are  identical and so conclude that $(\Q,+,<)$ has decidable $\FO$-theory. A breakthrough paper establishes that $(\Q,+)$ is not in $\waut$ \cite{Tsan11}. 
%For further work towards classifying the torsion-free abelian groups that are in $\waut$ see \cite{BrSr11}. 
It is not known whether or not $(\Q,+,<) \in \raut$. 
\end{example} 

Two structures with the same FO-theory are called {\em elementary equivalent}. Let $\frakA,\frakB$ have the same signature. 
Say that $\frakA$ is an {\em elementary substructure} of $\frakB$ if
$\A \subseteq \B$ and for all formulas $\Phi(\tup{x})$ and all $\tup{a}$ from $\A$,
\[
\frakA \models \Phi(\tup{a}) \mbox{ if and only if } \frakB \models \Phi(\tup{a}) \hspace{5pt} (\dagger)
\]

Then in particular: $\frakA$ and $\frakB$ are elementary equivalent (take $\Phi$ to be a sentence) 
and $\frakA$ is a substructure of $\frakB$ (they agree on the atomic relations of $\frakA$).
There is a simple characterisation of being an elementary substructure.

\begin{lemma}[Tarski-Vaught \cite{Hodg93}]
Let $\frakA$ be a substructure of $\frakB$.
Then $\frakA$ is an elementary substructure of $\frakB$ if and only if  for every FO-formula $\Phi(x,\tup{y})$ and all $\tup{a}$ from $\A$
\[
\frakB \models \exists x \Phi(x,\tup{a}) \,  \implies\,  \frakA \models \exists x \Phi(x,\tup{a}).
\]
\end{lemma}

Say $f:\frakA \simeq (\B,S_1,\dots,S_N)$ is an $\omega$-string automatic presentation of $\frakA$. %??? technically, a presentation includes the automata.
Write $\frakA_{\rm{up}}$ for the substructure of $\frakA$ isomorphic via $f$ to the substructure whose domain consists of the ultimately periodic strings from $\B$.
Similarly if $\frakA \in \raut$ define $\frakA_{\rm{reg}}$ as the substructure of $\frakA$ isomorphic via $f$ to the substructure consisting of regular trees from $\B$.

\begin{proposition} \cite{BKRu08,HKMN08}
\begin{enumerate}
\item Let $\frakA \in \baut$. The structure $\frakA_{\rm{up}}$ is an elementary substructure of $\frakA$.
\item Let $\frakA \in \raut$. The structure $\frakA_{\rm{reg}}$ is an elementary substructure of $\frakA$.
\end{enumerate}
\end{proposition}

\begin{proof}
Use the fact that an automaton --- possibly instantiated with ultimately periodic strings $\tup{a}$ --- has non-empty language only if it contains an ultimately periodic string. Similarly for the tree case with `regular tree' replacing 'ultimately periodic string'. 
\end{proof}



Similar reasoning shows that $\frakA_{\rm{reg}}$ and $\frakA_{\rm{up}}$ have decidable $\FOextcount$-theory.




%=====================================================
%=====================================================
%=====================================================
\section{Proving a structure has no automatic presentation}
% \Anatomy of automatic structures}
%=====================================================
%=====================================================
%=====================================================

\todo{give intro}

%Not much is known about uncountable automatic structures. For instance, the simplest technique for showing that a structure
%is not in $\raut$ is to show that its theory is sufficiently complex, for instance that the $\FOext$-theory is undecidable.
%In this section we ask and give partial answers to the question:
% \begin{quote}
% What do structures in $\waut$ or $\taut$ look like?
% \end{quote}
Here is a useful pumping observation:


%note needed... Write $h(x)$ for the height of a tree $x$, and $h(\tup{x})$ for the height of the largest tree in $\tup{x}$.

%?? generalise this to loc fin relations ?? where?
\begin{proposition} \label{AS:prop:locfin} \cite{KhNe95}
Suppose that the partial function $F:A^n \to A$ is finite-string/tree regular, and let $p$ be the number of states of the automaton.
If $\tup{x}$ is in the domain of $F$ then the length/height of the string/tree $F(\tup{x})$ is at most $p$ more than the length/height of the largest string/tree in $\tup{x}$.
\end{proposition}

\begin{proof}
Otherwise, take a counterexample  $\tup{x}$.
After all of $\tup{x}$ has been read, and while still reading $F(\tup{x})$, some path in the run must have a repeated state. 
So the automaton also accepts infinitely many tuples of the form $(\tup{x}, \cdot)$ contradicting the functionality of $F$.
\end{proof}

\subsubsection*{Growth of generation}

\begin{definition} \label{dfn:growth} \cite{KhNe95}
Let $\frakA$ be a structure with functions $f_1, \dots, f_k$ of arities $r_1, \dots, r_k$ respectively. 
Let $D \subset \A$ be a finite set.
Define the
{\em $n$th growth level}, written $G_n(D)$, inductively by $G_0(D) = D$
and $G_{n+1}(D)$ is the union of $G_n(D)$ and
\[
\bigcup_{i\leq k} \{f_i(x_1,\dots,x_{r_i}) \st x_j \in G_n(D) \text{ for } 1 \leq j \leq r_i\}.
\]
\end{definition}

How fast does $|G_n(D)|$ grow as a function of $n$?  For
example, consider the free group with generating
set $D = \{d_1, \dots, d_m\}$. For $m \geq 2$ the set $G_n(D)$ includes all strings over $D$ of length (in the generators)
at most $2^n$; so  $|G_n(D)|$ is at least $m^{2^{n}}$.

\begin{proposition} \cite{KhNe95} 
 \label{AS:prop:growth}
Let $\frakA \in \taut$ and $D \subset \A$ be a finite set. Then there is a
linear function $t:\N \rightarrow \N$ so that for all $e \in G_n(D)$ the tree $\code{e}$ has
height at most $t(n)$.
\end{proposition}
\begin{proof}
 Iterate Proposition~\ref{AS:prop:locfin}.
\end{proof}

\begin{corollary} 
If $\frakA \in \taut$ then $|G_n(D)| \leq 2^{2^{O(n)}}$. If $\frakA \in \waut$ then 
$|G_n(D)| \leq 2^{O(n)}$.
\end{corollary}
\begin{proof}
Count the number of $\{0,1\}$-labelled trees (strings) of height at most $k$.
\end{proof}

Thus the free group on more than one generator is not in $\waut$.

\subsubsection*{Growth of projections}

We will be considering structures $(\frakA,R)$ where $R$ is a relation of arity $> 1$. For a tuple $\tup{u}$ of elements from $\A$ define
$R(\cdot,\tup{u}) := \{a \in \A \st (a,\tup{u}) \in R\}$.

\begin{definition}
For finite $E \subset \A$ the  {\em shadow cast by $\tup{u}$ on $E$ via $R$} is the set 
$R(\cdot,\tup{u}) \cap E$ and the {\em shadow count of $E$ via $R$} is the number of distinct shadows
cast on $E$ via $R$ as $\tup{u}$ varies over tuples of elements of $\A$. We may suppress mention of $R$.
\end{definition}

\todo{"random graph" needs a reference and an explanation since
computer scientist often do not know this graph (and, even worth, have
their own understanding of a random graph in algorithmic considerations)}
For example the random graph $(\A,R)$ has the property that for every pair of disjoint finite sets $E,F \subset \A$ there is a point $x \in \A$ that has an edge to every element in $E$ and to no element of $F$. So for a given finite $E \subset \A$, the shadow count of $E$ via $R$ is 
the largest possible, namely $2^{|E|}$. The following propositions are due to Christian Delhomm{\'e} \cite{Delh04} (the first one independently due to Frank Stephan) and limit the possible shadow counts in automatic structures.

\begin{proposition}
Suppose $(\frakA,R) \in \waut$.
Then there is a constant $k$,  that depends on the automata for domain ${\A}$ and $R$, and arbitrarily large finite subsets $E \subset {\A}$ such
that the shadow count of $E$ via $R$ is at most $k|E|$.
\end{proposition}

\begin{proof} To simplify readability we suppose $R$ is binary.
Let $\A_n$ be the set of strings in ${\A}$ of length at most $n$. Let $Q$ be the state set of the automaton for $R$. 
First,  there is a constant $c := |Q|^{|Q|}$ such that for all $n$ and all $x \in {\A}$ there is a 
$y \in \A_{n+c}$ such that $x$ and $y$ cast the same shadow on $\A_n$. Indeed, consider the sequence of functions $f_i:Q \to Q$
with $f_i(q)$ defined to be the state reached when the automaton for $R$ starts in $q$ and reads the
string $\con(x[n+1,n+i],\lambda)$.  If $|x| > n + c$ then there are two positions  $k < l$ such that $f_k = f_l$. If we remove the segment $x[k,l-1]$ from $x$ we
get a shorter string $x'$ that casts the same shadow on $\A_n$ as $x$ does. Repeat until the string is short enough.
Second, consider the sequence of sets
$\A_{b+nc}$ where $b$ is fixed so that $\A_b \neq \emptyset$. Write $X_n$ for the cardinality of $\A_{b+nc}$ and $S_n$ for the shadow count of $\A_{b+nc}$. 
We know that $S_n \leq X_{n+1}$. Suppose $t$ were such that for almost all $n$, $S_n > t X_n$.  Then $2^{b+(n+1)c} \geq X_{n+1} \geq S_n > t^n X_0$ for almost all $n$. So  $t$ is smaller than a constant that depends on $b,c$ and $X_0$. So take $k$ larger than this constant and conclude that $S_n \leq k X_n$ for infinitely many $n$, as required.
\end{proof}

The proof of the following proposition is similar.
\begin{proposition}
Suppose $(\frakA,R) \in \taut$.
Then there is a constant $k$, 
that depends on the automata for $\A$ and $R$, and arbitrarily large finite subsets $E \subset {\A}$ such
that the shadow count of $E$ via $R$  is at most $|E|^k$.
\end{proposition}

These are used to show that certain structures are not automatic. An immediate application is that the random graph is not in $\taut$.

\subsubsection*{Sum- and product-decompositions}

All definitions and results in this section are due to Delhomm\'e \cite{Delh04}.
In these definitions all structures are over the same signature.

\begin{definition} 
Say that a structure $\frakB$ is {\em sum-decomposable} using a set of structures
$\mathbf{C}$ if there is a finite partition of $\B
= \B_1 \cup \cdots \cup \B_n$ such that for each $i$ the substructure $\frakB
\restriction \B_i$ is isomorphic to some structure in $\mathbf{C}$.
\end{definition}


\begin{theorem} \label{AS:thm:sumaug}
Suppose $(\frakA,R) \in \waut$.
There is a finite set of
structures $\mathbf{C}$ so that for every tuple of elements $\tup{u}$ from $\A$, the
substructure ${\frakA} \restriction R(\cdot,\tup{u})$ is sum-decomposable using $\mathbf{C}$.
\end{theorem}

\begin{proof}
To simplify readability we suppose that $\frakA = \left<\A, \prec \right>$ and for each $T \in \{\prec,R\}$ fix a deterministic automaton  $(Q_T,\iota_T,\Delta_T,F_T)$ recognising $\code{T^\frakA}$. Naturally extend $\Delta_T$ to all strings and so write
$\Delta_T(q,w)$. From now on we supress writing $\code{}$.
Given a tuple of strings $\tup{u}$ write $|\tup{u}| := \max\{|u_i|\}$. Observe that we can partition the set $R(\cdot,\tup{u})$
into the finitely many sets: the singletons $\{c\}$ such that $R(c,\tup{u})$ and  $|c| <
|\tup{u}|$; as well as the sets 
\[
R^{a}(\cdot,\tup{u}) := \{aw \in \A \st (aw,\tup{u}) \in R, w \in \twodom\}
\]
where $|a| = |\tup{u}|$. There are finitely
many isomorphism types amongst substructures of the form ${\A} \restriction
\{c\}$, for $c \in \A$. So, it is sufficient to show that as we vary the tuple
$(a,\tup{u})$ subject to $|a| = |\tup{u}|$, there are finitely many isomorphism
types amongst substructures of the form $\frakA \restriction R^{a }(\cdot,\tup{u})$.

We do this by bounding the number of isomorphism types in terms of the number
of states of the automata. To this end, define a function $f$ as
follows. Its domain consists of tuples $(a,\tup{u})$ satisfying $|a| = |\tup{u}|$; and $f$ sends $(a,\tup{u})$ to the pair of states
$$
\left<
\Delta_R(\iota_R,\con(a,\tup{u})),
\Delta_\prec(\iota_\prec,\con(a,a) ) \right>.
$$
The range of $f$ is bounded by $|Q_R| \times |Q_S|$; in particular, the range is finite.

To finish the proof, we argue that the isomorphism type of the
substructure $\frakA \restriction R^{a }(\cdot,\tup{u})$ depends
only on the value $f(a,\tup{u})$. This follows from the fact that if
$f(a,\tup{u}) = f(a',\tup{u'})$, then the corresponding substructures
are isomorphic via the mapping $aw \mapsto a'w$ ($w \in \twodom$).
For instance $(aw,\tup{u}) \in R$ if and only if the automaton for $R$ starting in state
$\Delta_R(\iota_R,\con(a,\tup{u}))$ and reading $\con(w,\lambda,\dots,\lambda)$ reaches
a final state if and only if starting in $\Delta_R(\iota_R,\con(a',\tup{u'}))$ and reading $\con(w,\lambda,\dots,\lambda)$
it reaches a final state if and only if $(a'w,\tup{u'}) \in R$. This established that $f$ is a bijection between the domains
$R^{a}(\cdot,\tup{u})$ and $R^{a'}(\cdot,\tup{u'})$.
A similar argument shows that $aw_1 \prec aw_2$ if and only if $a'w_1 \prec a'w_2$.
\end{proof}


\begin{corollary}  \label{AS:cor:omom}
The ordinal $(\omega^\omega,<)$ is not in $\waut$.
\end{corollary}

\begin{proof}
Suppose for a contradiction that $(\omega^\omega,<)$ has an automatic presentation and let $\mathbf{C}$ be the finite set of structures
guaranteed by the theorem using $<$ for $R$. For $n \in \N$ consider the substructure restricted to $R(\cdot,\omega^n)$. It has order type $\omega^n$.
It can be shown that if ordinal $\omega^\alpha$ is sum-decomposable using a finite set of ordinals $\mathbf{C}$ then at least one ordinal in $\mathbf{C}$ has order type $\omega^\alpha$ (see \cite{Delh06}).
Thus $\mathbf{C}$ must contain a structure with order type $\omega^n$, for each $n \in \N$, contradicting the finiteness of $\mathbf{C}$. 
%This means that $\mathbf{C}$ must contain (isomorphic copies of) $(\omega^n,<)$ for every $n \in \N$, contradicting
%the finiteness of $\mathbf{C}$.
\end{proof}


We now state the analogous results for $\taut$. In these definitions all structures are over the same signature and all sequences of structures are finite.

%\begin{definition}
%Let $(\frakC_i)_{i \leq k}$ be a non-empty finite sequence of structures.
%Say that $\B$ is product-decomposable using $(\frakC_i)_{i \leq k}$ if there exists a bijection $f:\prod_{i \leq k} C_i \to B$ such that
%for all $(c_1,\dots,c_k) \in \prod C_i$ and all $i \leq k$ the function sending $x \in C_i$
%to $(c_1,\dots,c_{i-1},x,c_{i+1},\dots,c_k)$ is an embedding of $\C_i$ into $\B$.
%\end{definition}
%
%
%\begin{example}
%Lets unpack this definition for the concrete signature of ordinals $\{<\}$. An ordinal $\beta$ is product-decomposable using the sequence
%$\alpha_1,\dots,\alpha_k$ if there exists a bijection $f:\alpha_1\times \dots \times \alpha_k \to \beta$ such that for all $\delta_i < \alpha_i$ ($i \leq k$) 
%and all $i \leq n$ and all $x < y < \alpha_i$ we have that $f(\delta_1,\dots,\delta_{i-1},x,\delta_{i+1},\dots,\delta_k) < f(\delta_1,\dots,\delta_{i-1},y,\delta_{i+1},\dots,\delta_k)$. It can be shown (following Theorem $2$ in \cite{Carr42}) that if ordinal $\omega^{\omega^\alpha}$ is product-decomposable using ordinals $\alpha_1,\dots,\alpha_k$
%then some $\alpha_i$ must be $\omega^{\omega^\alpha}$.
%\end{example}

\begin{definition}
Let $(\frakB_i)_{i \leq k}$ be a sequence of structures.
Their {\em synchronous product} is the structure with domain $\prod_{i \leq k} \B_i$ and relations defined as follows.
Write $\pi_j$ for the projection $\prod_{i \leq k} \B_i \to \B_j$. 
The interpretation of an $r$-ary relation symbol $R$
consists of those tuples $(\tup{x_1},\dots,\tup{x_r})$ such that for all $i \leq k$, 
$\frakB_i \models (\pi_i(\tup{x_1}),\dots,\pi_i(\tup{x_r})) \in R.$
If $\mathbf{C}$ is a set of structures write $\textsc{synch}(\mathbf{C})$ for the set of all
synchronous products of sequences (of arbitary finite length) of structures from $\mathbf{C}$.

Let $(\frakC_i)_{i \leq k}$ be a finite family of structures each with domain $A$.
Their {\em superposition} is the structure with domain $A$ and relation symbol $R$ interpreted as $\cup_{i \leq k} R^{\frakC_i}$.
Write $\textsc{super}(\mathbf{C})$ for the set of all superpositions formed from families of structures from $\mathbf{C}$.
\end{definition}

\begin{theorem} \label{AS:thm:treeautdecomp}
Suppose $(\frakA,R) \in \taut$.
There is a finite set of structures $\mathbf{C}$ so that for every tuple of elements $\tup{u}$ from $A$, the
substructure ${\frakA} \restriction R^{\frakA}(\cdot,\tup{u})$ is sum-decomposable using the set $\textsc{super}(\textsc{synch}(\textbf{C}))$.
\end{theorem}

\begin{proof}
We adapt the proof of Theorem~\ref{AS:thm:sumaug}. There $\frakA = (\A,\prec)$ and we do not distinguish between a tuple of elements of $\A$ and its code, or between a relation and the automaton defining it. We may assume all our automata $M$ are deterministic leaf to root.
If tree $t$ has $k$ leaves, say at positions $p_1,\dots,p_k$ ordered lexicographically write $\delta_M(q_1,\dots,q_k,t)$ 
for the label of the root of the run on input $t$ that for all $i \leq k$ labels position $p_i$ by state $q_i$. Write $\Delta_M(t)$ for $\delta_M(\iota_M,\dots,\iota_M,t)$; that is, for the state labelling the root of the run on input $t$ in which leaves are labelled with the initial state $\iota_M$.

For a fixed tuple of trees $\tup{u}$ partition $R^\frakA(\cdot,\tup{u})$ into finitely many sets $R^a(\cdot,\tup{u})$. Here $a$ is a tree whose domain is a subset of the domain of the tree $\conv(\tup{u})$ and $R^a(t,\tup{u})$ means that $t$ extends $a$ and that $\frakA \models R(t,\tup{u})$. Fix an $a$ and suppose for simplicity that $a$ has the same domain as $\conv(\tup{u})$ and that this domain has $k$ leaves at positions $p_1,\dots,p_k$. For a sequence $\tup{q} := q_1,\dots,q_k$ of states of $Q_R$ with  $\delta_R(\tup{q},\conv(a,\tup{u})) \in F_R$ write
$R^a_{\tup{q}}(\cdot,\tup{u})$ for those trees $t$ such that $R^a(t,\tup{u})$ and the run (of the automaton for $R$ on input $\conv(t,\tup{u})$) 
at position $p_i$ has label $q_i$ (for all $i \leq k$). Thus $R^a(\cdot,\tup{u})$ is itself partitioned into finitely many pieces each of the form $R^a_{\tup{q}}(\cdot,\tup{u})$. Fix one of these sequences $\tup{q}$. Recall our goal is to express $R^a_{\tup{q}}(\cdot,\tup{u})$ as a superposition of synchronous products. For $q \in Q_R$ define the set $\C_q$ of all trees $t$ (with any number of leaves) such that $\Delta_R(\conv(t,\lambda,\dots,\lambda)) = q$.
For $q \in Q_R, s \in Q_{\prec}$ define a structure $\frakC_{q,s}$ with domain $C_q$ and relation $t \prec_s t'$ if $\Delta_{\prec}(\conv(t,t')) = s$.
The required set $\mathbf{C}$ consists of the structures $\frakC_{q,s}$; and up to isomorphism there are at most $|Q_R| \times |Q_{\prec}|$. It is straightforward to check that $R^a_{\tup{q}}(\cdot,\tup{u})$ is in the set $\textsc{super}(\textsc{synch}(\textbf{C}))$. Indeed, the domain of $R^a_{\tup{q}}(\cdot,\tup{u})$ is $C_{q_1} \times \cdots \times C_{q_k}$ and $(R^a_{\tup{q}}(\cdot,\tup{u}),\prec)$ is the superposition of synchronous products $\frakC_{q_1,s_1} \times \cdots \times \frakC_{q_k,s_k}$ for $\tup{s} = (s_1,\dots,s_k)$ satisfying $\delta_\prec(\tup{s},\conv(a,a)) \in F_\prec$.
\end{proof}

%
%Let $C_i$ be the set of all trees $t$ such that the run of the automaton for $R$ on input $\conv(t,\lambda,\cdots,\lambda)$ has root labelled with state $q_i$. Define a binary relation $\prec_i$ on $C_i$ as follows: $t \prec_i t'$ if there exist 
%$s_j \in C_j$ (for all $j \neq i$) such that $(s_1,\cdots,s_{i-1},t,s_{i+1},\cdots,s_k) \prec (s_1,\cdots,s_{i-1},t',s_{i+1},\cdots,s_k)$. Write $\C_i$ for the structure $(C_i,\prec_i)$. Then $\frakA \restriction R^a_{\tup{q}}(\cdot,\tup{u})$ is a product-decomposition of the sequence $\C_1,\cdots,\C_k$. Indeed, the witnessing function
%$f$ sends $(t_1,\cdots,t_k) \in \prod_{i \leq k} C_i$ to the tree extending $a$ with $t_i$ rooted at position $p_i$ (for all $i \leq k$). Moreover there are at most
%many $\C_i$s.

%Finally, also fix one of the finitely many sequences $\tup{s} := s_1,\cdots,s_k$ of states of $Q_{\prec}$ such that $\delta_{\prec}(\tup{s},\conv(a,a)) \in F_\prec$.  
%
%
%Write $R^a_{\tup{q},\tup{s}}(\cdot,\tup{u})$ for those trees $t$ in $R^a_{\tup{q}}(\cdot,\tup{u})$ such that
%
%
%Call the resulting set of trees $R^a_{\tup{q},\tup{s}}$. It is now possible to see that $\frakA \restriction R^a_{\tup{q},\tup{s}}$ is a product-decomposition of the sequence $\frakC_1,\cdots,\frakC_k$ where $\frakC_i$ is the restriction of $\frakA$ to the domain consisting  of trees $t$ such that the run of the automaton for $R$ on $\conv(t,\lambda,\cdots,\lambda)$ has root-label $q_i$ and the run of the automaton for $\prec$ on $\conv(t,t)$ has root-label $s_i$. Indeed the required function $f$ sends the $k$-tuple $(t_1,\cdots,t_k)$ to the tree that extends $a$ at leaf $i$ by the tree $t_i$. There are at most $|Q_R| \times |Q_{\prec}|$ many $\frakC_i$s.

\begin{corollary} \label{AS:cor:omomom}
The ordinal $(\omega^{\omega^{\omega}},<)$ is not in $\taut$.
\end{corollary}

\begin{proof}
If $\omega^{\omega^{\omega}}$ were tree-automatic then take $R$ to be $<$ and consider the substructure restricted to $R(\cdot,\omega^{\omega^n})$ (for $n \in \omega$). It has order type $\omega^{\omega^n}$. Since this is a power of $\omega$, at least one substructure in $\textsc{super}(\textsc{sync}(\mathbf{C}))$ has order type $\omega^{\omega^n}$. It can be shown that if $\omega^{\omega^{\alpha}}$ is in $\textsc{super}(\mathbf{E})$ then $\mathbf{E}$ must contain $\omega^{\omega^{\alpha}}$ (see \cite[Remark $8$]{Delh06}). It is straightforward to check that if 
$\textsc{synch}(\mathbf{C})$ contains $\omega^{\omega^n}$ then already $\mathbf{C}$ contains $\omega^{\omega^n}$. This contradicts finiteness of $\mathbf{C}$.
\end{proof}




% The {\em height} of a binary relation $(V,E)$ is the least ordinal $\alpha$ into which it admits a morphism $f$ --- that is, $E(a,b)$ implies $f(a) < f(b)$. Thus $R(\cdot,\omega^{\omega^n})$  has height $\omega^{\omega^n}$. It can be shown that if $\omega^{\omega^{\alpha}}$ is in $\textsc{super}(\mathbf{E})$ then $\mathbf{E}$ must contain a structure of height $\omega^{\omega^{\alpha}}$ (see \cite[Remark $8$]{Delh06}). It is straightforward to check that if $\textsc{synch}(D)$ contains a str
%This contradicts finiteness of $\mathbf{C}$.

%Using this theorem it can be shown that the ordinal $(\omega^{\omega^{\omega}},<)$ is not in $\taut$.


%=====================================================
%=====================================================
%=====================================================
\section{Equivalent automatic presentations} \label{AS:sec:equiv}
%=====================================================
%=====================================================
%=====================================================

In this section we give a definition of when two finite-string automatic presentations of a fixed structure are ``equivalent''. 
The main result is due to B{\'a}r{\'a}ny \cite{Bara06}. We illustrate with base $k$ presentations of
$(\N,+)$ where $2 \leq k \in \N$. Write $\mu_k$ for the map sending $n \in \N$ to the base-$k$ representation of $n$.
The translation between bases $p$ and $q$ is the map $\mu_q \circ \mu_p^{-1}$. It sends a string in base-$p$ to that
string in base-$q$ that represents the same natural number.

Call two bases $p$ and $q$ {\em multiplicatively dependent} if for some positive integers
$k,l$
\[
 p^k = q^l.
\]

\todo{regular relation is undefined, here and later}

\begin{proposition} \label{AS:prop:multdep}
If $p$ and $q$ are multiplicatively dependent, then every relation $R \subseteq \N^r$
is regular when coded in base $p$ if and only if it is regular when coded in base $q$.
\end{proposition}

To see this we may use semi-synchronous rational relations: these can be thought of as being
recognised by a multi-tape automaton where each read-head advances at a
different, but still constant, speed. In the following definition the $i$th head moves $m_i$ symbols at a time.

\begin{definition}
Fix a finite alphabet $\Sigma$ and a vector of positive integers $\underline{m} = (m_1,\dots,m_r)$.  
Let $\blank$ be a symbol not in $\Sigma$ and write $\Sigma_{\blank}$ for $\Sigma \cup \{\blank\}$.
For each component $m_i$ introduce the alphabet 
$(\Sigma_{\blank})^{m_i}$.  The {\em $\underline{m}$-convolution of a tuple} 
$(w_1,\dots,w_r) \in (\Sigma^{\star})^r$ is formed as follows. First, consider the intermediate
string $(w_1\blank^{a_1}, \dots, w_r\blank^{a_r})$ where the $a_i$ are minimal
such that there is some $k \in \N$ so that for all $i$, $|w_i|+a_i = km_i$. Second,
partition each component $w_i\blank^{a_i}$ into $k$ many blocks of size $m_i$,
and view each block as an element of $(\Sigma_{\blank})^{m_i}$. Thus the string
$\con_{\underline{m}} (w_1,\dots,w_r)$ is formed over alphabet
$(\Sigma_{\blank})^{m_1} \times \cdots \times (\Sigma_{\blank})^{m_r}$.
The {\em $\underline{m}$-convolution of a relation} $R \subseteq (\Sigma^{\star})^r$ is the
set $\con_{\underline{m}} R$ defined as 
\[
\{\con_{\underline{m}} \tup{w} \st \tup{w} \in R\}.
\]
A relation $R$ is {\em \underline{m}-synchronous rational} if there is a finite automaton
recognising $\con_{\underline{m}} R$.
Call $R$ {\em semi-synchronous} if it is $\underline{m}$-synchronous rational for some $\underline{m}$.
\end{definition}

For example, if $\underline{m} = (1,\dots, 1)$ then $\con_{\underline{m}}$ is the same as $\con$.
For another example, the base-changing translation from base $p$ to base $q$ assuming $p^k = q^l$ is $(k,l)$-synchronous. Proposition~\ref{AS:prop:multdep} now follows
from the fact that the image of a regular relation under a semi-synchronous transduction is regular. The converse of Proposition~\ref{AS:prop:multdep} is also true and follows from the Cobham-Semenov theorem, see \cite{BHMCV94}. For instance, if $p$ and $q$ are multiplicatively independent then the set of powers of $p$ is regular in base-$p$ but not
regular in base-$q$. This discussion is the inspiration for the following generalisation.

For a given finite-string automatic presentation $\mu:\frakA \simeq (\B,S_1,\dots,S_N)$ 
write $\mu_{\reg}$ for the collection of relations
$$
\{\mu^{-1}(R) \st R \subseteq \B^k \text{ is a regular relation}, k \in \N\}.
$$

Let $\nu:\frakA \simeq (\C,R_1,\dots,R_N)$ be another finite-string automatic presentation of $\frakA$.

\begin{definition} \cite{Bara06}
The presentations  $\mu$ and $\nu$ of $\frakA$ are {\em equivalent} if $\mu_{\reg}  = \nu_{\reg}$.
\end{definition}

For instance if $p$ and $q$ are multiplicatively dependent then Proposition~\ref{AS:prop:multdep} says that the presentations $\mu_p$ and $\mu_q$ are equivalent.

\begin{theorem} \cite{Bara06}
The presentations $\mu$ and $\nu$ are equivalent if and only if the map $\nu \mu^{-1}:\B \to \C$, namely
 \[
  \{(\mu(x), \nu(x)) \in \B \times \C\st x \in \A\},
 \]
is semi-synchronous.
\end{theorem}

\begin{proof}
The interesting case is the forward direction. Let $f$ denote the translation $\nu\mu^{-1}:\B \to \C$. 
Here is an outline: starting with $x \in \B$, we apply $f$ to get $f(x) \in \C$, then pad to get $f'(x) \in \C'$,
then cut into blocks to get $f''(x) \in \C''$. 
Write $\pi$ for the padding $\C \to \C'$ and $\beta$ for the blocking $\C' \to \C''$.
Then $f$ can be decomposed into semi-synchronous maps
\[
 \B \stackrel{f''}{\to} \C'' \stackrel{\beta^{-1}}{\to} \C' \stackrel{\pi^{-1}}{\to} \C.
\]

We need some definitions. 
For a set $X$ of strings, write $\mathbb{L}_X$ for the regular relation of pairs $(x,y) \in X \times X$ such that $|x| \geq |y|$. 
The {\em growth} of a function $g$ between regular sets is the function 
$G:n \mapsto \max_{|a| \leq n} |g(a)|$. A bijection $g$ is {\em length preserving} if $|g(x)| = |x|$.
It is {\em length-monotonic} if $|x_1| \leq |x_2|$ implies $|g(x_1)| \leq |g(x_2)|$. It has {\em $\delta$-delay} 
if $|x_2| > |x_1| + \delta$ implies $|g(x_2)| > |g(x_1)|$.


\

\noindent
{\em Claim 1.} There is a constant $\delta$ such that $f$ has $\delta$-delay.

Since $f^{-1}(\mathbb{L}_\C) := \{(a,b) \st |f(a)| \geq |f(b)|\} \subseteq \B \times \B$ is regular (by assumption of equivalence) and locally finite (every $a$ is related to finitely many $b$s) 
there is a $\delta$ (by a pumping argument as in Proposition~\ref{AS:prop:locfin}) such that $(a,b) \in f^{-1}(\mathbb{L}_\C)$ implies $|b| \leq |a| + \delta$.

\

The next claim says that the strings shorter than $x$ are not translated into strings that are more than a constant longer than the string $x$ is translated into.

\

\noindent
{\em Claim 2.} There is a constant $K$ with $F(|x|) - |f(x)| \leq K$, where $F$ is the growth of $f$.

Since $f(\mathbb{L}_\B) := \{(f(a),f(b)) \st |a| \leq |b|\} \subseteq \C \times \C$ is regular and locally finite, $|a| \leq |x|$ implies $|f(a)| \leq |f(x)| + K$. Thus
$F(|x|) := \max_{|a| \leq |x|} |f(a)|$ is at most $|f(x)| + K$.

\

Let $\natural$ be a new symbol. Define $f':x \mapsto f(x) \natural^{F(|x|) - |f(x)|}$, for $x \in \B$. Thus $f'$ pads 
$f(x)$ by $\natural$ symbols so that the length of the translation by $f'$ of $x$ is no smaller than the length of the translation of any shorter string than $x$. 
Since the property $\{x \st F(|x|)-|f(x)| = i\}$ is regular for fixed $i$
 the graph of $f'$ is regular (by the previous claim). Write $\C'$ for $f'(\C)$. Write $F'$ for the growth of $f'$.

\

\noindent
{\em Claim 3.} The translation $f':\B \to \C'$ is length-monotonic and has $\delta$-delay.

Use the fact that $|f'(x)| = F(|x|) = F'(|x|)$.

\

\noindent
{\em Claim 4.} There exists $p,s \in \mathbb{N}$ such that the sequence  $F'(n+p) - F'(n) = s$ for almost all $n$.

Let $l_0 < l_1 < \cdots$ be the sequence of integers $l$ for which there is a string in $\B$ of length $l$. Let 
$u_n \in \C'$ denote the length-lexicographically smallest element amongst $\{f'(x) \st |x| = l_n, x \in \B\}$.
The set $L$ of all such $u_n$ is regular. Note that $|u_n| = F'(l_n)$ and so $|u_i| \leq |u_{i+1}|$ (length-monotonic) and $|u_i| < |u_{i+\delta}|$ ($\delta$-delay).
Thus $L$ has at most $\delta$ many strings of any given length.
So partition $L$ into regular sets $L_k$ for $k \leq \delta$:  $x \in L_k$ if there are exactly $k$ strings of length $|x|$ in $L$. The length-preserving projection of these onto $0^\ast$ results in 
unary presentations of $L_k$. These are ultimately periodic.


\

For simplicity assume the previous claim holds for all $n$. Now for $x \in \B$ of length $n$ write $f'(x) $ as $v_1 v_2 \dots v_n$ where $|v_i| = s$ (if $|f'(x)|$ is not a multiple of $s$, append
a (new) blank symbol until it is). For a string $w$ of length $s$ write $\widehat{w}$ for a new alphabet symbol. 
Define $f'':x \mapsto \widehat{v_1} \cdots \widehat{v_n}$. Write $\C''$ for $f''(\B)$. Clearly 
the translation $f'':\B \to \C''$ is length-preserving.

\

\noindent
{\em Claim 5.} Since $f''$ is length-preserving and preserves all regular relations, the graph of $f''$ is regular. 

The idea is that we can use lengths of elements of $\B$ (and $\C''$) as pointers to simultaneously identify the symbols in $x$ and $f''(x)$. For simplicity, suppose that for every $n \in \N$ there is an element in $\B$  of length $n$ (in general the gap between lengths is bounded). For a symbol $\sigma$ define the regular relation $S_\sigma(p,b) \subset \B \times \B$ saying that $\sigma$ occurs in $b$ at position $|p|$. Write $R_\sigma \subset \C'' \times \C''$ for the image of $S_\sigma$ under $f''$. It is also regular. Then $f''(x) = y$ if and only if $|x| = |y|$ and for all $p \in \B$
and $q \in \C''$ with $|p| = |q|$ and each symbol $\sigma$ we have $S_\sigma(p,x) \iff R_\sigma(q,y)$. This latter condition is regular.

\


Finally, write $\pi:\C \to \C'$ for the map sending $f(x) \mapsto f'(x)$, and $\beta:\C' \to \C''$ for the map sending $f'(x) \mapsto f''(x)$.
Since $\pi^{-1}$ is a projection its graph is regular.
Finally, $\beta^{-1}$ is semi-synchronous sending blocks of size $1$ to blocks of size $s$.
\end{proof}

%=====================================================
%=====================================================
%=====================================================


%=====================================================
%=====================================================
%=====================================================
\section{Automatic-like structures}  \label{AS:sec:gen} 
%=====================================================
%=====================================================
%=====================================================

\subsection{Expansions by predicates and automatic with advice}

 
Elgot and Rabin \cite{ElRa66} use automata theoretic arguments to show that certain expansions of $\frakT_1$ by unary
predicates have decidable MSO-theories.  For instance they showed that $(\N,+1,\textrm{Fact})$
with $\textrm{Fact} := \{n! \st n \in \N\}$ has decidable MSO-theory.  For a predicate $P \subseteq \N$, the {\em $P$-membership problem} is to decide, given a B\"uchi-automaton $M$,  whether or not $M$ accepts $\chi_P$. Recall that we write $(\frakA,P)$ for the structure $\frakA$ expanded by the predicate $P$.

\begin{lemma} \cite{ElRa66}
For every predicate $P \subseteq \N$, the structure $(\frakT_1,P)$ has decidable
MSO-theory if and only if the $P$-membership problem is decidable.
\end{lemma}

\begin{proof}
Let $\Phi$ be a sentence of $(\frakT_1,P)$. Let $X$ be a
variable not used in $\Phi$.  Build a formula $\Psi(X)$ from $\Phi$ in which
every occurence of $P$ has been replaced by the variable $X$.  By construction
$(\frakT_1,P) \models \Phi$ if and only if $\frakT_1 \models \Psi(P)$.  The
latter condition is equivalent to the problem of whether the automaton
corresponding to $\Psi$ accepts $P$ or not.
\end{proof}

We now briefly discuss how to find explicit predicates whose $P$-membership problem is decidable.

\subsubsection*{The contraction method for $P \subseteq \N$.}
Almost trivially, ultimately periodic $P$ have decidable $P$-membership problem.
For more general predicates, like the factorials $\mathrm{Fact}$, we use the {\em contraction method} of \cite{ElRa66}
and its generalisation by Carton and Thomas \cite{CaTh02} that we now explain.

Call $P$  {\em residually ultimately-constant} if there is an infinite sequence $x_0 < x_1 < \cdots$ of numbers such that
for every semigroup morphism $h:\twodom \to S$ with $S$ finite, the sequence $(h(u_i))_{i \geq 0}$ is ultimately constant,
where $u_i = \chi_P[x_i,x_{i+1})$.
Call $P$ {\em effectively residually ultimately-constant} if the function $i \mapsto x_i$ is computable and given $h(0),h(1)$ and $S$ one can compute an integer $l$, {\em a lag}, such that for all $m \geq l$, $h(u_l) = h(u_m)$. For example, it can be shown that the set of factorials $\textrm{Fact}$ is effectively residually ultimately-constant.

\begin{proposition} \cite{CaTh02}
If $P$ is effectively residually ultimately-constant then the $P$-membership problem is decidable.
\end{proposition}

\begin{proof}
There is a standard effective way to associate with a given automaton $M$ a morphism and finite semigroup $h:\twodom \to (S_M,\star)$ with the following property\footnote{The idea appeared in B\"uchi's complementation proof: define $h:\twodom \to 2^{Q \times Q \times \{=,\neq\}}$ so that $(q,q',\oplus) \in h(u)$ if
and only if there is a path in $M$ from state $q$ to state $q'$ labelled $u$ such that if $C$ are the states occurring on this path then $C \cap F \oplus \emptyset $. Note that if $h(u_i) = h(v_i)$ for $i = 1,2$ then $h(u_1u_2) = h(v_1v_2)$. Thus define the associative operation $\star$ on $S$ by
$s_1 \star  s_2 = h(u_1u_2)$ where $u_i$ is any element such that $h(u_i) = s_i$.
}: if a string $\alpha_0\alpha_1\alpha_2\cdots$ ($\alpha_i \in \twodom$) is accepted by $M$
then every string $\beta$ that can be decomposed into $\beta_0\beta_1\beta_2\cdots$ ($\beta_i \in \twodom$) with $h(\alpha_i)=h(\beta_i)$ (for all $i$) is also accepted by $M$.
Then $\chi_P$ is accepted by $M$ if and only if
the ultimately periodic string $uv^\omega$ is accepted by $M$, where $u = \chi_P[0,x_l)$ and $v = \chi_P[x_l,x_{l+1})$. The latter property is decidable since we can compute a lag $l$ and the elements $h(u),h(v)$.
\end{proof}

It turns out that $P$ being effectively residually ultimately-constant is also a necessary condition for $(\frakT_1,P)$ having a decidable MSO-theory. See \cite{RaTh06, rabinovich07} for a proof of this and other characterisations of $(\frakT_1,P)$ having decidable MSO-theories. See \cite{CaTh02,Bara07} for explicit effectively residually ultimately-constant predicates, including the morphic predicates.
  
Why restrict to expansions by unary predicates? The reason is that expansions by non-trivial binary relations result in undecidability. 

\begin{theorem} \cite{ElRa66}
Let $g:\N \to \N$ be a function such that $x < y$ implies $1 + g(x) < g(y)$.
The expansion of $\frakT_1$ by the relation $G_g:= \{(n,g(n)) \st n \in \N\}$ has undecidable $\wmso$-theory.
\end{theorem}

In fact, one shows that one can quantify over finite relations which in turn gives the power of recursion to define addition and then multiplication.
An example is taking $g(n) := 2n$.

What about expansions of $\frakT_2$? For predicate $P \subseteq \twodom$, define the $P$-membership problem as above but with Rabin automata instead. Then identical arguments show that $(\frakT_2,P)$ has decidable MSO-theory if and only if the $P$-membership problem is decidable.  The pushdown/Caucal hierarchy is a well studied collection of trees (and graphs) 
with decidable MSO-theory \cite{Cauc02} \cite{Thom03}. We illustrate an approach for decidability due to Fratani \cite{Frat05} (that can yield decidability of all trees in the hierarchy). To a semigroup $(\M,+)$ with finitely many generators $g_1,\dots,g_k$, associate the Cayley structure 
$(\M,S_1,\dots,S_k)$ where $S_i(m) = m+ g_i$. For instance, the semigroup of strings under concatenation $(\twodom,\cdot)$ viewed as a Cayley structure is
$\frakT_2$.  

\begin{theorem} \cite{Frat05}
Take a surjective semigroup morphism $\mu:\twodom \to \M$ and a set
$R \subset \M$. If  the MSO-theory of $(\M,S_1,\dots,S_k,R)$ is decidable then the MSO-theory of $(\frakT_2,\mu^{-1}(R))$ is decidable. 
\end{theorem}

\begin{proof}
Use the relationship between Rabin automata and parity games (see \cite{Thom90}) to show that a given tree automaton (with state set $Q$, initial state $\iota$, transition set $\Delta$) accepts $\mu^{-1}(R)$ if and only if the first player has a memoryless winning strategy in the parity game defined as follows: the arena is $\M \times (Q \cup \Delta)$; the priority of $(m,q)$ and $(m,(q,\sigma,q_0,q_1))$ is the priority of $q$, the starting node is $(\mu(\lambda),\iota)$, and for every transition $\delta = (q,\sigma,q_0,q_1) \in \Delta$ and $m \in \M$ such that $m \in R \iff \sigma = 1$ the first player's moves are of the form $(m,q)$ to $(m,\delta)$ and the second player's moves are of the form $(m,\delta)$ to $(m+\mu(i),q_{i})$ for $i \in \{0,1\}$. Having a memoryless winning strategy is expressible in MSO over $(\M,S_1,\dots,S_k,R)$.
\end{proof}

\begin{example}
Consider the semigroup morphism  $\mu:\twodom \to \N$ that sends
$u$ to the number of $1$s in $u$ (the operation on $\N$ is addition). We have seen that $(\frakT_1,\textrm{Fact})$ has a decidable MSO-theory and so conclude that
$(\frakT_2,\mu^{-1}(\textrm{Fact}))$ does too.
\end{example}


\subsubsection*{Automatic with advice}

If $(\frakT_2,P)$ has decidable MSO-theory then every structure FO-interpretable in $\Power[(\frakT_2,P)]$ has decidable FO-theory. This justifies
the following definitions (see \cite{CoLo07}).

\begin{definition}
A structure is  {\em Rabin-automatic with advice $P \subseteq \twodom$} if it is FO-interpretable in $\Power[(\frakT_2,P)]$.
A structure is  {\em B\"uchi-automatic with advice $P \subseteq \N$}  if it is FO-interpretable in $\Power[(\frakT_1,P)]$.
\end{definition}

A machine theoretic characterisation holds. A {\em Rabin-automaton with
advice $P \subseteq \twodom$} is one that, while in position $u \in \twodom$, can decide on its next
state using the additional information of whether or not $u \in P$.\footnote{The word `advice' is meant to connote that we can ask for a bit of information based on the current state and the current symbol being read. The other term found in the literature is 'oracle' which I choose not to use because in computability theory it means that the machine can ask if the whole content written on a tape is in the oracle language.}  In other words, the advice $P$ is simply read as part
of the input. Thus a structure has a presentation by Rabin-automata with advice $P$ if and only if it is FO-interpretable in $\Power[(\frakT_2,P)]$. An analogous statement holds for B\"uchi-automata with advice $P \subseteq \N$.


The theory of Rabin-automatic structures with advice is yet to be developed. These generalise Rabin-automatic structures which themselves still hold some mystery, for example, are the countable quotients already finite-tree automatic? It is known that the extension of the fundamental theorem  (Theorem~\ref{AS:thm:FOext}) holds: if $\frakA$ is Rabin-automatic with advice $P$ then the code of every $\FOext$-definable relation is recognised by a Rabin-automaton with advice $P$, see \cite{BKRa}. 

On the other hand, structures that are B\"uchi-automatic with advice have received some attention. For instance, the ordinal $\omega^\omega$ is not B\"uchi-automatic with advice \cite{RaRu12}.
%  \cite{Reinhardt13} \cite{AbuZaid08}

Structures in which elements are coded by {\em finite} strings/trees have received some consideration. 

\begin{definition} 
A structure is  {\em finite-tree automatic with advice $P \subseteq \twodom$} if it is FO-interpretable in $\Power_f[(\frakT_2,P)]$.
A structure is  {\em finite-string automatic with advice $P \subseteq \N$}  if it is FO-interpretable in $\Power_f[(\frakT_1,P)]$.
\end{definition}

Again, if $(\frakT_i,P)$ has decidable WMSO then $\Power_f[(\frakT_i,P)]$ has decidable FO-theory. Machine models, namely finite-string/tree automata with advice, are studied (sometimes implicitly) in \cite{ElRa66,CaTh02,RaTh06,Bara06Hierarchy,CoLo07,KRSZ12,Fratani12}. 
%have been studied  for, say finite-tree automatic with advice $P$, would
%have to have an infinitary acceptance condtions (such as the Rabin acceptance condition) since the automaton has to process $P$ which is typically infinite (cf. \cite{CoLo07}).



\begin{example} The structure $(\Q,+)$, although not finite-string automatic \cite{Tsan11}, is finite-string automatic with 
advice.\footnote{This was communicated independently by Frank Stephan and Joe Miller and reported in \cite{Nies07}.}
To simplify the exposition we give a presentation $([0,1) \cap \Q,+)$ by finite strings
over the alphabet $\{0,1,\#\}$ where the automata have access to the advice string
\[
10\#11\#100\#101\#110\#111\#1000\#\cdots
\]
which is a version of the Champernowne-Smarandache string and known to have decidable MSO-theory~\cite{Bara07}. 
To every rational in $[0,1)$ there is a unique {\em finite}
sequence of integers  $a_1, \dots, a_n$ such that $0 \leq a_i < i$ 
and $\sum_{i=2}^n \frac{a_i}{i!}$ and $n$ minimal. The presentation codes this rational as
$f(a_2)\#f(a_3)\#f(a_4) \cdots \#f(a_n)$ where $f$ sends $a_i$ to the binary string of length $\lceil  \log_2 i \rceil +1$  representing $a_i$. Addition $a+b$ is performed least significant digit first (right to left) based on the fact that 
\[
\frac{a_i + b_i + c}{i!} = \frac{1}{(i-1)!} + \frac{a_i + b_i + c - i}{i!}
\]
where $c \in \{0,1\}$ is the carry in. In other words, if $a_i+b_i+c \geq i$ then write $a_i+b_i+c-i$ in the $i$th segment and carry a $1$ into the $(i-1)$st segment; and if $a_i +b_i + c < i$ then write this under the $i$th segment and carry a $0$ into the $(i-1)$st segment.  These comparisons and additions can be performed since the advice tape is storing $i$ in the same segment as $a_i$ and $b_i$. Of course since the automaton reads the input and advice from left to right it should non-deterministically guess the carry bits and verify the addition.
\end{example}

Structures that are finite-tree automatic with advice were first studied in \cite{CoLo07}.

\begin{theorem} \cite{CoLo07}
If $(\frakA,\equiv)$ is finite-tree automatic with advice $P$ and $\equiv$ is a congruence on $\frakA$ then $\frakA/_\equiv$ is also finite-tree automatic with advice $P$.
\end{theorem}

\begin{theorem} \cite{CoLo07} \label{AS:thm:colo}
If $\Power_f[\frakS]$ is finite-tree automatic with advice $P$ then $\frakS$ is WMSO-interpretable in $(\frakT_2,P)$.
\end{theorem}

Consequently the following structures are not finite-tree automatic with any advice: the free monoid on two or more generators; the random graph; the structure $\Power_f[(\N,+)]$. For instance, the last item follows from the fact that $(\N,+)$ is not WMSO-interpretable in any tree. It is not known what happens in these examples and theorems if we replace WMSO by MSO and $\Power_f$ by $\Power$.

\subsection{Descriptive set theory and Borel presentations}

A standard reference for classical descriptive set theory is \cite{Kech95}. See \cite{NiMo11} for a short survey about Borel presentable structures.
There is a natural topology, called the Cantor topology, on $\{0,1\}^\omega$, namely the one whose basic open
sets are of the form $\{\alpha \st \tau \prefeq \alpha\}$ for $\tau \in \twodom$.  A subset $X \subseteq \{0,1\}^\omega$ is called {\em Borel (over $\{0,1\}$)} if it is in the smallest
class of subsets of $\{0,1\}^\omega$ containing the basic open sets and closed under
countable unions and complementation. 

\begin{example}
\begin{enumerate}
 \item A set $X$ is a  countable union of basic open sets if and only if there exists $W \subseteq \twodom$ such that
\[
 \alpha \in X \iff (\exists i) \alpha[i] \in W.
\]
These are the {\em open} sets. 

\item Complements of open sets are called {\em closed}.  Every singeleton is closed and thus
every countable subset of $\{0,1\}^\omega$ is Borel.

\item A set $X$ is a countable intersection of open sets if and only if there exists $W \subseteq \twodom$ such that
\[
\alpha \in X \iff (\forall j) (\exists i > j)\alpha[i] \in W
\]

\item A set $X$ is a countable union of closed sets if and only if there exists $W \subseteq \twodom$ such that
\[
\alpha \in X \iff (\exists j) (\forall i > j) \alpha[i] \in W
\]
 
\item A language $X$ recognised by a {\em deterministic} Muller automaton is a boolean combination of sets of the previous two forms. Thus 
every $\omega$-regular language is Borel.
\end{enumerate}
\end{example}

In a similar way we can form Borel subsets of $A^\omega$ where $A$ is a finite set, not just $\{0,1\}$.
Thus we can define Borel relations: we call $S \subseteq (\{0,1\}^\omega)^r$ Borel if $\conv S$ is a Borel subset of $(\{0,1\}^r)^\omega$.

\begin{lemma} \label{AS:lem:Borelclosure}
Borel relations are closed under Boolean combinations and instantiation, that is, if $R$ is Borel over $X$ and $x \in X^\omega$ is fixed, then

\[
\{(x_1,\dots,x_{i-1},x_{i+1},\dots,x_r) \st (x_1,\dots,x_{i-1},x,x_{i+1},\dots,x_r) \in R\}
\]
 is Borel for every $i$.
\end{lemma}

\begin{definition}[Borel presentation] \label{AS:dfn:bap}
A {\em Borel structure} $(\B,S_1,\dots,S_N)$ is one for which $\B \subseteq \{0,1\}^\omega$ is Borel 
and each of the relations $S_1,\dots, S_N$ are Borel. Any structure isomorphic to it is called {\em Borel presentable}.
\end{definition}

\begin{example} The following structures are Borel presentable.
 \begin{enumerate}
  \item Every structure with a countable domain.
  \item The field $(\mathbb{C},+,\times)$.
  \item The order $(\Power(\{0,1\}^\star),\subseteq)$.
  \item The power structure of $(\mathbb{N},+,\times)$.
  \item Every  B\"uchi-automatic structure.
 \end{enumerate}
\end{example}

Although Borel presentable structures do not neccessarily have decidable first-order theories, methods of descriptive set theory can
be used to answer questions about automatic presentations. To illustrate we will make use of expansions and extensions of the Borel presentable
structure $(\Power(\{0,1\}^\star),\subseteq)$.

\begin{lemma} [\cite{HKMN08}] \label{AS:lem:borel}
Let $C$ be a countable set.
If $\Phi$ is an isomorphism between two Borel presentations of $(\Power(C),\subseteq)$, say $(D,L)$ and $(D',L')$, 
then the graph of $\Phi$ is Borel.
\end{lemma}

\begin{proof}
Every Borel presentation of  $(\Power(C),\subseteq)$ can be expanded to include the countable unary predicate $\text{Sing} := \{\{c\} \st c \in C\}$.
So consider expansions $(D,L,S)$ and $(D',L',S')$ that are Borel.
Note that $\Phi(x)=y$ if and only if for every $u$ in the countable set $S$, $(u,x) \in L \iff (\Phi(u),y) \in L'$.  For fixed $(u,\Phi(u))$ the relation $(u,x) \in L \iff (\Phi(u),y) \in L'$ is Borel by Lemma \ref{AS:lem:Borelclosure}. Thus the graph of $\Phi$ is the intersection over a countable set of Borel relations, and so is itself Borel.
\end{proof}

We will use the following facts. 

{\em Fact $1.$} The graph of a function is Borel if and only if the preimage of every Borel set is Borel.

{\em Fact $2.$}  Suppose a function $F:\{0,1\}^\omega \to \{0,1\}^\omega$ satisfies that for all $X,Y$ 
the strings $X$ and $Y$ are eventually equal if and only if $F(X) = F(Y)$. Then the graph of $F$ is not Borel.

The following theorem separates $\raut$ from $\baut$.

\begin{theorem}[\cite{HKMN08}] \label{AS:thm:sep}
There is a structure in $\raut$ that has no Borel presentation; in particular this structure is not in $\baut$.
\end{theorem}

\begin{proof}
Consider the structure $\mathfrak{V} = (\Power(\{0,1\}^\star),\subseteq,V)$ where $V$ is the unary relation consisting of those sets $X$ such that
the characteristic tree of $X$ has the property that every infinite path is labelled with only finitely many $1$s. The structure $\mathfrak{V}$ has a natural presentation placing it in $\raut$. A bijection between $\{0,1\}^\star$ and $\N$ allows us to identify $\Power(\{0,1\}^\star)$ with $\{0,1\}^\omega$.
Write $\mathfrak{V'} = (\{0,1\}^\omega,\subseteq,V')$ for the corresponding structure. 
Now if $\mathfrak{V'}$ were Borel presentable, say via isomorphism $\Psi$, then by Lemma \ref{AS:lem:borel} the graph of $\Psi$ would be Borel and so would $V'$ be Borel. But this contradicts the fact that $V'$ is $\Pi_1^1$-complete. \todo{Say why $V'$ is $\Pi_1^1$-complete}
\end{proof}

The following theorem says that $\baut$ is not closed under quotients.  A more sophisticated argument using the same counter-example structure shows that $\raut$ is not closed under quotients \cite{HKMN08}.


\begin{theorem}[\cite{HKMN08}] \label{AS:thm:borel}
There is a structure $(\frakA,\approx)$ in $\baut$ whose quotient $\frakA/_{\approx}$ has no Borel presentation, and is thus not in $\baut$.
\end{theorem}

\begin{proof}
Let $\frakB_0$ and $\frakB_1$ be structures with disjoint domain, each isomorphic to $(\Power(\mathbb{N}),\subseteq)$.
Define $\frakA$ to be the structure with domain $B_0 \cup B_1$ and relations $\leq$, $U$, $f$ where $\leq$ is
$\subseteq$ restricted to each $B_i$, $U$ holds on the elements of $B_0$ and $f:B_0 \to B_1$ is the identity. Let $=^*$ be an equivalence
relation on the $B_0 \cup B_1$ which is the identity on $B_0$ and `the symmetric difference is finite' on $B_1$. Clearly 
$(\frakA,=^*) \in \baut$. We now sketch the proof that $\frakA/_{=^*}$ is not Borel presentable. 
Assume that $\frakA/_{=^*}$ is isomorphic to a Borel structure $(B',\leq',U',f')$ via $\Phi$ and let $\Phi_0$ be the restriction of 
$\Phi$ to $[B_0]_{=^*}$ (which is $B_0$). By Lemma \ref{AS:lem:borel} $\Phi_0$ is Borel.
Then  by Fact $1$ the composition $F: = f' \circ \Phi_0$ is Borel, contradicting Fact $2$.
\end{proof}

\todo{$\M$ for presentation?}
Methods from (descriptive) set theory have also been used to study the isomorphism problem for automatic structures.
Specifically, writing $\mathcal{M} = (M_B,M_1,\dots,M_N)$ for a tuple of automata presenting a structure, the {\em isomorphism problem} for a collection
$\mathbf{C}$ is the set of pairs $(\mathcal{M},\mathcal{M'})$ such that both $\M$ and $\M'$ present the same structure from $\mathbf{C}$. \todo{really? this is the dfn of isomo problem? ``same structure''} See \cite{Rubi04,KLL10LICS,KLL13,KLM14,Kuske14} in which various natural isomorphism problems can be placed within the arithmetic and analytic hierarchy (see \cite{Roge67} for the definitions of these terms). For instance, the complexity of the
isomorphism problem for $\waut$ is $\Sigma^1_1$-complete.

We end with a stunning result:

\begin{theorem}[\cite{FiTo10}]
The isomorphism problem for the collection 
\[
\{\frakA/_\approx \st (\frakA,\approx) \in \raut\}
\]
is not determined by the axiomatic system \textrm{ZFC}.
\end{theorem}

\begin{proof}
Here is the barest of sketches. 
Write $\mathrm{FIN}$ for the ideal of finite subsets of $\N$, and $\mathrm{ANC}$ for the ideal of subsets of $\{l,r\}^\star$ with no infinite antichain with respect to the prefix-order.
Both $(\Power(\N),\cap,\cup,\neg,\mathrm{FIN})$ and $(\Power(\{l,r\}^\star),\cap,\cup,\neg,\mathrm{ANC})$ are Rabin-automatic. Write $\frakB_1$ for the Boolean algebra $\Power(\N)/\mathrm{FIN}$ and $\frakB_2$ for $\Power(\{l,r\}^\star)/\mathrm{ANC}$.
Using results from descriptive set theory it can be proved  that $\frakB_1$ and $\frakB_2$  are isomorphic under ZFC+CH, but  not isomorphic under ZFC+OCA (open colouring axiom).
\end{proof}


%=====================================================
\section{Questions and directions} \label{AS:sec:summary}
%=====================================================

This chapter focused on what I consider foundational problems of automatic structures. 
Here are some directions and questions raised along the way.
\begin{enumerate}
\item Theorem \ref{AS:thm:FOext} describes extensions of the fundamental theorem by certain unary generalised quantifiers. Which other generalised quantifiers can be added? What if one restricts to a specific automatic structure? 
\item Section \ref{AS:subsub:quotient} states that Rabin-automatic structures are not closed under quotient. The counterexample mentioned has size continuum. Is every counterexample this large? That is, if $(\frakA,\epsilon)$ is Rabin-automatic and the quotient $\frakA/\epsilon$ is countable, then is the quotient Rabin-automatic (and hence finite-tree automatic)? If not, then which countable ordinals are regular quotients of Rabin-automatic structures?
\item Can automata be used to explain why the countable random graph $\G$, which is not finite-tree automatic with any advice,  has a decidable FO-theory? Although this question seems hard to formalise we may ask: Is $\G$, or an elementarily equivalent structure, the regular quotient of a Rabin-automatic structure with advice? The same question can be asked of other important structures such as real arithmetic $(\R,+,\times,<)$ which is decidable in double exponential time. In \cite{AGKW13} it is proved that real arithmetic is not the regular quotient of a B\"uchi-automatic structure.
 %This starts to address the general problem of proving that certain structures of size continuum are not B\"uchi-automatic.
 %\item Find techniques to show that certain structures are not Rabin-automatic or that certain uncountable structures are not B\"uchi-automatic. %mention new results
\item Theorem \ref{AS:thm:colo} allows one to prove that certain structures of the form $\Power_f[\frakS]$ are not finite-tree automatic with advice $P$. Is it the case that if the power structure $\Power[\frakS]$ is Rabin-automatic with advice $Q$ then $\frakS$ is
MSO-interpretable in $(\frakT_2,Q)$? \todo{cite CoLo for this question}
\end{enumerate}

%TO ADD: real field not omega-automatic. 
%thanks to  Martin Huschenbett for discussions regarding theorem \ref{AS:thm:treeautdecomp}.

