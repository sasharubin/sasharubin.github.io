%Part 2

%%%%%%%%%%%%%%%%%%%%%%%%%%%%%%%%%%%%%%%%%%%%%%%%%%%%%%%%%%%%%%%%%%%%%%%%%%%%%%%%%%%%%%%%%%%%%%%%%%%%%%%%%%%%%%%%%%%%%%%%%5
%%%%%%%%%%%%%%%%%%%%%%%%%%%%%%%%%%%%%%%%%%%%%%%%%%%%%%%%%%%%%%%%%%%%%%%%%%%%%%%%%%%%%%%%%%%%%%%%%%%%%%%%%%%%%%%%%%%%%%%%%5
%%%%%%%%%%%%%%%%%%%%%%%%%%%%%%%%%%%%%%%%%%%%%%%%%%%%%%%%%%%%%%%%%%%%%%%%%%%%%%%%%%%%%%%%%%%%%%%%%%%%%%%%%%%%%%%%%%%%%%%%%5
\section{Other classes of automatic structures}
%%%%%%%%%%%%%%%%%%%%%%%%%%%%%%%%%%%%%%%%%%%%%%%%%%%%%%%%%%%%%%%%%%%%%%%%%%%%%%%%%%%%%%%%%%%%%%%%%%%%%%%%%%%%%%%%%%%%%%%%%5
%%%%%%%%%%%%%%%%%%%%%%%%%%%%%%%%%%%%%%%%%%%%%%%%%%%%%%%%%%%%%%%%%%%%%%%%%%%%%%%%%%%%%%%%%%%%%%%%%%%%%%%%%%%%%%%%%%%%%%%%%5
%%%%%%%%%%%%%%%%%%%%%%%%%%%%%%%%%%%%%%%%%%%%%%%%%%%%%%%%%%%%%%%%%%%%%%%%%%%%%%%%%%%%%%%%%%%%%%%%%%%%%%%%%%%%%%%%%%%%%%%%%5

We introduce subclasses of $\raut$ related to automata on finite strings/trees and infinite-strings. 
Each has a machine theoretic charactersation as in Definition~\ref{AS:dfn:rap}.
A member of any of the four classes is said to be {\em automatic}. The relationships are summarised in figure \ref{}. %??? ref to pic

\subsubsection{Finite-string automatic structures}

\begin{definition}
A structure is called {\em finite-string automatic} if it is FO-interpretable in $\Power_f[\frakT_1]$. This collection of structures is written $\waut$.
\end{definition}

We have seen in the section \ref{} that $(\N,+)$ is {\em finite-string automatic}. 
Note that such structures have countable domain. %??? ref

\begin{definition}[convoluting finite strings]
Let $\tup{w} = (w_1,\cdots,w_k)$
be a $k$-tuple of $\{0,1\}$-labelled finite strings. Let $l:= \max_i |w_i|$.
The {\em convolution} $\conv(\tup{w})$ is the $\{0,1\}^k$-labelled string of length $l$ such that for all positions $n \leq l$
the $i$th component of $\conv(\tup{w})(n)$ is equal to $w_i(n)$ if $n \leq |w_i|$ and the blank symbol $\blank$ otherwise.
\end{definition} %??? make sure w(n) is consistent notation for nth symbol of w

\begin{definition}[characteristic finite-string]
For a finite set $Y \subset \N$ define
its {\em characteristic string} $\chi_Y$ as the $\{0,1\}$-labeled string of length $\max_{y \in Y} y + 1$ with a $1$ in position $n$ if and only if $n \in Y$.
%For a tuple $(Y_1,\cdots,Y_n)$ write $\chi_{\tup{Y}}$ for $(\chi_{Y_1},\cdots,\chi_{Y_n})$.
\end{definition}

\begin{definition}
Say $f$ is an isomorphism witnessing $\frakA \in \waut$. For any relation $R \subseteq A^k$ denote by
$\code{R}$ the set of finite-strings  $\{\conv(\chi_{f(a_1)},\cdots,\chi_{f(a_k)}) \st  \tup{a} \in R\}$.  %??? say what \chi is in this case
\end{definition}

Just as for Rabin-automatic structures, there is a fundamental theorem for finite-string automatic structures. We do not state it in full;
simply replace Rabin-automatic by finite-string automatic in the statement.
However, we do slightly generalise the analogous definition of automatic presentation.

\begin{definition}[finite-string automatic presentation] \label{AS:dfn:fsap}
Fix a finite alphabet $\Sigma$. Suppose that $f: \frakA \simeq  (B,S_1,\cdots,S_N)$ and
\begin{enumerate}
\item the elements of $B$ are finite strings from $\Sigma^\ast$;
\item the set $B$ is recognised by a finite-string automaton, say $M_B$; 
\item the set $\{\conv(\tup{t}) \st \tup{t} \in S_i\}$ is recognised by a finite-string automaton, say $M_i$, for $i \leq N$.
\end{enumerate}
Then the data $\left<(M_B,M_1,\cdots,M_N), f \right>$ is called a {\em finite-string automatic presentation} of $\frakA$.
\end{definition}

\begin{proposition}[Machine theoretic characterisation]
A structure is FO-interpretable in $\Power_f[\frakT_1]$ if and only if it 
has a finite-string automatic presentation over an alphabet with $|\Sigma| \geq 2$.
\end{proposition}

\begin{proof}
 We simply need to point out what to do in case $|\Sigma| > 2$.
\end{proof}

%\subsubsection{$\omega$-string automatic structures}

%\begin{definition}
%A structure is called {\em $\omega$-string automatic} (also called {\em \buchi-automatic}) if it is FO-interpretable in $\Power[\frakT_1]$. This collection is written $\baut$.
%\end{definition}

%Since `$x$ is finite' is a FO-definable property in $\frakT_1$, examples of $\omega$-string automatic structures 
%include all finite-string automatic structures.
%The structure $([0,1],+,<)$, where addition is taken modulo one, is an uncountable example.


%\subsubsection{finite-tree automatic structures}

%\begin{definition}
%A structure is called {\em finite-tree automatic} if it is FO-interpretable 
%in $\Power_f[T_2]$.  This collection is written $\taut$.
%\end{definition}

%Since $\Power_f[\frakT_1]$ is FO-interpretable in $\Power_f[\frakT_2]$, every
%finite-string automatic structure is also finite-tree automatic.  Also
%$(\N,\times)$ is finite-tree automatic (decompose $n$ into prime powers,
%$n=\prod_{i} p_i^{e_i}$, and code it as a tree with $e_i$ written in binary on
%the $i$th branch). It is not finite-string automatic \cite{}.


\subsubsection{Relationships amongst the classes of automatic structures}


One can similarly define and characterise $\omega$-string automatic structures, also called {\em B\"uchi-automatic} and denoted $\baut$, as well as
finite-tree automatic structures, denoted $\taut$.

We make do with two examples: $([0,1),+,<)$, where $+$ is taken modulo $1$, is $\omega$-string automatic 
(the usual binary coding works); $(\N,\times)$ is finite-tree automatic (decompose $n$ into prime powers,
$n=\prod_{i} p_i^{e_i}$, and code it as a tree with $e_i$ written in binary on
the $i$th branch). It is not finite-string automatic \cite{Blum99}.

The following lemma justifies figure \ref{}.

\begin{lemma} \label{AS:lem:relations}
$\Power_f(\frakT_1)$ is FO-interpretable in both $\Power(\frakT_1)$ and $\Power_f(\frakT_2)$, each of which in turn is
FO-interpretable in $\Power(\frakT_2)$.
\end{lemma} %???

We will see (Theorem \ref{AS:thm:sep}) that there is a structure separating $\raut$ from $\baut$, namely $(\Power(\{l,r\}^\star),\subset,V)$ where $V$ is the unary relation consisting of those sets $X$ such that
the characteristic tree of $X$ has the property that every infinite path is labelled with only finitely many $1$s.

\subsubsection{Automatic with advice} ??? This section needs work ???

A structure FO-interpretable in $\Power[(\2,\suc_0,\suc_1,P)]$ for a predicate $P \subseteq \2$ is called
{\em $\omega$-tree automatic with advice $P$}. Similar definitions hold for the other subclasses.
Of course if $(\frakT_2,P)$ has decidable MSO-theory then every $\omega$-tree automatic structure with advice $P$
has decidable FO-theory. %%% ??? fix notation (\frakT_2,P)

Colcombet and L\"oding prove a quite general result about that says, in particular, that if $\Power_f[\frakT_2,P]$ is FO-interpretable in $\Power_f[\frakT_2,Q]$
then already $(\frakT_2,P)$ is WMSO-interpretable in $(\frakT_2,Q)$. 

%??? say something about the (W)MSO hierarchy...


%??? in dfn have f be part of interpretation $\I$ ???
%
%\begin{definition}
%For an automatic structure $\frakA$ (isomorphic to some $\I(\T_2)$ via $f$) and relation $R$ in $\frakA$ denote by
%$[R]$ the set of $\omega$-trees $\{T_{f(a_1),\cdots,f(a_k)} \st  \tup{a} \in R\}$.
%\end{definition}

%The following theorem says that FO-definable relations in automatic structures are, modulo coding into strings or trees, regular.

%\begin{theorem}[Fundamental theorem of automatic structures]
%Let $\frakA$ be set-interpretable in $\T_2$.
%\begin{enumerate}
%\item For every first-order definable relation $R$ in $\frakA$ the set of trees $[R]$ is recognised by an automaton.
%\item The first-order theory of an automatic structure is decidable.
%\end{enumerate}
%\end{theorem}

%\begin{proof}
%Combine the translation lemma for set-interpretations and Rabin's theorem.
%\end{proof}:q

%look again at the finite-set interpretation of $(\N,+)$ in $\one$. An element $n$ of
%the interpreted structure can be viewed as a finite binary string $(n)_2$.
%What happens to definable sets and relations on $\N$ under $\code:\N \to \2$?
%The fundamental theorem says that image under $f$ of an FO-definable relation $R$ of $(\N,+)$ is regular.
%Write $\code(R)$ for $\{f(a_1,\cdots,a_k) \st (a_1,\cdots,a_k) \in R\}$.
%In particular, $f(\N)$ is a regular set and $f(+)$ is a regular relation.

%=====================================================
\subsection{Universal automatic structures}
%=====================================================

A structure $\frakU$ is called {\em universal} for the $\av$-automatic structures
if it is $\av$-automatic and every $\av$-automatic structure is FO-interpretable in $\frakU$.
For instance, $\Power(\frakT_2)$ is universal for $\raut$. Clearly if $\frakU$ is universal for $\av$-automatic structures,
and $\frakA$ is $\av$-automatic, then  $\frakA$ is universal if and only if $\frakU$ is FO-interpretable in $\frakA$.

In the following $\Sigma$ is a finite set. Define the structure 
\[\mathfrak{S}_\Sigma := (\Sigma^\ast,\{\sigma_a\}_{a \in \Sigma},\pref,\el)\]
where $\sigma_a$ holds on pairs $(w,wa)$, $\el$ holds on pairs $(u,v)$ such that $|u|=|v|$, and $\pref$ is the prefix relation.
For $k \geq 2$ define the structure
\[
 \frakN_k := (\N,+,|_k)
\]
where $|_k$ is the binary relation on $\N$ with $x |_k y$ if $x$ is a power of $k$ and $x$ divides $y$.

\begin{proposition}
If $|\Sigma| \geq 2$ then $\mathfrak{S}_\Sigma$ is universal for $\waut$ and $\mathfrak{S}_{\Sigma}^\omega$ is universal for $\baut$.
 For $k \geq 2$, $\frakN_k$ is universal for $\waut$.
%If $|\Sigma| \geq 2$ then 
\end{proposition}

{\bf Remark.} For a discussion of $\mathfrak{S}_\Sigma$ for an infinite set $\Sigma$ see \cite{}.
%%% see later?
 
Define the structure
\[
\mathfrak{S}_{\Sigma}^\omega := (\Sigma^{\omega} \cup \Sigma^\ast,\{\sigma_a\}_{a \in \Sigma},\pref,\el)
\]
whose domain consists of all finite and $\omega$-strings, $\sigma_a$ and $\el$ are only defined on the finite strings, 
and $u \pref v$ if $u$ is a prefix of $v$ where $u$ is finite and $v$ is finite or infinite.


Define a structure $\frakR_\Sigma$ with domain consisting of all finite $\Sigma$-labelled trees
  and has operations
\[
        (\exteq, \edom, (\suc_a^l)_{a \in \Sigma}, (\suc_a^r)_{a \in \Sigma}, (\epsilon_a)_{a \in \Sigma} )
\]
  where
$T \exteq S$ if $\dom(T) \subseteq \dom(S)$ and $S(\alpha) = T(\alpha)$ for $\alpha \in \dom(T)$;
  $T \edom S$ if $\dom(T) = \dom(S)$;
  $\suc_a^d(T) = S$ if $S$ is formed from $T$ by extending its leaves in direction $d$
  and labeling each new such node by $a$; and
  $\epsilon_a$ is the tree with a single node labelled $a$.

  Similarly the structure $\frakR^\omega_\Sigma \in \raut$ has domain consisting of
  all finite and infinite trees and operations
  \[
        (\exteq, \edom, (\suc_a^l)_{a \in \Sigma}, (\suc_a^r)_{a \in \Sigma}, (\epsilon_a)_{a \in \Sigma} )
  \]
  that are restricted to finite trees, except that $T \exteq S$ is
  defined as above but allows $S$ to be an infinite tree.

\begin{proposition}
 if $|\Sigma| \geq 2$ then $\frakR_\Sigma$ is universal for $\taut$ and $\frakR^{\omega}_\Sigma$ is universal for $\raut$.
\end{proposition}


\begin{proof}
 Details rqd.
\end{proof}

What about the cases $|\Sigma| = 1$? Well, $\mathfrak{S}_\Sigma \in \uaut$ is isomorphic to $(\N,+1,<,=)$.
However, $\mathfrak{S}_\Sigma$ is not universal for $\uaut$.

Define the structure $\frakO := (\N,+1,<,\{m\N\}_{m > 1})$ where $m\N$ is the unary predicate consisting of multiples of $m$.

\begin{proposition}[\cite{Nabe76,Blum99}]
The structure $\frakO$ is universal for $\uaut$.
\end{proposition}

{\bf Remark.} $\frakO$ has infinite signature. In order for us to conclude that it has decidable FO-theory  
note that every FO-formula consists of finitely many predicates and there is a unary 
presentation of $\frakO$ in which $m \mapsto \code{m\N}$ is computable.


\begin{proposition}
 If $\frakA \in \uaut$ then $\Power_f(\frakA) \in \waut$.
\end{proposition}

\begin{proof}
Every element $x$ of $\frakA$ corresponds to a unary string $f(x)$. Thus subsets of $\frakA$ correspondence
to binary strings. The required presentation codes $X \subseteq\A$ by its characteristic string.
\end{proof}

The converse follows from a much more general result, see \cite{CoLo06}. The present proof is from \cite{Bara07}.

\begin{proposition}
 If $\Power_f(\frakA) \in \waut$ then $\frakA \in \uaut$.
\end{proposition}

\begin{proof}
Take a finite-string automatic presentation $f: \Power_f(\frakA) \simeq (B,S_1,\cdots,S_N)$. 
Let $S \subset \Power_f(\calA)$ denote the set of singeletons, and
$S_n$ denote the set $\{x \in S \st |f(x)| \leq n\}$.

As in the proof of proposition \cite{} we can show that $|S_n| = O(n)$. So $f(S)$ is a regular language
of linear growth. By a result from \cite{SYZS92} this implies that $f(S)$ is a finite union of sets of the form $uv^\ast w$.
In particular the size of $D_n := S_{n+1} \setminus S_n$ is bounded, say by $K$.

Construct a unary presentation of $\frakA$ as follows. For $a \in D_n$ suppose the position of $f(a)$ in the set
$f(D_n)$ under the lexicographic order is $j$. Then $a$ is coded by the unary string of length $Kn + j$.
\end{proof}

Another characterisation of $\uaut$ is that it is those structure that are MSO-interpretable in $\frakT_1$, see \cite{}.
% \begin{theorem}
%  The following are equivalent.
% \begin{itemize}
%  \item $\frakA$ has a finite-string unary-automatic presentation.
%  \item $\frakA$ is FO-interpretable in $(\N,+1,<,\{m\N\}_{m \in \N})$.
%  \item $\Power_f(\frakA)$ is finite-string automatic.
%  \item $\frakA$ is MSO-interpretable in $\frakT_1$.
% \end{itemize}
% 
% \end{theorem}

%that there is a constant $K$ such that
%the number of subsets 

%By proposition \ref{} there is a constant $K$ such that
%\[
% x \in \Power(S_n) \implies |f(x)| \leq n + k
%\]


\subsection{Countable elementary substructures}

%In algebra one looks at structures up to isomorphism. In logic 
Two structures with the same FO-theory are called {\em elementary equivalent}.
In this section we show a way of producing, from a B\"uchi- or Rabin-automatic structure $\frakB$, an elementary equivalent 
substructure $\frakA$. Thus although $\frakA$ may not itself be automatic, it has decidable FO-theory.

Let $\frakA,\frakB$ have the same signature. 
Say that $\frakA$ is an {\em elementary substructure} of $\frakB$ if 
$\A \subseteq \B$ and for all formulas $\phi(\tup{x})$ and all $\tup{a}$ from $\A$,
\[
\frakA \models \phi(\tup{a}) \mbox{ if and only if } \frakB \models \phi(\tup{a}).
\]

Note in particular that $\frakA$ and $\frakB$ agree on the atomic relations of $\frakA$, and thus $\frakA$ is a substructure of $\frakB$.
There is a simple characterisation of being an elementary substructure.

\begin{lemma}[Tarski-Vaught]
Say $\A \subseteq \B$.
Then $\frakA$ is an elementary substructure of $\frakB$ if and only if  for every FO-formula $\phi(x,\tup{y})$ and all $\tup{a}$ from $\A$
\[
\frakB \models \exists x \phi(x,\tup{a}) \,  \implies\,  \frakA \models \exists x \phi(x,\tup{a}).
\]
\end{lemma}

\begin{proof}
%For `only if' note that 
Details rqd.
\end{proof}

Say $f:\frakA \simeq (B,S_1,\cdots,S_N)$ is an $\omega$-string automatic presentation of $\frakA$. %??? abuse of notation. fix/mention?
Write $\frakA_{\rm{up}}$ for the substructure of $\frakA$ isomorphic via $f$ to the substructure whose domain consists of the ultimately periodic strings from $B$.
Similarly if $\frakA \in \raut$ define $\frakA_{\rm{reg}}$ as the substructure of $\frakA$ isomorphic via $f$ to the substructure consisting of regular trees from $B$.

\begin{proposition}
\begin{enumerate}
\item Let $\frakA \in \baut$. The structure $\frakA_{\rm{up}}$ is an elementary substructure of $\frakA$.
\item Let $\frakA \in \raut$. The structure $\frakA_{\rm{reg}}$ is an elementary substructure of $\frakA$.
\end{enumerate}
\end{proposition}

\begin{proof}
Use the fact that an automaton --- possibly instantiated with ultimately periodic strings $\tup{a}$ --- is non-empty only if it contains an ultimately periodic string. Similarly for the tree case.
\end{proof}

Let $\frakA$ be the $\omega$-string automatic structure $(\R,+,<)$ where reals are coded in binary. 
Then $\frakA_{\rm{up}}$ is isomorphic to $(\Q,+,<)$. It is not known whether $(\Q,+,<) \in \raut$.
%It is known that $(\Q,+)$ has no finite-string automatic presentation (and hence by ???, no $\omega$-string automatic presentation).

??? tree ex ???


%=====================================================
%=====================================================
%=====================================================
\subsection{Algebraic operations on automatic structures}
%=====================================================
%=====================================================
%=====================================================


{\bf Notation.}

  Let $\av$ stand for `finite-string', `$\omega$-string', `finite-tree', or `$\omega$-tree'.
Thus we write $\av$-automatic. 

\subsubsection{Closure under interpretations}

The $\av$-automatic structures are closed under FO-interpretations (proposition \ref{AS:prop:compose}).
In particular, if $\frakA$ is $\av$-automatic, and $R$ is FO-definable in $\frakA$ then $(\frakA,R)$ is $\av$-automatic.

There is a more general notion of interpretation called a $d$-dimensional FO-interpretation.
Here $\delta$ has $d$ free variables and each $\Phi_i$ has $d\times r_i$ free variables.

\begin{proposition}
$\av$-automatic structures are closed under $d$-dimensional FO-interpretations.
\end{proposition}

\begin{proof}
Details rqd.
% We illustrate the idea for Rabin-automatic structures.
% Say $\frakA$ is $d$-dimensionally interpretable in $\frakT_2$. Then $\code{a}$, for an element $a$, is
% a $\{0,1\}^d$-labeled tree. Recode $a$ by replacing the node $u$
% by a left-most path whose $i$th label is the $i$th coordinate of $\code{a}(u)$. For each successor $u0$ and $u1$ do the same starting at the end of the path just constructed for $u$.
% This transformation sends regular relations of $\{0,1\}^d$-labeled trees to regular relations of $\{0,1\}$-labeled trees.
\end{proof}

Say $\frakA$ and $\frakB$ are FO-interpretable in $\frakU$. Then the disjoint union of $\frakA$ and $\frakB$ is $2$-dimensionally interpretable in $\frakU$. Similarly for their direct product.

\begin{corollary}
$\av$-automatic structures are closed under disjoint union and direct product.
\end{corollary}

The {\em (weak) direct power} of $\frakA$ is a structure with the same signature as $\frakA$, its domain consists 
of (finite) sequences of $A$, and the 
interpretation of a relation symbol $R$ is the set of sequences $\sigma$ such that $R^\frakA(\sigma(n))$ holds for all $n \in \N$.
For example the weak direct power of $(\N,+)$ is isomorphic to $(\N,\times)$; the isomorphism sends $n$ to  the finite sequence $(e_i)$ where
$ \prod p_i^{e_i}$ is the prime power decomposition of $n$ ($p_i$ is the $i$th prime).
 
\begin{proposition}
The class $\raut$ is closed under (weak) direct power. The class $\taut$ is closed under weak direct power.
\end{proposition}

\begin{proof}
We illustrate the idea for Rabin-automatic structures.
Let $\sigma = (a_i)$ be an element of the direct power of $\frakA$.
Code the sequence $\sigma$ by the tree $t(\sigma)$ whose subtree at $0^n1$ is the tree $\code(a_n)$.
Let $R$ be a relation symbol. The interpretation of $R$ in the direct power is recognised by a tree automaton: it  processes $t(\sigma)$ by checking
that each subtree rooted at $0^n1$ is recognised by the automaton for $R^\frakA$.
\end{proof}

\subsubsection{Closure under quotients} \label{AS:subsub:quotient}
Let $\frakA = ({A},R_1,\cdots,R_N)$ be a structure.
An equivalence relation $\epsilon$ on the domain ${A}$ is called a {\em congruence for $\frakA$} if each relation $R_i$ satisfies the following property:
for every pair of $r_i$-tuples $\tup{x},\tup{y}$ of elements of ${A}$, if $x_j \epsilon y_j$ for $1 \leq j \leq r_i$ then $R_i(\tup{x})$ if and only if $R_i(\tup{y})$.

The {\em quotient of $\frakA$ by $\epsilon$}, written $\frakA/\epsilon$ is the structure whose domain is the set of equivalence classes of $\epsilon$ and whose $i$th relation
is the image of $R_i$ by the map sending $u \in {A}$ to the equivalence class of $u$.

\begin{quote}
If $(\frakA,\epsilon)$ is $\av$-automatic, is $\frakA/\epsilon$ $\av$-automatic?
\end{quote}

\subsubsection*{$\waut$:}
There is a regular well-ordering of the set of finite strings, for instance the length-lexicographic
ordering $\llex$. Use this order to define a regular set $D$ of unique $\epsilon$-representatives.
Then restrict the presentation of $\frakA$ to $D$.

\subsubsection*{$\taut$:}
Except in the finite word case, there is no regular well ordering of the set of
all finite trees \cite{CL07csl}. 
However one can still convert a finite-tree automatic presentation of $(\frakA,\epsilon)$ into
one for $\frakA/\epsilon$ \cite{CL07LMCS}. The idea is to associate with each tree $t$ a
new tree $\hat{t}$ of the following form: the domain is the intersection 
of the prefix-closures of the domains of all trees that are $\eL({A}\epsilon)$-equivalent to $t$; 
a node is labelled $\sigma$ if $t$ had label $\sigma$ in that position; 
a leaf $x$ is additionally labelled by those states $q$ from which the 
automaton ${A}_{\epsilon}$ accepts the pair consisting of the subtree of $t$ 
rooted at $x$ and the tree with empty domain.
Using transitivity and symmetry of $\eL({A}_\approx)$, if $\hat{t} = \hat{s}$ 
then $t$ is $\eL({A}_\approx)$-equivalent to $s$. 
Moreover each equivalence class is associated with finitely many new trees, 
and so a representative may be chosen using any fixed regular linear ordering 
of the set of all finite trees.\footnote{The construction 
given in \cite{CL07LMCS} is slightly more general and allows one to effectively 
factor finite-subset interpretations in any tree.}


%In fact, Arnaud says he has a proof that every finite-tree regular equivalence
%relation has a regular set of representatives

\subsubsection*{$\baut$:} %??? change \omega-regular to ...?
Kuske and Lohrey \cite{} observed that there is no unique set of representatives of the equal almost-everywhere relation that is $\omega$-regular.
Thus we can't quotient using the trick that worked for $\waut$. In fact, there is a structure in $\baut$ whose quotient is not $\omega$-word 
automatic \cite{HKMN08}. The proof actually shows that the structure has no Borel presentation, see \ref{AS:thm:borel}.
 
However, every $\omega$-string regular equivalence relation with countable index has an $\omega$-regular set of unique representatives \cite{KRB08}.
Thus if $\frakA /\epsilon$ is countable and $(\frakA,\epsilon)$ is $\omega$-string automatic then so too is $\frakA/\epsilon$; and consequently the quotiented structure
is also finite-string automatic since every countable $\omega$-regular set consists only of ultimately periodic strings with a uniform bound on the length of the periods.

\subsubsection*{$\raut$:}
Nothing is known.

~\\

Nonetheless, quotients still have decidable theory.

\begin{proposition}
If $(\frakA,\epsilon)$ is $\av$-automatic then the quotient $\frakA/\epsilon$ has decidable FO-theory.
\end{proposition}

\begin{proof} 
details rqd.
\end{proof}

%=====================================================
%=====================================================
%=====================================================
\section{Anatomy of finite-string and finite-tree automatic structures}
%=====================================================
%=====================================================
%=====================================================



Not much is known about Rabin-automatic structures. For instance, the only known techniques for showing that a structure
$\frakA$ is not in $\raut$ is to show $\frakA$ has undecidable FO-theory.\footnote{A refined method would show that the $\FOext$-theory
of $\frakA$ is of high complexity. ??? } So instead we focus on the other classes.

{\bf Notation.} Let $\frakA \in \waut$. For every ${\frakA}$-formula $\psi(x_1,\cdots,x_k)$, fix a
deterministic automaton %??? unify DFA notation
$(Q_\psi,\iota_\psi,\Delta_\psi,F_\psi)$ recognising $\code{\psi^{\frakA}}$ and write
$\Gamma_\psi(w)$ for $\Delta_\psi(\iota_\psi,w)$.
??? what about $\taut$???
Write $h(x)$ for the height of a tree $x$, and $h(\tup{x})$ for the height of the largest tree in $\tup{x}$.

??? generalise this to loc fin relations ???
\begin{proposition} \label{AS:prop:locfin}
Suppose that the partial function $F:A^n \to A$ is finite-tree regular, and let $p$ be the number of states of the automaton.
Then
$$h(F(\tup{x}))
\leq h(\tup{x}) + p
$$
for all $\tup{x}$ in the domain of $F$.
\end{proposition}

\begin{proof}
Otherwise, take a counterexample  $\tup{x}$.
After all of $\tup{x}$ has been read, and while still reading $F(\tup{x})$, some path in the run must have a repeated state. 
So the automaton also accepts infinitely many tuples of the form $(\tup{x}, \cdot)$.
\end{proof}

\subsubsection{Growth of generation}

\begin{definition} \label{dfn:growth}
Let $\frakA$ be a structure with functions $f_1, \cdots, f_k$ of arities $r_1, \cdots, r_k$ respectively. 
Let $D \subseteq A$ be a finite set.
Define the
{\em $n$th growth level}, written $G_n(D)$, inductively by $G_0(D) = D$
and $G_{n+1}(D)$ is the union of $G_n(D)$ and
\[
\bigcup_{i\leq k} \{f_i(x_1,\cdots,x_{r_i}) \st x_j \in G_n(D) \text{ for } 1 \leq j \leq r_i\}.
\]
\end{definition}
%??? fix spacing in eqnarray

How fast does $|G_n(D)|$ grow as a function of $n$ ?  For
example, consider the free semigroup $(\2,\cdot)$ with generating
set $D = \{d_1, \cdots, d_m\}$. For $m \geq 2$ the set $G_n(D)$ includes all strings over $D$ of length
at most $2^n$; thus the cardinality of $G_n(D)$ is at least $2^{2^{n}}$.

\begin{proposition} %{\rm \cite{BlGr00}, cf. \cite{KhNe95}}
 \label{prop:growth}
Let $\frakA \in \taut$ and $D \subset A$ be a finite set. Then there is a
linear function $t:\N \rightarrow \N$ so that for all $e \in G_n(D)$ the tree $\code{e}$ has
height at most $t(n)$.
\end{proposition}
\begin{proof}
 Iterate proposition~\ref{AS:prop:locfin}.
\end{proof}

\begin{corollary} 
If $\frakA \in \taut$ then $|G_n(D)| \leq 2^{2^{O(n)}}$. If $\frakA \in \waut$ then 
$|G_n(D)| \leq 2^{O(n)}$.
\end{corollary}
\begin{proof}
Count the number of $\{0,1\}$-labelled trees (strings) of height (length) at most $k$.
\end{proof}

Thus the free semigroup on more than two generators is not in $\waut$. Moreover, since it has undecidable FO-theory \cite{}, %QUINE 46
it is not in $\taut$.

\subsubsection{Growth of definable sets}

\begin{definition}
Let $\phi(x,y)$ be a FO-formula in the signature of $\frakA$ and $E \subset{A}$ finite. 
Define the {\em shadow of $\phi$ on $E$} as the set $\{\phi^\frakA(u,\cdot) \cap E \st u \in {A}\}$ where $\phi^\frakA(u,\cdot)$
is the subset of $A$ defined by $\phi$ with first component-fixed to $u$.
\end{definition}

For example let $\frakA$ be the random graph and $\phi(x,y)$ the formula that there is an edge between $x$ and $y$.
Then for every finite subset $E$ the shadow of $\phi$ on $E$ consists of all subsets of $E$.

??? generalise to $\tup{y}$ ???
\begin{proposition}[\cite{} \cite{}]
Suppose $\frakA \in \waut$ and $\phi(x,y)$ is a FO-formula in the signature of $\frakA$.
Then there is a constant $k$, 
that depends on the automata for ${A}$ and $\phi$, and infinitely many finite subsets $E \subset {A}$ such
that the cardinality of the shadow of $\phi$ on $E$ is at most $k|E|$.
\end{proposition}

\begin{proof}
Let $E_n$ be the set of strings in ${A}$ of length at most $n$. For every $x \in {A}$ there is a 
$y \in E_{n+c}$ such that $\phi(x,\cdot) \cap E_n = \phi(y,\cdot) \cap E_n$. Indeed $c$ is chosen so that
if $|x| \geq n+c$ then we can pump $x$ down, repeatedly, to get the string $y$.  Now use the fact that
$|E_{n+1}| \leq c'|E_n|$ for some constant $c'$ that depends on the automaton for ${A}$.
\end{proof}

Consequently the random graph is not in $\waut$.

\begin{proposition}[\cite{Delh04}]
Suppose $\frakA \in \taut$ and $\phi(x,y)$ is a FO-formula in the signature of $\frakA$.
Then there is a constant $k$, 
that depends on the automatic presentation of $\frakA$, and infinitely many finite subsets $E \subset {A}$ such
that the log of the cardinality of the shadow of $\phi$ on $E$ is at most $k\log|E|$.??? 
\end{proposition}

Thus the random graph is not in $\taut$.

\subsubsection{Sum- and box-decompositions}

All results in this section are due to Delhomm\'e \cite{}, \cite{Delh01a}.
\begin{definition} 
Say that a structure $\frakB$ is a {\em sum-decomposition} of a set of structures
$\eS$ (each having the same signature as $\frakB$) if there is a finite partition of $B
= B_1 \cup \cdots \cup B_n$ such that for each $i$ the substructure $\frakB
\restriction B_i$ is isomorphic to some structure in $\eS$.
\end{definition}

\begin{theorem} \label{thm:sumaug}
Suppose $\frakA \in \waut$ and $\phi(x,\tup{y})$ is a FO-formula in the signature of $\frakA$.
There is a finite set of
structures $\eS$ so that for every tuple of elements $\tup{b}$ from $A$, the
substructure ${\frakA} \restriction \phi^{\frakA}(\cdot,\tup{b})$ is a sum-decomposition
of $\eS$.
\end{theorem}

\begin{proof}
Let $\frakA = \left<\A, R_1,\cdots,R_N\right>$. We will use the following property $(P_{\psi})$:
For all strings $c_i,d_i$ with the $c_i$s all the same length,
$$
\Delta_\psi(\Gamma_\psi(\con(c_1,\cdots,c_k)),\con(d_1,\cdots,d_k)) \in F_\psi
$$
if and only if $\frakA \models \psi(c_1d_1,\cdots,c_kd_k)$.

Now, given an $\frakA$-formula $\phi(x,y_1,\cdots,y_k)$ as in the hypothesis,
and tuple $\tup{b}$,
observe that for $m = \max\{|b_i|\}$, we can partition $\phi^{{A}}(\cdot,\tup{b})$
into the finitely many singletons $\{c\}$ such that $\phi(c,\tup{b})$ and  $|c| <
m$, and the finitely many sets 
\[
\phi^{a \2}(\cdot,\tup{b}) := \{aw \in \A \st {\frakA} \models \phi(aw,\tup{b}), w \in \2\}
\]
such that $|a| = m$. There are finitely
many isomorphism types amongst substructures of the form ${A} \restriction
\{c\}$, for $c \in \A$. So, it is sufficient to show that as we vary the tuple
$(a,\tup{b})$ subject to $|a| = \max\{|b_i|\}$, there are finitely many isomorphism
types amongst substructures of the form $\frakA \restriction \phi^{a \2}(\cdot,\tup{b})$.

We do this by bounding the number of isomorphism types in terms of the number
of states of the automata. To this end, define a function $f_\phi$ as
follows. Its domain consists of tuples $(a,\tup{b})$ satisfying $|a| =
\max\{|b_i|\}$; and $f_\phi$ sends this tuple to the tuple of states
$$
\left<\Gamma_\phi(\con(a,b_1,\cdots,b_k)),
(\Gamma_{R_i}(\con(a, \cdots, a)))_{i\leq N}\right>.
$$
The range of $f_\phi$ is bounded by $|Q_\phi| \times |Q_\A| \times \prod_{i
\leq r} |Q_{R_i}|$. In particular, the range is finite.

To finish the proof, we argue that the isomorphism type of the
substructure $\frakA \restriction \phi^{a \2}(\cdot,\tup{b})$ depends
only on the value $f_\phi(a,\tup{b})$. This follows from the fact that if
$f_\phi(a,\tup{b}) = f_\phi(a',\tup{b'})$, then the corresponding substructures
are isomorphic via the mapping $I:aw \mapsto a'w$ ($w \in \2$).
Indeed, by property $(P_\phi)$ we get that $\frakA \models \phi(aw,\tup{b})$ if and only if $\frakA \models \phi(a'w,\tup{b})$.  This gives the bijective property.
By the properties $(P_\psi)$ where $\psi$ is taken to be
$\tup{x} \in R_i$, we get that $I$ is an isomorphism of the substructures.
\end{proof}


\begin{corollary}  \label{cor:omegaomega}
The ordinal $(\omega^\omega,\leq)$ is not in $\waut$.
\end{corollary}

\begin{proof}
Suppose for a contradiction that $(\omega^\omega,\leq)$ has an automatic presentation.
In theorem \ref{thm:sumaug}, take $\phi(x,y)$ to be $x < y$ and
consider the following fact (proved by induction): if the domain of a well-order,
isomorphic to some ordinal of the form $\omega^n$ for $n \in \N$, is
partitioned into finitely many pieces $\{B_i\}_i$, then there is some $i$ so
that the substructure on domain $B_i$ is isomorphic to $\omega^n$.

This means that the set of structures $\eS$ must contain $(\omega^n,<)$ for every $n \in \N$, contradicting
the finiteness of $\eS$.
\end{proof}

Since the set of ordinals in $\waut$ are closed downwards, this means that $\omega^\omega$ is the least ordinal not in $\waut$.

\begin{definition}
Let $(\frakB_i)_{i \in I}$ be a non-empty finite sequence of structures over the same signature. Their {\em asynchronous product} is a structure over the same signature 
defined as follows. Its domain is $\prod_{i \in \I} \B_i$. Write $\pi_j$ for the projection $\prod_{i \in I} \B_i \to \B_j$. The interpretation of an $r$-ary relation symbol $R$
consists of those tuples $(x_1,\cdots,x_r)$ such that there exists $k \in \I$ such that
\[
\frakB_k \models (\pi_k(x_1),\cdots,\pi_k(x_r)) \in R
\]
and all $m \neq k$ satisfy $\frakB_m \models \pi_m(x_1) = \cdots = \pi_m(x_r)$.
\end{definition}

The asynchronous product of the singleton $(\frakB_1)$ is simply the structure $\frakB_1$. 

\begin{theorem}
Suppose $\frakA \in \taut$ and $\phi(x,\tup{y})$ is a FO-formula in the signature of $\frakA$.
There is a finite set of structures $\eS$ so that for every tuple of elements $\tup{b}$ from $A$, the
substructure ${\frakA} \restriction \phi^{\frakA}(\cdot,\tup{b})$ is a sum-decomposition of the set
of all asynchronous products of sequences from $\eS$.
\end{theorem}

The proof is similar to that of \ref{}.

\begin{corollary}
The least ordinal not in $\taut$ is $\omega^{\omega^{\omega}}$.
\end{corollary}


%=====================================================
%=====================================================
%=====================================================
\section{Equivalent presentations of finite-string automatic structures}
%=====================================================
%=====================================================
%=====================================================

In this section we focus on finite-string automatic presentations of a fixed structure. We illustrate with
$(\N,+)$.
Write $\mu_k$ for the co-ordinate map corresponding to the base-$k$ presentation of $(\N,+)$, $k \in \N, k \geq 2$.
The translation between bases $p$ and $q$ is the map $\mu_q \circ \mu_p^{-1}$. It sends a string in base-$p$ to that
string in base-$q$ that represents the same natural number.

Call two bases $p$ and $q$ {\em multiplicatively dependent} if for some positive integers
$k,l$
\[
 p^k = q^l.
\]

Here the translation sends blocks of $p$-ary digits of size $l$ to blocks of $q$-ary digits of 
size $k$.

\begin{proposition} \label{AS:prop:multdep}
If $p$ and $q$ are multiplicatively dependent and $R \subseteq \N^k$ is a relation, then 
$R$ is regular when coded in base $p$ if and only if $R$ is regular when coded in base $q$.
\end{proposition}

To see this we use semi-synchronous rational relations: these can be thought of as being
recognised by a multi-tape automaton where each read-head advances at a
different, but still constant, speed.

\begin{definition}
Fix a finite alphabet $\Sigma$ and a vector of positive integers $\underline{m} = (m_1,\cdots,m_r)$.  
Let $\blank$ be a symbol not in $\Sigma$ and write $\Sigma_{\blank}$ for $\Sigma \cup \{\blank\}$.
For each component $m_i$ introduce the alphabet 
$(\Sigma_{\blank})^{m_i}$.  The {\em $\underline{m}$-convolution of a tuple} 
$(w_1,\cdots,w_r) \in (\Sigma^{\star})^r$ is formed as follows: First, consider the intermediate
string $(w_1\blank^{a_1}, \cdots, w_r\blank^{a_r})$ where the $a_i$ are minimal
such that there is some $k \in \N$ so that for all $i$, $|w_i|+a_i = km_i$. Second,
partition each component $w_i\blank^{a_i}$ into $k$ many blocks of size $m_i$,
and view each block as an element of $(\Sigma_{\blank})^{m_i}$. Thus the string
$\con_{\underline{m}} (w_1,\cdots,w_r)$ is formed over alphabet
$(\Sigma_{\blank})^{m_1} \times \cdots \times (\Sigma_{\blank})^{m_r}$.
The {\em $\underline{m}$-convolution of a relation} $R \subseteq (\Sigma^{\star})^r$ is the
set $\con_{\underline{m}} R$ defined as 
\[
\{\con_{\underline{m}} \tup{w} \st \tup{w} \in R\}.
\]
A relation $R$ is {\em \underline{m}-synchronous rational} if there is a finite automaton
recognising $\con_{\underline{m}} R$.
Call $R$ {\em semi-synchronous} if it is $\underline{m}$-synchronous rational for some $\underline{m}$.
\end{definition}

For example, if $\underline{m} = (1,\cdots, 1)$ then $\con_{\underline{m}}$ is the same as $\con$.
For another example, the base-changing translation above is $(l,k)$-synchronous. Proposition \ref{AS:prop:multdep} now follows
from the fact that the image of a regular relation under a semi-synchronous transduction is regular.
??? say more ???

On the other hand, if $p$ and $q$ are multiplicatively independent then, for instance, the set of powers of $p$ is regular in base-$p$ but not
regular in base-$q$. The Cobham-Semenov theorem is a beautiful generalisation of this fact, see \cite{}.
This discussion is the inspiration for the following generalisation.

%The Cobham-Semenov theorem deals with the other case.

%\begin{theorem}[Cobham-Semenov cf. \cite{BHMV94,Bes00,Much03}] \label{AS:thm:CS}
%If $p$ and $q$ are multiplicatively independent then a relation $R \subseteq \N^r$ is regular in
%both base $p$ and base $q$ if and only if $R$ is $\FO$-definable in $(\N,+)$.
%\end{theorem}

%Informally multiplicatively dependent bases are similar, while multiplicatively independent bases are as different
%from each other as can be. For instance, the set of powers of $p$ is a regular set when coded in base $p$ but is not regular when
%coded in bases multiplicatively independent from $p$.

For a given automatic presentation $\mu:\frakA \simeq (B,S_1,\cdots,S_N)$ 
write $\mu_{\reg}$ for the collection of relations
$$
\{\mu^{-1}(R) \st R \subseteq B^k \text{ is a regular relation}, k \in \N\}.
$$

\begin{definition}[\cite{Bara06}]
Let $\mu$ and $\nu$ be co-ordinate maps from two automatic presentations of $\frakA$.
Then these presentations are {\em equivalent} if $\mu_{\reg}  = \nu_{\reg}$.
\end{definition}

\begin{theorem}[\cite{Bara06}]
Fix two automatic presentations of $\frakA$; say $\mu$ with domain $B$ and $\nu$ with domain $C$.
The presentations are equivalent if and only if the map $\nu \mu^{-1}:B \to C$, namely
 \[
  \{(\mu(x), \nu(x)) \in B \times C\st x \in \A\},
 \]
is semi-synchronous.
\end{theorem}

\begin{proof}
The interesting case is `only if'. %First some definitions.
% A bijection $f:X \to Y$ between regular sets {\em preserves regularity} ({\em preserves non-regularity})
% if the image under $f$ (under $f^{-1}$) of every regular relation is regular. 
 Let $f$ denote the translation $\nu\mu^{-1}:B \to C$. 
% First we define $\pi:C \to C'$ such that 
% 
% %$\mu$ is equivalent to the presentation $\pi\nu$ and 
% 
% the translation  $f':=\pi\nu\mu^{-1}:B \to C'$ is length monotonic (if $x \leq y$ then $f'(x) \leq f'(y)$)
% and has $\delta$-delay (if $|y| - |x| > \delta$ then $|f'(y)| > |f'(x)|$).
% Next, define $\beta:C' \to C''$ such that %$\mu$ is equivalent to $\beta\pi\nu$ and 
% the translation $f'':= \beta f':B \to C''$ is length-preserving ($|f''(x)| =|x|$).
Here is an outline: starting with $x \in B$, we apply $f$ to get $f(x) \in C$, then pad to get $f'(x) \in C'$,
then cut into blocks to get $f''(x) \in C''$. 
Write $\pi$ for the padding $C \to C'$ and $\beta$ for the blocking $C' \to C''$.
Then $f$ can be decomposed into semi-synchronous maps
\[
 B \stackrel{f''}{\to} C'' \stackrel{\beta^{-1}}{\to} C' \stackrel{\pi^{-1}}{\to} C.
\]

We need some definitions. 
For a set $X$ of strings, write $\mathbb{L}_X$ for the regular relation of pairs $(x,y) \in X \times X$ such that $|x| \geq |y|$. 
The {\em growth function} of a function $g$ between regular sets is the function 
$G:n \mapsto \max_{|x| \leq n} |g(x)|$. A bijection $f$ is {\em length preserving} if $|f(x)| = |x|$.
It is {\em length-monotonic} if $x_1 \leq x_2$ implies $f(x_1) \leq f(x_2)$. It has {\em $\delta$-delay} 
if $|x_2| > |x_1| + \delta$ implies $|f(x_2)| > |f(x_1)|$.


\

\noindent
{\em Claim 1.} There is a constant $\delta$ such that $f$ has $\delta$-delay.

Since $f^{-1}(\mathbb{L}_C) \subseteq B \times B$ is regular and locally finite, by proposition \ref{AS:prop:locfin} there is a $\delta$
such that $(a,b) \in f^{-1}(\mathbb{L}_C)$ implies $|b| \leq |a| + \delta$.% That is, $|f(a)| \geq |f(b)|$ implies $|b| -|a| \leq \delta$.)

\

\noindent
{\em Claim 2.} There is a constant $K$ such that $F(|x|) - |f(x)| \leq K$.

Since $f(\mathbb{L}_B) \subseteq C \times C$ is regular and locally finite, $|x| \geq |y|$ implies $|f(y)| \leq |f(x)| + K$.

\

Let $\natural$ be a new symbol. Define $f':x \mapsto f(x) \natural^{F(|x|) - |f(x)|}$, $x \in B$. Thus $f'$ pads 
$f(x)$ by $\natural$ symbols. Since the property $\{x \st F(|x|)-|f(x)| = i\}$ is regular for fixed $i$, 
by claim 2. the graph of $f'$ is regular. Write $C'$ for $f'(C)$.

\

\noindent
{\em Claim 3.} The translation $f':B \to C'$ is length-monotonic and has $\delta$-delay.

Use the fact that $|f'(x)| = F(|x|) = F'(|x|)$.

\

\noindent
{\em Claim 4.} There exists $p,s \in \mathbb{N}$ such that the sequence  $F'(n+p) - F'(n) = s$ for all $n$.

Let $u_n$ be the length-lexicographically least word amongst $\{f'(x) \st |x| = n, x \in B\}$ 
if this set is non-empty, and undefined otherwise. 
Since  $|u_{n+1}| - |u_n|$ is bounded and the set of pairs $(1^n,u_n)$ is regular,
the set of pairs $(1^n,1^{|u_{n+1}| - |u_n|})$ is regular. In particular
the sequence $\left<|u_{n+1}|-|u_n|\right>$, or what is the same $\left<F'(n+1) - F'(n)\right>$,
is ultimately periodic.  Take $p$ large enough so that $\left<F'(n+p) -F'(n)\right>$ is constant.

\

Now for $x \in B$ of length $n$ write $f'(x) $ as $v_1 v_2 \cdots v_n$ where $|v_i| = s$ (if $|f'(x)|$ is not a multiple of $s$, append
a (new) blank symbol until it is). For a word $w$ of length $s$ write $\widehat{w}$ for a new alphabet symbol. 
Define $f'':x \mapsto \widehat{v_1} \cdots \widehat{v_n}$. Write $C''$ for $f''(B)$. Clearly 
the translation $f'':B \to C''$ is length-preserving.

\

\noindent
{\em Claim 5.} The graph of $f''$ is regular. 

Use the fact that $f''$ is length-preserving and preserves all regular relations. Details rqd. %???

\


Finally, write $\pi:C \to C'$ for the map sending $f(x) \mapsto f'(x)$, and $\beta:C' \to C''$ for the map sending $f'(x) \mapsto f''(x)$.
Since $\pi$ is a projection its graph is regular, hence $\pi^{-1}$ is semi-synchronous.
Finally, $\beta^{-1}$ is semi-synchronous sending blocks of size $1$ to blocks of size $s$.
\end{proof}