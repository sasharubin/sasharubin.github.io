\documentclass{book}
\usepackage{tlgart}
\usepackage{amsthm}
%assumption definition observation
%axiom      example    problem
%claim      exercise   proposition
%conjecture fact       remark
%convention lemma      subclaim
%corollary  notation   theorem

\input{BGRsurvey.macro}
\newif\iffilled
\filledfalse
\title{Automatic Structures}{Automatic Structures}
\author{S. ~Rubin}{Sasha Rubin}
\address{310 Malott Hall\\
Cornell University\\
Ithaca \\
NY 14853-4201 USA}
\email{srubin@math.cornell.edu}


\begin{document}
\maketitle

\begin{abstract}
The field of automatic structures concerns those mathematical structures (such
as groups, orders, algebras, etc) that can be computed by automata. A structure
is called {\em automatic} if it can be coded in such a way that the induced
domain and relations and operations can be recognised by automata.  There are
as many types of automatic structures as there are models of automata, eg.,
automata operating (a)synchronously on (in)finite words or trees, possibly with
oracle.

This chapter discusses fundamental logical, computational and algebraic
properties of the main types of automatic structures. 

Equivalent ways of defining automatic structures are via formulas (in the
language of first-order or monadic-second order logic) or as solutions of
equations.  Automatic structures provide a framework in which to view other
classes of infinite structures, eg. context-free graphs. The central themes in
the literature include the classification of classes of automatic structures
and identifying essentially different presentations of a given structure.


\end{abstract}

\tableofcontents

\section{Introduction}

%Like any good story there are a number of ways of looking at automatic structures.

\subsection{From computable structures}
Automatic structures are instantiations of the more general notion of
computable structure but have better properties including decidability (of the
first-order theory) and closure under natural operations on structures (like
product).
 
\iffilled The standard logical terminology for talking about mathematical
objects is that of a {\em structure}: a {\em domain} (set of elements) and {\em
atomic} operations or relations on this domain.  From a description of a
structure we try to identify its properties.

For instance the free group on one generator $(\mathbb{Z}, +)$ is a structure.
We can also describe this structure computationally: code the integers, say in
base $10$, and supply an algorithm for computing the sum of two integers.

A dream (for some) would be an algorithm that given a query $\Phi$ decides
whether or not the structure satisfies $\Phi$. This is called a {\em
decidability procedure} (for the given structure and the class of queries which
are usually taken to be those expressible in a given logical language such as
first-order logic).

%Of course in general this is hopeless.  
Unfortunately decidability occurs rarely. For instance, the negative solution
to Hilbert's $10$th problem states that there is no algorithm that decides
whether a given Diophantine equation has a solution in the integers. So
although there is a simple computational description of the structure
$(\mathbb{Z},+,\times)$, even single quantifier queries of the form `Do there
exist integers $\tup{x}$ such that $D(\tup{x}) = 0$' are unmanageable
(just the same, there is a rich theory of such {\em computable structures}).  %idiom 

Since structures describable by arbitrary algorithms are too general, what if
we restrict the model of computation?  A structure is called {\em
automatic} if its domain and atomic operations are computable by
automata.  The precise details of the automata  - (in)finite words/trees,
(a)synchronous, with oracle, etc -  govern the general properties of
the resulting structures.  Automata operating synchronously on their inputs,
first studied by B\"uchi and Elgot on finite words leading to the work of Rabin
on infinite trees, are robust: they are effectively closed under logical
operations. This leads immediately to a fundamental observation.
\begin{theorem}[Fundamental Theorem]
The first-order theory of an automatic structure is decidable.
\end{theorem}

Thus from a description of a structure by automata we can decide certain
(first-order) queries about it. There are extensions of this theorem to include
generalised quantifiers such as `there exists infinitely many', however the
theorem fails for allowing set quantification (monadic-second order logic) or
extensions by fixed-point operators.  It is worth mentioning other important
techniques for producing decidability: eliminating quantifiers \cite{?} and
Shelah's composition-method \cite{?} which is based on ideas of Feferman and
Vaught (see \cite{?} for their relationship to the automata theoretic method).

\fi

\subsection{Graph-grammars and other automata based presentations}
Traditionally graph-grammars describe sets of finite-graphs. However they may
also be used to describe single, typically infinite, graphs. Standard formalism
include HR-equational graphs and VR-equational graphs. Other approaches are
ground-term rewriting systems. Tree-interpretable structures are
those that are MSO-definable in the full binary tree. All of these are examples
of previously-studied robust classes of tree-automatic structures.

\subsection{Other notions of automaticity}
Specific automatic presentations have been employed in other mathematical
fields: computational group theory, symbolic dynamics, numeration
systems (of integers or reals), and infinite sequences represented in natural
numeration systems (eg. morphic words).

\subsubsection{Thurston's automatic groups}
Motivated by work of Cannon on hyperbolic groups, Thurston
introduced `automatic groups'. These are finitely generated groups displaying
tractable algorithmic properties that are undecidable in the general case.

In a seminal paper Khoussainov and Nerode introduce automatic presentations as
a generalisation of Thurston's automatic groups.

\section{Definitions}
\subsection{Synchronous automata}
(Recall or cite) definition and closure properties of automata operating on finite or infinite words or trees (with oracle).

\subsection{Automatic presentations}
Definition of finite-word, infinite-word, finite-tree, infinite-tree automatic structures.
\subsection{Examples}
Include examples of reducts of arithmetics, (well)-orders, trees, Boolean algebras, universal structures.

\section{Basic properties}
\subsection{The fundamental theorem}
Explanation of first-order logic.

\begin{theorem}[Fundamental Theorem]
The first-order theory of an automatic structure is decidable.
\end{theorem}

\subsubsection{Extensions of the fundamental theorem}

Brief explanation of generalised quantifiers (cardinality, ramsey, etc.) and the extended theorem.
Limitations (eg. MSO, fixed-point operators).
\subsection{Equivalent definitions}
Explain MSO, interpretability. 
State equivalent definition: those structures interpretable in universal structures (for MSO and FO).  
Explain VRS operations (give example) and state equivalent definition of
word-automatic as VRS-equational structures. 

\subsection{Generalisations of the automaton model}
\subsubsection{Automata with oracle}
Define automata with (tree) oracle. State extension of fundamental theorem and the Colcombet-L\"oding theorem.

\subsubsection{Asynchronous automata}
Define rational graphs. Give basic properties and relationship to traces.


\section{Central themes}
\subsection{Proving non-automaticity}
Usually it is quite straightforward to show that a structure has an automatic
presentation (assuming it does indeed have one). Showing that a structure has
no automatic presentation is a significant challenge. Here are the basic techniques.

\subsection{Classification of classes of automatic structures}
In some cases a class of similar structures (such as groups or linear orders)
may be classically described by (full) invariants (often a natural number, or a
sequence of natural numbers).  We seek to identify those invariants that
correspond to the automatic members of the class. Examples: well-orders, Boolean algebras, trees, linear orders, groups.

\subsubsection{The isomorphism problem}
Deciding whether two presentations present isomorphic objects is called the
isomorphism problem. The complexity of this problem (in the analytic hierarchy)
is a measure of the complexity/richness of a class of structures. Examples:
general case, well-orders, Boolean-algebras, trees, linear orders, groups.

\subsection{Presentations up to equivalence}
An automatic structure has many presentations. Discuss equivalent presentations and results of B\'ar\'any, Colcombet, L\"oding.

\subsection{Summary of open questions}
Specific questions and general directions (eg. automatic model theory).
%\subsection{Automatic model theory}
%\section{Relationship with other presentations of infinite structures}
%\subsection{Computable presentations}
%\subsection{Automata based-presentations}
\end{document}

                                Abstract
General: The theory of automatic structures is an attempt to de-
scribe the properties of those mathematical structures (such as groups,
orders, algebras, etc) that can be computed by automata. A struc-
ture is called automatic if it can be coded in such a way that the
induced domain and relations and operations can be recognised by
automata. The precise details of the automata (words/trees, syn-
chronous/asynchronous, with oracle, etc) govern the general proper-
ties of the resulting structures. Automata operating synchronously
on their inputs are robust - they are effectively closed under logical
operations. Consequently, structures presented by these automata
have a decidable first-order theory. This chapter focuses on these
synchronously-automatic structures and their logical and algebraic
properties.
Simplified: The standard logical terminology for talking about math-
ematical objects is that of a structure: a set of elements (domain) and
so called atomic operations and relations on this domain (for instance
the semigroup (N, +) is a structure). An automatic structures is one
that can be described by automata, in particular it can be coded
so that the induced domain and atomic operations and relations are
recognised by finite automata. There are as many types of automatic
structures as there are models of automata, eg., automata operating
(a)synchronously on (in)finite words or trees, possibly with oracle.
This chapter discusses fundamental logical and algebraic properties of
the main types of automatic structures. Equivalent ways of definining
automatic structures are via formulas (in the language of first-order
or monadic-second order logic), and as solutions of equations in the
language of graphs.

