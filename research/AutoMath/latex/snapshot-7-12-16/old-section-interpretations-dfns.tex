A good reference for interpretations is \cite{Hodg93}.
Let $\eL$ denote FO or MSO or WMSO. Let  $\I = (\delta, \Phi_1, \dots, \Phi_N)$ be $\eL$-formulas in the signature of a structure $\frakB$ in 
which the free variables are individual variables.
Suppose that $\delta$ has $1$ free variable and $\Phi_i$ has $r_i$ free variables.
If $\Phi_i^\frakB$ are relations over $\delta^\frakB$ then we can define the structure 
$\I(\frakB) := (\delta^\frakB, \Phi_1^\frakB, \dots, \Phi_N^\frakB).$ 
This structure is said to be {\em $\eL$-definable in $\frakB$}. The tuple $\I$ is called an {\em $\eL$-definition}.

{\bf Remark.} We have overloaded the phrase ``$\eL$-definable'': here one structure is definable in another; 
and the earlier meaning is that a relation is definable in a structure.

The following lemma says, in particular, that every relation FO-definable in $\I(\frakB)$ is 
FO-definable in $\frakB$.

\begin{lemma} \label{AS:lem:translation} \todo{It is important for Prop. 2.9 that $\Phi_{\mathcal I}$ can be
computed from $\Phi$.}
Fix an $\eL$-definition $\I$ and an $\eL$-formula $\Phi(x_1,\dots,x_k)$ in the signature of $\I(\frakB)$.
There is an $\eL$-formula $\Phi_\I(x_1,\dots,x_k)$ in the signature of $\frakB$ such that for all elements $b_i$ of $\delta^\frakB$,
\[
\I(\frakB) \models \Phi(b_1,\dots,b_k) \mbox{ if and only if } \frakB \models \Phi_\I(b_1,\dots,b_k).
\]
\end{lemma}

\begin{proof}
The idea is to relativise all quantifiers to $\delta$ and replace the $i$th atomic formula $R_i$ in the signature of $\I(\frakB)$ 
by $\Phi_i$.
Formally, define $\Phi_\I$ inductively by $(R_i)_\I := \Phi_i$ and for the other cases:

\begin{align*}
(\Psi \wedge \Xi)_\I & := \Psi_\I \wedge \Xi_\I  					&(\neg \Psi)_\I & := \neg \Psi_\I \\
(\exists x_i \Psi)_\I & := \exists x_i [\delta(x_i) \wedge \Psi_\I]  		&(\exists X_i \Psi)_\I & := \exists X_i [(\forall x \in X_i \delta(x)) \wedge \Psi_\I]\\
(x \in X)_\I &:= (x \in X)  & (X \subseteq Y)_\I &:= (X \subseteq Y) .
\end{align*}
\end{proof}

\begin{definition}
Let $\I$ be an $\eL$-definition. If $\frakA$ is isomorphic to $\I(\frakB)$, say via $f$, then say that {\em $\frakA$ is  $\eL$-interpretable in $\frakB$ via co-ordinate map $f$}.
\end{definition}

\begin{proposition}
Suppose that $\frakA$ is $\eL$-interpretable in a structure with decidable $\eL$-theory. Then $\frakA$ has decidable $\eL$-theory.
\end{proposition}

\begin{proof}
For a sentence $\phi$ of $\frakA$, Lemma \ref{AS:lem:translation} produces a sentence $\phi^\I$ preserving truth. 
Apply the given decision procedure to $\phi^\I$. 
\end{proof}

\todo{update: missing $T_r$}
For example  for $r < \omega$ the structure $\frakT_r$ is MSO-interpretable in $\frakT_2$ and consequently the MSO-theory of $\frakT_r$ is decidable.

There are other types of interpretations that could be introduced here, namely
(finite)-set interpretations \cite{ElRa66,CoLo07}.  For the sake
of filling in undefined notions in the introduction we define a {\em (finite)-set
interpretation} of $\frakA$ in $\frakB$ to be like a (W)MSO-interpretation except
that the free variables are (finite) set
variables. Thus elements of $\frakA$ are coded by (finite) subsets of the domain
of $\frakB$.
