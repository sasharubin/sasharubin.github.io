%% old Rabin automatic
\subsection{Rabin-automatic structures}

\todo{better to define autstr as those structures that are set-interpretable in $T_2$}
MSO-definability can be thought of as FO-definability in a power structure. We use this idea to define the automatic structures.
Let $\Power(X)$ denote the set of subsets of $X$. 

\begin{definition} \label{AS:dfn:powerset} \cite{CoLo07}
The {\em power} of a structure $\frakA = ({\A},R_1,\dots,R_N)$ is the structure
\[
\Power[\frakA] := (\Power({\A}),{R'_1}, \dots,{R'_N}, \subseteq)
\]
where 
\[
 R'_i := \{ (\{x_1\},\dots,\{x_{r_i}\}) \st (x_1,\dots,x_{r_i}) \in R_i\}. 
\]
\end{definition}

For example, $\Power[\one]$ is the structure with domain $\Power(\N)$, the subset relation $\subseteq$, and the binary relation $\{(\{n\},\{n+1\}) \st n \in \N\}$.
The following says that FO definability in the power of a structure is the same as MSO definability in the structure.

\begin{proposition} \label{AS:prop:translation}
For every FO-formula $\phi(\tup{x})$ there is an MSO-formula $\Psi(\tup{X})$ (and {\it vice versa}) such that for all structures $\frakA$ and all $U_i \in \Power({\A})$
\[
\Power[\frakA] \models \phi(\tup{U}) \ \mbox{ if and only if } \  \frakA \models \Psi(\tup{U}) .
\]
\end{proposition}

As a consequence we can transfer MSO-decidability of $\frakT_r$ to FO-decidability of the power structure.
\begin{corollary}
The FO-theory of $\Power[\frakT_r]$ is decidable.
\end{corollary}

Since FO-interpretations preserve decidability we make the following definition.

\begin{definition} \label{AS:dfn:raut} \cite{Blum99} \todo{ do one-dimensional interpretations suffice?}
A structure FO-interpretable in $\Power[\frakT_2]$ is called 
{\em Rabin-automatic} or {\em $\omega$-tree automatic}. The collection of these structures is written $\raut$.
\end{definition}

Note that any structure that is FO bi-interpretable with $\Power[\frakT_2]$ could be used in this definition. For instance we can use the following structure \cite{Blum99}:
the domain consists of all finite and infinite trees\footnote{Here a tree is a function $T$ from a prefix-closed subset of $\2$ to $\{0,1\}$.}  and the atomic relations are
 \[
       \exteq, \, \edom,\,  (\suc_a^d)_{a \in \{0,1\}}^{d \in \{0,1\}},  \epsilon_0, \epsilon_1, \F 
 \]
where  $T \exteq S$ if $\dom(T) \subseteq \dom(S)$
and $S(w) = T(w)$ for $w \in \dom(T)$; $T \edom S$ if $\dom(T) = \dom(S)$;  $\suc_a^d(T) = S$ if the finite tree $S$ is formed from the finite tree $T$ by extending its leaves in direction $d$
and labeling each new such node by $a$; $\epsilon_a$ is the tree with domain $\{\lambda\}$ labelled $a$; and $\F$ is the set of finite trees.

The elements of the Rabin-automatic structure are naturally viewed as $\omega$-trees. 

\begin{definition}
For a Rabin-automatic structure $\frakA$ (isomorphic via $f$ to a FO-definable structure $\I(\Power[\frakT_2])$) and any relation $R \subseteq \A^k$, denote by
$\code{R}$ the set of $\omega$-trees 
\[\{\conv(\chi_{f(a_1)},\dots,\chi_{f(a_k)}) \st  (a_1,\dots,a_k) \in R\}.\]
\end{definition}

The following is called the {\em fundamental theorem of automatic structures} and says that FO-definable relations in Rabin-automatic structures are, modulo coding, regular. 

\begin{theorem}[Fundamental theorem, cf. \cite{KhNe95,BlGr00}] \label{AS:thm:fundthm} 
Let $\frakA$ be Rabin-automatic.
\begin{enumerate}
\item For every first-order definable relation $R$ in $\frakA$ the set of trees $\code{R}$ is recognised by an $\omega$-tree automaton.
\item The first-order theory of a Rabin-automatic structure is decidable.
\end{enumerate}
\end{theorem}

\begin{proof}
For the first item apply Proposition~\ref{AS:prop:translation} to get that $f(R)$, being FO-definable in $\Power[\frakT_2]$, is 
MSO definable in $\frakT_2$. Now apply Rabin's theorem.
For the second item use the fact that it is decidable whether or not an $\omega$-tree automaton accepts some tree.
\end{proof}

Since formulas are seen as automata, we may view a Rabin-automatic structure as being presented by automata.



\begin{definition}[$\omega$-tree automatic presentation \cite{BlGr00}] \label{AS:dfn:rap} 
Suppose that $f: \frakA \simeq  (B,S_1,\dots,S_N)$ and
\begin{enumerate}
\item the elements of $B$ are $\{0,1\}$-labeled trees;
\item the set $B$ is recognised by an $\omega$-tree automaton, say $M_B$; 
\item the set $\{\conv(\tup{t}) \st \tup{t} \in S_i\}$ is recognised by an $\omega$-tree automaton, say $M_i$, for $i \leq N$.
\end{enumerate}
Then the data $\left<(M_B,M_1,\dots,M_N), f \right>$ is called an {\em $\omega$-tree automatic presentation} of $\frakA$.
\end{definition}

The following characterisation is typically taken as a definition of a structure being Rabin-automatic. 

\begin{proposition}[Machine theoretic characterisation \cite{BlGr00}]
A structure $\frakA$ is FO-interpretable in $\Power[\frakT_2]$ if and only if $\frakA$ has an $\omega$-tree automatic presentation.
\end{proposition}

\begin{proof}
The forward direction follows from the fundamental theorem (Theorem~\ref{AS:thm:fundthm}). For the reverse direction convert regular sets and relations in the presentation
into MSO formulas over $\frakT_2$ (using Rabin's theorem) and then into FO-formulas using Proposition~\ref{AS:prop:translation}.
\end{proof}

\begin{example}
 The following structures are Rabin-automatic. \todo{ you should, at least, provide proof ideas or references for
these claims.}
\begin{enumerate}
 \item The power structure $\Power[\frakT_r]$ ($r \geq 1$) as well as its substructure $\Power_f[\frakT_r]$ whose domain consists of the finite subsets of $\frakT_r$.
 \item The power structure of the ordering of the rationals, namely $\Power[(\mathbb{Q},<)]$.
 \item Presburger arithmetic $(\N,+)$.
 \item Skolem arithmetic $(\N,\times)$.
 \item The structure $(\Power(\N),\subseteq, =^*)$ where $X =^* Y$ means that $X$ and $Y$ have finite symmetric difference.
 \item Every ordinal $(\alpha,<)$ where $\alpha < \omega^{\omega^\omega}$.
\end{enumerate}
