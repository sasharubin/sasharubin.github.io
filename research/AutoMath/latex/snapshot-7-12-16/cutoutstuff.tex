








\iffalse




\subsubsection{Automata with infinite alphabet}

We saw in Proposition \ref{AS:prop:powerclosure} that the weak direct power of a structure in $\waut$ is in $\taut$.

We follow \cite{Bes08} 






\subsection{MSO interpretations}

Fix a structure $\frakB$ and a tuple of MSO-formulas $\I = (\delta, \Phi_1, \cdots, \Phi_N)$ in the signature of $\frakB$.
Suppose all the free variables are individual variables, $\delta$ has exactly one free variable, and $\Phi_i$ has $r_i$ free variables.
If $\Phi_i^\frakB$ are relations over $\delta^\frakB$ then the structure 
\[
\I(\frakB) := (\delta^\frakB, \Phi_1^\frakB, \cdots, \Phi_N^\frakB)
\] 
is {\em MSO-definable in $\frakB$}. The tuple $\I$ is called a {\em MSO-definition}.

For example $(0^\ast,\suc_0)$ is MSO-definable in $\two$ via the formulas
\[
\delta(x) := (\forall z)\, \left[z \pref x \wedge z \neq x \implies (\exists y \pref x)\,  \suc_0(z,y)\right]
\]
and $\Phi_1(x_1,x_2) := \suc_0(x_1,x_2)$. 


The following lemma says one can translate formulas about the definable structure $\I(\frakB)$ into formulas about the defining structure $\frakB$.

??? say something about fo vars being replaced by mso vars.
or just change all set vars in TL to fo vars...???
\begin{lemma}[Translation Lemma] 
Fix an MSO-definition $\I$ and an MSO-formula $\Phi(X_1,\cdots,X_k)$ in the signature of $\I(\frakB)$.
There is an MSO-formula $\Phi_\I(X_1,\cdots,X_k)$ in the signature of $\frakB$ such that for all subsets $B_i$ of $\delta^\frakB$,
\[
\I(\frakB) \models \Phi(B_1,\cdots,B_k) \iff \frakB \models \Phi_\I(B_1,\cdots,B_k).
\]
\end{lemma}

\begin{proof}
The idea is to relativise all quantifiers to $\delta$ and replace and the $i$th atomic formula by $\Phi_i$.
Formally, define $\Phi_\I$: $(\Psi \wedge \Xi)_\I$ is defined by $\Psi_\I
\wedge \Xi_\I$; $(\neg \Psi)_\I$ by $\neg \Psi_\I$; $(\exists X_i \Psi)_\I$ by
$\exists X_i [ (\forall x \in X_i \delta(x)) \wedge \Psi_\I]$, and similarly for individual quantifiers. 
\end{proof}


A structure $\frakA$ isomorphic to $\I(\frakB)$ is called {\em
MSO-interpretable in $\frakB$}. An isomorphism from $\frakB$ to $\frakA$ is called a {\em co-ordinate map} of the interpretation. 

For example $\one$ is MSO-interpretable in $\two$ with co-ordinate map sending the string $0^n$ to $n$.

Applying the translation lemma to sentences shows that we can transfer MSO-decidability to the interpreted structure.

\begin{proposition}
Suppose that $\frakA$ is MSO-interpretable in a structure with decidable MSO-theory. Then $\frakA$ has decidable MSO-theory.
\end{proposition}

\subsection{Set interpretations}

??? Instead of set interpretation, introduce powerset and FO interpretation ???

Fix a structure $\frakB$ and a tuple of MSO-formulas $\I = (\delta, \Phi_1, \cdots, \Phi_N)$ in the signature of $\frakB$.
Suppose all the free variables are set variables, $\delta$ has exactly one free variable, and $\Phi_i$ has $r_i$ free variables.
In case this defines a structure with domain 
\[
\{U \subseteq {A} \st \frakB \models \delta(U)\}
\]
and relations
\[
\{(U_1,\cdots,U_{r_i}) \st \frakB \models \Phi_i(U_1,\cdots,U_{r_i})\}
\]
call it  $\I(\frakB)$.
A structure $\frakA$ isomorphic to $\I(\frakB)$, say via map $f$, is called {\em
set-interpretable in $\frakB$}. The {\em interpretation} consists of the data $\left<\I,f\right>$.
%Abuse notation and write $\I$ for an isomorphism, called a {\em co-ordinate map}, from $\I(\frakB)$ to $\frakA$.
For example, in section \ref{AS:sec:motivation} we argue that $(\N,+)$ is set-interpretable in
$\one$ via the map sending the non-empty set $X$ to the number $\sum\{2^i \st i \in X\}$.

???change notation $\I(B_i)$???
\begin{lemma}[Translation for set-interpretation]
Fix a set-interpretation $\I$ and a FO-formula $\Phi(x_1,\cdots,x_k)$ in the signature of $\I(\frakB)$.
There is an MSO-formula $\Phi_\I(X_1,\cdots,X_k)$ in the signature of $\frakB$ such that for all sets $B_i$ satisfying $\delta(B_i)$,
\[
\I(\frakB) \models \Phi(\I(B_1),\cdots,\I(B_k))  \iff 
\frakB \models \Phi_\I(B_1,\cdots,B_k). 
\]
\
\end{lemma}

\begin{proof}
The proof is similar to the translation lemma for MSO-interpretations except that first replace all free variables $x_i$ by (unused) 
set variables $X_i$ and define $(\exists x_i \Psi)_\I$ instead as
$\exists X_i [ \delta(X_i) \wedge \Psi_\I]$.
\end{proof}

\begin{proposition}[Transfer decidability]
If $\frakA$ is set-interpretable in a structure with decidable MSO-theory, then $\frakA$ has decidable FO-theory.
\end{proposition}


Here is a straightforward proposition connecting MSO interpretations and set-interpretations.

\begin{lemma}
If $\frakA$ is MSO-interpretable in $\frakB$ and $\frakB$ is set-interpretable
in $\frakC$ then $\frakA$ is set-interpretable in $\frakC$.  
\end{lemma}

The discussion in this section makes sense if we replace every occurence of MSO
by WMSO and every occurence of set-interpretable by finite-set interpretable.

\subsection{Automata with advice}
Fix a predicate $P \subseteq \2$ (a similar definition holds for $P \subseteq \N$).
An {\em $\omega$-tree-automaton with advice $P$} is an $\omega$-tree-automaton $\M$ that, while in
position $w \in \2$ can decide on its next state using the additional
information of whether or not $w \in P$. Formally, the automaton accepts an
input $X \subseteq \2$ if it has a successful run on the characterstic tree $T_{(X,P)}$.

%The {\em emptiness problem} for a collection of automata is to decide, given an automaton $M$, whether or not
%it accepts any string whatsoever.

\fi
