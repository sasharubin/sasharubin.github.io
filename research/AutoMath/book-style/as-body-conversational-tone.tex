%=====================================================
\section{Introduction} \label{AS:sec:introduction}
%=====================================================

\subsubsection{Decidability.}
I imagine that high-school students dream of having a systematic procedure
answering all of their mathematical problems. A lot of high-school algebra can be
expressed in the first-order (FO) language of the structure
$(\R,+,\times,<,+1,=,0,1,-1)$. This simply means
that we can write formulas that allow quantification over elements of $\R$ and
make use of the predicates $+, \times, < ,+1, =$ and the constants $0,1$ and $-1$.
Tarski proved that there is an algorithm that decides the truth or falsity of
every such FO-sentence (a sentence is a formula with no free variables). We say
that the FO-theory of real arithmetic is decidable. Tarski's proof uses a
technique called effective quantifier elimination. The approach is familiar:
the statement that a quadratic has two real roots
\[
\exists x \exists y [x \neq y \wedge ax^2 + bx + c = 0 \wedge ay^2 + by + c =0]
\]
(written with abbreviations such as $x^2$ for $x \times x$) 
can be replaced by the simpler quantifier free condition
\[
a \neq 0 \wedge 0 < b^2 - 4ac \ .
\]

There are a number of other techniques for proving decidability. These include
providing a finite (or computably enumerable) axiomatisation of the theory; and  the composition method
of Feferman, Vaught and Shelah that uses a decomposition of a structure into pieces with simpler theories.\footnote{Theories also arise without having a specific structure in mind. Often they are a satisfiable computably-enumerable set of sentences closed under logical deduction. Decidability follows from establishing that the theory is complete (every sentence or its negation is in the set). Showing completeness is bread and butter for model theorists.}
A good technique will do more. It will
also give some insight into the nature of the sets and relations definable by
formulas of the language.  For instance, Tarski's procedure transforms a
formula into an equivalent formula with no quantifiers and without changing its
free variables.  This guarantees, for instance, that every FO-definable set in real arithmetic is
a finite union of intervals with algebraic endpoints. For a discussion of Tarski's
result see \cite{Drie88}. 

Of course not all theories are decidable. A good heuristic is that if a long
unsolved problem is expressible in the language of a certain structure then its theory
is probably undecidable. A case in point is the FO-theory of integer
arithmetic $(\N,+,\times,<,=,0,1)$.  For a treatment of decision problems see \cite{}.

\subsubsection{Enter automata.}
In the late 1950s, Church summarised the known decision methods for restricted
systems of arithmetic. He asked, in particular, whether the weak monadic second-order
(WMSO) logic of $\one$ is decidable.  The term (weak) monadic second-order
means that variables vary over (finite) subsets of the domain $\N$. After
Church's lectures B\"uchi and Elgot leveraged the natural correspondence
between finite subsets of $\N$ and certain finite strings over the alphabet $\{0,1\}$, namely
let $\chi(A)$ be the shortest binary string with a $1$ in position $n \in \N$ if and only if $n \in A$.
They proved a fundamental result,  independently established by Trahtenbrot, 
that identifies logical definability with recognisability by automata.

\begin{theorem}[\cite{Buch60, Elgo61, Trah62}] \label{AS:thm:BET}
A set of tuples $(A_1,\cdots,A_n)$ of finite sets of natural numbers is weak monadic
second-order definable in $\one$ if and only if the corresponding $n$-ary
relation of strings $(\chi(A_1),\cdots,\chi(A_n))$ is a synchronous rational relation. Moreover, the translation between formulas and automata is effective.
\end{theorem}

The proof of this theorem can be found in ???. As a simple illustration the reader can find 
a formula with one free set variable defining the set $E$
\[\{A \subset \N \st \mbox{the cardinality of $A$ is even}\}
\]
and an automaton accepting the corresponding set $\{\chi(A) \in \{0,1\}^\ast \st A \in E\}$.

\begin{corollary} \label{AS:cor:ws1s}
The WMSO-theory of $\one$ is decidable.
\end{corollary}

\begin{proof}
%To get this decidability we use one direction of the theorem. 
A formula $\phi$
with no free variables is (logically equivalent to) one of the form $\exists X
\psi(X)$ or $\neg \exists X \psi(X)$. Use the effective procedure guaranteed by
theorem \ref{AS:thm:BET} to build an automaton $M$ corresponding to $\psi$. One can
effectively decide whether an automaton accepts any string whatsoever. If $M$
does then $\exists X \psi(X)$ is true, otherwise it is false.
\end{proof}

B\"uchi and Elgot noticed that the {\it first-order theory} of $(\N,+)$ is
decidable\footnote{This theory is called \emph{Presburger arithmetic} after ...
Presburger who proved it decidable via effective QE?} as it is interpretable in
the WMSO-theory of $\one$. Lets explain what this means (formal definitions
appear later).
%There is a natural first-order variation on WMSO of $\one$. Namely, the structure 
%\[
%\power_f(\one) := (\power_f(\Nat),S',\subset)
%\]
%where $\power_f(\Nat)$ consists of the finite subsets of $\Nat$, 
%$S'$ is a binary relation holding on
%pairs $(\{n\},\{n+1\})$ for $n \in \N$, and $\subset$ is the subset relation. 
%It is easy to see that a WMSO formula $\Phi(X_1,\cdots,X_N)$ in $\one$ defines
%exactly the same set as the FO formula $\phi(x_1,\cdots,x_N)$ in $\power_f(\one)$
%where we get $\phi$ from $\Phi$ by replacing the set variables $X_i$ by individual variables $x_i$, 
%individual variables $x$ are constrained to be singeletons (a FO definable property),
%and relations of the form $x \in Y$ are replaced by $x \subset y$.
Interpretations appear everywhere
in mathematics --- a familiar example of an interpretation is the definition of
rational arithmetic in terms of integer arithmetic, with the rationals defined
as certain pairs of integers. Interpretations are studies systematically in
model theory, see \cite{}.

\begin{corollary} \label{AS:cor:PRES}
\begin{enumerate}
\item Every first-order definable relation of $(\N,+)$ is, modulo coding into binary strings, recognised by a finite-automaton.

\item The first-order theory of $(\N,+)$ is decidable.
\end{enumerate}
\end{corollary}

Here is the argument.
Code a natural number as a \emph{finite set} by first
representing it in binary (least significant digit first) and then taking the
characteristic set. Thus the number eleven is $1101$ in binary which corresponds
to the set $\{0,1,3\}$. Although this coding might seem unnatural, there is a
simple WMSO-formula $\phi_+(X_1,X_2,X_3)$ of the structure $\one$ expressing that the $X_i$ are non-empty (hence $X_i$ codes a natural, say $x_i$) 
and that $x_1 + x_2 = x_3$. The formula implements the usual bit-carry procedure for addition:
it guesses the existence of the carry set and uses the successor to scan the
sets one place at a time verifying the addition. 

In this way we find a copy of $(\N,+)$ inside $\one$. Formally we have a
{\em finite-set interpretation} of $(\N,+)$ in $\one$ (\ref{}). This allows us to
translate a FO-formula $\phi(x_1,\cdots,x_n)$ of $(\N,+)$ into a WMSO-formula
$\phi'(X_1,\cdots,X_n)$ of $\one$ (\ref{}) with the property that
the two formulas define the same relation, modulo the coding. Finally, theorem \ref{AS:thm:BET} ensures
the relation, viewed as a relation on strings, is recognised by a
finite-automaton. Also,  the decidability follows as in the proof of
corollary \ref{AS:cor:ws1s}.

This chapter concerns structures such as $(\N,+)$ whose theories are
decidable via techniques involving automata. The main technique,
set-interpretation in a structure such as $\one$ or more generally $\two$,
gives us the \emph{automatic structures}. 

Why focus on automatic structures?  Formulas can be treated as automata.  This
point of view is expressed in the {\em fundamental theorem of automatic
structures} which says that a FO-definable relation in an automatic structure
is, modulo coding, recognised by a finite-automaton. This chapter will
illustrate that many results about automatic structures exploit this theorem.

Why the shift from MSO to FO? By downgrading the expressive power of the logical
language we increase the complexity of the structures whose theories are
decidable (for instance, from successor on $\N$ to addition on $\N$).

Here is an outline of the types of problems discussed in this chapter.

\begin{enumerate}
\item Which structures have the property that FO-definability corresponds exactly with regularity? 
This is the problem of finding universal automatic structures.
\item To what extent can we increase the expressive power of the logical language (FO) and still retain decidability? 
This is the problem of extending the fundamental theorem of automatic structures.
\item Fix a collection of similar mathematical structures, say linear orders. Which linear orders are automatic? 
This is the problem of characterising classes of automatic structures.
\item An automatic structure may be interpreted in, say, $\one$, in a number of ways. How are these ways different?
This is the problem of studying non-equivalent automatic presentations.
\end{enumerate}

The story just told is one way to motivate and define the study of automatic
structures. See \cite{KhNe95,Rubi08} for related accounts. Finally, here is a
concrete problem: is the structure of real arithmetic automatic?

%A note about the presentation: proofs are sketched if given at all; the focus is on Rabin-automatic structures; ???

{\bf Acknowledgements.} I have drawn on the exposition in \cite{BlCoLo07}.

%=====================================================
\section{Background: logic, automata and interpretations}
%=====================================================

Mathematical logic looks at the nature of definability. For this we need a logical language and a structure in which to interpret
formulas written in the language.
 
\subsubsection{Logical languages}

A \emph{structure}  $\frakA = (\mathcal{U}, R_1, \cdots, R_N)$ consists of a set $\mathcal{U}$ called the {\em domain} and relations\footnote{We deal with relational structures. This is no real handicap since we can replace an operation, such as $+$, by its graph.} $R_i \subset \mathcal{U}^{r_i}$. The names of the $R_i$ together with their arities $r_i \in \N$ form the \emph{signature} of the structure.

\begin{example}
The structure  $\frakT_r$ has domain $\{0,1, \cdots,r\}^\star$ and for $i \in \{0,1, \cdots, r\}$ a binary relation $\suc_i$ consisting of pairs $(w,wi)$. 
Note that $\frakT_1$ is isomorphic to $\one$.
\end{example}


To talk about structures we need a logical language. Formulas of monadic
second-order logic (MSO) are constructed using logical connectives (`and',
`or', `not'), individual variables $x,y,z$ (that are intended to range over
elements of the domain), set variables $X,Y,Z,\cdots$ (that are intended to
range over subsets of the domain), quantification over these variables, and the
subset $X \subset Y$ and membership $x \in X$ relations.

Formulas of weak monadic second-order logic (WMSO) are defined as for MSO
except that set variables are intended to range over finite subsets of the
domain. Formulas of first-order logic (FO) are defined as for MSO but without 
set variables. A formula written $\phi(X_1,\cdots,X_n,x_1,\cdots,x_m)$ means that $\phi$'s
free variables are included in the set $\{X_1,\cdots,X_n,x_1,\cdots,x_m\}$.
A {\em sentence} is a formula without free variables.

Formulas may also use the names of the relations in a fixed signature. 
Here is a formula of two free individual variables $x,y$ in the signature of $\frakT_1$
\[
(\forall Z) [x \in Z \wedge (\forall z)(z \in Z \implies \suc(z) \in Z) \implies y \in Z].
\]
We can see that it is satisfied by those pairs of natural numbers $(x,y)$ such that $x \leq y$. 
One can similarly define the prefix relation $\pref$ in $\frakT_2$.

Note that we are appealing to our natural sense of what it means for a
sentence to be true of a structure. The discussion above has conflated syntax
and semantics. For a rigourous definition of truth and satisfaction in
mathematical logic see \cite{}. We will use the shorthand $\frakA \models \Phi$
(read $\frakA$ models $\Phi$) to mean that the sentence $\Phi$ is true in
$\frakA$.

An MSO-formula $\phi(x_1,\cdots,x_n)$, in the signature of $\frakA$, with
individual free variables defines the relation $\phi^\frakA := \{\tup{a} \in A^n\st
\frakA \models \phi(\tup{a})\}$. 


Let $\eL$ be one of MSO, WMSO, or FO. The {\em $\eL$-theory} of a structure is the set
of $\eL$-sentences true in that structure.  A set $X$ of sentences is {\em
decidable} if there is an algorithm that correctly decides, given a sentence
$\phi$ in the language of $\eL$, whether or not $\phi \in X$.  We say that a
structure has {\em decidable $\eL$-theory} if its $\eL$-theory is decidable.  A
central problem in mathematical logic has been establishing the decidability
(or not) of theories.

\subsubsection{Rabin's tree theorem}

The Church/\buchi/Elgot/Trahtenbrot revolution led to increasingly complex
structures whose decidability follows from methods of automata. A cornerstone
is Rabin's tree theorem which states that MSO definability in $\two$ coincides
with recognisability by $\omega$-tree automata. We state this theorem precisely
and introduce notation along the way.

Tree automata operate on {\em $A$-labeled trees} $T:\{0,1\}^\ast \to A$. 
For a definition of Rabin tree automaton see \cite{}.

To state Rabin's theorem we code tuples of sets as labelled trees.

\begin{definition}[characteristic tree]
For sets $X_0, \cdots,  X_{n-1} \subset \2$ define their {\em characteristic
tree} $T_{\tup{X}}$ as $\{0,1\}^n$-labeled binary tree such that the $i$th
component of $T_{\tup{X}}(w)$ is $1$ if and only if $w \in X_i$.
\end{definition}

\begin{theorem}[Rabin's tree theorem]
For each MSO-formula $\phi(\tup{X})$ in the signature of $\frakT_2$ there is a tree-automaton (and vice-versa) such that the language recognised by the automaton is
\[
\{T_{\tup{X}} \st \frakT_2 \models \phi(\tup{X})\}.
\]
\end{theorem}

Elgot and Rabin showed that certain extensions of $\one$ and $\two$ by unary
predicates have decidable MSO.  For instance they showed that $(\N,+1,F)$
with $F := \{n! \st n \in \N\}$ has decidable MSO-theory. ???WMSO???
 
The {\em $P$-membership problem} is to decide, given an automaton $M$,
whether or not $M$ accepts $P$.

\begin{lemma}
For every predicate $P \subset \2$, the structure $(\frakT_2,P)$ has decidable
MSO-theory if and only if the $P$-membership problem is decidable.
\end{lemma}

\begin{proof}
Let $\Phi$ be a sentence of $(\frakT_2,P)$. Let $X$ be a
variable not used in $\Phi$.  Build a formula $\Psi(X)$ from $\Phi$ in which
every occurence of $P$ has been replaced by the variable $X$.  By construction
$(\frakT_2,P) \models \Phi$ if and only if $\frakT_2 \models \Psi(P)$.  The
latter condition is equivalent to the problem of whether the automaton
corresponding to $\Psi$ accepts $P$ or not.
\end{proof}

See \cite{} for a discussion of other characterisations of $(\frakT_2,P)$ having decidable MSO.

One may be wondering why we restrict to expansions of $\frakT_2$ only by unary predicates.
The reason is that expansions by non-trivial binary relations result in undecidability. 

\begin{theorem} \cite{ElRa66}
Let $f:\N \to \N$ be a function such that $x < y$ implies $1 + f(x) < f(y)$.
The extension of $\frakT_1$ by the relation $G_f:= \{(n,g(n)) \st n \in \N\}$ has undecidable $\wmso$-theory.
\end{theorem}

For example $(\N,+1,G_{n \mapsto 2n})$ has undecidable $\wmso$-theory.

\subsubsection{From MSO to FO}

%The exposition in this section borrows from \cite{BlCoLo07}. 
Let $\Power(X)$ denote the set of subsets of $X$. The {\em power set} of a structure $\frakA = (\U,R_1,\cdots,R_N)$ is the structure
\[
\Power[\frakA] := (\Power(\U),{R'_1}, \cdots,{R'_N}, \subset)
\]
where 
\[
 R'_i := \{ (\{x_1\},\cdots,\{x_{r_i}\}) \st (x_1,\cdots,x_{r_i}) \in R_i\}. 
\]

For example, $\Power[\one]$ is the structure with domain $\Power(\N)$, the subset relation $\subset$, and the binary relation $\{(\{n\},\{n+1\}) \st n \in \N\}$.

\begin{proposition}[Translation]
For every FO-formula $\phi(\tup{x})$ there is an MSO-formula $\Psi(\tup{X})$ (and {\it vice versa}) such that for all structures $\frakA$ and all $U_i \in \Power(\U)$
\[
\Power[\frakA] \models \phi(\tup{U}) \ \mbox{ if and only if } \  \frakA \models \Psi(\tup{U})
\]
\end{proposition}

\begin{corollary}
The FO-theory of $\Power[\frakT_r]$ is decidable.
\end{corollary}

This whole section is valid if we replace MSO by WMSO, replace $\Power$ by $\Power_f$ (called the {\em finite-power set}), and restrict all subsets to be finite.

\subsubsection{FO Interpretations.}

Let  $\I = (\delta, \Phi_1, \cdots, \Phi_N)$ be FO-formulas in the signature of a structure $\frakB$.
Suppose that $\delta$ has $1$ free variable and $\Phi_i$ has $r_i$ free variables.
If $\Phi_i^\frakB$ are relations over $\delta^\frakB$ then the structure 
\[
\I(\frakB) := (\delta^\frakB, \Phi_1^\frakB, \cdots, \Phi_N^\frakB)
\] 
is {\em FO-definable in $\frakB$}.\footnote{We have overloaded the phrase `FO-definable': here one structure is definable in another; 
and the other meaning is that a relation is definable in a structure.} The tuple $\I$ is called an {\em FO-definition}.

For example ???

\begin{proposition}[Interpretations compose] \label{AS:prop:compose}
If $\frakA$ is FO-interpretable in $\frakB$, and $\frakB$ is FO-interpretable in $\frakC$, then $\frakA$ is FO-interpretable in $\frakC$.
\end{proposition}

The following lemma says that every relation FO-definable in $\I(\frakB)$ is FO-definable in $\frakB$.
%one can translate formulas about the definable structure $\I(\frakB)$ into formulas about the defining structure $\frakB$.

\begin{lemma}[Translation Lemma] 
Fix a FO-definition $\I$ and a FO-formula $\Phi(x_1,\cdots,x_k)$ in the signature of $\I(\frakB)$.
There is a FO-formula $\Phi_\I(x_1,\cdots,x_k)$ in the signature of $\frakB$ such that for all elements $b_i$ of $\delta^\frakB$,
\[
\I(\frakB) \models \Phi(b_1,\cdots,b_k) \mbox{ if and only if } \frakB \models \Phi_\I(b_1,\cdots,b_k).
\]
\end{lemma}

\begin{proof}
The idea is to relativise all quantifiers to $\delta$ and replace and the $i$th atomic formula by $\Phi_i$.
Formally, define $\Phi_\I$: $(\Psi \wedge \Xi)_\I$ is defined by $\Psi_\I
\wedge \Xi_\I$; $(\neg \Psi)_\I$ by $\neg \Psi_\I$; $(\exists x_i \Psi)_\I$ by
$\exists x_i \delta(x_i) \wedge \Psi_\I]$.
\end{proof}


A structure $\frakA$ isomorphic to $\I(\frakB)$ is called {\em FO-interpretable in $\frakB$}.
An isomorphism from $\frakB$ to $\frakA$ is called a {\em co-ordinate map} of the interpretation. 

For example ???


\begin{proposition}
Suppose that $\frakA$ is FO-interpretable in a structure with decidable FO-theory. Then $\frakA$ has decidable FO-theory.
\end{proposition}

\begin{proof}
For a sentence $\phi$ of $\frakA$, the translation lemma produces a sentence $\phi^\I$ preserving truth. Apply the decision procedure for the interpreting structure to $\phi^\I$. \end{proof}

There are other types of interpretations that could be introduced here, namely
(W)MSO-interpretations and (finite)-set interpretations \cite{}.  For the sake
of filling in undefined notions in the introduction we define a {\em (finite)-set
interpretation} of $\frakA$ in $\frakB$ to be like a FO-interpretation except
that the formulas are (W)MSO formulas, and the free variables are finite set
variables. Thus elements of $\frakA$ are coded as finite subsets of the domain
of $\frakB$. An immediate consequence is that $\frakA$ is (finite)-set
interpretable in $\frakB$ if and only if $\frakA$ is FO-interpretable in
$\Power_{(f)}[\frakB]$.

\iffalse
??? ALT DFN ???

\begin{definition}
\begin{enumerate}
\item A structure FO-interpretable in $\Power_f(\frakT_1)$ is called {\em finite-string automatic}.
\item A structure FO-interpretable in $\Power(\frakT_1)$ is called {\em $\omega$-string automatic}.
\item A structure FO-interpretable in $\Power_f(\frakT_2)$ is called {\em finite-tree automatic}.
\item A structure FO-interpretable in $\Power(\frakT_2)$ is called {\em $\omega$-tree automatic}.
\end{enumerate}
\end{definition}
\fi


%=====================================================
\section{Definition of automatic structures}
%=====================================================

\begin{definition}
A structure FO-interpretable in $\Power[\frakT_2]$ is called {\em $\omega$-tree automatic} (also called {\em Rabin-automatic}). The class of such structures is written $\raut$.
\end{definition}

\begin{corollary}
The FO-theory of every Rabin-automatic structure is decidable.
\end{corollary}

Members of the domain of Rabin-automatic structures are naturally viewed as $\omega$-trees. 
%Indeed an element $X$ of $\Power[\frakT_2]$
%is a subset of $\2$. A FO-definition in $\Power[\frakT_2]$ produces a structure $\frakB$ whose domain is a collection of such sets.
%But each set $X$ may be viewed as a tree $T_X$ and so the relations of $\frakB$ may be viewed as relations on trees. Now pass to an isomorphic copy $\frakA$ via $f$.
%Lets formalise this.

\begin{definition}
For a Rabin-automatic structure $\frakA$ (isomorphic to some $\I(\Power[\frakT_2])$ via $f$) and relation $R$ in $\frakA$ denote by
$\code{R}$ the set of $\omega$-trees $\{T_{f(a_1),\cdots,f(a_k)} \st  \tup{a} \in R\}$.
\end{definition}

FO-definable relations in Rabin-automatic structures are, modulo coding, regular. 
This may be called the {\em fundamental theorem of automatic structures}.

\begin{theorem}[Fundamental theorem]
Let $\frakA$ be Rabin-automatic.
\begin{enumerate}
\item For every first-order definable relation $R$ in $\frakA$ the set of trees $\code{R}$ is recognised by an automaton.
\item The first-order theory of an automatic structure is decidable.
\end{enumerate}
\end{theorem}

\begin{proof}
 For the first item combine Rabin's theorem and the translation lemma.
 For the second use the fact that it is decidable whether or not a Rabin automaton accepts any tree whatsoever.
\end{proof}

\begin{definition} \label{AS:dfn:ap}
Suppose that $f: \frakA \simeq  (B,S_1,\cdots,S_N)$ and
\begin{enumerate}
\item elements of $B$ are $\{0,1\}$-labeled binary trees,
\item each of the sets $B$, $S_1, \cdots, S_N$ is recognised by an $\omega$-tree automaton, say the automata are $M_B,M_1, \cdots, M_N$.
\end{enumerate}
Then the data $\left<(M_B,M_1,\cdots,M_N), f \right>$ is called an {\em $\omega$-tree automatic presentation} of $\frakA$.
\end{definition}

It is convenient to use the following characterisation.\footnote{Indeed,
definition \ref{AS:dfn:ap} is usually taken as the definition of
Rabin-automatic structure though we will not do so here.} 


\begin{theorem}[Machine theoretic characterisation]
A structure is Rabin-automatic (i.e. FO-interpretable in $\Power[\frakT_2]$) if and only if it has an $\omega$-tree automatic presentation.
\end{theorem}

\begin{proof}
details rqd
%For the forward direction define $\frakB = ([\U], [R_1], \cdots, [R_N])$ and apply the fundamental theorem.
%For the reverse direction use Rabin's tree theorem to convert the regular set $B$ to a formula $\delta(s)$ and regular sets $S_i$ to formulas $\Phi(s_1,\cdots,s_{r_i})$, giving
%a FO-interpretation. ???details???
\end{proof}

\subsubsection{Other classes of automatic structures}

\begin{definition}
A structure is called {\em finite-string automatic} if it is FO-interpretable in $\Power_f[\frakT_1]$. This collection of structures is written $\waut$.
\end{definition}

We have seen in the first section that $(\N,+)$ is {\em finite-string automatic}. 
Note that such structures have countable domain.

\begin{definition}
A structure is called {\em $\omega$-string automatic} (also called {\em \buchi-automatic}) if it is FO-interpretable in $\Power[\frakT_1]$. This collection is written $\baut$.
\end{definition}

Since `$x$ is finite' is a FO-definable property in $\frakT_1$, examples of $\omega$-string automatic structures 
include all finite-string automatic structures.
The structure $([0,1],+,<)$, where addition is taken modulo one, is an uncountable example.

\begin{definition}
A structure is called {\em finite-tree automatic} if it is FO-interpretable 
in $\Power_f[T_2]$.  This collection is written $\taut$.
\end{definition}

Since $\Power_f[\frakT_1]$ is FO-interpretable in $\Power_f[\frakT_2]$, every
finite-string automatic structure is also finite-tree automatic.  Also
$(\N,\times)$ is finite-tree automatic (decompose $n$ into prime powers,
$n=\prod_{i} p_i^{e_i}$, and code it as a tree with $e_i$ written in binary on
the $i$th branch). It is not finite-string automatic \cite{}.

These are subclasses of the Rabin-automatic structures. Each has a machine theoretic charactersation as in Definition~\ref{AS:dfn:ap}: simply replace Rabin-automata by
finite-string automata, $\omega$-string automata, or finite-tree automata.
A member of any of the four classes is said to be {\em automatic}.

???picture???

\begin{lemma} \label{AS:lem:relations}
The structure $\Power_f(\frakT_1)$ is FO-interpretable in $\Power(\frakT_1)$ and $\Power_f(\frakT_2)$. These latter two structures 
are each FO-interpretable in $\Power(\frakT_2)$.
\end{lemma}

A structure FO-interpretable in $\Power[(\2,\suc_0,\suc_1,P)]$ for a predicate $P \subset \2$ is called
{\em Rabin-automatic with advice $P$}. Similar definitions hold for the other subclasses.
Of course if $(\frakT_2,P)$ has decidable MSO-theory then every Rabin-automatic structure with advice $P$
has decidable FO-theory.

{\bf Notation.} Let $\av$ stand for `finite-string', `$\omega$-string', `finite-tree', or `$\omega$-tree'.
Thus we write $\av$-automatic.

%??? in dfn have f be part of interpretation $\I$ ???
%
%\begin{definition}
%For an automatic structure $\frakA$ (isomorphic to some $\I(\T_2)$ via $f$) and relation $R$ in $\frakA$ denote by
%$[R]$ the set of $\omega$-trees $\{T_{f(a_1),\cdots,f(a_k)} \st  \tup{a} \in R\}$.
%\end{definition}

%The following theorem says that FO-definable relations in automatic structures are, modulo coding into strings or trees, regular.

%\begin{theorem}[Fundamental theorem of automatic structures]
%Let $\frakA$ be set-interpretable in $\T_2$.
%\begin{enumerate}
%\item For every first-order definable relation $R$ in $\frakA$ the set of trees $[R]$ is recognised by an automaton.
%\item The first-order theory of an automatic structure is decidable.
%\end{enumerate}
%\end{theorem}

%\begin{proof}
%Combine the translation lemma for set-interpretations and Rabin's theorem.
%\end{proof}

%look again at the finite-set interpretation of $(\N,+)$ in $\one$. An element $n$ of
%the interpreted structure can be viewed as a finite binary string $(n)_2$.
%What happens to definable sets and relations on $\N$ under $\code:\N \to \2$?
%The fundamental theorem says that image under $f$ of an FO-definable relation $R$ of $(\N,+)$ is regular.
%Write $\code(R)$ for $\{f(a_1,\cdots,a_k) \st (a_1,\cdots,a_k) \in R\}$.
%In particular, $f(\N)$ is a regular set and $f(+)$ is a regular relation.
\subsection{Algebraic operations on automatic structures}

The $\av$-automatic structures are closed under FO-interpretations (proposition \ref{AS:prop:compose}).
In particular, if $\frakA$ is $\av$-automatic, and $R$ is FO-definable in $\frakA$ then $(\frakA,R)$ is $\av$-automatic.

There is a more general notion of interpretation called a $d$-dimensional FO-interpretation.
Here $\delta$ has $d$ free variables and each $\Phi_i$ has $d\times r_i$ free variables.

\begin{proposition}
$\av$-automatic structures are closed under $d$-dimensional FO-interpretations.
\end{proposition}

\begin{proof}
Details rqd.
% We illustrate the idea for Rabin-automatic structures.
% Say $\frakA$ is $d$-dimensionally interpretable in $\frakT_2$. Then $\code{a}$, for an element $a$, is
% a $\{0,1\}^d$-labeled tree. Recode $a$ by replacing the node $u$
% by a left-most path whose $i$th label is the $i$th coordinate of $\code{a}(u)$. For each successor $u0$ and $u1$ do the same starting at the end of the path just constructed for $u$.
% This transformation sends regular relations of $\{0,1\}^d$-labeled trees to regular relations of $\{0,1\}$-labeled trees.
\end{proof}

Say $\frakA$ and $\frakB$ are FO-interpretable in $\frakU$. Then the disjoint union of $\frakA$ and $\frakB$ is $2$-dimensionally interpretable in $\frakU$. Similarly for their direct product.

\begin{corollary}
$\av$-automatic structures are closed under disjoint union and direct product.
\end{corollary}

The {\em (weak) direct power} of $\frakA$ is a structure with the same signature as $\frakA$, its domain consists of (finite) sequences of $A$, and the 
interpretation of a relation symbol $R$ is the set of sequences $\sigma$ such that $R^\frakA(\sigma(n))$ holds for all $n \in \N$.

For example the weak direct power of $(\N,+)$ is isomorphic to $(\N,\times)$; the isomorphism sends $n$ to  the finite sequence $(e_i)$ where
$ \prod p_i^{e_i}$ is the prime power decomposition of $n$ ($p_i$ is the $i$th prime).
 
\begin{proposition}
The Rabin-automatic structures are closed under (weak)-direct power. The finite-tree automatic structures are closed under weak-direct power.
\end{proposition}

\begin{proof}
We illustrate the idea for Rabin-automatic structures.
Let $\sigma = (a_i)$ be an element of the direct power of $\frakA$.
Code the sequence $\sigma$ by the tree $t(\sigma)$ whose subtree at $0^n1$ is the tree $\code(a_n)$.
Let $R$ be a relation symbol. The interpretation of $R$ in the direct power is recognised by a tree automaton: it  processes $t(\sigma)$ by checking
that each subtree rooted at $0^n1$ is recognised by the automaton for $R^\frakA$.
\end{proof}

Quotienting is a familiar operation in mathematics.
Let $\frakA = (\U,R_1,\cdots,R_N)$ be a structure.
An equivalence relation $\sim$ on the domain $\U$ is called a {\em congruence for $\frakA$} if each relation $R_i$ satisfies the following property:
for every pair of $r_i$-tuples $\tup{x},\tup{y}$ of elements of $\U$, if $x_j \sim y_j$ for $1 \leq j \leq r_i$ then $R_i(\tup{x})$ if and only if $R_i(\tup{y})$.

The {\em quotient of $\frakA$ by $\sim$}, written $\frakA/_\sim$ is the structure whose domain is the set of equivalence classes of $\sim$ and whose $i$th relation
is the image of $R_i$ by the map sending $u \in \U$ to the equivalence class of $u$.

\begin{quote}
If $(\frakA,\sim)$ is $\av$-automatic, is $\frakA/_\sim$ $\av$-automatic?
\end{quote}

\subsubsection*{$\waut$:}
There is a regular well-ordering of the set of finite strings, for instance the length-lexicographic
ordering $\llex$. Use this order to define a regular set $D$ of unique $\sim$-representatives.
Then restrict the presentation of $\frakA$ to $D$.

\subsubsection*{$\taut$:}
Except in the finite word case, there is no regular well ordering of the set of
all finite trees \cite{CL07csl}. 
However one can still convert a finite-tree automatic presentation of $(\frakA,\sim)$ into
one for $\frakA/_\sim$ \cite{CL07LMCS}. The idea is to associate with each tree $t$ a
new tree $\hat{t}$ of the following form: the domain is the intersection 
of the prefix-closures of the domains of all trees that are $\eL(\A_\sim)$-equivalent to $t$; 
a node is labelled $\sigma$ if $t$ had label $\sigma$ in that position; 
a leaf $x$ is additionally labelled by those states $q$ from which the 
automaton $\A_{\sim}$ accepts the pair consisting of the subtree of $t$ 
rooted at $x$ and the tree with empty domain.
Using transitivity and symmetry of $\eL(\A_\approx)$, if $\hat{t} = \hat{s}$ 
then $t$ is $\eL(\A_\approx)$-equivalent to $s$. 
Moreover each equivalence class is associated with finitely many new trees, 
and so a representative may be chosen using any fixed regular linear ordering 
of the set of all finite trees.\footnote{The construction 
given in \cite{CL07LMCS} is slightly more general and allows one to effectively 
factor finite-subset interpretations in any tree.}


%In fact, Arnaud says he has a proof that every finite-tree regular equivalence
%relation has a regular set of representatives

\subsubsection*{$\baut$:}
There is a structure in $\baut$ whose quotient is not $\omega$-word 
automatic \cite{HKMN08}. The proof actually shows that the structure has no Borel presentation.
 
However, if $\frakA$ has countable domain then the quotient is in $\baut$ (and consequently is also in
$\waut$). This follows from the more general result that every $\omega$-word
regular equivalence relation with countable index has a regular set of
representatives \cite{KRB08}.

\subsubsection*{$\raut$:}
It is not yet known whether $\raut$ is closed under quotients, even for countable structures. ~\\

Nonetheless, quotients still have decidable theory.

\begin{proposition}
If $(\frakA,\sim)$ is $\av$-automatic then the quotient $\frakA/_\sim$ has decidable FO-theory.
\end{proposition}

\begin{proof} 
details rqd.
\end{proof}

%=====================================================
\section{Extensions of FO}
%=====================================================

%The exposition in this section borrows from \cite{}.
We show that we can enrich MSO by certain additional set quantifiers (such as `there are finitely many sets $X$ such that')
and still get decidability  for $\frakT_2$. 
We do this by showing that the newly quantified formulas are actually equivalent to vanilla MSO formulas. We can then push this to 
automatic structures.

%finite-string/tree: exists infinitely many, exists modulo

%infinite-string: as well as exists countably many

%infinite-tree:
%For a cardinal $\kappa$ write $\msokappa$ for the extension of MSO with the
%quantifier `There are at least $\kappa$ many sets $X$'.
For a cardinal $\kappa$ let $\exists^{\geq \kappa}$ denote the quantifier `there exists at least $\kappa$ many sets $X$ such that'.
Write $\msokappa$ for $\mso$ enriched by the quantifier $\exists^{\geq \kappa}$.


\begin{lemma}
The property `$X$ is finite' is $\mso$-definable in $\frakT_2$.
\end{lemma}

\begin{proof}
As a consequence
of K\H{o}nig's tree lemma $X$ is finite if and only if every path of $\frakT_2$
contains only finitely many elements from $X$.  Being a path is definable in
$\mso$ using the prefix relation.  The $\mso$ formula
\[
\exists p \in P \, \forall x \in X\, (x \in P \implies x \prefeq p).
\]
expresses, for a path $P$, that it contains only finitely many elements from $X$.
\end{proof}

Now, $\msocount$-definable relations in $\frakT_2$ are already $\mso$-definable.

\begin{proposition}
For every $\msocount$ formula $\phi(\tup{X})$ there is an $\mso$ formula 
$\phi'(\tup{X})$ equivalent to $\phi(\tup{X})$ over $\frakT_2$.
\end{proposition}

\begin{proof}
The following are equivalent:
\begin{enumerate}
\item There are only finitely many $X$ satisfying $\phi(X,\tup{Y})$. 
\item There is a set $Z$ such that every pair of different sets $X_1,X_2$ which both satisfy $\phi(X_i,\tup{Y})$ differ on $Z$.
\end{enumerate}
The second condition can be expressed in $\mso$.
\end{proof}

The proof of the following theorem uses the composition method and basic ideas from descriptive set theory.
\begin{theorem} \cite{BKRa}
For every $\msounc$ formula $\phi(\tup{X})$ there is an $\mso$ formula
$\phi'(\tup{X})$ equivalent to $\phi(\tup{X})$ over $\frakT_2$.
\end{theorem}

The proof of the following theorem ???
\begin{theorem} \cite{}
For every $\msomod$ formula $\phi(\tup{X})$ there is an $\mso$ formula
$\phi'(\tup{X})$ equivalent to $\phi(\tup{X})$ over $\frakT_2$.
\end{theorem}

We overload notation and let $\exists^\kappa$ denote the quantifier `there exists at least $\kappa$ many individual elements $x$ such that'.
Similarly for other quantifiers. 

%Denote by $\FOext$ the enrichment of the FO-language with the quantifiers $\existsmod$, $\exists^\aleph_0$ and $\exists^\aleph_1$.

\begin{theorem}[Extension of fundamental theorem] \label{AS:thm:FOext}
Let $\frakA$ be $\av$-automatic.
\begin{enumerate}
\item For every $\FOext$-definable relation $R$ in $\frakA$ the set $\code{R}$ is recognised by an automaton.
\item The $\FOext$ theory of an $\av$-automatic structure is decidable.
\end{enumerate}
\end{theorem}

\begin{proof}
This is immediate for Rabin-automatic structures. For the other classes use lemma~\ref{AS:lem:relations} and relativise
the quantifiers.
\end{proof}
 
How far can we push this? First we need a rigourous definition of quantifier.
This is neatly provided by Lindstr\"om's definition of `generalised quantifier',
see \cite{}. However for our purpose it is enough to define the
{\em cardinality quantifier parameterised by $C$}, for $C$  a set of
cardinals, as `there exists exactly $\alpha$ many elements $x$ such that \ldots,
where $\alpha \in C$'. Examples include $\existsmod,\exists^{\geq \kappa}$ as well as
`there exist a prime number of elements such that'.

It turns out that the only cardinality quantifiers we can add to FO and still get the fundamental theorem are, essentially, 
the ones mentioned in \cite{AS:thm:FOext}. We illustrate the idea:

\begin{theorem}
Let $Q_C$ be a cardinality quantifier parameterised by $C$.  Suppose for every
Rabin-automatic $\frakA$ and every $\FO(Q)$-definable relation $R$ in $\frakA$,
the set $\code{R}$ is recognised by a Rabin-automaton.  Then $R$ is
$\FOext$-definable in $\frakA$. 
\end{theorem}

\begin{proof}
We illustrate the proof for $C \subset \N$. Let $\frakA$ be the Rabin-automatic structure $\Power_f[\T_1]$.
The set $D: = \{[n] \st n \in C\}$ is $\FO(Q_C)$-definable in $\frakA$. 
By assumption $\code{D}$ is recognised by a tree-automaton. The trees in $\code{D}$ are unary strings,
so we may view $\code{D}$ as a regular subset of $\{1\}^\ast$. But every regular unary language is ultimately periodic.
So $D$ is already $\FO(\existsmod)$-definable in $\frakA$.
\end{proof}

What about extensions of FO by set quantification?  Unfortunately $\wmso$ is
too much to hope for.  Since the configuration graph $\frakG$ of a Turing machine (with the
single-transition edge relation) is finite-string automatic, and reachability
is expressible in $\wmso$, the halting problem reduces to the $\wmso$-theory of $\frakG$.

??? Connections to Pushdown hierarchy ???
%=====================================================
\section{Structural properties of automatic structures}
%=====================================================

Not much is known about Rabin-automatic structures. For instance, there are no non-trivial methods for proving a structure
is not Rabin-automatic (besides showing it has undecidable theory, or that it interprets a structure already known not to be
Rabin-automatic). So instead we focus on the other classes.

Write $h(x)$ for the height of a tree $x$, and $h(\tup{x})$ for the height of the largest tree in $\tup{x}$.

\begin{proposition} \label{AS:prop:locfin}
Suppose that function $F:A^n \to A$ is finite-tree regular, and let $p$ be the number of states of the automaton.
Then
$$h(F(\tup{x}))
\leq h(\tup{x}) + p
$$
for all $\tup{x}$ in the domain of $F$.
\end{proposition}

\begin{proof}
Otherwise, take a counterexample  $\tup{x}$.
After all of $\tup{x}$ has been read, and while still reading $F(\tup{x})$, some path in the run must have a repeated state. 
So the automaton also accepts infinitely many tuples of the form $(\tup{x}, \cdot)$.
\end{proof}

\subsubsection*{Growth of generation}

\begin{definition} \label{dfn:growth}
Let $\frakA$ be a structure with functions $f_1, \cdots, f_k$ of arities $r_1, \cdots, r_k$ respectively. 
Let $D \subseteq A$ be a finite set.
Define the
{\em $n$th growth level}, written $G_n(D)$, inductively by $G_0(D) = D$
and $G_{n+1}(D)$ is the union of $G_n(D)$ and
\[
\bigcup_{i\leq k} \{f_i(x_1,\cdots,x_{r_i}) \st x_j \in G_n(D) \text{ for } 1 \leq j \leq r_i\}.
\]
\end{definition}
%??? fix spacing in eqnarray

How fast does $|G_n(D)|$ grows as a function of $n$ ?  For
example, consider the free semigroup $(\Sigma^{\star},\cdot)$ with generating
set $D = \{d_1, \cdots, d_m\}$. For $m \geq 2$ the set $G_n(D)$ includes all strings over $D$ of length
at most $2^n$; thus the cardinality of $G_n(D)$ is at least $2^{2^{n}}$.

\begin{proposition} %{\rm \cite{BlGr00}, cf. \cite{KhNe95}}
 \label{prop:growth}
Let $\frakA \in \taut$ and $D \subset A$ be a finite set. Then there is a
linear function $t:\N \rightarrow \N$ so that for all $e \in G_n(D)$ the tree $\code{e}$ has
height at most $t(n)$.
\end{proposition}
\begin{proof}
 Iterate proposition~\ref{AS:prop:locfin}.
\end{proof}

\begin{corollary}
If $\frakA \in \taut$ then $|G_n(D)| \leq 2^{2^{O(n)}}$. If $\frakA \in \waut$ then 
$|G_n(D)| \leq 2^{O(n)}$.
\end{corollary}
\begin{proof}
Count the number of $\{0,1\}$-labelled trees (strings) of height (length) at most $k$.
\end{proof}

Thus the free semigroup on more than two generators is not in $\waut$. It is unknown whether it is in $\taut$. ???

\subsubsection*{Growth of definable sets}

\begin{definition}
Let $\phi(x,y)$ be a FO-formula in the signature of $\frakA$ and $E \subset \U$ finite. 
Define the {\em shadow of $\phi$ on $E$} as the set $\{\phi(u,\cdot) \cap E \st u \in \U\}$.
\end{definition}

For example let $\frakA$ be the random graph and $\phi(x,y)$ the formula that there is an edge between $x$ and $y$.
Then for every finite subset $E$ the shadow of $\phi$ on $E$ consists of all subsets of $E$.

\begin{proposition}
Suppose $\frakA \in \waut$ and $\phi(x,y)$ is a FO-formula in the signature of $\frakA$.
Then there is a constant $k$, 
that depends on the automata for $\U$ and $\phi$, and infinitely many finite subsets $E \subset \U$ such
that the cardinality of the shadow of $\phi$ on $E$ is at most $k|E|$.
\end{proposition}

\begin{proof}
Let $E_n$ be the set of strings in $\U$ of length at most $n$. For every $x \in \U$ there is a 
$y \in E_{n+c}$ such that $\phi(x,\cdot) \cap E_n = \phi(y,\cdot) \cap E_n$. Indeed $c$ is chosen so that
if $|x| \geq n+c$ then we can pump $x$ down, repeatedly, to get the string $y$.  Now use the fact that
$|E_{n+1}| \leq c|E_n|$ ???.
\end{proof}

Thus the random graph is not in $\waut$.

\begin{proposition}
Suppose $\frakA \in \taut$ and $\phi(x,y)$ is a FO-formula in the signature of $\frakA$.
Then there is a constant $k$, 
that depends on the automatic presentation of $\frakA$, and infinitely many finite subsets $E \subset \U$ such
that the log of the cardinality of the shadow of $\phi$ on $E$ is at most $k\log|E|$.??? 
\end{proposition}

Thus the random graph is not in $\taut$.

\subsubsection*{Sum-augmentation}

\begin{definition}
Say that a structure $\frakB$ is a {\em sum-augmentation} of a set of structures
$\eS$ (each having the same signature as $\frakB$) if there is a finite partition of $B
= B_1 \cup \cdots \cup B_n$ such that for each $i$ the substructure $\frakB
\restriction B_i$ is isomorphic to some structure in $\eS$.
\end{definition}

\begin{theorem} \label{thm:sumaug}
If $\A$ is an automatically presentable
then for every $\A$-formula $\phi(x,\tup{y})$, there is a finite set of
structures $\eS$ so that for every tuple of elements $\tup{b}$ from $A$, the
substructure $\A \restriction \phi^{\A}(\cdot,\tup{b})$ is a sum-augmentation
of $\eS$.
\end{theorem}

\begin{proof}
Say $\A = (A;R^{\A}_1, \cdots, R^{\A}_r)$, $A \subseteq \Sigma^{\star}$, is an
automatic copy of a structure.  For any given $\A$-formula $\psi$, fix a
deterministic automaton
$(Q_\psi,\iota_\psi,\Delta_\psi,F_\psi)$ recognising $\psi^{\A}$, and write
$\Gamma_\psi(w)$ for $\Delta_\psi(\iota_\psi,w)$.
We will use the following property $(P_{\psi})$:
For all strings $c_i,d_i$ with the $c_i$s all the same length,
$$
\Delta_\psi(\Gamma_\psi(\con(c_1,\cdots,c_k)),\con(d_1,\cdots,d_k)) \in F_\psi.
$$
if and only if $\psi(c_1d_1,\cdots,c_kd_k)$ holds in $\A$.

Now, given an $\A$-formula $\phi(x,y_1,\cdots,y_l)$ as in the hypothesis,
and tuple $\tup{b}$,
observe that for $m = \max\{|b_i|\}$, we can partition $\phi^{\A}(\cdot,\tup{b})$
into the finitely many singletons $\{c\}$ for $\phi(c,\tup{b})$ with $|c| <
m$, and the finitely many sets $\phi^{a \Sigma^{\star}}(\cdot,\tup{b}) := \{aw \in A \st \A \models \phi(aw,\tup{b}), w \in \Sigma^{\star}\}$ for
$|a| = m$. Since the signature is assumed to be finite, there are finitely
many isomorphism types amongst substructures of the form $\A \restriction
\{a\}$, for $a \in A$. So, it is sufficient to show, that as we vary
$(a,\tup{b})$ subject to $|a| = \max\{|b_i|\}$, there are finitely many isomorphism
types amongst substructures of the form $\A \restriction \phi^{a \Sigma^{\star}}(\cdot,\tup{b})$.

The idea is to bound this number of isomorphism types in terms of the number
of states of the automata involved. To this end, define a function $f_\phi$ as
follows. Its domain consists of tuples $(a,\tup{b})$ where $|a| =
\max\{|b_i|\}$; and $f_\phi$ sends this tuple to the tuple of states
$$
(\Gamma_\phi(\con(a,b_1,\cdots,b_l)),\Gamma_A(a),
(\Gamma_{R^{\A}_i}(\con(a, \cdots, a)))_{i\leq r}).
$$
The range of $f_\phi$ is bounded by $|Q_\phi| \times |Q_A| \times \prod_{i
\leq r} |Q_{R^{\A}_i}|$. In particular, the range is finite.

To finish the proof, one needs to argue that the isomorphism type of the
substructure $\A \restriction \phi^{a \Sigma^{\star}}(\cdot,\tup{b})$ depends
only on the value $f_\phi(a,\tup{b})$. This follows from the fact that if
$f_\phi(a,\tup{b}) = f_\phi(a',\tup{b'})$, then the corresponding substructures
are isomorphic via the mapping $I:aw \mapsto a'w$ ($w \in \Sigma^{\star}$).
Indeed, by property $(P_{x \in A})$ we get that $aw \in A$ if and only if $a'w
\in A$, for every $w$.  This means that $I$ is a bijection between the sets
$A \cap a\Sigma^{\star}$ and $A \cap a'\Sigma^{\star}$.
Similarly, by the properties $(P_\psi)$ where $\psi$ is taken to be
$\tup{x} \in R_i^{\A}$,
we get that $I$ is an isomorphism between the substructures $\A \restriction
a\Sigma^{\star}$ and $\A \restriction a'\Sigma^{\star}$.
Finally, by $(P_\phi)$ we get that $I$ is an isomorphism between the
substructures $\A \restriction \phi^{a \Sigma^{\star}}(\cdot,\tup{b})$ and $\A
\restriction \phi^{a' \Sigma^{\star}}(\cdot,\tup{b})$.
\end{proof}

\begin{corollary} {\rm \cite{Delh01a}} \label{cor:omegaomega}
The ordinal $(\omega^\omega,\leq)$ is not automatically presentable.
\end{corollary}

\begin{proof}
Suppose for a contradiction that $(\omega^\omega,\leq)$ has an automatic presentation.
In Theorem \ref{thm:sumaug}, take $\phi(x,y)$ to be $x < y$ and
consider the following fact (proved by induction): If the domain of a well-order,
isomorphic to some ordinal of the form $\omega^n$ for $n \in \N$, is
partitioned into finitely many pieces $\{B_i\}_i$, then there is some $i$ so
that the substructure on domain $B_i$ is isomorphic to $\omega^n$.

This means that the set of structures $\eS$ must contain $(\omega^n,<)$ for every $n \in \N$, contradicting
the finiteness of $\eS$.
\end{proof}

\begin{definition}
 
\end{definition}


%=====================================================
\section{???Equivalent presentations}
%=====================================================


%=====================================================
\section{???Summary and open questions}
%=====================================================




















\iffalse










\subsection{MSO interpretations}

Fix a structure $\frakB$ and a tuple of MSO-formulas $\I = (\delta, \Phi_1, \cdots, \Phi_N)$ in the signature of $\frakB$.
Suppose all the free variables are individual variables, $\delta$ has exactly one free variable, and $\Phi_i$ has $r_i$ free variables.
If $\Phi_i^\frakB$ are relations over $\delta^\frakB$ then the structure 
\[
\I(\frakB) := (\delta^\frakB, \Phi_1^\frakB, \cdots, \Phi_N^\frakB)
\] 
is {\em MSO-definable in $\frakB$}. The tuple $\I$ is called an {\em MSO-definition}.

For example $(0^\ast,\suc_0)$ is MSO-definable in $\two$ via the formulas
\[
\delta(x) := (\forall z)\, \left[z \pref x \wedge z \neq x \implies (\exists y \pref x)\,  \suc_0(z,y)\right]
\]
and $\Phi_1(x_1,x_2) := \suc_0(x_1,x_2)$. 


The following lemma says one can translate formulas about the definable structure $\I(\frakB)$ into formulas about the defining structure $\frakB$.

??? say something about fo vars being replaced by mso vars.
or just change all set vars in TL to fo vars...???
\begin{lemma}[Translation Lemma] 
Fix an MSO-definition $\I$ and an MSO-formula $\Phi(X_1,\cdots,X_k)$ in the signature of $\I(\frakB)$.
There is an MSO-formula $\Phi_\I(X_1,\cdots,X_k)$ in the signature of $\frakB$ such that for all subsets $B_i$ of $\delta^\frakB$,
\[
\I(\frakB) \models \Phi(B_1,\cdots,B_k) \iff \frakB \models \Phi_\I(B_1,\cdots,B_k).
\]
\end{lemma}

\begin{proof}
The idea is to relativise all quantifiers to $\delta$ and replace and the $i$th atomic formula by $\Phi_i$.
Formally, define $\Phi_\I$: $(\Psi \wedge \Xi)_\I$ is defined by $\Psi_\I
\wedge \Xi_\I$; $(\neg \Psi)_\I$ by $\neg \Psi_\I$; $(\exists X_i \Psi)_\I$ by
$\exists X_i [ (\forall x \in X_i \delta(x)) \wedge \Psi_\I]$, and similarly for individual quantifiers. 
\end{proof}


A structure $\frakA$ isomorphic to $\I(\frakB)$ is called {\em
MSO-interpretable in $\frakB$}. An isomorphism from $\frakB$ to $\frakA$ is called a {\em co-ordinate map} of the interpretation. 

For example $\one$ is MSO-interpretable in $\two$ with co-ordinate map sending the string $0^n$ to $n$.

Applying the translation lemma to sentences shows that we can transfer MSO-decidability to the interpreted structure.

\begin{proposition}
Suppose that $\frakA$ is MSO-interpretable in a structure with decidable MSO-theory. Then $\frakA$ has decidable MSO-theory.
\end{proposition}

\subsection{Set interpretations}

??? Instead of set interpretation, introduce powerset and FO interpretation ???

Fix a structure $\frakB$ and a tuple of MSO-formulas $\I = (\delta, \Phi_1, \cdots, \Phi_N)$ in the signature of $\frakB$.
Suppose all the free variables are set variables, $\delta$ has exactly one free variable, and $\Phi_i$ has $r_i$ free variables.
In case this defines a structure with domain 
\[
\{U \subset \U \st \frakB \models \delta(U)\}
\]
and relations
\[
\{(U_1,\cdots,U_{r_i}) \st \frakB \models \Phi_i(U_1,\cdots,U_{r_i})\}
\]
call it  $\I(\frakB)$.
A structure $\frakA$ isomorphic to $\I(\frakB)$, say via map $f$, is called {\em
set-interpretable in $\frakB$}. The {\em interpretation} consists of the data $\left<\I,f\right>$.
%Abuse notation and write $\I$ for an isomorphism, called a {\em co-ordinate map}, from $\I(\frakB)$ to $\frakA$.
For example, in section \ref{AS:sec:motivation} we argue that $(\N,+)$ is set-interpretable in
$\one$ via the map sending the non-empty set $X$ to the number $\sum\{2^i \st i \in X\}$.

???change notation $\I(B_i)$???
\begin{lemma}[Translation for set-interpretation]
Fix a set-interpretation $\I$ and a FO-formula $\Phi(x_1,\cdots,x_k)$ in the signature of $\I(\frakB)$.
There is an MSO-formula $\Phi_\I(X_1,\cdots,X_k)$ in the signature of $\frakB$ such that for all sets $B_i$ satisfying $\delta(B_i)$,
\[
\I(\frakB) \models \Phi(\I(B_1),\cdots,\I(B_k))  \iff 
\frakB \models \Phi_\I(B_1,\cdots,B_k). 
\]
\
\end{lemma}

\begin{proof}
The proof is similar to the translation lemma for MSO-interpretations except that first replace all free variables $x_i$ by (unused) 
set variables $X_i$ and define $(\exists x_i \Psi)_\I$ instead as
$\exists X_i [ \delta(X_i) \wedge \Psi_\I]$.
\end{proof}

\begin{proposition}[Transfer decidability]
If $\frakA$ is set-interpretable in a structure with decidable MSO-theory, then $\frakA$ has decidable FO-theory.
\end{proposition}


Here is a straightforward proposition connecting MSO interpretations and set-interpretations.

\begin{lemma}
If $\frakA$ is MSO-interpretable in $\frakB$ and $\frakB$ is set-interpretable
in $\frakC$ then $\frakA$ is set-interpretable in $\frakC$.  
\end{lemma}

The discussion in this section makes sense if we replace every occurence of MSO
by WMSO and every occurence of set-interpretable by finite-set interpretable.

\subsection{Automata with advice}
Fix a predicate $P \subset \2$ (a similar definition holds for $P \subset \N$).
An {\em $\omega$-tree-automaton with advice $P$} is an $\omega$-tree-automaton $\M$ that, while in
position $w \in \2$ can decide on its next state using the additional
information of whether or not $w \in P$. Formally, the automaton accepts an
input $X \subset \2$ if it has a successful run on the characterstic tree $T_{(X,P)}$.

%The {\em emptiness problem} for a collection of automata is to decide, given an automaton $M$, whether or not
%it accepts any string whatsoever.

\fi
