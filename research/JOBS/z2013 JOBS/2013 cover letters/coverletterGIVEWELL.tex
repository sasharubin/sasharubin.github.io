\documentclass{letter}

\usepackage[margin=4cm]{geometry}

\begin{document}

\begin{letter}{}
%Lectureship in Computer Science\\
%AUT}

%\address{Department of Computer Science\\University of Auckland\\Private Bag 92019\\New Zealand}

\signature{\mbox{ }\\[-1.5 cm]Sasha Rubin\\\texttt{sasha.rubin@gmail.com}\ \texttt{forsyte.at/people/rubin/}}

%\opening{}

\hfill December 31, 2013.

I would like to visit your offices in January to discuss working for you as a research analyst. 
I learned about GiveWell from a friend at the end of 2009. What strikes me as impressive, apart from your focus on strength of evidence, are your regular self-evaluations and cataloguing of your own mistakes. These actions seem crucial for an organisation that is striving for  efficiency. 

As my resume indicates, I have limited experience in non-profits compared with academic pursuits. My discussions with members of various human rights groups and non-profits --- Amnesty International (London), B'Tselem (Jerusalem),  EqualEducation (South Africa), Anani Memorial International School (Accra) --- have led me to decide to apply my analytical and writing skills towards a cause with positive consequences for others.

%I admire your no-nonsense advice, even if at first it comes across as counter-intuitive (eg. some charities don't have room for more funding).
%
%[Say more about their company.]

As a research analyst for GiveWell I can offer the following qualities I've cultivated in my academic career:
\begin{enumerate}
\item As a researcher I identify feasible problems, and break them down until each part is well understood. This process sometimes involves questioning ideas that might be otherwise taken for granted.
%\item I've developed general analytic skills. I've published in top journals and conferences in my area
\item I take pride in my writing --- I aim for readability and relevance. I have written a number of survey articles of my field, each for a different audience, and am assisting with the editing of a technical handbook. I enjoy finding good ways to communicate difficult ideas.
% I am currently assisting with the editing of technical handbook.
%\item I try to learn from others. When I read a solution to a problem that I failed to solve, I try establish what I could have done differently. What questions could I have asked myself that would have led me to think in a way to find that solution?
\item I enjoy working with bright people who are passionate about similar things. I have a number of collaborators around the world and successfully supervised a number of undergraduate projects.
\end{enumerate}

%I'll be frank about my shortcomings. I like to question foundational issues, which is not always the best strategy for moving forward. I am slow to absorb completely new ideas -- I usually need to think about things for a while before I've internalised them. I work best in a group where we can throw ideas around until we find some clarity of direction. Although I am not the strongest problem solver of my peers, nor the weakest, the main obstacle to a successful academic career seems to be my current inability to identify feasible problems and research directions to work on.

%I am looking for a work environment where people talk clearly and without jargon about their specialities, and where deep conversations take place with concrete outcomes. I imagine this is the sort of environment you are creating.

%My own charity giving has been intermittent. Besides giving to Oxfam and Doctors without Borders (I choose these based on their reputation, rather than on a close analysis of their effectiveness), I have also helped fund a film project (\textsf{www.sahelcalling.com}) that aims to raise awareness about refugees and internally displaced people in Mali. I chose to give to this cause because the producer is a friend who convinced me that the film could help fill in the shortfall in reporting about the crisis in Mali.

%Although I have not volunteered or worked for a charity, I have taken an interest in understanding the mechanics of human rights organisations (I did this by meeting with members of Amnesty International in London and B'Tselem in Jerusalem). I also have an interest in education, specifically science and mathematics. I briefly volunteered to teach school-level mathematics in 2010 in Ghana and South Africa. In Ghana I observed and taught some mathematics classes and subsequently communicated my thoughts to the principal. Although the principal thanked me for my thoughts, I sensed that the school's financial problems took priority over improving the quality of education. 

%If you think I would fit well in your organisation I would be happy to visit you.
The main purpose of the visit, on my side, would be for me to get a clearer idea of what it is like to work for GiveWell, and to learn to what extent I can contribute to the work of researching and recommending charities.

I am travelling from Vienna (my current base) to California for a conference in January. I will be in San Francisco between January 13 and January 17, and then again on January 29 and January 30. I would be glad if I could visit you during those times.

%
%Finally, if we do work together, you will have a problem convincing me to use American spelling conventions or engaging me in long conversations about issues we clearly know nothing about.

\closing{sincerely,}
\end{letter}

\end{document} 