\documentclass{letter}

\begin{document}

\begin{letter}{}
%Lectureship in Computer Science\\
%AUT}

%\address{Department of Computer Science\\University of Auckland\\Private Bag 92019\\New Zealand}

\signature{\mbox{ }\\[-1.5 cm]Sasha Rubin\\\texttt{sasha.rubin@gmail.com}}

\opening{}

I would like to apply for the position of lecturer in computing sciences.

%I completed my doctoral studies under Bakhadyr Khoussainov at the University of Auckland in $2004$ in theoretical computer science, specifically on describing mathematical systems by automata.

My research area is in automata theory, a branch of theoretical computer science. It has immediate connections with mathematical logic, game theory, and formal verification. Although my work is theoretical and mathematical in nature I recently took a joint postdoctoral position at IST Austria and TU Vienna with the aim of extending my reach: my current work involves the description and analysis of distributed algorithms and systems. 

As an expositor I have written or co-written a number of surveys of my subfield, each for a different audience (logicians, automata theorists, computer scientists using logical methods). I have cultivated collaborators in New Zealand (Bakhadyr Khoussainov was my PhD supervisor), the United States (I collaborated with Moshe Vardi as a graduate student), Singapore and Europe.

My CV reflects that I have taught at a variety of levels, including courses in discrete mathematics, mathematical logic, and computation. My undergraduate teaching evaluations include statements such as 
\begin{quote}
Professor was very well prepared and made us really understand what we were learning \dots 
Instead of memorizing formulas, he helped us understand what we were learning, why we
were learning it and why it was useful.
\end{quote}

I would like to bring your attention to my supervision of an NSF funded Research Experience for Undergraduates.
We produced two papers (accepted to the journal {\em Theoretical Computer Science} and the conference {\em GandALF 2012}).
I am currently co-supervising an undergraduate who visited us from India.

On a personal note, I have an inclination to engage with others about their research and teaching. I enjoy environments where people are happy to talk clearly and without jargon about their work. I am particularly interested in talking with scientists. To illustrate, I recently created a computer science seminar at IST Austria that tries to deepen the conversations between computer scientists as well as with those natural scientists using computational ideas. I am also on the organising committee of a one-day symposium to be held at IST Austria devoted to describing and analysing shape in mathematics and biology. 

\closing{Thank you for your consideration,}
\end{letter}

\end{document} 