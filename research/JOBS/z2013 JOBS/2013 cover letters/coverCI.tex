\documentclass{letter}

\begin{document}

\begin{letter}{}
%Lectureship in Computer Science\\
%AUT}

%\address{Department of Computer Science\\University of Auckland\\Private Bag 92019\\New Zealand}

\signature{\mbox{ }\\[-1.5 cm]Sasha Rubin\\\texttt{sasha.rubin@gmail.com}}

\opening{}

I would like to apply for the position of Assistant Professor -- Mathematics.

I recently visited the campus and informally met with Ivona Grzegorczyk and Cindy Wyels. Even in that short visit I was impressed by the friendly atmosphere, a rigorous undergraduate training, outreach, and openness to new ideas. I also paid a brief visit to the computer science department.

%I completed my doctoral studies under Bakhadyr Khoussainov at the University of Auckland in $2004$ in theoretical computer science, specifically on describing mathematical systems by automata.

My research is in theoretical computer science. My work touches on mathematical logic, automata theory, game theory, and formal verification.  I hold a joint postdoctoral position at IST Austria and TU Vienna. My research in Vienna is an extension of my training: I am working on the description and analysis, via logical methods, of distributed algorithms and systems. This semester I will teach a graduate level course, open to students from IST Austria and TU Vienna, on the automata-theoretic approach to verification, a topic not covered in their training.

As an expositor I have written or co-written a number of surveys of my subfield, each for a different audience (logicians, automata theorists, computer scientists using logical methods). I have cultivated collaborators in New Zealand (Bakhadyr Khoussainov was my PhD supervisor), the United States (I collaborated with Moshe Vardi as a graduate student), Singapore and Europe.

My CV reflects that I have taught at a variety of levels, including courses in discrete mathematics, mathematical logic, and calculus. My undergraduate teaching evaluations include statements such as 
\begin{quote}
Professor was very well prepared and made us really understand what we were learning \dots 
Instead of memorizing formulas, he helped us understand what we were learning, why we
were learning it and why it was useful.
\end{quote}

I would like to bring your attention to my supervision of undergraduate research. In total we produced three papers and the undergraduates are now in graduate programs or working in industry.

On a personal note, I have an inclination to engage with others about their research and teaching. I enjoy environments where people are happy to talk clearly and without jargon about their work. I am particularly interested in talking with scientists. To illustrate, I created a computer science seminar at IST Austria that tries to deepen the conversations between computer scientists as well as with natural scientists who use computational ideas. 
I was also on the organising committee of a one-day symposium held at IST Austria devoted to describing and analysing shape in mathematics and biology. 

\closing{Thank you for your consideration,}
\end{letter}

\end{document} 