\documentclass[margin,line,a4paper]{resume}
 
 \usepackage{multicol}

\usepackage[latin1]{inputenc}
\usepackage[english,danish]{babel}
\usepackage[T1]{fontenc}
\usepackage{graphicx,wrapfig}
\usepackage{url}
\usepackage[colorlinks=true, a4paper=true, pdfstartview=FitV,
linkcolor=blue, citecolor=blue, urlcolor=blue]{hyperref}
\pdfcompresslevel=9
 
\begin{document}
\textsc{\Large \textbf{Sasha Rubin} -- Curriculum Vitae, January 2014} \vspace{1mm}
\begin{resume}
%    \vspace{0.5cm}
%    \begin{wrapfigure}{R}{0.6\textwidth}
%         \vspace{-1cm}
%%        \begin{center}
%        \includegraphics[width=0.6\textwidth]{me}
%        \end{center}
%         \vspace{-1cm}
%    \end{wrapfigure}

\section{\mysidestyle Contact}%\vspace{2mm}
% Sasha Rubin\\
 41/14 Alser Stra{\ss}e,\\
 Vienna, 1080, Austria \\
 \textsf{sasha.rubin@gmail.com}\\
\textsf{forsyte.at/people/rubin/}\\
% 	  +43 0 66488 349201
	   %+$64$ $9$ $486$ $1196$\\

%\section{\mysidestyle Current Aim}%\vspace{2mm}
%
%I am interested in using my analytical skills to work, as part of a team, on projects that will have broad impact improving the well being of as many people as possible.
%\section{\mysidestyle Personal Data}   
%   Date of Birth:  $16.02.1976$\\
%    Place of Birth: Johannesburg, South Africa\\
%   Citizenship: New Zealand \\

%\section*{\center{Personal Data}}
%\section{\mysidestyle{Personal Data}
%\begin{tabular}{@{}ll}
%
%  \bf Date of Birth 
%		 & $16$ February $1976$\\
%\ifcut
% \bf Place of Birth 
%		 & Johannesburg, South Africa\\
%\fi
%  \bf Citizenship & New Zealand \\
%%  \bf Language   & English\\
%  \bf Address & IST Austria \\
%  &Am Campus $1$\\
% & $3400$ Klosterneuburg, Austria \\
%  \bf Contact 
%    %  & Web  www.math.cornell.edu / $\sim$ srubin
%\end{tabular}

\section{\mysidestyle University}%\vspace{2mm}

\begin{description}
 
\item[Postdoctoral Researcher] ($3.2012$ -- present)\\ IST Austria and TU Vienna, Austria.

\item[Visiting Lecturer] ($2.2010$ -- $5.2010$)\\ Department of Mathematics, University of Cape Town, South Africa.

\item[Visiting Assistant Professor] ($08.2008$ -- $12.2009$)\\ Department of Mathematics, Cornell University, USA.

\item[Honorary Research Fellow] ($12.2004$ -- $02.2008$)\\ Department of Computer
Science, University of Auckland, New Zealand.\\
Supported by New Zealand Science and Technology Postdoctoral Fellowship.

%\end{description}
%
%\section{\mysidestyle Ph.D.}%\vspace{2mm}
%
%\begin{description}
 
\item[PhD] Mathematics and Computer Science ($2004$)\\
University of Auckland, New Zealand.\\
Supervisor: Bakhadyr Khoussainov\\
Title: Automatic Structures\\
Awards: Vice-chancellor's prize for the best doctoral thesis in the Faculty of Science, and
Montgomery memorial prize in logic from the Department of Philosophy.\\
%
%
%\item[MSc] Mathematics and Computer Science ($1997$ -- $1998$)\\
%University of Auckland, New Zealand.\\
%   %March $1997$ - November $1998$ \\
% First class
%
%\item[BSc] Mathematics and Computer Science ($1994$ -- $1996$)\\
% University of Cape Town, South Africa.\\
% Deans Honour List
%  %\bf March $1994$ - November $1996$\\
%%  Mathematics and Computer Science
%%  
\end{description}



\section{\mysidestyle {Research Interest}}%\vspace{2mm}

I work in theoretical computer science studying the power of automata theory and mathematical logic for describing mathe\-matical structures.  Concretely, I have contributed to the following areas: automatic structures, formal verification, and finite model
theory. I am currently working on the theory of distributed systems and distributed algorithms using logical and automata-theoretic methods.
%in particular verification of probabilistic finite state systems.

\section{\mysidestyle{Recent Invited Workshop-Talks}}

%\subsection*{Recent Seminar Talks}  IST Austria and TU Vienna ($2011,2012$), CNRS Liafa Paris 7 ($2011$), Tel Aviv University ($2011$), University of Cape Town ($2010$) \\
%Cornell ($2007,2008,2009$),
%\ifcut LSV Cachan ($2008$), CNRS LIAFA Paris 7 ($2007$), Heidelberg ($2007$)\\
%\fi
%\subsection*{Invited Talks}
 {\it Finite and Algorithmic Model Theory}, Les Houches, France ($05.2012$)\\
 {\it Automata theory and Applications}, IMS programme,  Singapore ($09.2011$)\\
{\it Computational Model Theory}, CNRS SIG, Bordeaux, France ($06.2008$)\\
 {\it Algorithmic-Logical Theory of Infinite Structures}, Dagstuhl, Germany ($10.2007$)\\
%{\it Finite and Algorithmic Model Theory}, Newton Institute programme, Durham, England ($01.2006$)
%, Workshop on Automata, Structures and Logic in Auckland ($12.2004$).

%\subsection*{Conference Talks} CiE ($2008$),

%$2008 - 2009$: Logic Seminar, Cornell Talks in the Logic Seminar at Cornell. \\
%Ongoing work with Anil Nerode and Dexter Kozen.\\

%$2.2007$: Attended the 'Model Theory and Computable Model Theory' workshop, part
%of the University of Florida's Special Year in Logic.\\

%\section{\mysidestyle{Recent research visits}}
%
%Topic: Application of Logic to AI\\
%Host: \L ukasz Kaiser, Universit\'e Paris Diderot, France ($10.2011$)
%
%Topic: Games of Imperfect Information and Pushdown Automata\\
%Host: Aniello Murano, Universit\`a degli Studi di Napoli Federico II., Italy ($08.2011$).
%
%Topic: Logical-Interpretability and Trees\\
%Host: Alexander Rabinovich, Tel Aviv University, Israel ($5.2011$ -- $8.2011$)

%Erich Gr\"adel and students at RWTH Aachen ($08.2006-01.2007$).\\
%\section{\mysidestyle{Refereed}}
%\begin{description}
%\item[Journals]\
%
%Journal of Symbolic Logic, Logical Methods in Computer Science, Central European Journal of Mathematics, Information and Computation, Journal of Logic and Computation, Theory and Practice of Logic Programming, Handbook of Model Checking
%
%\item[Conferences]\
%
%LICS, STACS, FoSSaCS, CONCUR, FSTTCS, CSL, CiE, LATA
%
%\end{description}


%
%\subsubsection*{In preparation}
%
%Chapter on Automatic Structures in Handbook {\it Automata: From mathematics to
%applications}, to be published by EMS and AutomathA network.\\
% 
%{\it Regularity preserving quantifiers}, with V. Goranko and M. Vardi.\\
%
%%{\it Automatic linear orders} with V. Goranko and T. Knapik. \\
%
%{\it Myhill-Nerode with oracle} with A. Kruckman, J. Sheridan, B. Zax.\\
%
%{\it Trap depth of parity games} with P. Phalitnonkiat, A. Grinshpun, A.Tarfulea.\\

\iffalse

{\it Finite Automata and Algebraic Structures},
with Bakhadyr Khoussainov, 
Abstracts of Symposium on Logic and Applications, Novosibirsk, May $2000$. \\

{\it On Automata Presentable Structures},
with Bakhadyr Khoussainov, 
Abstracts of papers presented to the American Mathematical Society, 
$20(2), 1999$. \\ 
\fi

%\section*{\center{Selected Talks}}

%\pagebreak

%
%
%$1998$: Competed as part of a team of three, in the world finals of the $1998$
%ACM Programming Contest in Atlanta, Georgia USA, representing the University of Auckland
%and New Zealand. 
%%The same team won the Regional Programming Contest in $1997$.\\ 
%\fi

\section{\mysidestyle{Undergraduate Teaching philosophy}}

My goal as a teacher is to guide students through the material (eg. I point out which ideas are fundamental and which are technicalities), show students how the material is relevant to their degree, and help students think deeply.  I regularly self-evaluate and engage colleagues in order to discover good teaching principles.
% (eg. is there a better way of breaking the material into chunks to help students follow?).
%and engage colleagues in discussions
% in order to improve my presentation %of the material, 
I employ questions which encourage students to express themselves clearly and internalise the material, eg. `can anyone help A with her answer?', `can you explain B's idea to me?', `what do you mean by X?', `are you sure?'.
Another technique I have used is administering an easy online quiz that requires students to read the relevant section of the textbook before coming to class; as a result students ask deeper questions than they otherwise would, a sign that they are better prepared to understand the material discussed in class.
%,   `who will summarise today's class?'.
%,  `if all I do is teach you skills I'm short-changing you'.



\section{\mysidestyle{Recent Supervision and Teaching}}

\begin{description}
\item[Supervision]\

%Ongoing PhD project\\
%Topic: Synthesis of Distributed Systems.\\
%{\em IST Austria} $(2013)$\\

Summer undergraduate project\\
Topic: Edit-distance and Formal Languages.\\
{\em IST Austria}  $(2012)$\\

Summer research experience for undergraduates\\
Topic 1: Parity Games.\\
Topic 2: Automatic Structures with Advice.\\
{\em  Cornell University, Department of Mathematics} $(2009)$\\

\item[Teaching]\

Logic and Computation (undergraduate)\\
{\it University of Cape Town, Department of Mathematics} $(2010)$\\

 Logical Definability and Random Graphs (graduate) \\
{\it Cornell University, Department of Mathematics} $(2009)$\\

Calculus for Engineers (undergraduate)\\
 {\it Cornell University, Department of Mathematics}  $(2008 - 2009)$\\

%Computational Biology (undergraduate, TA)\\
%{\it University of Auckland, Department of Computer Science} $(2008)$\\

%Discrete Structures in Mathematics and Computer Science (undergraduate)\\
%Mathematical Foundations of Software Engineering (undergraduate) \\
%{\it University of Auckland, Department of Computer Science} $(2007)$\\
%
%Logic and Computation in Finitely Presentable Infinite Structures (co-taught five day advanced course)\\
%{\it European Summer School in Logic, Language and Information} $(2006)$\\
%
% Introduction to Formal Verification (advanced undergraduate)\\
%  {\it University of Auckland, Department of Computer Science} $(2003)$\\
%
%Automata Theory (undergraduate)\\
%  {\it University of Auckland, Department of Computer Science} $(2002)$\\
%
%Pre-calculus\\
% {\it University of Wisconsin, Madison, Department of Mathematics} $(2001)$ \\
% 
 \end{description}
 

%  {\bf Tutoring}\\
%  {\it University of Auckland, Department of Mathematics} \\
%  $1998$ and $1999$: Undergraduate Mathematics\\


\iffalse $1998$ and $1999$: Stages $1$, $2$ and $3$ in the Mathematics \\
   	 %Assistance Room \\
	 $1999$: Assistant tutor for `Discrete Mathematics', Stage $2$
	Mathematics\\
	 $1999$: Demonstrator for `Combinatorial Computing', Stage $3$
	Mathematics\\
\fi

%\pagebreak 

\section{\mysidestyle{Recent community service}}

%%%HBMC
Besides ongoing refereeing for journals and conferences, I am currently involved in reviewing and assisting the editors with the Handbook of Model Checking.

%One reviewer wrote "We are extremely grateful to this reviewer for his/her careful reading of the paper, and for his/her constructive suggestions."

In $2012/2013$ I was one of the organisers of the IST Austria Young Scientist Symposium on the topic `Understanding Shape: {\it in silico} and {\it in vivo}'.\\
\textsf{ist.ac.at/young-scientist-symposium-2013/}

In $2012$ I formed the computer science seminar at IST Austria whose goal is to foster collaborations within the institute.\\
\textsf{ist.ac.at/computer-science-seminar/}

%In $2010$ I briefly volunteered at a secondary school school in Accra, Ghana, teaching, observing and commenting on grade $5$ mathematics classes. I also  briefly volunteered in Khayelitsha, South Africa, helping high-school students prepare for their high-school mathematics exams.
%equaleducation.org.za


\section{\mysidestyle{Publications}}

\begin{description}
\item[Book chapters]\

 {\it Automatic Structures} in {\it Automata: From mathematics to applications}, J.E. Pin, Ed., to be published by EMS.

 {\it Automata based presentations of infinite structures} with V. B{\'a}r{\'a}ny and E. Gr{\"a}del,
in {\it Finite and Algorithmic Model Theory}, J. Esparza, C. Michaux, and C. Steinhorn, Eds.,
Series: London Mathematical Society Lecture Note Series (379), $1-76$, $2011$.

\item[LICS Proceedings]\

{\it Interpretations in trees with countably many branches}, with A. Rabinovich, $551-560$, $2012$. 


{\it Automatic Structures: Richness and Limitations}, with B. Khoussainov, A. Nies and F. Stephan, 
$44-53$, $2004$. 

{\it Automatic Partial Orders}, with B. Khoussainov and F. Stephan, $168-177$, $2003$.

{\it Some Results on Automatic Structures}, with B. Khoussainov
and H. Ishihara,  $235-244$, $2002$.

\item[STACS Proceedings]\

{\it Cardinality and counting quantifiers on omega-automatic structures}, with V.  B{\'a}r{\'a}ny and \L. Kaiser, $385-396$, $2008$.  

{\it Order invariant MSO is stronger than counting MSO},with T. Ganzow, $313-324$,  
 $2008$.  
 
{\it Definability and Regularity in Automatic Structures}, with B. Khoussainov
and F. Stephan,  $440-451, 2004$.  

%International Workshop on Logic and Computational Complexity $2007$.\\

\item[CAV Proceedings]\

{\it Verifying $\omega$-regular Properties of Markov Chains}, with D. Bustan and
M. Vardi, $189-201, 2004$. 

\item[VMCAI Proceedings]\

{\it Parameterized Model Checking of Token-Passing Systems}, with B. Aminof, S. Jacobs and A. Khalimov.
to appear, Jan $2014$. 

\item[Other Conference Proceedings]\

{\it Finite Cycle Games} with B. Aminof, {\it Strategic Reasoning}, $2014$.

{\it How to Travel Between Languages} with  K. Chatterjee and S. Chaubal, {\it LATA}, $2013$.

{\it A Myhill-Nerode Theorem for Automata with Advice} with A. Kruckman, J. Sheridan and B. Zax, {\it GandALF}, $238-246$, $2012$.

%{\it Quantifiers on Automatic Structures} with V. Goranko, {\it CiE}, $2008$.

\item[Journals]\

{\it Alternating Traps in Parity Games} with P. Phalitnonkiat, A. Grinshpun, A.Tarfulea, Theoretical Computer Science,  $(521), 73-91, 2014$.

{\it Automata presenting structures: A survey of the finite-string case}, The Bulletin of Symbolic Logic, 
$14 (2), 169-209, 2008$.

{\it Automatic Structures: Richness and Limitations}, with B. Khoussainov, A. Nies and F. Stephan, 
Logical Methods in Computer Science, Vol $3$, $2007$.  

{\it Automatic linear orders and trees}, with B. Khoussainov and F. Stephan, 
ACM Transactions on Computational Logic,
$6 (4), 675-700, 2005$.  


%{\it Automatic Linear Orders and Trees: Revised}, CDMTCS Technical Report $208$,
%Department of Computer Science, University of Auckland, $2003$.   
 
 %{\it Definability and Regularity in Automatic Presentations of Subsystems of
%Arithmetic}, CDMTCS Technical Report $209$,
%Department of Computer Science, University of Auckland, $2003$.   
 
{\it Automatic Structures - Overview and Future Directions}, with 
B. Khoussainov,
Journal of Automata, Languages and Combinatorics, $8(2), 287-301, 2003$.   

{\it Graphs with Automatic Presentations over a Unary Alphabet}
Journal of Automata, Languages and Combinatorics, $6(4), 467-480, 2001$.   

{\it Finite Automata and Well Ordered Sets},
New Zealand Journal of Computing, $7(2), 39-46, 1999$. 

\end{description}


\pagebreak

\section{\mysidestyle{Academic References}}

\begin{multicols}{2}
%\subsection*{Primary}


{\bf Bakhadyr Khoussainov}\\
Department of Computer Science\\
University of Auckland, New Zealand\\
bmk@cs.auckland.ac.nz\\
+$64$ $9$ $373$ $7599$ Ext $85120$
%Fax   : +$64$ $9$ $373$ $7453$\\
%- PhD supervisor\\
%- Project supervisor for `Finite Automaton Presentable Unary Structures', 1998


{\bf Helmut Veith}\\
% Full Professor\\
 Faculty of Informatics\\
 TU Wien, Austria\\
% Institut fur Informationssysteme 184/4\\
%AB Formal Methods in Systems Engineering\\
%Favoritenstra�e 9, 1040 Wien, Austria\\
veith@forsyte.tuwien.ac.at\\
 +$43$ $1$ $58801$ $18441$
\columnbreak


{\bf Erich Gr\"adel}\\
Mathematische Grundlagen der Informatik,\\
RWTH Aachen, Germany\\
%D-52056 Aachen\\
graedel@logic.rwth-aachen.de\\
	+$49$ $241$ $80$ $21730$


%\columnbreak
{\bf Moshe Y. Vardi}\\
Department of Computer Science\\
Rice University, USA\\
vardi@cs.rice.edu\\
 +$1$ $713$ $348$ $5977$
%Fax   : +$1$ $713$ $348$ $5930$\\

%\columnbreak
%{\bf Dr. Frank Stephan}\\
%School of Computing\\
%National University of Singpore\\
%3 Science Drive 2, Singapore 117543\\
%fstephan@comp.nus.edu.sg\\
%Phone :  +$65$ $6516$ $2759$\\
%Fax   :  +$65$ $7795$ $5452$\\

\iffalse
{\bf Anil Nerode}\\
Goldwin Smith Professor of Mathematics\\
%545 Malott Hall\\
Cornell University\\
	 Ithaca, NY 14853\\
	 anil@math.cornell.edu\\
Phone: +$1$  $607$ $255$ $3577$\\
\fi
%\columnbreak

%{\bf Helmut Veith}\\
%% Full Professor\\
% Faculty of Informatics\\
% TU Wien, Austria\\
%% Institut fur Informationssysteme 184/4\\
%%AB Formal Methods in Systems Engineering\\
%%Favoritenstra�e 9, 1040 Wien, Austria\\
%veith@forsyte.tuwien.ac.at\\
%Phone +$43$ $1$ $58801$ $18441$
 
 \end{multicols}

\section{\mysidestyle{Teaching References}}

\begin{multicols}{2}

{\bf Maria Terrell}\\
Director of Teaching Assistant Programs\\
% 	 225 Malott Hall \\
	 Cornell University, USA\\
%	 Ithaca, NY 14853\\
	 maria@math.cornell.edu\\
	   +$1$  $607$ $255$ $3905$


{\bf David Way}\\
Associate Director of
Instructional Support
Center for Teaching Excellence\\
%420 D CCC\\
Cornell University, USA\\
%Ithaca, NY 14853\\
dgw2@cornell.edu\\
+$1$ $607$ $255$ $2663$
%www.cte.cornell.edu\\
%fax: (607) 255 1562\\

\end{multicols}

\section{\mysidestyle{Supervision References}}
\begin{multicols}{2}

{\bf Bob Strichartz}\\
Department of Mathematics\\
 Cornell University, USA\\
%	 Ithaca, NY 14853\\
str@math.cornell.edu\\
+$1$  $607$ $255$ $3509$

{\bf Krishnendu Chatterjee}\\
IST Austria\\
Krishnendu.Chatterjee@ist.ac.at\\ 
+$43$ $2243$ $9000$ $3201$

\end{multicols}

\end{resume}
\end{document}


{\bf Dr. Frank Stephan}\\
School of Computing\\
National University of Singpore\\
3 Science Drive 2, Singapore 117543\\
fstephan@comp.nus.edu.sg\\
Phone :  +$65$ $6516$ $2759$
%Fax   :  +$65$ $7795$ $5452$\\


%----------------------------------------------------
%----------------------------------------------------
%----------------------------------------------------
%----------------------------------------------------
%----------------------------------------------------
%----------------------------------------------------


\iffalse
{\bf Dr. Valentin Goranko}\\
School of Mathematics\\
University of Witwatersrand\\ 
Private Bag 3, WITS 2050\\
Johannesburg, South Africa\\
goranko@maths.wits.ac.za\\
Phone : +$27$ $11$ $717$ $6243$ \\
%Fax   : +$27$ $11$ $717$ $6259$ \\
\fi
%- Tutorial co-ordinator for undergraduate paper `Discrete Mathematics', %1999 \\
%- Project and summer scholarship supervisor for `Application of Elementary \\
%\phantom{- }Submodels to Topology', 1998\\ 
%- Lecturer for `Logic and Set Theory', 1998\\ 
% \enlargethispage*{1cm}

%\section*{Conferences Attended}
%\begin{tabular}{@{}ll}
% {\bf 2000} & Computational Group Theory, Sydney\\
% {\bf 1999} & Third New Zealand Computer Science Research Students'
%Conference \\
%	    & April, Waikato\\
% {\bf 1999} & ACSC - DMTCS/CATS \\
%	    & January, Auckland \\
% {\bf 1999} & NZMRI summer workshop, Harmonic Analysis\\
%	    & January, Raglan, New Zealand\\
% {\bf 1998} & Second Japan-New Zealand Workshop on 
%		{\em Logic in Computer Science}\\
%	    & October, Auckland\\
% {\bf 1998} & First International Conference on 
%		{\em Unconventional Models of Computation}\\
%	    & January, Auckland \\
% {\bf 1997} & First Japan-New Zealand Workshop on {\em Logic in Computer Science}\\
%	    & August, Auckland\\
% {\bf 1997} & Fifth Australasian Mathematics Convention\\
%	    & July, Auckland\\
%\end{tabular}


\iffalse
\subsection*{Additional}
{\bf Dr. David McIntyre}\\
Department of Mathematics\\
University of Auckland, New Zealand\\
mcintyre@math.auckland.ac.nz\\
Phone : (+64 9) 373 7599 Ext 8763\\
%- Tutorial co-ordinator for undergraduate paper `Discrete Mathematics',
%1999 \\
%- Project and summer scholarship supervisor for `Application of Elementary \\
%\phantom{- }Submodels to Topology', 1998\\
%- Lecturer for `Logic and Set Theory', 1998\\
\enlargethispage*{1cm}


%\pagebreak
\section*{Non Academic}
\subsection*{University of Auckland}
{\it Membership}\\
  Auckland University Dramatic Society (1998/1999/2000)\\
  Auckland University Comedy and Improvisation Club (1997) Secretary (1998)\\


{\it Performance}\\
  `Morte Accidentale Di Un Anarchico', (Dario Fo) (1999)\\
  Celebration of Performing Arts, Auckland Town Hall (Stoppard, Simon) (1999)\\
  Three comedy shows, Auckland University (1997 and 1998)\\
  A duet in an evening of short plays, Auckland University (Stoppard) (1998)\\
  Cultural Mosaic Festival, Auckland University (Stoppard) (1997)\\
\begin{center}
--------------------
\end{center}
\fi


\section*{Seminars/Talks} 

 
`Techniques to prove non-automaticity', University of Heidelberg Logic Seminar, 2002 \\


`Some Results on Automatic Structures', with Bakhadyr Khoussainov
and Hajime Ishihara, 17th Annual IEEE Symposium on Logic in Computer Science, 2002. \\


`Automata-theoretic approach to verification of probabilistic systems',
Rice University Computer Science Theory Seminar, 2001. \\


`Automatic Structures', University of Notre Dame Logic Seminar, 2001, and \\
University of Madison, Wisconsin, Logic Seminar, 2001.\\


`Finite Automata and Relational Structures', with Bakhadyr Khoussainov, 
DCAGRS, July 2000, London, Ontario \\


 `Finite Automata and Well Ordered Sets', 
3rd New Zealand Computer Science Research Conference, 1999, Waikato, New
Zealand \\


{\em Auckland Department of Computer Science:}\\
 `Finite Model Theory - Ehrenfeucht-Fraisse Theorem', 2000\\
 `Extracting Algebraic Information from Finite State Machines', 1999\\
 `Finite Automata and Regular Languages', 1999\\


{\em Auckland Department of Mathematics:}\\
 `Algebraic Structures and Finite Automata', 1999\\
 `Applications of Elementary Submodels to Topology', 1999\\ 
\fi

\iffalse
		
  {\bf Marking -} {\it University of Auckland} \\
	 $1998$: Stage $3$ Assignments for the Department of Mathematics\\
  	 1997: Stage 1 Assignments for the Departments of 
	 Computer Science and Mathematics\\
\fi
\iffalse
