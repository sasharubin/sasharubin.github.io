\documentclass[11pt]{article}

\usepackage{amsmath}
\usepackage{amssymb}
\usepackage{amsthm}
\usepackage{amscd}
\usepackage{amsfonts}
\usepackage{graphicx}%
\usepackage{fancyhdr}
\def\buchi{B{\"u}chi}
\def\str{\mbox{\tt str}}
\def\wstr{\mbox{\tt wstr}}
\def\tree{\mbox{\tt tree}}
\def\trees{\mbox{\tt trees}}
\def\FO{\mbox{\rm FO}}
\def\MSO{\mbox{\rm MSO}}
\def\WMSO{\mbox{\rm WMSO}}
%\def\FO{FO}
%\def\MSO{MSO}
%\def\WMSO{WMSO}

%%%%%%%%%USED IN BSL
%\def\waut{\mbox{\tt W-AutStr}}
%\def\taut{\mbox{\tt T-AutStr}}
%\def\baut{\mbox{\tt $\omega$W-AutStr}}
%\def\raut{\mbox{\tt $\omega$T-AutStr}}
%\def\aut{\mbox{\tt AutStr}}
%%%%%%%%%%%%%%

\def\waut{\mbox{\tt S-AutStr}}

\def\taut{\mbox{\tt T-AutStr}}
\def\injtaut{\mbox{\tt inj-T-AutStr}}

\def\baut{\mbox{\tt $\omega$S-AutStr}}
\def\injbaut{\mbox{\tt inj-$\omega$S-AutStr}}

\def\raut{\mbox{\tt $\omega$T-AutStr}}
\def\injraut{\mbox{\tt inj-$\omega$T-AutStr}}

\def\aut{\mbox{\tt AutStr}}

%\newcommand{\aut}{\diamondsuit}

\def\suc{\text{suc}}
\def\edom{\equiv_{\text{dom}}}
\def\el{\text{el}}

\def\qed{\nolinebreak \hfill $\lhd$}
\def\qex{\hfill $\lhd$}


\newcommand{\dom}{\mbox{dom}}
\newcommand{\st}{\ \vert \ }
\def\Fraisse{Fra\"{\i}ss\'{e} }

\newcommand{\U}{{\mathcal U}}
\newcommand{\Power}{{\mathbb P}}
%\renewcommand{\L}{{\mathcal L}}
\newcommand{\eL}{{\mathcal L}}

\newcommand{\A}{{\mathcal A}}
\newcommand{\B}{{\mathcal B}}
\newcommand{\C}{{\mathcal C}}
\newcommand{\D}{{\mathcal D}}
\newcommand{\E}{{\mathcal E}}
\newcommand{\F}{{\mathcal F}}
\newcommand{\G}{{\mathcal G}}
\newcommand{\J}{{\mathcal J}}
\newcommand{\M}{{\mathcal M}}
\newcommand{\Hch}{{\mathcal H}}
\newcommand{\T}{{\mathcal T}}
\newcommand{\Oh}{{\mathcal O}}
\newcommand{\I}{{\mathcal I}}
\newcommand{\R}{{\mathcal R}}
\newcommand{\V}{{\mathcal V}}
\newcommand{\W}{{\mathcal W}}
\newcommand{\eS}{{\mathcal S}}

\newcommand{\frakA}{\mathfrak{A}}
\newcommand{\frakD}{\mathfrak{D}}

\newcommand{\NS}{\mbox{$(\N,\textrm{S})$}}     %successor
\newcommand{\NL}{\mbox{$(\N,\leq)$}}  %ordering
\newcommand{\IR}{\mbox{\rm IR}}

\newcommand{\tup}[1]{\mbox{$\overline {#1}$}}

\newcommand{\K}{{\mathbf K}}
\newcommand{\mb}[1]{{\mathbf {#1}}}

\newcommand{\class}{\mathsf C}
\newcommand{\klass}{\mathsf K}
\newcommand{\klassaut}{{\mathsf K}^{\rm aut}}

\newcommand{\0}{{\bf 0}}
%\newcommand{\1}{{\bf 1}}


%\newcommand{\kb}{<_{kb}}
%\newcommand{\kbeq}{\leq_{kb}}
%\newcommand{\kbg}{>_{kb}}
%\newcommand{\val}{{\rm val}}
%\newcommand{\num}{{\rm num}}
%\newcommand{\base}{{\rm base}}
%\newcommand{\size}{{\rm size}}
%\newcommand{\last}{{\rm last}}
%\newcommand{\first}{{\rm first}}
%\newcommand{\WMSO}{\mbox{${\rm WMSO}$}}

%\newcommand{\modm}{\, (\rm{mod \,} m) \,}
%\newcommand{\modk}{\, (\rm{mod \,} k) \,}
%\newcommand{\modp}{\, (\rm{mod \,} p) \,}
%\newcommand{\mult}{\mbox{\rm mult}}
%\newcommand{\el}{\mbox{\rm el}}
\newcommand{\FOADD}{\mbox{\rm FO} + \exists^\infty + \exists^{\rm mod}}
%\def\true{\mbox{${\rm {\bf true}}$}}
%\def\false{\mbox{${\rm {\bf false}}$}}

\newcommand{\N}{{\mathbb N}}
\newcommand{\Nt}{\mathcal (\N,+,|_2)}
\def\Nn{\mathcal N}
\newcommand{\Np}{{\mathbb N}^{+}}
\newcommand{\Q}{{\mathbb Q}}
\renewcommand{\R}{{\mathbb R}}
\newcommand{\Quan}{{\mathbf Q}}
\newcommand{\Z}{{\mathbb Z}}

\newcommand{\pair}[2]{\langle{#1},{#2}\rangle}
%\renewcommand{\diamond}{{\perp}}
\newcommand{\blank}{{\perp}}
\newcommand{\zz} {{0 \choose 0}}
\newcommand{\oo} {{1 \choose 1}}
\newcommand{\oz} {{1 \choose 0}}
\newcommand{\zo} {{0 \choose 1}}
\newcommand{\zd} {{0 \choose \blank}}
\newcommand{\dz} {{\blank \choose 0}}
\newcommand{\oD} {\mbox{$\pmatrix{1\cr \blank \cr}$}}
\newcommand{\Do} {\mbox{$\pmatrix{\blank \cr 1\cr}$}}

\newcommand{\szz} {\mbox{$ {0 \choose 0} $}}
\newcommand{\soo} {\mbox{$ {1 \choose 1} $}}
\newcommand{\szo} {\mbox{$ {0 \choose 1} $} }
\newcommand{\soz} {\mbox{$ {1 \choose 0} $} }
\newcommand{\szd} {\mbox{$ {0 \choose \diamond} $}}
\newcommand{\sdz} {\mbox{$ {\diamond \choose 0} $}}
\newcommand{\sdo} {\mbox{$ {\diamond \choose 1} $}}
\newcommand{\sdd} {\mbox{$ {\diamond \choose \diamond} $}}
\newcommand{\sod} {\mbox{$ {1 \choose \diamond} $}}
\newcommand{\stack}[2] {\mbox{$ {#1 \choose #2} $} }

\newcommand{\PR}{{\mathbb P}}
\newcommand{\PA}{\mbox{$(\N,+)$}}
\newcommand{\pref}{\prec_{\tt prefix}}
\newcommand{\prefeq}{\preceq_{\tt prefix}}
\newcommand{\subtree}{\preceq_{\tt ext}}
\newcommand{\con}{\otimes}

%\newcommand{\lex}{\preceq_{\tt lex}} %lexeq used in bsl
\newcommand{\lex}{<_{\mathrm{lex}}} %THIS ONE USED FOR DAGSTUHL
\newcommand{\llex}{<_{\text{llex}}} %THIS ONE USED FOR DAGSTUHL

\newcommand{\ext}{\prec_{\tt ext}}
\newcommand{\exteq}{\preceq_{\tt ext}}


%\newcommand{\ll}[1]{ {\, <_{{#1}\text{-llex}}\,} }
%\newcommand{\lleq}[1]{ {\, \leq_{{#1}\text{ -llex}} \,} }
\newcommand{\kllex}{\, <_{k\text{-llex}}\,}
\newcommand{\kllexeq}{\, \leq_{k\text{-llex}}\,}


%\newcommand{\FO}{\mathrm FO}
\newcommand{\existsinfty}{\exists^{\infty}}
\newcommand{\existsmod}{\exists^{\text{mod}}}
\newcommand{\existscount}{\exists^{\leq \aleph_0}}
\newcommand{\existsuncount}{\exists^{> \aleph_0}}

\newcommand{\FOEXT}{\mbox{\rm FO} + \exists^\infty + \exists^{\rm mod} + \existscount + \existsuncount}
\newcommand{\FOinfty}{\FO[\existsinfty]}
\newcommand{\FOmod}{\FO[\existsmod]}
\newcommand{\qreg}{{\mathbf Q}^{\text{reg}}}

\newcommand{\op}{\mathrm{O}}
\newcommand{\one}{\mbox{$(\N,\mathrm{S})$}}
\newcommand{\two}{\mbox{$(\{0,1\}^{\star},\mathrm{s}_0, \mathrm{s}_1)$}}
\newcommand{\twolr}{\mbox{$(\{l,r\}^{\star},\suc_l, \suc_r)$}}
\newcommand{\oracleB}{\mbox{$(\{0,1\}^{\star},\mathrm{s}_0,
\mathrm{s}_1,\tup{B},Y)$}}
\newcommand{\oracle}{\mbox{$(\{0,1\}^{\star},\mathrm{s}_0, \mathrm{s}_1,\op)$}}



\theoremstyle{plain} \numberwithin{equation}{section}
\newtheorem{theorem}{Theorem}[section]
\newtheorem{corollary}[theorem]{Corollary}
\newtheorem{conjecture}{Conjecture}
\newtheorem{lemma}[theorem]{Lemma}
\newtheorem{proposition}[theorem]{Proposition}
\theoremstyle{definition}
\newtheorem{definition}[theorem]{Definition}
\newtheorem{finalremark}[theorem]{Final Remark}
\newtheorem{remark}[theorem]{Remark}
\newtheorem{example}[theorem]{Example}
\newtheorem{question}{Question} \topmargin-2cm

\textwidth6in

\setlength{\topmargin}{0in} \addtolength{\topmargin}{-\headheight}
\addtolength{\topmargin}{-\headsep}

\setlength{\oddsidemargin}{0in}

\oddsidemargin  0.0in \evensidemargin 0.0in 
%\parindent0em
\pagestyle{plain}

\lhead{Research Proposal} \rhead{January 2008}
\chead{{\large{\bf Sasha Rubin}}} \lfoot{} \rfoot{} \cfoot{\thepage}

\begin{document}

%\raisebox{1cm}
\thispagestyle{fancy}

\begin{center}
{\bf Is Kolmogorov complexity a reasonable parameter for parameterised complexity?}
\end{center}

\noindent
{\bf Plan}: I propose to study whether knowing something of the Kolmogorov complexity of a finite graph (or any structure) can be exploited to improve the time or space complexity for algorithms that solve graph-theoretic problems.

A graph with high Kolmogorov complexity (KC) does not have those properties that would allow it to be compressed. This information could be exploited in graph algorithms. For instance, a graph on $n$ vertices and with high KC can not have a clique of size $k > log n$ (for otherwise instead of listing the $k^2$ edges of the clique, one could simply list the vertices using only $k \log n$ bits). 
%This information can be used to save some time in finding the largest clique of the graph.

This leads to our first question:
\begin{question}
Which graph properties can (or alternatively can not) be used to give shorter descriptions of a graph than simply listing all its edges?
\end{question}

Is there a logical characterisation of such properties?
Is their a relationship to properties that hold in almost all graphs (0-1 laws)? 
%Intuitively, if a property fails to hold in almost all graphs, then it holds in almost all graphs with high KC.

On the other hand, a graph has low KC ostensibly because it does have certain properties that allow it to be compressed.

\begin{question}
Which properties can be tested efficiently on graphs with low KC?
\end{question}

Again, is there a logical characterisation of such properties?

[\begin{question}
Fix the collection $\C$ of strings whose KC is within some range. Which properties can be efficiently tested on strings from $\C$, and which properties can not? 
\end{question}

%For instance consider strings $w$ such that $KC(w|l(w)) \leq c$ for fixed $c$.

General question 2: What is the relationship between structural properties of an object and its KC? ]

\noindent
{\bf Background}: Kolmogorov complexity (KC) is a measure of information content of a finite string \cite{Solo64, Kolm65, Chai69}. Roughly, the information content of a finite string is the size, measured in bits, of the shortest program that produces it. This notion can be extended to other objects, such as infinite strings and graphs. Intuitively, an object with low KC has regularity that can be exploited to give it a short description. KC has found applications in proving lower bounds in complexity theory, average case analysis of algorithms, formal language theory and random graphs.

Parameterised complexity \cite{DoFe99} is a framework that classifies problems by the amount of resources (time, space) required by algorithms solving them. It measures complexity in terms of input size {\it and} a parameter. This is usually motivated by considering cases when the parameter is comparatively smaller than the input size. For instance in the realm of databases, a query produced by a human is typically very small compared to the size of the database. The central complexity class in this area is fixed-parameter tractability which consists of those problems whose complexity can be split into a polynomial - that depends on the size of the input -  and some arbitrary function - that depends on the parameter.

Our starting point are theorems of the following sort: Certain algorithmic problems are feasible on certain subclasses of graphs. For instance, 
%if parameterised by some structural information of the input graph. For instance $k$-dominating set is fixed-parameter tractable for the class of planar graphs, while it is unlikely
%that this is the case for the class of all graphs. 
%find better example. simpler (no k) and intractable in general case.
The typical subclasses of graphs (trees, graphs of finite tree-width, \ldots)

This leads to the first intuition: algorithmic problems on graphs with low KC can be solved efficiently. To test this intuition, I propose investigating the following questions:
\begin{enumerate}
    \item How low is the KC of members of naturally occuring classes of graphs (trees, finite tree-width, \ldots)?
    \item Is knowing that a graph have low KC enough? or is the value also required?
\end{enumerate}
    
On the other hand, a graph with high KC does not have certain properties that would allow a shorter description (for instance it would not have a large clique). Thus, simply knowing a graph has high KC, could be exploited to save time/space (for instance in finding the largest clique in the graph).

Thus leads to the second intuition: algorithmic problems on graphs with high KC can be solved faster than the worst case. 

\begin{enumerate}
\item Which naturally occurring graph properties do not occur in graphs with high KC? Can they be characterised logically?
\end{enumerate}

Also, it might be the case that one cannot reasonably compare arbitrary properties with arbitrary classes of graphs within the framework of KC. For instance, Courcelle's theorem states that all problems definable in monadic-second order logic are fixed-parameter tractable if parameterised by the tree width
of the input graph. 

\noindent
{\bf Related work}: 
\begin{itemize}
\item random input analysis. phase transitions.
\item 0-1 laws
\end{itemize}



I propose to analyse under what circumstances

\end{document}