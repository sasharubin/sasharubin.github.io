\documentclass[a4paper]{article}
\usepackage{fullpage}
\usepackage{multicol}
%\usepackage{times}
%\textwidth  = 16.0cm
%\textheight = 22.0cm
%\voffset    = -1.0cm
%\hoffset    = -1.5cm

\setlength{\parindent}{0em}
\pagestyle{empty}
%\pagenumbering{roman}

\def\tit#1{\medskip \begin{center}  {\Large {\bf -- #1 -- }} \end{center}}

\begin{document}

\begin{center}
{\Huge \bf Sasha Rubin}
\end{center}

\begin{center}
{\huge \it Recent Publications} \\
\vspace{1em}
{\large March 2011}
\end {center}


%\section*{\center{Personal Data}}
\begin{enumerate}

\item {\it Automata based presentations of infinite structures} with V. B{\'a}r{\'a}ny and E. Gr{\"a}del,
in {\it Finite and Algorithmic Model Theory},
Series: London Mathematical Society Lecture Note Series (No. 379), $2011$.\\

\item {\it Cardinality and counting quantifiers on omega-automatic structures}, with V.  B{\'a}r{\'a}ny and \L. Kaiser, 
$25$th Annual Symposium on Theoretical Aspects of Computer Science, $2008$.  \\

\item {\it Order invariant MSO is stronger than counting MSO}, with T. Ganzow, 
$25$th Annual Symposium on Theoretical Aspects of Computer Science, $2008$.  \\
%International Workshop on Logic and Computational Complexity $2007$.\\

\item {\it Automata presenting structures: A survey of the finite-string case}, The Bulletin of Symbolic Logic, 
Volume 14, Issue 2, 2008, 169-209.\\

\item {\it Automatic Structures: Richness and Limitations}, with B. Khoussainov, A. Nies and F. Stephan, 
Logical Methods in Computer Science, Vol $3$, $2007$. Appeared in Proceedings of 
$19$th Annual {IEEE} Symposium on Logic in Computer Science, $44-53$, $2004$. \\

\end{enumerate}
\end{document}




{\it Automatic Structures: Richness and Limitations}, with B. Khoussainov, A. Nies and F. Stephan, 
Logical Methods in Computer Science, Vol $3$, $2007$. Appeared in Proceedings of 
$19$th Annual {IEEE} Symposium on Logic in Computer Science, $44-53$, $2004$. \\



%{\it Automatic Linear Orders and Trees: Revised}, CDMTCS Technical Report $208$,
%Department of Computer Science, University of Auckland, $2003$. \\
 
{\it Definability and Regularity in Automatic Structures}, with B. Khoussainov
and F. Stephan, $21$st Annual Symposium on Theoretical Aspects of 
Computer Science, $440-451, 2004$.  \\
%{\it Definability and Regularity in Automatic Presentations of Subsystems of
%Arithmetic}, CDMTCS Technical Report $209$,
%Department of Computer Science, University of Auckland, $2003$. \\
 
{\it Automatic Structures - Overview and Future Directions}, with 
B. Khoussainov,
Journal of Automata, Languages and Combinatorics, $8(2), 287-301, 2003$. \\

{\it Some Results on Automatic Structures}, with B. Khoussainov
and H. Ishihara, $17$th Annual IEEE Symposium on Logic in Computer Science,
$2002$. \\

{\it Graphs with Automatic Presentations over a Unary Alphabet}
Journal of Automata, Languages and Combinatorics, $6(4), 467-480, 2001$. \\

{\it Finite Automata and Well Ordered Sets},
New Zealand Journal of Computing, $7(2), 39-46, 1999$. 

%\subsubsection*{In preparation}
%
%Chapter on Automatic Structures in Handbook {\it Automata: From mathematics to
%applications}, to be published by EMS and AutomathA network.\\
% 
%{\it Regularity preserving quantifiers}, with V. Goranko and M. Vardi.\\
%
%%{\it Automatic linear orders} with V. Goranko and T. Knapik. \\
%
%{\it Automata based presentations of infinite structures} with V. B{\'a}r{\'a}ny and E. Gr{\"a}del.\\
%
%{\it Myhill-Nerode with oracle} with A. Kruckman, J. Sheridan, B. Zax.\\
%
%{\it Trap depth of parity games} with P. Phalitnonkiat, A. Grinshpun, A.Tarfulea.\\

\end{document}


\columnbreak
{\bf Dr. Frank Stephan}\\
School of Computing\\
National University of Singpore\\
3 Science Drive 2, Singapore 117543\\
fstephan@comp.nus.edu.sg\\
Phone :  +$65$ $6516$ $2759$
%Fax   :  +$65$ $7795$ $5452$\\


%----------------------------------------------------
%----------------------------------------------------
%----------------------------------------------------
%----------------------------------------------------
%----------------------------------------------------
%----------------------------------------------------


\iffalse
{\bf Dr. Valentin Goranko}\\
School of Mathematics\\
University of Witwatersrand\\ 
Private Bag 3, WITS 2050\\
Johannesburg, South Africa\\
goranko@maths.wits.ac.za\\
Phone : +$27$ $11$ $717$ $6243$ \\
%Fax   : +$27$ $11$ $717$ $6259$ \\
\fi
%- Tutorial co-ordinator for undergraduate paper `Discrete Mathematics', %1999 \\
%- Project and summer scholarship supervisor for `Application of Elementary \\
%\phantom{- }Submodels to Topology', 1998\\ 
%- Lecturer for `Logic and Set Theory', 1998\\ 
% \enlargethispage*{1cm}

%\section*{Conferences Attended}
%\begin{tabular}{@{}ll}
% {\bf 2000} & Computational Group Theory, Sydney\\
% {\bf 1999} & Third New Zealand Computer Science Research Students'
%Conference \\
%	    & April, Waikato\\
% {\bf 1999} & ACSC - DMTCS/CATS \\
%	    & January, Auckland \\
% {\bf 1999} & NZMRI summer workshop, Harmonic Analysis\\
%	    & January, Raglan, New Zealand\\
% {\bf 1998} & Second Japan-New Zealand Workshop on 
%		{\em Logic in Computer Science}\\
%	    & October, Auckland\\
% {\bf 1998} & First International Conference on 
%		{\em Unconventional Models of Computation}\\
%	    & January, Auckland \\
% {\bf 1997} & First Japan-New Zealand Workshop on {\em Logic in Computer Science}\\
%	    & August, Auckland\\
% {\bf 1997} & Fifth Australasian Mathematics Convention\\
%	    & July, Auckland\\
%\end{tabular}


\iffalse
\subsection*{Additional}
{\bf Dr. David McIntyre}\\
Department of Mathematics\\
University of Auckland, New Zealand\\
mcintyre@math.auckland.ac.nz\\
Phone : (+64 9) 373 7599 Ext 8763\\
%- Tutorial co-ordinator for undergraduate paper `Discrete Mathematics',
%1999 \\
%- Project and summer scholarship supervisor for `Application of Elementary \\
%\phantom{- }Submodels to Topology', 1998\\
%- Lecturer for `Logic and Set Theory', 1998\\
\enlargethispage*{1cm}


%\pagebreak
\section*{Non Academic}
\subsection*{University of Auckland}
{\it Membership}\\
  Auckland University Dramatic Society (1998/1999/2000)\\
  Auckland University Comedy and Improvisation Club (1997) Secretary (1998)\\


{\it Performance}\\
  `Morte Accidentale Di Un Anarchico', (Dario Fo) (1999)\\
  Celebration of Performing Arts, Auckland Town Hall (Stoppard, Simon) (1999)\\
  Three comedy shows, Auckland University (1997 and 1998)\\
  A duet in an evening of short plays, Auckland University (Stoppard) (1998)\\
  Cultural Mosaic Festival, Auckland University (Stoppard) (1997)\\
\begin{center}
--------------------
\end{center}
\fi


\section*{Seminars/Talks} 

 
`Techniques to prove non-automaticity', University of Heidelberg Logic Seminar, 2002 \\


`Some Results on Automatic Structures', with Bakhadyr Khoussainov
and Hajime Ishihara, 17th Annual IEEE Symposium on Logic in Computer Science, 2002. \\


`Automata-theoretic approach to verification of probabilistic systems',
Rice University Computer Science Theory Seminar, 2001. \\


`Automatic Structures', University of Notre Dame Logic Seminar, 2001, and \\
University of Madison, Wisconsin, Logic Seminar, 2001.\\


`Finite Automata and Relational Structures', with Bakhadyr Khoussainov, 
DCAGRS, July 2000, London, Ontario \\


 `Finite Automata and Well Ordered Sets', 
3rd New Zealand Computer Science Research Conference, 1999, Waikato, New
Zealand \\


{\em Auckland Department of Computer Science:}\\
 `Finite Model Theory - Ehrenfeucht-Fraisse Theorem', 2000\\
 `Extracting Algebraic Information from Finite State Machines', 1999\\
 `Finite Automata and Regular Languages', 1999\\


{\em Auckland Department of Mathematics:}\\
 `Algebraic Structures and Finite Automata', 1999\\
 `Applications of Elementary Submodels to Topology', 1999\\ 
\fi

\iffalse
		
  {\bf Marking -} {\it University of Auckland} \\
	 $1998$: Stage $3$ Assignments for the Department of Mathematics\\
  	 1997: Stage 1 Assignments for the Departments of 
	 Computer Science and Mathematics\\
\fi
\iffalse
