\documentclass[a4paper]{article}
\usepackage{fullpage}
\usepackage{multicol}
%\usepackage{times}
%\textwidth  = 16.0cm
%\textheight = 22.0cm
%\voffset    = -1.0cm
%\hoffset    = -1.5cm

\setlength{\parindent}{0em}
\pagestyle{empty}
%\pagenumbering{roman}

\def\tit#1{\medskip \begin{center}  {\Large {\bf -- #1 -- }} \end{center}}

\begin{document}

\begin{center}
{\Huge \bf Sasha Rubin}
\end{center}

\begin{center}
{\huge \it Curriculum Vitae} \\
\vspace{1em}
{\large December 2010}
\end {center}


%\section*{\center{Personal Data}}
\tit{Personal Data}
\begin{tabular}{@{}ll}

  \bf Date of Birth 
		 & $16$ February $1976$\\
  \bf Citizenship & New Zealand \\
  \bf Language   & English\\
  \bf Residence & $2/15$ Hororata Rd., Takapuna \\
  		 & Auckland, New Zealand\\
  \bf & Phone 	  +$64$ $9$ $486$ $1196$\\
   & Email   sasha.rubin at gmail.com
 %  & Web  www.math.cornell.edu / $\sim$ srubin
\end{tabular}

\tit{University Education}

\subsection*{Postdoctoral}

{$2.2010$--$5.2010$: Visiting Researcher and Lecturer, Department of Mathematics, University of Cape Town.}\\

{$08.2008$--$12.2009$: Visiting Assistant Professor, Department of Mathematics, Cornell University.} \\

{$12.2005$--$02.2008$: Honorary Research Fellow in the Department of Computer
Science, University of Auckland. Supported by a New Zealand Science and Technology Postdoctoral
Fellowship $\textrm{UOAX}0413$.}

\subsection*{{PhD -- Mathematics and Computer Science}}
{$1999$ -- $2004$: University of Auckland, New Zealand.\\
Supervisor: Bakhadyr Khoussainov\\
Thesis Title: Automatic Structures}

\subsection*{{MSc -- Mathematics (first class)}}
{$1997$ -- $1998$: University of Auckland, New Zealand.}
   %March $1997$ - November $1998$ \\
   %Pure Mathematics with Theoretical Computer Science (first class) 

\subsection*{BSc -- Mathematics and Computer Science}
{$1994$ -- $1996$: University of Cape Town, South Africa.} 
  %\bf March $1994$ - November $1996$\\
  %Majored in Pure Mathematics and Computer Science\\

\tit{Summary of Research Interests}

My main interest is in the expressiveness of abstract models of computation such as
finite automata. The bulk of my work has been theoretical and related to logic (automatic structures,
finite model theory). I have also worked on  automatically verifying properties of probabilistic systems.
%My main interest is in the expressiveness and limitations of automata in
%relation to logical systems and structures. In particular I have worked in the
%area of automatic structures with a focus on %model theoretic questions and
%classifying classes of automatic structures up to isomorphism.  I have also
%worked on the automata-theoretic approach to verification, and in finite model
%theory.
%in particular verification of probabilistic finite state systems.

\tit{Research Activities}

\subsection*{Recent Seminar Talks} Cape Town ($2010$), Cornell ($2007,2008,2009$), LSV Cachan ($2008$), CNRS LIAFA Paris 7 ($2007$), Heidelberg ($2007$)\\

\subsection*{Invited Talks}
Working group reunion at LABRI Bordeaux ($06.2008$), Dagstuhl workshop $07441$ `Algorithmic-Logical Theory of Infinite Structures' ($10.2007$), Workshop on Logic and Algorithms
satellite meeting `Finite and Algorithmic Model Theory' in Durham ($01.2006$), Workshop on Automata, Structures and Logic in Auckland ($12.2004$).

%\subsection*{Conference Talks} CiE ($2008$),

%$2008 - 2009$: Logic Seminar, Cornell Talks in the Logic Seminar at Cornell. \\
%Ongoing work with Anil Nerode and Dexter Kozen.\\

%$2.2007$: Attended the 'Model Theory and Computable Model Theory' workshop, part
%of the University of Florida's Special Year in Logic.\\

\subsection*{Research visits} Erich Gr\"adel and students at RWTH Aachen ($08.2006-01.2007$).\\

Valentin Goranko at the University of Witwatersrand ($04.2006$).\\

Workshop on Logic and Algorithms at Newton Institute ($01.2006 - 02.2006$).\\

Anil Nerode and students at Cornell University ($10.2005$).\\

Frank Stephan at University of Heidelberg (part of $2003$).\\

Moshe Vardi at Rice University (part of $2001$).\\

Stephan Lempp at University of Wisconsin-Madison (part of $2000$).

\subsection*{Refereed}
{Journal of Symbolic Logic}, {Information and Computation}, {Journal of Logic and Computation}, {IEEE Symposium on Logic in Computer Science}, {Central European Journal of Mathematics}, {Logical methods in Computer Science}, {IARCS Annual Conference on
Foundations of Software Technology and Theoretical Computer Science}, {Theory and Practice of Logic Programming}.

\subsection*{Publications}
{\it Automatic Structures} in {\it Automata: from Mathematics to Applications}, to be published by the European Mathematical Society.\\

{\it Alternating traps in Parity Games} with P. Phalitnonkiat, A. Grinshpun, A.Tarfulea, {\it submitted $2010$}.\\

{\it Automata based presentations of infinite structures} with V. B{\'a}r{\'a}ny and E. Gr{\"a}del,
in {\it Finite and Algorithmic Model Theory},
Series: London Mathematical Society Lecture Note Series (No. 379), $2011$.\\

{\it Cardinality and counting quantifiers on omega-automatic structures}, with V.  B{\'a}r{\'a}ny and \L. Kaiser, 
$25$th Annual Symposium on Theoretical Aspects of Computer Science, $2008$.  \\

{\it Order invariant MSO is stronger than counting MSO}, with T. Ganzow, 
$25$th Annual Symposium on Theoretical Aspects of Computer Science, $2008$.  \\
%International Workshop on Logic and Computational Complexity $2007$.\\

{\it Automata presenting structures: A survey of the finite-string case}, The Bulletin of Symbolic Logic, 
Volume 14, Issue 2, 2008, 169-209.\\

{\it Automatic Structures: Richness and Limitations}, with B. Khoussainov, A. Nies and F. Stephan, 
Logical Methods in Computer Science, Vol $3$, $2007$. Appeared in Proceedings of 
$19$th Annual {IEEE} Symposium on Logic in Computer Science, $44-53$, $2004$. \\

{\it Automatic linear orders and trees}, with B. Khoussainov and F. Stephan, 
ACM Transactions on Computational Logic,
$6 (4), 675-700, 2005$. Appeared in 
Proceedings of $18$th Annual IEEE Symposium on Logic in Computer Science, $2003$,
as {\it Automatic Partial Orders}.  \\

{\it Verifying $\omega$-regular Properties of Markov Chains}, with D. Bustan and
M.Y.~Vardi, $16$th International conference on Computer Aided Verification,
$189-201, 2004.$ \\

%{\it Automatic Linear Orders and Trees: Revised}, CDMTCS Technical Report $208$,
%Department of Computer Science, University of Auckland, $2003$. \\
 
{\it Definability and Regularity in Automatic Structures}, with B. Khoussainov
and F. Stephan, $21$st Annual Symposium on Theoretical Aspects of 
Computer Science, $440-451, 2004$.  \\
%{\it Definability and Regularity in Automatic Presentations of Subsystems of
%Arithmetic}, CDMTCS Technical Report $209$,
%Department of Computer Science, University of Auckland, $2003$. \\
 
{\it Automatic Structures - Overview and Future Directions}, with 
B. Khoussainov,
Journal of Automata, Languages and Combinatorics, $8(2), 287-301, 2003$. \\

{\it Some Results on Automatic Structures}, with B. Khoussainov
and H. Ishihara, $17$th Annual IEEE Symposium on Logic in Computer Science,
$2002$. \\

{\it Graphs with Automatic Presentations over a Unary Alphabet}
Journal of Automata, Languages and Combinatorics, $6(4), 467-480, 2001$. \\

{\it Finite Automata and Well Ordered Sets},
New Zealand Journal of Computing, $7(2), 39-46, 1999$. 

%
%\subsubsection*{In preparation}
%
%Chapter on Automatic Structures in Handbook {\it Automata: From mathematics to
%applications}, to be published by EMS and AutomathA network.\\
% 
%{\it Regularity preserving quantifiers}, with V. Goranko and M. Vardi.\\
%
%%{\it Automatic linear orders} with V. Goranko and T. Knapik. \\
%
%{\it Myhill-Nerode with oracle} with A. Kruckman, J. Sheridan, B. Zax.\\
%
%{\it Trap depth of parity games} with P. Phalitnonkiat, A. Grinshpun, A.Tarfulea.\\

\iffalse

{\it Finite Automata and Algebraic Structures},
with Bakhadyr Khoussainov, 
Abstracts of Symposium on Logic and Applications, Novosibirsk, May $2000$. \\

{\it On Automata Presentable Structures},
with Bakhadyr Khoussainov, 
Abstracts of papers presented to the American Mathematical Society, 
$20(2), 1999$. \\ 
\fi

%\section*{\center{Selected Talks}}

%\pagebreak

\tit{Awards}

{\em Foundation for Research, Science and Technology}\\ 
$12.2005-2.2008$: New Zealand Science and Technology Postdoctoral Fellowship.\\ 

{\em University of Auckland}\\
%$2004$: Vice-chancellor's prize for the best doctoral thesis in the Faculty of Science.
$2004$: Best doctoral thesis award.

$2004$: Montgomery memorial prize in logic from the Department of Philosophy.

%$2004$: Thesis was nominated for Sack's Prize.

%{\em University of Auckland}\\
%$2000 - 2003$: PhD Scholarship from the Departments of Computer Science and 
%Mathematics and University Graduate Research Grant for travel.\\

%
%$2000$, $2002$: University Graduate Research Grant for travel.\\

\iffalse

$1998$ and $1999$: Summer Scholarships from the Mathematics Department
 for vacational study and research.\\
\fi


$1998$: Competed as part of a team of three, in the world finals of the $1998$
ACM Programming Contest in Atlanta, Georgia USA, representing the University of Auckland
and New Zealand. 
%The same team won the Regional Programming Contest in $1997$.\\ 

\tit{Supervision/Teaching}

{\bf Supervision}\\
$2009$: Summer research experience for undergraduates (REU)\\
{\em  Cornell University, Department of Mathematics}\\
Topic 1: Parity games.\\
Topic 2: Automatic Structures in the presence of advice.\\

 {\bf Teaching}\\
{\it University of Cape Town, Department of Mathematics}\\
$2010$: Logic and computation, Stage $3$.\\

 {\it Cornell University, Department of Mathematics}\\
 Fall $2009$: Logical definability and Random graphs, Graduate course.\\ 
 $2008/2009$: Calculus for Engineers, Stage $1$.\\

 {\it University of Auckland, Department of Computer Science}\\
 $2007$: Discrete Structures in Mathematics and Computer Science, Stage $2$.\\
 Mathematical Foundations of Software Engineering, Stage $2$.\\

{\it $18$th European Summer School in Logic, Language and Information, University of Malaga, $2006$}\\
 Five day advanced course `Logic and computation in finitely presentable infinite structures'
 with V. Goranko.\\

 {\it University of Auckland, Department of Computer Science} \\
 $2003$: Introduction to Formal Verification, Stage $4$.\\
 $2002$: Automata Theory, Stage $3$.\\

 {\it University of Wisconsin, Madison, Department of Mathematics} \\
 $2001$: Pre-calculus\\
 

  {\bf Tutoring}\\
  {\it University of Auckland, Department of Mathematics} \\
  $1998$ and $1999$: Undergraduate Mathematics\\

\iffalse $1998$ and $1999$: Stages $1$, $2$ and $3$ in the Mathematics \\
   	 %Assistance Room \\
	 $1999$: Assistant tutor for `Discrete Mathematics', Stage $2$
	Mathematics\\
	 $1999$: Demonstrator for `Combinatorial Computing', Stage $3$
	Mathematics\\
\fi
\tit{Academic References}

%\begin{multicols}{2}
%\subsection*{Primary}

{\bf Bakhadyr Khoussainov}\\
Department of Computer Science\\
University of Auckland, New Zealand\\
bmk@cs.auckland.ac.nz\\
Phone : +$64$ $9$ $373$ $7599$ Ext $85120$\\
%Fax   : +$64$ $9$ $373$ $7453$\\
%- PhD supervisor\\
%- Project supervisor for `Finite Automaton Presentable Unary Structures', 1998

%\columnbreak
{\bf Moshe Y. Vardi}\\
Department of Computer Science\\
Rice University, Houston, USA\\
vardi@cs.rice.edu\\
Phone : +$1$ $713$ $348$ $5977$\\
%Fax   : +$1$ $713$ $348$ $5930$\\

{\bf Erich Gr\"adel}\\
Mathematische Grundlagen der Informatik\\
RWTH Aachen\\
D-52056 Aachen\\
graedel@logic.rwth-aachen.de\\
Phone :	+$49$ $241$ $80$ $21730$\\
%\columnbreak
%{\bf Dr. Frank Stephan}\\
%School of Computing\\
%National University of Singpore\\
%3 Science Drive 2, Singapore 117543\\
%fstephan@comp.nus.edu.sg\\
%Phone :  +$65$ $6516$ $2759$\\
%Fax   :  +$65$ $7795$ $5452$\\

{\bf Anil Nerode}\\
Goldwin Smith Professor of Mathematics\\
545 Malott Hall\\
Cornell University\\
	 Ithaca, NY 14853\\
	 anil@math.cornell.edu\\
Phone: +$1$  $607$ $255$ $3577$\\

%\columnbreak
\tit{Teaching References}

{\bf Maria Terrell}\\
 Director of Teaching Assistant Programs\\
 	 225 Malott Hall \\
	 Cornell University\\
	 Ithaca, NY 14853\\
	 maria@math.cornell.edu\\
Phone: +$1$  $607$ $255$ $3905$\\


{\bf David Way}\\
Director of Instructional Support
Center for Teaching Excellence\\
420 D CCC\\
Cornell University\\
Ithaca, NY 14853\\
dgw2@cornell.edu\\
Phone: +$1$ $607$ $255$ $2663$\\
%www.cte.cornell.edu\\
%fax: (607) 255 1562\\
%\end{multicols}
\end{document}


\columnbreak
{\bf Dr. Frank Stephan}\\
School of Computing\\
National University of Singpore\\
3 Science Drive 2, Singapore 117543\\
fstephan@comp.nus.edu.sg\\
Phone :  +$65$ $6516$ $2759$
%Fax   :  +$65$ $7795$ $5452$\\


%----------------------------------------------------
%----------------------------------------------------
%----------------------------------------------------
%----------------------------------------------------
%----------------------------------------------------
%----------------------------------------------------


\iffalse
{\bf Dr. Valentin Goranko}\\
School of Mathematics\\
University of Witwatersrand\\ 
Private Bag 3, WITS 2050\\
Johannesburg, South Africa\\
goranko@maths.wits.ac.za\\
Phone : +$27$ $11$ $717$ $6243$ \\
%Fax   : +$27$ $11$ $717$ $6259$ \\
\fi
%- Tutorial co-ordinator for undergraduate paper `Discrete Mathematics', %1999 \\
%- Project and summer scholarship supervisor for `Application of Elementary \\
%\phantom{- }Submodels to Topology', 1998\\ 
%- Lecturer for `Logic and Set Theory', 1998\\ 
% \enlargethispage*{1cm}

%\section*{Conferences Attended}
%\begin{tabular}{@{}ll}
% {\bf 2000} & Computational Group Theory, Sydney\\
% {\bf 1999} & Third New Zealand Computer Science Research Students'
%Conference \\
%	    & April, Waikato\\
% {\bf 1999} & ACSC - DMTCS/CATS \\
%	    & January, Auckland \\
% {\bf 1999} & NZMRI summer workshop, Harmonic Analysis\\
%	    & January, Raglan, New Zealand\\
% {\bf 1998} & Second Japan-New Zealand Workshop on 
%		{\em Logic in Computer Science}\\
%	    & October, Auckland\\
% {\bf 1998} & First International Conference on 
%		{\em Unconventional Models of Computation}\\
%	    & January, Auckland \\
% {\bf 1997} & First Japan-New Zealand Workshop on {\em Logic in Computer Science}\\
%	    & August, Auckland\\
% {\bf 1997} & Fifth Australasian Mathematics Convention\\
%	    & July, Auckland\\
%\end{tabular}


\iffalse
\subsection*{Additional}
{\bf Dr. David McIntyre}\\
Department of Mathematics\\
University of Auckland, New Zealand\\
mcintyre@math.auckland.ac.nz\\
Phone : (+64 9) 373 7599 Ext 8763\\
%- Tutorial co-ordinator for undergraduate paper `Discrete Mathematics',
%1999 \\
%- Project and summer scholarship supervisor for `Application of Elementary \\
%\phantom{- }Submodels to Topology', 1998\\
%- Lecturer for `Logic and Set Theory', 1998\\
\enlargethispage*{1cm}


%\pagebreak
\section*{Non Academic}
\subsection*{University of Auckland}
{\it Membership}\\
  Auckland University Dramatic Society (1998/1999/2000)\\
  Auckland University Comedy and Improvisation Club (1997) Secretary (1998)\\


{\it Performance}\\
  `Morte Accidentale Di Un Anarchico', (Dario Fo) (1999)\\
  Celebration of Performing Arts, Auckland Town Hall (Stoppard, Simon) (1999)\\
  Three comedy shows, Auckland University (1997 and 1998)\\
  A duet in an evening of short plays, Auckland University (Stoppard) (1998)\\
  Cultural Mosaic Festival, Auckland University (Stoppard) (1997)\\
\begin{center}
--------------------
\end{center}
\fi


\section*{Seminars/Talks} 

 
`Techniques to prove non-automaticity', University of Heidelberg Logic Seminar, 2002 \\


`Some Results on Automatic Structures', with Bakhadyr Khoussainov
and Hajime Ishihara, 17th Annual IEEE Symposium on Logic in Computer Science, 2002. \\


`Automata-theoretic approach to verification of probabilistic systems',
Rice University Computer Science Theory Seminar, 2001. \\


`Automatic Structures', University of Notre Dame Logic Seminar, 2001, and \\
University of Madison, Wisconsin, Logic Seminar, 2001.\\


`Finite Automata and Relational Structures', with Bakhadyr Khoussainov, 
DCAGRS, July 2000, London, Ontario \\


 `Finite Automata and Well Ordered Sets', 
3rd New Zealand Computer Science Research Conference, 1999, Waikato, New
Zealand \\


{\em Auckland Department of Computer Science:}\\
 `Finite Model Theory - Ehrenfeucht-Fraisse Theorem', 2000\\
 `Extracting Algebraic Information from Finite State Machines', 1999\\
 `Finite Automata and Regular Languages', 1999\\


{\em Auckland Department of Mathematics:}\\
 `Algebraic Structures and Finite Automata', 1999\\
 `Applications of Elementary Submodels to Topology', 1999\\ 
\fi

\iffalse
		
  {\bf Marking -} {\it University of Auckland} \\
	 $1998$: Stage $3$ Assignments for the Department of Mathematics\\
  	 1997: Stage 1 Assignments for the Departments of 
	 Computer Science and Mathematics\\
\fi
\iffalse
