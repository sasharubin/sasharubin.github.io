% LaTeX resume using res.cls
\documentclass{res}
%\usepackage{helvetica} % uses helvetica postscript font (download helvetica.sty)
%\usepackage{newcent}   % uses new century schoolbook postscript font  
\setlength{\topmargin}{-0.6in}  % Start text higher on the page 
\setlength{\textheight}{9.8in}  % increase textheight to fit more on a page
\setlength{\headsep}{0.2in}     % space between header and text
\setlength{\headheight}{12pt}   % make room for header
\usepackage{fancyhdr}  % use fancyhdr package to get 2-line header
\renewcommand{\headrulewidth}{0pt} % suppress line drawn by default by fancyhdr
\lhead{\hspace*{-\sectionwidth}Sasha Rubin} % force lhead all the way left
\rhead{Page \thepage}  % put page number at right
\cfoot{}  % the footer is empty
\pagestyle{fancy} % set pagestyle for the document

\begin{document} 
\thispagestyle{empty} % this page does not have a header
\name{SASHA RUBIN}
\address{Am Campus 1\\
Klosterneuburg 3400\\
Austria}


\begin{resume}
\vspace{0.1in}
\moveleft.5\sectionwidth\centerline{Objective: Position as research analyst.}  

\section{EDUCATION}
\vspace{0.1in} 
  
   University of Auckland, New Zealand\\
   Ph.D., Computer Science and Mathematics, 2004
    
   University of Cape Town, South Africa\\
   B.Sc., Computer Science and Mathematics, 1998
   
%{$3.2012$ -- present: Postdoctoral Researcher, }\\
%
%{$5.2011$ -- $8.2011$: Visiting Researcher, Department of Computer Science, Tel Aviv University.}\\
%


\section{RESEARCH INTERESTS}

My mathematical work studies the expressiveness and limitations of automata theory and mathematical logic for describing
computation.
%Concretely, I have worked in the
%area of automatic structures, the automata-theoretic approach to verification, and in finite model
%theory. I am currently working on the analysis of distributed systems and algorithms using logical and automata-theoretic methods.

 
 
\section{SPECIAL SKILLS} 
\vspace{0.1in}
  {\bf Writing}
    \begin{itemize} % Use \item[] to prevent a bullet from appearing
      \item[] Published academic papers: surveys, journal articles, conference proceedings; refereed academic papers.
      \end{itemize}

  {\bf Teaching and public speaking}
        \begin{itemize}
        \item[] Taught courses, gave seminars and invited talks in mathematics and computer science at a variety of levels: highschool,  undergraduate, postgraduate.
        
        \end{itemize}
 
\section{EMPLOYMENT} 
\vspace{0.1in} 

{\bf Postdoctoral Researcher,}  IST Austria and TU Vienna, Austria, 3.2012 --

   {\bf Visiting Researcher,} Tel Aviv University, 5.2011 -- 8.2011 
   
  {\bf Visiting Lecturer,} University of Cape Town, Department of Mathematics, $2.2010$ -- $5.2010$

{\bf   Visiting Assistant Professor,} Cornell University, Department of Mathematics,
 $08.2008$ -- $12.2009$

{\bf Postdoctoral Fellow,} University of Auckland,
 Department of Computer Science,
  supported by a New Zealand Science and Technology Postdoctoral
 Fellowship,
 $12.2005$ -- $02.2008$\\
 
\section{PUBLICATIONS} 
\vspace{0.1in}

{\it Automatic Structures} in {\it Automata: From mathematics to applications}, to be published by EMS and AutomathA network.

{\it A Myhill-Nerode Theorem for Automata with Advice} with A. Kruckman, J. Sheridan and B. Zax, {\it GandALF 2012}.

{\it Interpretations in trees with countably many branches} with A. Rabinovich,  appeared in {\it Proceedings of 
$27$th Annual {IEEE} Symposium on Logic in Computer Science, $551-560$, $2012$}. 

{\it Alternating Traps in Parity Games} with P. Phalitnonkiat, A. Grinshpun, A.Tarfulea, {\it accepted to {\em Theoretical Computer Science}}.

{\it Automata based presentations of infinite structures} with V. B{\'a}r{\'a}ny and E. Gr{\"a}del,
in {\it Finite and Algorithmic Model Theory},
Series: London Mathematical Society Lecture Note Series (No. 379), $2011$.

{\it Cardinality and counting quantifiers on omega-automatic structures}, with V.  B{\'a}r{\'a}ny and \L. Kaiser, 
$25$th Annual Symposium on Theoretical Aspects of Computer Science, $2008$.  

{\it Order invariant MSO is stronger than counting MSO}, with T. Ganzow, 
$25$th Annual Symposium on Theoretical Aspects of Computer Science, $2008$.  
%International Workshop on Logic and Computational Complexity $2007$.

{\it Automata presenting structures: A survey of the finite-string case}, The Bulletin of Symbolic Logic, 
Volume 14, Issue 2, 2008, 169-209.

{\it Automatic Structures: Richness and Limitations}, with B. Khoussainov, A. Nies and F. Stephan, 
Logical Methods in Computer Science, Vol $3$, $2007$. Appeared in Proceedings of 
$19$th Annual {IEEE} Symposium on Logic in Computer Science, $44-53$, $2004$. 

{\it Automatic linear orders and trees}, with B. Khoussainov and F. Stephan, 
ACM Transactions on Computational Logic,
$6 (4), 675-700, 2005$. Appeared in 
Proceedings of $18$th Annual IEEE Symposium on Logic in Computer Science, $2003$,
as {\it Automatic Partial Orders}.  

{\it Verifying $\omega$-regular Properties of Markov Chains}, with D. Bustan and
M.Y.~Vardi, $16$th International conference on Computer Aided Verification,
$189-201, 2004.$ 

%{\it Automatic Linear Orders and Trees: Revised}, CDMTCS Technical Report $208$,
%Department of Computer Science, University of Auckland, $2003$. 
 
{\it Definability and Regularity in Automatic Structures}, with B. Khoussainov
and F. Stephan, $21$st Annual Symposium on Theoretical Aspects of 
Computer Science, $440-451, 2004$.  
%{\it Definability and Regularity in Automatic Presentations of Subsystems of
%Arithmetic}, CDMTCS Technical Report $209$,
%Department of Computer Science, University of Auckland, $2003$. 
 
{\it Automatic Structures - Overview and Future Directions}, with 
B. Khoussainov,
Journal of Automata, Languages and Combinatorics, $8(2), 287-301, 2003$. 

{\it Some Results on Automatic Structures}, with B. Khoussainov
and H. Ishihara, $17$th Annual IEEE Symposium on Logic in Computer Science,
$2002$. 

{\it Graphs with Automatic Presentations over a Unary Alphabet}
Journal of Automata, Languages and Combinatorics, $6(4), 467-480, 2001$. 

{\it Finite Automata and Well Ordered Sets},
New Zealand Journal of Computing, $7(2), 39-46, 1999$. 
 
\end{resume}
\end{document}






























