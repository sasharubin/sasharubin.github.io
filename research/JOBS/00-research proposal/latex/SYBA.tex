\documentclass[a4paper,12pt]{scrartcl}


%% fiddle with list params 
\usepackage{enumitem}
\setlist[itemize]{parsep=0pt,itemsep=0pt}
\setlist[enumerate]{parsep=0pt,itemsep=0pt}


% \newcommand{\FileId}{${}$Id: a2.tex,v 1.1 2009/03/02 20:31:04 kaie Exp kaie ${}$}
\usepackage[nohead, nofoot, margin=1.72cm]{geometry}
\usepackage[german,english,australian]{babel}
\usepackage[latin1]{inputenc}
\usepackage[T1]{fontenc} 
\usepackage{textcomp}
\usepackage{xspace,amsmath,amssymb,url,pslatex,mathptmx,courier%
}
\usepackage[tight,hang]{subfigure}
\usepackage{graphicx%,floatflt
}
%% needs to go back to empty or setting the page no. later needs to be fixed
\pagestyle{empty}
\makeatletter
\renewcommand*\bib@heading{}%
%  \subsection{\refname}\@mkboth{\refname}{\refname}
%  \small See Sections B10.2 and 10.3 for citations    [e\ldots], [m\ldots], [M\ldots], and [W\ldots].}
\renewenvironment{thebibliography}[1]{\bib@heading%
  \list{\@biblabel{\@arabic\c@enumiv}}%
  {\settowidth\labelwidth{\@biblabel{#1}}%
    \leftmargin\labelwidth
    %% K begin
    \advance\leftmargin\labelsep
    \small%
    \setlength\lineskip{0pt}%
    \setlength\parsep{0pt}%
    \setlength\itemsep{0pt}%
    %% K end
    \@openbib@code
    \usecounter{enumiv}%
    \let\p@enumiv\@empty
    \renewcommand*\theenumiv{\@arabic\c@enumiv}}%
  \sloppy\clubpenalty4000\widowpenalty4000%
  \sfcode`\.=\@m}
{\def\@noitemerr
  {\@latex@warning{Empty `thebibliography' environment}}%
  \endlist}
\makeatother

\newcommand{\Springer}{}
\newcommand{\bogocite}[2]{[#1]}
\newcommand{\orderblurb}{In theoretical computer science, authors are typically ordered alphabetically.}
\newcommand{\commentout}[1]{}
\newcommand{\CIs}{CIs\xspace}
\newcommand{\PIs}{PIs\xspace}
% \newcommand{\Ron}{CI$_1$\xspace}
% \newcommand{\Kaile}{CI$_2$\xspace}
%\newcommand{\Yoram}{PI$_1$\xspace}
%\newcommand{\Thomas}{PI$_2$\xspace}
% \newcommand{\SRA}{SRA\xspace}
\newcommand{\Token}[1]{\textsf{#1}\xspace}

\newcounter{myenumi}
\newenvironment{myenumone}{\begin{trivlist}}{\end{trivlist}}

\hyphenation{non-de-ter-min-is-tic}
\hyphenation{non-de-ter-min-ism}
\hyphenation{non-in-ter-fer-ence}
\sloppy


\selectlanguage{australian}
\marginparwidth10pt
\marginparsep10pt
\marginparpush2pt
\reversemarginpar
\parsep0pt
\topsep0pt

\setlength{\fboxrule}{1pt}
\newcommand{\BOX}[1]{\noindent\fbox{\parbox{\textwidth}{#1}}}

%% 
%% part B -- Personnel
%% 
% % \renewcommand{\thesection}{\Alph{section}}
% % \renewcommand{\thesubsection}{\thesection\arabic{subsection}}
% \setcounter{section}{2}
%% 
%% parts B10 - one for each CI and PI = 4 altogether
%% 
% \newcommand{\boast}[1]{\newblock \\{\textsf{#1}}}
% %% ideally I'd slot in the templates I sent out
% \newcounter{ronbibitem}
% \newcounter{kaibibitem}
% \newcounter{yorambibitem}
% \newcounter{thomasbibitem}

% old start
% \documentclass[12pt,a4paper,times]{article}       
% \usepackage{xcolor}
%  \usepackage[margin=4cm]{geometry}

\def\TITLE{Synthesis of Trustworthy Behaviour of Artificial Agents (SYBA)} 
% \title{\TITLE}
\def\TSE{Temporal-Strategic-Epistemic\xspace}

\usepackage{framed}

%% PACKAGES %%
\usepackage{latexsym}
\usepackage{amsmath}
\usepackage{amssymb}
%\usepackage{amsthm} %


\usepackage{color}
\usepackage{graphicx}
\usepackage{tikz,pgf}
  \usetikzlibrary{automata,positioning,matrix,calc,petri,arrows}

%% ENVIRONMENTS %%
%\theoremstyle{plain}
%\newtheorem{theorem}{Theorem}
%\newtheorem{lemma}{Lemma}
%\newtheorem{fact}{Fact}
%\newtheorem{example}{Example}
%\newtheorem{definition}{Definition}
%%\newtheorem{corollary}{Corollary}
%\newtheorem{proposition}{Proposition}
%%\newtheorem*{proof*}{Proof}

%% COMMENTS %%

\newcommand\note[1]{{\color{red}{#1}}}
\newcommand{\todo}[1]{{{\color{blue} #1}}}
%\renewcommand{\todo}[1]{}

%% LATEX SHORTCUTS %%
% cause problems with aamas style file
%\def\it{\begin{itemize} }
%\def\-{\item[-] }
%\def\ti{\end{itemize} }
%\def\en{\begin{enumerate} }
%\def\ne{\end{enumerate} }

%% COMPLEXITY CLASSES %%

\def\UPTime{\textsc{up}\xspace}
\def\CoUPTime{\textsc{coup}\xspace}


\def\exptime{\textsc{exptime}\xspace}
\def\exptimeC{\exptime-complete}

\def\pspace{\textsc{pspace}\xspace}
\def\pspaceC{\pspace-complete}

\def\logspace{\textsc{logspace}\xspace}
\def\nlogspace{\textsc{nlogspace}\xspace}

\def\ptime{\textsc{ptime}}
\def\np{\textsc{np}}



%% LOGIC %%
\def\fol{\mathsf{FOL}}
\def\SL{\textsf{SL}}

\def\msol{\mathsf{MSOL}}
\def\fotc{\mathsf{FOL+TC}}

\def\know{\mathbb{K}}
\def\dknow{\mathbb{D}}
\def\cknow{\mathbb{C}}
\def\eknow{\mathbb{E}}

\def\dualknow{\widetilde{\mathbb{K}}}
\def\dualdknow{\widetilde{\mathbb{D}}}
\def\dualcknow{\widetilde{\mathbb{C}}}
\def\dualeknow{\widetilde{\mathbb{E}}}

\renewcommand\implies{\rightarrow}

%% TEMPORAL LOGIC %%
\newcommand{\sqsq}[1]{\ensuremath{[\negthinspace[#1]\negthinspace]}}

\DeclareMathOperator{\ctlE}{{\mathsf{E}}}
\DeclareMathOperator{\ctlA}{\mathsf{A}}


\newcommand{\atlE}[1][A]{\ensuremath{\langle\!\langle{#1}\rangle\!\rangle}}
\newcommand{\atlA}[1][A]{\ensuremath{[[{#1}]]}}

\DeclareMathOperator{\nextX}{\mathsf{X}}
\DeclareMathOperator{\yesterday}{\mathsf{Y}}
\DeclareMathOperator{\until}{\mathbin{\mathsf{U}}}
\DeclareMathOperator{\weakuntil}{\mathbin{\mathsf{W}}}
\DeclareMathOperator{\since}{\mathbin{\mathsf{S}}}
\DeclareMathOperator{\releases}{\mathbin{\mathsf{R}}}
\DeclareMathOperator{\always}{\mathsf{G}}
\DeclareMathOperator{\hitherto}{\mathsf{H}}
\DeclareMathOperator{\eventually}{\ensuremath{\mathsf{F}}\xspace}
\DeclareMathOperator{\previously}{\mathsf{P}}
\newcommand{\true}{\mathsf{true}}
\newcommand{\false}{\mathsf{false}}


\newcommand{\LTL}{\ensuremath{\mathsf{LTL}}\xspace}
\newcommand{\PLTL}{\textsf{PROMPT-}\LTL}

\newcommand{\CTL}{\ensuremath{\mathsf{CTL}}\xspace}
\newcommand{\CTLS}{\ensuremath{\mathsf{CTL}^*}\xspace}
\newcommand{\PCTLS}{\textsf{PROMPT-}\CTLS}
\newcommand{\PCTL}{\textsf{PROMPT-}\CTL}
\newcommand{\CLTL}{\ensuremath{\textsf{C-}\LTL}\xspace}
\newcommand{\PCLTL}{\ensuremath{\textsf{PROMPT-C}\LTL}\xspace}

\newcommand{\ATL}{\ensuremath{\mathsf{ATL}}\xspace}
\newcommand{\ATLS}{\ensuremath{\mathsf{ATL}^*}\xspace}
\newcommand{\PATLS}{\textsf{PROMPT-}\ATLS}
\newcommand{\PATL}{\textsf{PROMPT-}\ATL}

\newcommand{\KATL}{\ensuremath{\mathsf{KATL}}\xspace}
\newcommand{\KATLS}{\ensuremath{\mathsf{KATL}^*}\xspace}
\newcommand{\PKATLS}{\textsf{PROMPT-}\KATLS}
\newcommand{\PKATL}{\textsf{PROMPT-}\KATL}


\def\red{{red}}
\def\col{{col}}
\def\alt{\, | \,}

%% PROMPT 
\def\kmodels{\models^k}
\def\twokmodels{\models^{2k}}
\DeclareMathOperator{\Fp}{\eventually_\mathsf{P}}
\DeclareMathOperator{\Gp}{\always_\mathsf{P}}
\DeclareMathOperator{\within}{\mathsf{within}}

\newcommand{\AP}{{AP}}
\def\Ag{{Ag}}
\def\Act{{Act}}

%% MATH OPERATIONS %%
\newcommand{\tpl}[1]{\langle {#1} \rangle }
\newcommand{\tup}[1]{\overline{#1}}
\def\proj{\mathsf{proj}}
\newcommand{\defeq}{\ensuremath{\triangleq}}

%% STRUCTURES and STRATEGIES %%
\newcommand{\cgs}{\ensuremath{\mathsf{S}}}
\newcommand{\LTS}{\mathsf{S}}
\newcommand{\Comp}{\mathsf{cmp}}
\newcommand{\Hist}{\mathsf{hist}}
\newcommand{\out}{{out}}

\newcommand{\Paths}{\mathsf{pth}}

\newcommand{\nat}{\mathbb{N}}
\def\int{\mathbb{Z}}
\newcommand{\natzero}{\mathbb{N}_0}

\newcommand{\trans}[3]{#1 \stackrel{\mathsf{#3}}{\rightarrow} #2}


%% HEADINGS ETC %%
\newcommand{\head}[1]{\noindent {\bf #1}.}

%% COUNTER MACHINES %%
\newcommand{\cm}{M}
\newcommand{\CMinc}{\mathsf{inc}}
\newcommand{\CMdec}{\mathsf{dec}}
\newcommand{\CMzero}{\mathsf{ifzero}}
\newcommand{\CMnonzero}{\mathsf{nzero}}
\newcommand{\CMcommit}{\mathsf{end}}


%% CLTL %%
\def\var{{\sf var}}
\def\ovar{{\sf ovar}}
\def\avar{{\sf avar}}
\def\svar{{\sf svar}}
\def\bvar{{\sf bvar}}

\def\MOD{\equiv}

%% PVP %%
\def\PVP{\mathsf{PVP}}




\usepackage{parskip}
\setlength{\parindent}{10pt}

\newcommand\aside[1]{\textcolor{red}{#1}}
\newcommand\pubact{\textsf{PUBACT}\xspace}
\newcommand\pomdp{\textsf{POMDP}}

\renewcommand{\labelitemi}{\tiny$\blacksquare$}

% \date{October 2017}

% CV REFS
\def\BMMRV17{[C3]}
\def\DBLPconfatalBelardinelliLMR17{[C2]}
\def\DBLPconfijcaiBelardinelliLMR17{[C1]}
\def\DBLPconfvmcaiAminofJKR14{[C17]}
\def\DBLPconfconcurAminofKRSV14{[C18]}
\def\DBLPconficalpAminofRZS15{[C13]}
\def\DBLPconfcadeAminofR16{[C9]}
\def\AKRSV17{[J1]}
\def\DBLPseriessynthesis2015Bloem{[B1]}
\def\DBLPjournalssigactBloemJKKRVW16{[J6]}
\def\DBLPconfatalRubin15{[C16]}
\def\DBLPconfprimaRubinZMA15{[C12]}
\def\DBLPconfatalAminofMRZ16{[C7]}
% \def\DBLPconfprimaMuranoPR15{[C]}

\def\AR16{[J5]}
\def\traps13{[J7]}
\def\GMRS16IJCAI{[C10]}
\def\BDGRICAPS{[W1]}
\def\BDGR17{[C4]}

\def\GMPRW17{[C5]}
\author{}
\date{}

% \def\newhead#1{\noindent\textbf{#1.}}

\renewcommand{\todo}[1]{}
\begin{document}

% \maketitle


\todo{
Even if you are
not planning to do implementation yourself, talking about a collaboration with others
who are doing that might help.
}

\todo{structure the first page so that it reads as an executive summary of the
proposal that conveys to a *non-expert* reader an overview of what it is about, }



\todo{ 1. Who will review the proposal? It seems to be written to EXPRTS  in the field. It may need be well read by people outside the field.

2. "We need to establish scalable algorithms and tools for Temporal-Strategic-Epistemic reasoning."

Your record is purely theoretical, so your claim that you will develop tools requires some elaboration.


4. You need to motivate the focus on systems with public actions. Give examples of such systems.
}

% \section*{ARC Future Fellowship}
% 
% \section*{Part A}
% 
% \paragraph{Summary (max 750 chars)}
% %INTRO< CONTEXT < OUTCOME < BENEFIT
% 
% Artificial Intelligence systems are increasingly deployed in the world as agents, e.g., software negotiating on the internet, autonomous cars, robots in dangerous environments, etc. Humans need to be able to trust decisions made by such artificial agents. The goal of this project is to develop computational foundations and techniques for building trustworthy agents, by leveraging insights from recent results on synthesis and strategic reasoning for single and multiagent systems. The benefit will be techniques, useable by computer scientists and engineers, for building trustworthy agents in realms such as high-level robot control, trustworthy social-media bots, collusion-free e-auctions, and safe and secure cloud storage.
% 
% % high-level control of lightweight swarms, 
% 
% % 
% % AI systems are increasingly deployed in the world as \emph{agents}, 
% % e.g., software negotiating on our behalf on the internet, driverless cars, 
% % robots exploring dangerous environments, etc. There is a recently articulated need for humans to 
% % be able to \emph{trust} the decisions made by such artificial agents~\cite{ACMStatement07}. 
% % The goal of this project is to develop mathematical foundations and computational techniques for building trustworthy artificial agents, by leveraging the insights from  
% % recent results, developed by the proposer, on synthesis and strategic reasoning for single and multi-agent systems. The projected benefit will be techniques, useable by computer scientists and engineers, for building trustworthy agents in realms such as high-level robot control including lightweight swarms, concurrent manufacturing in industry 4.0, trustworthy social-media and -news bots, safe and secure could storage facilities.
% 
% \paragraph{Benefit and Impact (max 750 chars)}
% 
% The anticipated benefit of this project to science is that it will advance the state-of-the-art of the verification and synthesis of artificial agents. The potential impact to UNSW's Strategic Theme ``Future Intelligence'' will be tighter integration with world-renowned experts in AI and Autonomous Systems, including attracting short- and long-term leaders in Automated Planning and Knowledge Representation. Safer and securer interactions with artificial agents are in the interest of society. 
% 
% \newpage

% \section*{Part C1 Project Description}
% max 10 pages)}
% \vspace{-0.5cm}

\subsection*{1. PROJECT TITLE: \TITLE}
% \vspace{-0.3cm}
% 
% \TITLE
% 
 \vspace{-0.2cm}

\subsection*{2. AIMS AND BACKGROUND}
 \vspace{-0.3cm}


%  Briefly outline the aims and background of this Proposal.
%  Include information about national/international progress in this field of research and
% its relationship to this Proposal.
%  Refer only to publications or non-traditional equivalents (outputs) that are accessible
% to the national and international research communities

% Use separate sub- headings. 
% Provide overarching research ambition/aim of this research along with any sub aims/objectives if applicable.
% Beware of slipping into research activity rather than aims. If your aims are framed as ‘Analysing’, ‘Investigating’, ‘Documenting’ etc you are detailing activity. Your aims are why you are undertaking these activities – what are you going to achieve, solve, enhance, change etc?
% Refer directly to the relevant literature in a way that demonstrates your knowledge is up to date. Your own published work should be cited to demonstrate you are at the cutting edge.
% Include reference to any research that assessors will expect you to be familiar with even if you are choosing not to use it in your project. In this case reflect your familiarity and indicate why this work will not inform your project.
% The background must be past tense. Avoid looking forward to your project in this section.



% EXEC SUMMARY
% • What is the nature of the challenge?
% • Which part is the focus of the project and why is it important to address it?
% • How are others trying to address it?
% • How do you propose to address it and why is your approach different, better and more exciting/innovative
% • Why you /your Team are particularly suited?


\emph{Nature of the challenge.} Systems built on the insights of Artificial Intelligence are increasingly deployed in the world as \emph{agents}, 
e.g., software agents negotiating on our behalf on the internet, driverless cars,  
robots exploring new and dangerous environments, bots playing games with humans. There is an obvious need for humans to 
be able to \emph{trust} the decisions made by artificial agents, 
the need for {meaningful interactions} between humans and agents, and the need for {transparent} agents~\cite{ACMStatement07}. 
	
{This need can only be met if humans are able to model, control and predict the {behaviour} of agents.} This challenge is made 
all the more complicated since: 1) agents are often deployed with \emph{other} agents leading to \emph{multi-agent systems}, 2) agent behaviour is complex, and extends into the future, leading to \emph{temporal reasoning}, 3)  agents are often ``self-interested'', leading to \emph{strategic reasoning}, 4)
agents may have uncertainty about the state, or even the structure, of other agents and the environment, leading to \emph{epistemic reasoning}.

\emph{Focus of this project.} Synthesising and analysing trustworthy artificial agents requires \emph{\TSE reasoning on Multi-agent Systems}.
{The aim of this project is to develop the mathematical foundations and computational techniques for building and analysing 
trustworthy artificial agents, by leveraging the insights from recent results developed by the candidate on modeling, control and analysis of single and multi-agent systems.} 
There are three specific objectives: 1) discover new classes of systems for which \TSE reasoning is decidable and tractable, 
2) develop the theory of reasoning about optimal strategies and socially optimal equilibria, and 3) establish scalable algorithms and tools for \TSE reasoning. 


\emph{State of the art.} Logic-based techniques are a standard approach to modeling, building and analysing computational systems. Indeed, simply {formalising} the reasoning tasks 
unambiguously requires a formal language. Not surprisingly, such reasoning is computationally \emph{undecidable} when it involves epistemic reasoning, a fact known since the late 1970s~\cite{DBLP:conf/focs/PetersonR79}. The historical approach to ameliorate this 
is to restrict to classes of multi-agent systems in which agents' private knowledge is hierarchical (typically, one assumes some sort of hierarchy on agent observation or information~\cite{DBLP:conf/focs/PetersonR79,DBLP:conf/focs/PnueliR90,DBLP:conf/lics/KupfermanV01, DBLP:conf/atva/BerwangerMB15,BMMRV17}). 
Although {mathematically elegant} and well-explored, the \emph{applicability of such assumptions is not very high} since in almost all meaningful scenarios, agents' private knowledge are not hierarchical.

\emph{Proposed approach.}
In a remarkable recent discovery~\cite{BLMR17IJCAI,BLMR17} the candidate defined and explored a very general class of systems that does not suffer from this long-standing limitation, i.e., the class in which \emph{agent actions are fully observable}. He proved that \emph{\TSE reasoning is decidable and not harder than the non-epistemic case}. Many scenarios already fall into this class, e.g., distributed computing and multi-party computation based on broadcast communication~\cite{DBLP:books/mk/Lynch96, ADGH06}, multi-player games with public play such as poker~\cite{Bowling145}, e-auctions with public bidding~\cite{EasleyK10}. 

Moreover, the importance of this recent discovery is that it charts an unanticipated path for applying logic-based methods to 
\emph{meaningful classes} of artificial agents in a \emph{large variety of fields}, for instance: models of collaborative robot exploration in controlled but dynamic environments~\cite{amazon};  models of cloud manufacturing~\cite{DBLP:conf/ijcai/FelliSLR17};  models of collusion in e-auctions and auction-based mechanisms~\cite{EasleyK10};  models of social networks that use broadcast communication, and thus also formalisations of \emph{twitter}~\cite{DeNicola2015,DBLP:journals/jlp/MaggiPST17};  models of secure cloud-storage that use data-dispersal~\cite{DBLP:journals/internet/LiQLL16} and secret-sharing protocols~\cite{ADGH06};  models of multi-player games in which bidding and play is public, such as poker~\cite{Bowling145}.

\emph{The team.} The candidate is particularly suited to meet this challenge. His background in mathematical logic and formal methods has enabled him to devise effective conceptual frameworks to address problems in AI~\cite{DBLP:conf/atal/Rubin15,DBLP:conf/kr/AminofMRZ16,DeGiacomoMRS16,DBLP:conf/atal/AminofMMR16,BDGR17,GMPRW17,BDGR17,BLMR17IJCAI,BLMR17,BMMRV17}. His individual expertise is complemented by his close integration with world experts in logic and automata-based verification and synthesis (M. Y. Vardi), multi-agent systems (M. Wooldridge), knowledge representation (G. De Giacomo), and automated planning (H. Geffner).





\paragraph{Overall Aim} The aim of this project is to develop the mathematical foundations and computational techniques for building and analysing 
trustworthy artificial agents, by leveraging the insights from recent results developed by the candidate on automated reasoning for single and multi-agent systems.


% We need to establish meaningful classes of agents that are amenable to automatic computational analysis.
\paragraph{Specific Aims}

We state and justify $3$ specific aims.


\BOX{1. We need to discover meaningful \emph{new classes} of multi-agent systems for which \TSE reasoning is decidable and tractable.}

Synthesis is a seemingly simple form of \TSE-reasoning that asks if a given coalition of agents have a strategy ensuring some joint objective. It is known synthesis is intractable and even \textbf{undecidable} on multi-agent systems, a fact that has been discovered in multiple contexts, i.e., decentralised \pomdp s~\cite{DBLP:journals/mor/BernsteinGIZ02}, multiplayer non-cooperative games of imperfect information~\cite{peterson2001lower}, distributed synthesis~\cite{DBLP:conf/focs/PnueliR90}).
 
Since the 1970s researchers have tried to find decidable fragments. The standard approach is to assume some sort of hierarchy on the information or 
observation sets, e.g.,~\cite{DBLP:conf/focs/PetersonR79,DBLP:conf/focs/PnueliR90,DBLP:conf/lics/KupfermanV01, DBLP:conf/atva/BerwangerMB15,BMMRV17}. Although mathematically elegant, the applicability of such assumptions is not very high since agents' private information is typically not hierarchical.

However, we recently~\cite{BLMR17IJCAI,BLMR17} defined and explored a class of systems in which {all agent actions are fully observable to all agents}, and proved that one can do analysis, i.e., \TSE reasoning is \textbf{decidable and not harder than the non-epistemic case}.\footnote{Previously it was only known that, in a similar setting, one can do multi-agent epistemic planning~\cite{DBLP:conf/aips/KominisG15} and synthesis~\cite{vanderMeyden2005}.} Many scenarios already fall into this class, e.g., distributed computing protocols and multi-party computation that are based on broadcast communication~\cite{DBLP:books/mk/Lynch96, ADGH06}, multi-player games with public play such as poker~\cite{Bowling145}, e-auctions with public bidding~\cite{EasleyK10}. That said, the importance of this result is that it lays the algorithmic and theoretical foundations for analysis of \emph{many meaningful} classes of agents since, notably, we are no longer bound by the restriction that agents' observations need be hierarchical.

% \begin{enumerate}
%  \item various models of collaborative robot exploration in controlled but dynamic environments~\cite{amazon},
%  \item various models of cloud manufacturing~\cite{DBLP:conf/ijcai/FelliSLR17},
%  \item various models of collusion in e-auctions and auction-based mechanisms~\cite{EasleyK10},
%  \item various models of social networks that use broadcast communication, and thus also formalisations of \emph{twitter}~\cite{DeNicola2015,DBLP:journals/jlp/MaggiPST17},
%  \item various models of secure cloud-storage that use data-dispersal~\cite{DBLP:journals/internet/LiQLL16} and secret-sharing protocols~\cite{ADGH06},
%  \item parts of multi-player games in which bidding and play is public, such as poker~\cite{Bowling145}.
% \end{enumerate}

%  \item various epistemic puzzles of interest to computer scientists~\cite{}.
%  such as the  muddy-children problem and the problem of russian cards, %% NOT STRONG POINT FOR PROPOSAL!


%  \todo{another use case that people not in area of agents say ``yes, i like it!'' think big and crazy. then say something meaningful. be bold and concrete. car, hacker attack, something concrete that we could do in principle}


% In contrast, general forms of planning can be used to capture multiple agents, 
% imperfect information, incomplete information, and temporally extended goals.

% I propose to systematically study how to reduce behaviour synthesis to classical planning. This will be done in two steps:
% \begin{enumerate}
%  \item Study how to 
%  reduce behaviour synthesis to general forms of planning.
%  \item Study how to 
%  reduce general forms of planning to classical planning.
% \end{enumerate}
% 
% Both steps will be done using insights from automated synthesis~\cite{Vard96,KuVa97,DeGiacomoFPS10,DeVa15,DeVa16}, generalised planning~\cite{HuG11,DeGiacomoMRS16,BDGR17}, and reductions of planning with LTL-goals to classical planning~\cite{}. Moreover, the practical aspects of such reductions will be done 
% in collaboration with leading planning experts Hector Geffner and Blai Bonet. \aside{ask Hector/Blai}

% I plan to ground the practical considerations developed in the second phase to real application domains such as cognitive robotics, 
% multiplayer card games such as poker, analysis of multi-party computation.
% \aside{these can be done with people at UNSW}

 
%  \item We need to discover new ways of dealing with the state-explosion problem for systems with imperfect-information.
 
\BOX{2. We need to define, analyse, and tackle the problem of reasoning about {optimal strategies} and \emph{socially optimal equilibria}.}
% in systems of agents that also have \emph{quantitative objectives}.}

In order to have evidence that one agent's behaviour is better or worse than another, or whether a collection of agents are acting in the good of society, 
we need to be able to measure the quality of agent strategies against each other.
This, in turn, would be facilitated by endowing agent objectives with a quantitative component. Although most work in verification deals with qualitative objectives, there has been a recent focus on verification of quantitative models of programs~\cite{DBLP:journals/ife/Henzinger13,ABK16}. However, this has yet to be clearly generalised to \TSE reasoning for multi-agent systems. Building on more classic work~\cite{DBLP:journals/jacm/AlurHK02,MogaveroMPV14}, the candidate recently introduced expressive logics that can be used to reason about socially optimal equilibria in cases agents have qualitative objectives~\cite{DBLP:conf/atal/AminofMMR16,BLMR17IJCAI,BMMRV17}. This, together with recent insights from quantitative verification \cite{DBLP:conf/concur/UmmelsW11,DBLP:journals/ife/Henzinger13,DBLP:journals/tocl/MarchioniW15,ABK16}, lays the foundation for designing useful logics and measures of strategy-quality for reasoning about socially optimal equilibria. 

% 
% a measure of quality of agent strategy.
%  Most existing work in Automated Planning and verification deals with agents with qualitative objectives.  This, in turn, likely requires that objectives be formalised with a quantitative component. However, most work in Automated Planning focuses on finding optimal strategies or optimal plans for qualitative (i.e., reachability) objectives~\cite{GeffnerBo13,Penna15,TorralbaAKE17}. Some work deals with multiple agents with possibly different but overlapping objectives, which revolves around finding stable solutions (e.g., Nash equilibria)~\cite{DBLP:journals/amai/KupfermanPV16,DBLP:journals/ai/GutierrezHW17}. Techniques for synthesising and reasoning about equilibria are all the more needed when agents have a mix of qualitative and quantitative objectives. 
 
 
 
%  \item We need to discover new parameters based on the recently introduced ``width'', that measure the ``complexity'' of the synthesis problem, and prove bounds on the computational complexity wrt these parameters, of synthesis problem.


\BOX{3. We need to establish \emph{scalable algorithms and tools} for \TSE reasoning.}

To accomplish \TSE reasoning for multi-agent systems, including reasoning about social equilibria, we need 
scalable algorithms and tools. Since the worst-case complexity of such reasoning is typically very high, 
we need tools that can deal with large but ``easy'' cases. This grand challenge is being met by a number 
of branches of computer science, notably the Automated Planning community in Artificial Intelligence.

\textbf{Automated Planning is a form synthesis} (and thus a form of temporal-strategic reasoning) that is central to the development of agents. 
It is a branch of Artificial Intelligence that addresses the problem of generating a course of action to achieve
a desired goal, given a description of the domain of interest and its initial state. 
The Automated Planning community has developed a ``science of search'', based on heuristic-search and symbolic methods, 
that efficiently plans for most problems of practical interest~\cite{GeffnerBo13,DBLP:conf/aaai/LipovetzkyG17}. 
The most successful of this technology  is for ``classical planning'', i.e., single agent, deterministic environment, with perfect information, 
and simple reachability goals, and ``fully observable non-deterministic planning'' (which amounts to the case of one agent in an adversarial environment).

Previous work has reduced planning with temporal goals or epistemic goals to classical and fully-observable 
nondeterministic planning~\cite{DBLP:conf/aaai/BaierM06,TorresB15,DBLP:conf/aips/KominisG15,Camacho17}. This lays the foundation for refining and extending the translations to handle full \TSE reasoning for multi-agent systems.


\subsection*{3. CANDIDATE}

% For Candidates applying for Future Fellowship Level 1:
%  Describe the Future Fellowship Candidate’s research opportunity and performance
% evidence (ROPE).
%  Provide evidence that the Future Fellowship Candidate has the capacity and
% leadership to undertake the proposed research
%  Provide evidence that the Future Fellowship Candidate has a record of high quality
% Research Outputs appropriate to the discipline/s
%  Provide evidence the Future Fellowship Candidate’s research training, mentoring
% and supervision
%  Provide evidence of the Future Fellowship Candidate’s national research standing.

\paragraph{Leadership}

% Reflect on your suitability to undertake the Future Fellowship in relation to the level of fellowship you are requesting.
% In reflecting your leadership present evidence with reference to the previous projects that you have led or on which you have played a significant role. 
% These do not have to be ARC projects only and, if appropriate, can include projects with industry. 
% In reflecting your leadership you reflect your research leadership as well as your project management abilities. As a Future Fellow you are wholly responsible for the project management over the entire 4yrs of funding – demonstrating these skills will also be important.

I have previously held two individual fellowships: a 3-year New Zealand Science and Technology Postdoctoral Fellowship (NZ\$ 224532), and a 2-year Marie-Curie Postdoctoral Fellowship confunded by the National Institute of Higher Mathematics (\texteuro 107000). I was project co-ordinator for the \emph{Handbook of Model Checking}, edited by Ed Clarke et. al., and published by Springer (Dec 2017).

\paragraph{High quality research outputs}


% The ROPE section offers you the opportunity to list publications so this section is not about restating this information.
% Provide a narrative of the resonance of your outputs – detail how they have been received, the standard of journal or other outlet or forum you regularly use for dissemination.
% If you have received positive comments and endorsements from reviewers or key people in the field you can quote them here - a few words only.
% Discuss how your outputs have been picked up by other researchers within or beyond your discipline.

In theoretical computer science and Artificial Intelligence, top conferences are highly prestigious venues for dissemination. I regularly publish in conferences of the highest level, e.g., I published 5 CORE A* papers in 2017 and 5 CORE A* papers in 2016. According to google scholar: I have 785 citations (391 since 2012), my H-index is 15 (12 since 2012), and my Phd Thesis has 72 citations. In my field it is expected that authors are ordered alphabetically. That said, my contribution to my published papers always meets and often far exceeds an equal division of labour. 
% \todo{More detailed contributions for each paper are given in the References section of the accompanying CV.}

I quote the following endorsements:
\begin{quotation}
\emph{I give this application my wholehearted support. Sasha is an highly motivated, extremely active researcher, with a superb technical grasp of issues in logic, AI, and game theory.} \\  Michael Wooldridge (Oxford University)
\end{quotation}
\begin{quotation}
\emph{His research papers on
these lines are effectively changing the discipline by providing very significant results in the
community of formal aspects of Artificial Intelligence. } \\ Alessio Lomuscio (Imperial College London)
\end{quotation}


% \todo{add endorsements from key people in field}

% I have deep and extensive knowledge in logics for temporal, strategic and epistemic reasoning, automata-theory and synthesis. 
% I have contributed foundational work on synthesis and graph-games~\AR16,\traps13 as well as 
% on the connections between synthesis and general forms of planning~\GMRS16IJCAI,\BDGRICAPS,\BDGR17. 
% 
% A first step towards synthesis is usually verification, and I have contributed deep work on verification of multi-agent systems~\DBLPconfvmcaiAminofJKR14,\DBLPconfconcurAminofKRSV14,\DBLPconficalpAminofRZS15,\DBLPconfcadeAminofR16,\AKRSV17, including a book on the topic~\DBLPseriessynthesis2015Bloem,\DBLPjournalssigactBloemJKKRVW16, and with a focus on verification of parameterised systems \DBLPconfatalRubin15,\DBLPconfprimaRubinZMA15,\DBLPconfatalAminofMRZ16.


\paragraph{Research training, mentoring and supervision}

% Detail the numbers of HDR students you have supervised and their successes.
% If any students have established careers in academia or industry of note mention these.
% Discuss any innovative or successful supervision/mentoring strategies you have developed and/or used.
% Indicate instances where your mentorship has been sought out and why.

I worked closely with a PhD student of Erich Gradel's (Tobias Ganzow) and solved a 12-year open problem~\cite{DBLP:conf/stacs/GanzowR08}.
I have been sought as a referee for the IRISA Master Research Internship 2017 (France).
I have mentored 1 Msc Internship (2017), 1 Undergraduate thesis (2017), and 7 undergraduate 
students doing research (2012, 2009).

\paragraph{National and International standing}

% Detail invitations for collaborations, expert opinion and engagement.
% Reflect on growing/established reputation (dependent on level) within both academia and industry, government policy etc.
% If you have received awards prizes these will be noted in D6 however explain how these reflect your reputational value here.


I serve as PC member of CORE A* conference in Artificial Intelligence and Multi-agent systems (i.e., IJCAI 2017, AAAI 2017, AAAI 2018, AAMAS 2018). I have chaired one national conference on theoretical computer science (ICTCS 2017, Italy) and one international workshop on strategic reasoning (SR 2017). I conceived and organised a Workshop on Formal Methods in Artificial Intelligence (FMAI 2017). 
I have served as an external reviewer for the Icelandic Research Fund (IRF 2017). I have been invited to talk at various universities, including UNSW (Australia), IMT Lucca (Italy), Sapienza University of Rome (Italy), Universit\'e Paris-Diderot (France), and Oxford University (UK).

I already have close connections with international experts with expertise and interest in the topic of this project, all of whom have agreed to collaborate on the project: Giuseppe De Giacomo (knowledge representation, artificial intelligence, verification, synthesis), Hector Geffner and Blai Bonet (Planning), Moshe Vardi and Aniello Murano (logics for strategic reasoning, automata-theory for synthesis and verification), and Michael Wooldridge and Alessio Lomuscio (multi-agent systems).



\subsection*{4. PROSPOSED PROJECT QUALITY AND INNOVATION}

Every society that has embraced the digital world faces the issue of whether, or to what extent, it can trust the behaviour 
of artificial agents. The challenge of building trustworthy artificial agents cannot be met without having some formal guarantees on their 
behaviour. There are two fundamental parts to this problem: automatically synthesise agent behaviour, or part of their behaviour, from unambigious declarative 
specifications; analyse behaviour of built or existing agents. This project will advance the state of the art of the mathematical foundations and computational techniques 
for building and analysing trustworthy agents from \TSE specifications. In particular, it will provide extend the use of logical methods to the analysis of multi-agent systems coming 
from a variety of fields, including high-level robot control, trustworthy social-media bots, collusion-free e-auctions, and safe and secure cloud storage. 
Safer and securer interactions with artificial agents are clearly in the interests of society. 

% You may choose to start from a ‘big picture’ perspective (i.e. global and national importance and potential benefits in the context of the broader discipline/area) and then focus on the specific outcomes of your project.
% Be realistic – assessors have complained of exaggerated or grandiose claims. Try to come up with several different dimensions. These can include the following areas, in whatever order is best for your case:
% Training – students and post-docs gain specific knowledge and important generic skills, potential workforce capability for Australia in new smart/innovation economy).
% Social or cultural impact/benefit particularly if beyond the immediate discipline.
% Creating intellectual linkages and leadership – national, regional and international.
% Direct economic benefits (e.g. wealth creation through commercialisation)
% Contribution to cutting-edge national and/or international research knowledge creation.
% 

%  Explain how the research addresses a significant problem.
%  Outline the conceptual/theoretical framework, and demonstrate that these are
% adequately developed, well integrated, innovative and original.
%  Explain how the aims, concepts, methods and results advance knowledge.
%  Describe how the design and methods are appropriate for the proposed research.
%  Describe how the proposed research may result in maximising economic,
% environmental, social, and/or cultural benefits to Australia. This statement should
% align with the Impact Statement.
%  If the research has been nominated as focussing on a topic or outcome that falls
% within one of the Science and Research Priorities, explain how it addresses one or
% more of the associated Practical Research Challenges (as selected in question B1 of
% this Proposal form).
%  Describe how the proposed Project involves interdisciplinary research, if appropriate.
%  Describe how the proposed Project will push the boundaries of research and open up
% new research opportunities.
%  Explain how the proposed Project will contribute to public policy formulation and
% debate.

\paragraph{Project Quality}
% In this section show HOW you will undertake the project. Not what, but HOW? 
% It needs energy and therefore must be future tense and preferably active voice throughout.  
% Check that the start of each paragraph is strong and convincing (you know exactly what you will do)
% Break up the text as appropriate with relevant figures and diagrams, bullet points, selective use of bold/italics to highlight key statements.
% Provide a brief overview/introduction of the different aspects of your project – which may be theoretical, experimental, numerical/computational, quantitative and qualitative.
% Write for an expert audience - you must include sufficient technical detail to demonstrate your credibility and suitability to successfully achieve your aims. Make sure you explain how the aims, concepts, methods and results advance knowledge.
% If there are risks to the research acknowledge them and detail your contingency plans.
% In detailing what you will do remember this is a fellowship – a 1 person project – ‘I’ should figure predominantly in your articulation of activities.


The objectives of the project are to generate new mathematics, algorithms, and tools for describing, reasoning-about and building trustworthy agents. 
This will be done using methods and insights from Logic and Formal Methods (including program synthesis), and 
Game Theory (including its development in multi-agent systems). 


\BOX{1a. In order to discover richer decidable classes, I propose to \textbf{generalise} systems in which all actions are fully observable.}

I recently explored the class of systems in which all actions are fully observable~\cite{BLMR17IJCAI,BLMR17}. This class already includes 
formalisations of many important scenarios, notably distributed computing protocols that use broadcast communication~\cite{DBLP:books/mk/Lynch96}, including rational distributed computing and multiparty computation~\cite{ADGH06}, as well as multi-player games of imperfect information that use public bidding~\cite{Bowling145}. I now outline the first directions I will pursue in order to expand this theory to encompass even more scenarios:
\begin{enumerate}
 \item Incorporate stochastic initial states, which is \textbf{widely applicable}. Indeed, not only do finite horizon stochastic systems 
 fall into this setting~\cite{DBLP:conf/uai/LittmanDK95}, but so do probabilistic multi-agent systems, called decentralised partially observable Markov decision processes~\cite{DBLP:series/sbis/OliehoekA16}, which are a framework for modeling uncertainty with respect to outcomes, environmental information and communication.  That is, this extension will addresses the problem of ensuring \textbf{agents behave well in unknown environments}.
 
 \item Incorporate symmetric and asymmetric encryption, which is applicable to online \textbf{privacy and security}. Indeed, private-keys can be stored in an agent's private state, and thus private-key encryption can already be simulated by fully observable actions. Furthermore, public-key encryption consists of public keys that can be widely disseminated, and thus encrypting with a public key can be modeled as a public action. 
 
 \item Limiting the number of non-public actions, which is applicable to design and analysis of \textbf{collusion analysis} in e-auctions. For instance, the Vickrey auction~\cite{EasleyK10} is a type of sealed-bid auction, in which each bidder submits a written bid without knowing the bids of the other bidders. The highest bid wins but the price paid is the second-highest bid. Vickrey-Clarke-Grove (VCG) auctions are generalisations of Vickrey's
auctions to multiple items. It is known that Vickrey-Clarke-Grove auctions provide bidders with an incentive to bid their true value. It is also known that these auctions are vulnerable to \emph{collusion}: if all bidders reveal their true values to each other, using a \emph{limited number of non-public actions}, they can lower some or all of these values, while preserving who wins the auction. Thus, collusion analysis gives the auctioneer, and thus the market, confidence that bidders cannot game the system.

 \item Tuning the amount of observability of actions. Indeed, although systems with hidden actions are undecidable and fully observable actions are decidable, there is likely a measure of ``action observability'' that can be tuned so that one can incorporate systems in which certain actions are partially observable (but not completely hidden). Whatever this measure will look like, the result will be a \textbf{deeper understanding} of the borders between decidability and undecidability for various systems.
\end{enumerate}

The resulting classes that could be handled by such extensions are truly impressive. We list just some:
\begin{itemize}
 \item various models of collaborative robot exploration in controlled but dynamic environments~\cite{amazon,DBLP:journals/trob/Kress-GazitFP09},
 \item various models of cloud manufacturing~\cite{DBLP:conf/ijcai/FelliSLR17},
 \item various models of collusion in e-auctions and auction-based mechanisms~\cite{EasleyK10},
 \item various models of social networks that use broadcast communication, and thus also formalisations of \emph{twitter}~\cite{DeNicola2015,DBLP:journals/jlp/MaggiPST17},
 \item various models of secure cloud-storage that use data-dispersal~\cite{DBLP:journals/internet/LiQLL16} and secret-sharing protocols~\cite{ADGH06},
 \item various models of multi-player games in which bidding and play is public, such as poker~\cite{Bowling145}.
\end{itemize}

\BOX{1b. In order to discover tractable classes of agents, I propose to \textbf{restrict} to sub-systems of those in 1a.}

The complexity of \TSE reasoning for multi-agent systems identified in 1a is expected to be high. To achieve better computational complexity I propose to restrict them to sub-systems, 
while still maintaining the features in 1a that allow one to model systems from a wide variety of fields (e.g., that agent's observations need not be hierarchical). 
In particular, I will start by restricting to classes in which:
\begin{enumerate}
 \item the set of initial states is homogenous~\cite{DBLP:journals/tocl/LomuscioMR00}. This is applicable to situations in which agents are initially ignorant of each others local states;
 \item the size of the epistemic states is bounded, which is applicable to situations in which each agent has full observation except of its \emph{own} finite state;
 \item the strategies considered do not depend on the full history, but on a bounded summary of the history. Besides lowering the complexity, this assumption reflects the assumption of \textbf{bounded rationality}~\cite{simon1982models}.
 \end{enumerate}
 
 

%  \item explore the middle-ground between belief-space (which is exponentially large but accurate) and observation-space (which is linear but coarse) using trajectory constraints (which we recently pioneered~\cite{}), in order to find new ways of dealing with the state-explosion problem for systems of agents with imperfect-information.



\BOX{2. In order to reason about socially optimal equilibria, I propose to {enrich} the models {and} specification 
languages with costs/rewards and analyse these with measures of \textbf{strategy quality}.}

Agents are typically ``self-interested'', and thus they may not act in a way that is socially optimal. Moreover, it is often not possible to ascribe agent behaviour as simply being good or bad. Thus, I will explore measures of strategy quality and algorithms for synthesising socially optimal strategies. Although many game-theoretic solution concepts, such as Nash equilibria, can be expressed in recently introduced strategic logics~\cite{MogaveroMPV14}, and  their epistemic extensions~\cite{BMMRV17,BLMR17}, these logics can only express qualitative agent objectives. 
Thus, I will define and explore logics that can reason about quality of agent behaviour. In particular I will extend and evaluate state-of-the-art proposals for measuring quality of strategies to 
full \TSE reasoning for multi-agent systems, i.e.: the logic $\LTL[\mathcal{F}]$, and extension of \LTL with a set of quality operators $\mathcal{F}$~\cite{ABK16}, that was designed to reason about the quality of programs and can be used to reason about the \textbf{quality of agent behaviour};  the logic $\LTL_f$ with costs~\cite{DBLP:journals/corr/BrafmanGP17}, that allows one to reason about 
non-Markovian objectives; logics that combine qualitative behaviour (expressed for instance in $\LTL$ or $\LTL_f$) and quantitative, expressed for instance as long-term average of the cost of some resource~\cite{GMPRW17} or as the total cost of some resource~\cite{DBLP:journals/acta/BrihayeGHM17}.
% \item min-cost reachability games~\cite{DBLP:journals/acta/BrihayeGHM17} that can be used to compute a measure of strategy-quality in fully-observable non-deterministic planning;
% % by a preference over a mix of qualitative  and quantitative. Recent work by the candidate~\cite{GMPRW17} paves the way for reasoning about multi-agent systems in which agent objectives are a mix between the % 
 
 \BOX{3. In order to establish scalable tools and algorithms, I propose to \textbf{translate} \TSE reasoning to Automated Planning.}

I will extend and refine the translations that handle temporal goals and epistemic goals~\cite{DBLP:conf/aaai/BaierM06,TorresB15,Camacho17} to full \TSE reasoning for the multi-agent setting.
I will do this by leveraging the automata-theoretic approach to 
model-checking of strategic epistemic logics~\cite{BLMR17,BMMRV17}, as well as search through strategy-space~\cite{DBLP:conf/aips/BonetPG09}. 
One parallel direction to meet this objective is to explore generalisations of specification formalisms over finite traces~\cite{DeVa13,DeVa15,DeVa16}, adopted in automated planning~\cite{DBLP:conf/aaai/BaierM06,TorresB15,DBLP:conf/ijcai/AguasCJ16,Camacho17}.
Indeed, specifications on finite traces allow one to avoid notorious difficulties of infinite-traces~\cite{Vardi-symbolic17}, namely complementation of B\"uchi-automata~\cite{TsaiFVT14}. In particular, I will define and study ``epistemic strategy logic over finite traces'', and extend the mentioned translations to this logic. 

This objective will be achieved with \textbf{tight co-ordination} with four world-experts in automated planning, namely Hector Geffner (Pompeu Fabra University), Blai Bonet (Sim\'on Bol\'ivar University), Sebastian Sardina (Associate Professor at RMIT University), and Nir Lipovetzky (Lecturer at the University of Melbourne). The budget includes one international visit to Hector Geffner in Spain, as well as multiple international visits to Europe where the candidate will meet both H. Geffner and B. Bonet, as well as yearly national visits between the candidate and his students and S. Sardina and N. Lipovetzky in Melbourne. One Msc student student will be hired for 2 years to help with refining and implementing the translations. 
 

\paragraph{Project Innovation}
% Make sure you confront the innovation, not discuss.
% The way in which you believe the project is innovative (how/why) rather than describing the innovation itself (what). The innovation may be conceptual, technological, or methodological – use subheadings of these aspects of the innovation as flags for assessors. Novel concepts and/or original thought relevant. E.g. This project is innovative as for the first time it will…

The scientific innovation of the project is demonstrated in three ways.

\BOX{1. The project will develop a new mathematical and algorithmic theory for the design and analysis of meaningful multi-agent systems.}

\BOX{2. The project will advance the study of automatically finding socially optimal equilibria in which agents have a mix of qualitative and quantitative objectives.}

\BOX{3. The project will leverage the successful theory and technology of Automated Planning for the design and analysis of multi-agent systems.}  

Thus, the project has conceptual innovation by supplying viable frameworks for computer scientists and engineers to think about the behaviour of the agents they build, 
as well as technological innovation in leveraging Automated Planning to automated reasoning about artificial agents. The project will train one Phd student and two Msc students by research in cutting edge research in methods for building and analysing artificial agents. 
The project will further integrate the Administering Organisation with world-leaders in a number of fields of computer science, namely multi-agent systems (Michael Wooldridge and Alessio Lomuscio), knowledge representation (Giuseppe De Giacomo), logic and automata theory (Moshe Y. Vardi), and automated planning (Hector Geffner). 




% This will be done by building on recent breakthroughs in reasoning about multi-agent systems~\cite{}, 
% 
% In particular, I propose to systematically study how to reduce \TSE reasoning 
% of multi-agent systems to these planning problems (e.g., I propose a careful study of the effects that structure of the 
% agent goals have on the resulting translations). This will be done using insights from distributed synthesis~\cite{Vard96,KuVa97,DeGiacomoFPS10,DeVa15,DeVa16} 
% and generalised planning~\cite{HuG11,DeGiacomoMRS16,BDGR17}. 
% 







\subsection*{REFERENCES}
\bibliographystyle{abbrv}
\bibliography{researchplan,FWF,References,rubin}

  
\end{document}
