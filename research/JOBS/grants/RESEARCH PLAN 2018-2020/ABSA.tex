\documentclass[a4paper,12pt,smallheadings]{scrartcl}


%% fiddle with list params 
\usepackage{enumitem}
\setlist[itemize]{parsep=0pt,itemsep=0pt}
\setlist[enumerate]{parsep=0pt,itemsep=0pt}


% \newcommand{\FileId}{${}$Id: a2.tex,v 1.1 2009/03/02 20:31:04 kaie Exp kaie ${}$}
\usepackage[nohead, nofoot, margin=2cm]{geometry}
\usepackage[german,english,australian]{babel}
\usepackage[latin1]{inputenc}
\usepackage[T1]{fontenc} 
\usepackage{textcomp}
\usepackage{xspace,amsmath,amssymb,url,pslatex,mathptmx,courier%
}
\usepackage[tight,hang]{subfigure}
\usepackage{graphicx%,floatflt
}
%% needs to go back to empty or setting the page no. later needs to be fixed
\pagestyle{empty}
\makeatletter
\renewcommand*\bib@heading{}%
%  \subsection{\refname}\@mkboth{\refname}{\refname}
%  \small See Sections B10.2 and 10.3 for citations    [e\ldots], [m\ldots], [M\ldots], and [W\ldots].}
\renewenvironment{thebibliography}[1]{\bib@heading%
  \list{\@biblabel{\@arabic\c@enumiv}}%
  {\settowidth\labelwidth{\@biblabel{#1}}%
    \leftmargin\labelwidth
    %% K begin
    \advance\leftmargin\labelsep
    \small%
    \setlength\lineskip{0pt}%
    \setlength\parsep{0pt}%
    \setlength\itemsep{0pt}%
    %% K end
    \@openbib@code
    \usecounter{enumiv}%
    \let\p@enumiv\@empty
    \renewcommand*\theenumiv{\@arabic\c@enumiv}}%
  \sloppy\clubpenalty4000\widowpenalty4000%
  \sfcode`\.=\@m}
{\def\@noitemerr
  {\@latex@warning{Empty `thebibliography' environment}}%
  \endlist}
\makeatother

\newcommand{\Springer}{}
\newcommand{\bogocite}[2]{[#1]}
\newcommand{\orderblurb}{In theoretical computer science, authors are typically ordered alphabetically.}
\newcommand{\commentout}[1]{}
\newcommand{\CIs}{CIs\xspace}
\newcommand{\PIs}{PIs\xspace}
% \newcommand{\Ron}{CI$_1$\xspace}
% \newcommand{\Kaile}{CI$_2$\xspace}
%\newcommand{\Yoram}{PI$_1$\xspace}
%\newcommand{\Thomas}{PI$_2$\xspace}
% \newcommand{\SRA}{SRA\xspace}
\newcommand{\Token}[1]{\textsf{#1}\xspace}

\newcounter{myenumi}
\newenvironment{myenumone}{\begin{trivlist}}{\end{trivlist}}

\hyphenation{non-de-ter-min-is-tic}
\hyphenation{non-de-ter-min-ism}
\hyphenation{non-in-ter-fer-ence}
\sloppy


\selectlanguage{australian}
\marginparwidth10pt
\marginparsep10pt
\marginparpush2pt
\reversemarginpar
\parsep0pt
\topsep0pt

\newcommand{\BOX}[1]{\noindent\fbox{\parbox{\textwidth}{#1}}}

%% 
%% part B -- Personnel
%% 
\renewcommand{\thesection}{\Alph{section}}
\renewcommand{\thesubsection}{\thesection\arabic{subsection}}
% \setcounter{section}{2}
%% 
%% parts B10 - one for each CI and PI = 4 altogether
%% 
% \newcommand{\boast}[1]{\newblock \\{\textsf{#1}}}
% %% ideally I'd slot in the templates I sent out
% \newcounter{ronbibitem}
% \newcounter{kaibibitem}
% \newcounter{yorambibitem}
% \newcounter{thomasbibitem}

% old start
% \documentclass[12pt,a4paper,times]{article}       
% \usepackage{xcolor}
%  \usepackage[margin=4cm]{geometry}

\def\TITLE{Formal Methods for Automated Synthesis of Trustworthy Behaviour of Artificial Agents (TRUST)} 
\title{\TITLE}

\usepackage{framed}

%% PACKAGES %%
\usepackage{latexsym}
\usepackage{amsmath}
\usepackage{amssymb}
%\usepackage{amsthm} %


\usepackage{color}
\usepackage{graphicx}
\usepackage{tikz,pgf}
  \usetikzlibrary{automata,positioning,matrix,calc,petri,arrows}

%% ENVIRONMENTS %%
%\theoremstyle{plain}
%\newtheorem{theorem}{Theorem}
%\newtheorem{lemma}{Lemma}
%\newtheorem{fact}{Fact}
%\newtheorem{example}{Example}
%\newtheorem{definition}{Definition}
%%\newtheorem{corollary}{Corollary}
%\newtheorem{proposition}{Proposition}
%%\newtheorem*{proof*}{Proof}

%% COMMENTS %%

\newcommand\note[1]{{\color{red}{#1}}}
\newcommand{\todo}[1]{{{\color{blue} #1}}}
%\renewcommand{\todo}[1]{}

%% LATEX SHORTCUTS %%
% cause problems with aamas style file
%\def\it{\begin{itemize} }
%\def\-{\item[-] }
%\def\ti{\end{itemize} }
%\def\en{\begin{enumerate} }
%\def\ne{\end{enumerate} }

%% COMPLEXITY CLASSES %%

\def\UPTime{\textsc{up}\xspace}
\def\CoUPTime{\textsc{coup}\xspace}


\def\exptime{\textsc{exptime}\xspace}
\def\exptimeC{\exptime-complete}

\def\pspace{\textsc{pspace}\xspace}
\def\pspaceC{\pspace-complete}

\def\logspace{\textsc{logspace}\xspace}
\def\nlogspace{\textsc{nlogspace}\xspace}

\def\ptime{\textsc{ptime}}
\def\np{\textsc{np}}



%% LOGIC %%
\def\fol{\mathsf{FOL}}
\def\SL{\textsf{SL}}

\def\msol{\mathsf{MSOL}}
\def\fotc{\mathsf{FOL+TC}}

\def\know{\mathbb{K}}
\def\dknow{\mathbb{D}}
\def\cknow{\mathbb{C}}
\def\eknow{\mathbb{E}}

\def\dualknow{\widetilde{\mathbb{K}}}
\def\dualdknow{\widetilde{\mathbb{D}}}
\def\dualcknow{\widetilde{\mathbb{C}}}
\def\dualeknow{\widetilde{\mathbb{E}}}

\renewcommand\implies{\rightarrow}

%% TEMPORAL LOGIC %%
\newcommand{\sqsq}[1]{\ensuremath{[\negthinspace[#1]\negthinspace]}}

\DeclareMathOperator{\ctlE}{{\mathsf{E}}}
\DeclareMathOperator{\ctlA}{\mathsf{A}}


\newcommand{\atlE}[1][A]{\ensuremath{\langle\!\langle{#1}\rangle\!\rangle}}
\newcommand{\atlA}[1][A]{\ensuremath{[[{#1}]]}}

\DeclareMathOperator{\nextX}{\mathsf{X}}
\DeclareMathOperator{\yesterday}{\mathsf{Y}}
\DeclareMathOperator{\until}{\mathbin{\mathsf{U}}}
\DeclareMathOperator{\weakuntil}{\mathbin{\mathsf{W}}}
\DeclareMathOperator{\since}{\mathbin{\mathsf{S}}}
\DeclareMathOperator{\releases}{\mathbin{\mathsf{R}}}
\DeclareMathOperator{\always}{\mathsf{G}}
\DeclareMathOperator{\hitherto}{\mathsf{H}}
\DeclareMathOperator{\eventually}{\ensuremath{\mathsf{F}}\xspace}
\DeclareMathOperator{\previously}{\mathsf{P}}
\newcommand{\true}{\mathsf{true}}
\newcommand{\false}{\mathsf{false}}


\newcommand{\LTL}{\ensuremath{\mathsf{LTL}}\xspace}
\newcommand{\PLTL}{\textsf{PROMPT-}\LTL}

\newcommand{\CTL}{\ensuremath{\mathsf{CTL}}\xspace}
\newcommand{\CTLS}{\ensuremath{\mathsf{CTL}^*}\xspace}
\newcommand{\PCTLS}{\textsf{PROMPT-}\CTLS}
\newcommand{\PCTL}{\textsf{PROMPT-}\CTL}
\newcommand{\CLTL}{\ensuremath{\textsf{C-}\LTL}\xspace}
\newcommand{\PCLTL}{\ensuremath{\textsf{PROMPT-C}\LTL}\xspace}

\newcommand{\ATL}{\ensuremath{\mathsf{ATL}}\xspace}
\newcommand{\ATLS}{\ensuremath{\mathsf{ATL}^*}\xspace}
\newcommand{\PATLS}{\textsf{PROMPT-}\ATLS}
\newcommand{\PATL}{\textsf{PROMPT-}\ATL}

\newcommand{\KATL}{\ensuremath{\mathsf{KATL}}\xspace}
\newcommand{\KATLS}{\ensuremath{\mathsf{KATL}^*}\xspace}
\newcommand{\PKATLS}{\textsf{PROMPT-}\KATLS}
\newcommand{\PKATL}{\textsf{PROMPT-}\KATL}


\def\red{{red}}
\def\col{{col}}
\def\alt{\, | \,}

%% PROMPT 
\def\kmodels{\models^k}
\def\twokmodels{\models^{2k}}
\DeclareMathOperator{\Fp}{\eventually_\mathsf{P}}
\DeclareMathOperator{\Gp}{\always_\mathsf{P}}
\DeclareMathOperator{\within}{\mathsf{within}}

\newcommand{\AP}{{AP}}
\def\Ag{{Ag}}
\def\Act{{Act}}

%% MATH OPERATIONS %%
\newcommand{\tpl}[1]{\langle {#1} \rangle }
\newcommand{\tup}[1]{\overline{#1}}
\def\proj{\mathsf{proj}}
\newcommand{\defeq}{\ensuremath{\triangleq}}

%% STRUCTURES and STRATEGIES %%
\newcommand{\cgs}{\ensuremath{\mathsf{S}}}
\newcommand{\LTS}{\mathsf{S}}
\newcommand{\Comp}{\mathsf{cmp}}
\newcommand{\Hist}{\mathsf{hist}}
\newcommand{\out}{{out}}

\newcommand{\Paths}{\mathsf{pth}}

\newcommand{\nat}{\mathbb{N}}
\def\int{\mathbb{Z}}
\newcommand{\natzero}{\mathbb{N}_0}

\newcommand{\trans}[3]{#1 \stackrel{\mathsf{#3}}{\rightarrow} #2}


%% HEADINGS ETC %%
\newcommand{\head}[1]{\noindent {\bf #1}.}

%% COUNTER MACHINES %%
\newcommand{\cm}{M}
\newcommand{\CMinc}{\mathsf{inc}}
\newcommand{\CMdec}{\mathsf{dec}}
\newcommand{\CMzero}{\mathsf{ifzero}}
\newcommand{\CMnonzero}{\mathsf{nzero}}
\newcommand{\CMcommit}{\mathsf{end}}


%% CLTL %%
\def\var{{\sf var}}
\def\ovar{{\sf ovar}}
\def\avar{{\sf avar}}
\def\svar{{\sf svar}}
\def\bvar{{\sf bvar}}

\def\MOD{\equiv}

%% PVP %%
\def\PVP{\mathsf{PVP}}




\usepackage{parskip}
\setlength{\parindent}{10pt}

\newcommand\aside[1]{\textcolor{red}{#1}}
\newcommand\pubact{\textsf{PUBACT}\xspace}
\newcommand\pomdp{\textsf{POMDP}}

\renewcommand{\labelitemi}{\tiny$\blacksquare$}

% \date{October 2017}

% CV REFS
\def\BMMRV17{[C3]}
\def\DBLPconfatalBelardinelliLMR17{[C2]}
\def\DBLPconfijcaiBelardinelliLMR17{[C1]}
\def\DBLPconfvmcaiAminofJKR14{[C17]}
\def\DBLPconfconcurAminofKRSV14{[C18]}
\def\DBLPconficalpAminofRZS15{[C13]}
\def\DBLPconfcadeAminofR16{[C9]}
\def\AKRSV17{[J1]}
\def\DBLPseriessynthesis2015Bloem{[B1]}
\def\DBLPjournalssigactBloemJKKRVW16{[J6]}
\def\DBLPconfatalRubin15{[C16]}
\def\DBLPconfprimaRubinZMA15{[C12]}
\def\DBLPconfatalAminofMRZ16{[C7]}
% \def\DBLPconfprimaMuranoPR15{[C]}

\def\AR16{[J5]}
\def\traps13{[J7]}
\def\GMRS16IJCAI{[C10]}
\def\BDGRICAPS{[W1]}
\def\BDGR17{[C4]}

\def\GMPRW17{[C5]}
\author{}
\date{}

\begin{document}

\maketitle


%%%%%%%%%%%%%%%%%%%%%%%%%%%%%%%%%%%%%%%%%%%%%%%%%%%%%%%%%%%%%55
%%%%%%%%%%%%%%%%%%%%%%%%%%%%%%%%%%%%%%%%%%%%%%%%%%%%%%%%%%%%%55
% Why is my research important?
% How will I approach it?
% What are my long-term research goals?
% What are my career goals?
%%%%%%%%%%%%%%%%%%%%%%%%%%%%%%%%%%%%%%%%%%%%%%%%%%%%%%%%%%%%%55
%%%%%%%%%%%%%%%%%%%%%%%%%%%%%%%%%%%%%%%%%%%%%%%%%%%%%%%%%%%%%55
% 
% - SHOW OFF YOUR KNOWLEDGE IN FM 
% - I see lots of connections between logic based AI and FM.
% - I see that AI is looking at very crucial problems and these problems are not 
% the same, but they resonate with FM. these two fields developed separately, I think I am 
% the person that can show these connections.
% - show I want to collaborate with everybody. as with GDG, HG, ...
% - think how i interact with GDG not NM (somehow lucky that FM to MAS).
% - show why DEEP knowledge is needed; not just to know that there are connections. show there 
% are differences. superficially looks the same. 
%%%%%%%%%%%%%%%%%%%%%%%%%%%%%%%%%%%%%%%%%%%%%%%%%%%%%%%%%%%%%55
%%%%%%%%%%%%%%%%%%%%%%%%%%%%%%%%%%%%%%%%%%%%%%%%%%%%%%%%%%%%%55
% RESEARCH PLAN
% - start talking about AI
% - FM can help AI
% - show you are working FOR AI.
% - show how FM can help AI
% e.g., planning for TEG is based on FM
% e.g., generalised planning/synthesis based on FM
%%%%%%%%%%%%%%%%%%%%%%%%%%%%%%%%%%%%%%%%%%%%%%%%%%%%%%%%%%%%%55
%%%%%%%%%%%%%%%%%%%%%%%%%%%%%%%%%%%%%%%%%%%%%%%%%%%%%%%%%%%%%55


% % 10 key sentences
% % Key sentence 1, the Summary sentence
% % 
% %     This project will [your own description of  how it will make partial progress towards solving a  huge, important problem – in the example it’s “develop a potential solution”]
% %     by [ your own much more specific description of what it will actually do]
% %     [your own assertion that the project is novel or timely – e.g.”novel synthetic metabolic inhibitors”]
% %     [your own claim to “ownership” of the project – e.g.”that we have discovered”].
% % 

% \section*{Project Plan}

\todo{
Even if you are
not planning to do implementation yourself, talking about a collaboration with others
who are doing that might help.
}

\todo{structure the first page so that it reads as an executive summary of the
proposal that conveys to a *non-expert* reader an overview of what it is about, }



% \section*{ARC Future Fellowship}
% 
\subsection*{Part A}

\paragraph{Summary (max 750 chars)}
%INTRO< CONTEXT < OUTCOME < BENEFIT

AI systems are increasingly deployed in the world as \emph{agents}, 
e.g., software negotiating on our behalf on the internet, driverless cars, 
robots exploring dangerous environments, etc. There is a recently articulated need for humans to 
be able to \emph{trust} the decisions made by such artificial agents~\cite{ACMStatement07}. 
The goal of this project is to develop mathematical foundations and computational techniques for building trustworthy artificial agents, by leveraging the insights from  
recent results, developed by the proposer, on synthesis and strategic reasoning for single and multi-agent systems. The projected benefit will be techniques, useable by computer scientists and engineers, for building trustworthy agents in realms such as high-level robot control including lightweight swarms, concurrent manufacturing in industry 4.0, trustworthy social-media and -news bots, safe and secure could storage facilities.

\paragraph{Benefit and Impact (max 750 chars)}

The anticipated benefit of this project to science is that it will advance the state-of-the-art of the verification and synthesis of artificial agents. The potential impact to UNSW's Strategic Theme ``Future Intelligence'' will be tighter integration with world-renowned experts in AI and Autonomous Systems, including attracting short- and long-term leaders in Automated Planning and Knowledge Representation. Safer and securer interactions with artificial agents are in the interest of society. 

\newpage

\subsection*{Part C (max 10 pages)}
\paragraph{PROJECT TITLE}
\TITLE

\paragraph{AIMS AND BACKGROUND}

%  Briefly outline the aims and background of this Proposal.
%  Include information about national/international progress in this field of research and
% its relationship to this Proposal.
%  Refer only to publications or non-traditional equivalents (outputs) that are accessible
% to the national and international research communities


% %     
% % Key sentence 2, the Importance sentence
% % 
% %     The [huge important problem] is [your own statement that demonstrates with evidence that the problem is very important for one or more of health, society, the economy and the advance of knowledge and understanding];
% %     [your own statement that the project outcome will contribute to solving the huge important problem].

Systems built on the insights of AI are increasingly deployed in the world as \emph{agents}, 
e.g., software agents negotiating on our behalf on the internet, driverless cars, bots playing games with humans, 
robots exploring new and dangerous environments, etc. There is an obvious and recently articulated need for humans to 
be able to \emph{trust} the decisions made by such artificial agents, 
the need for {meaningful interactions} between humans and agents, and the need for {transparent} agents~\cite{ACMStatement07}. 

In other words, humans must be able to model, control and predict the \emph{behaviour} of agents. This challenge is made 
all the more complicated since:
\begin{itemize} 
 \item agents are often deployed with other agents leading to \emph{multi-agent systems},
%  \item agents may not know the structure of the system or environment, 
 \item agent behaviour is complex, and extends into the future, leading to \emph{temporal reasoning},
 \item agents are required to behave strategically, leading to \emph{strategic reasoning},
 \item agents may have uncertainty about the state, or even the structure, of other agents and the environment, leading to \emph{epistemic reasoning}.
\end{itemize}

The \textbf{aim} of this project is to develop mathematical foundations and computational techniques for building trustworthy artificial agents, by leveraging the insights from  
recent results (developed by the candidate) on synthesis and strategic reasoning for single and multi-agent systems.

% We need to establish meaningful classes of agents that are amenable to automatic computational analysis.
Specifically: 

\BOX{1. We need to discover meaningful \emph{new classes} of agents for which temporal-strategic-epistemic reasoning is decidable and tractable.}

It is known that synthesis for systems composed of multiple agents having imperfect information is \textbf{undecidable} (this has been discovered in the 
 multiple contexts, i.e., decentralised \pomdp s~\cite{DBLP:journals/mor/BernsteinGIZ02}, multiplayer non-cooperative games of imperfect information~\cite{peterson2001lower}, distributed synthesis~\cite{DBLP:conf/focs/PnueliR90}). 
Since the 1990s researchers have tried to find meaningful decidable fragments. The standard approach is to assume some sort of hierarchy on the information or 
observation sets, e.g.,~\cite{DBLP:conf/atva/BerwangerMB15}. Although mathematically elegant, the applicability of such assumptions is not very high. 
However, we recently defined and explored a class of games in which all agent moves are public, and proved that one can do analysis, i.e., the model-checking problem for such games against temporal-strategic-epistemic logics is \textbf{decidable and not harder than the case of perfect information}~\DBLPconfatalBelardinelliLMR17,\DBLPconfijcaiBelardinelliLMR17~\footnote{References in the form [C\#\#], [J\#\#] and [B\#\#] refer to conferences, journals and books by the candidate, and can be found in 
the ``References'' section of the accompanying CV.} (previously it was only known that, in a similar setting, one can do multi-agent epistemic planning~\cite{DBLP:conf/aips/KominisG15} and synthesis~\cite{vanderMeyden2005}). 
The importance of this result is that it lays the algorithmic and theoretical foundations for analysing temporal-strategic-epistemic properties of \emph{meaningful} classes of agents such as: % don't care what we can do already... what we gonna do
\begin{itemize}
%  \item various epistemic puzzles such as the 
%  muddy-children problem and the problem of russian cards, %% NOT STRONG POINT FOR PROPOSAL!
 \item various models of distributed computing using broadcast communication, and thus also formalisations of \emph{twitter}~\cite{DeNicola2015,DBLP:journals/jlp/MaggiPST17}. 
 \item various models of secure cloud-storage that use data-dispersal~\cite{DBLP:journals/internet/LiQLL16} and secret-sharing protocols~\cite{ADGH06}.
 \item various multi-player games such as poker, stratego and bridge~\cite{SCIENCEPAPER}.

%  \todo{another use case that people not in area of agents say ``yes, i like it!'' think big and crazy. then say something meaningful. be bold and concrete. car, hacker attack, something concrete that we could do in principle}
\end{itemize}

\BOX{2. We need to reduce temporal-strategic-epistemic reasoning about agents to \emph{scalable tools} such as classical and fully-observable non-deterministic planners developed in the Planning in AI community.}

Planning is a branch of AI that addresses the problem of generating a course of action to achieve
a desired goal, given a description of the domain of interest and its initial state. \textbf{Planning is a form of synthesis} 
that is central to the development of agents. Besides theoretical insights, Planning provides practical tools 
based on heuristic search and symbolic methods~\cite{GeffnerBo13}. The most successful of this technology 
is for ``classical planning'', i.e., single agent, deterministic environment, with perfect information, 
and simple reachability goals, and ``fully observable non-deterministic planning'' (which amounts to the case of one agent in an adversarial environment).
Previous work has reduced planning with temporal goals to classical and fully-observable 
nondeterministic planning~\cite{DBLP:conf/aaai/BaierM06,TorresB15,Camacho17}. This lays the foundation for refining and extending the translations to handle 
temporal-strategic-epistemic reasoning.

% In contrast, general forms of planning can be used to capture multiple agents, 
% imperfect information, incomplete information, and temporally extended goals.

% I propose to systematically study how to reduce behaviour synthesis to classical planning. This will be done in two steps:
% \begin{enumerate}
%  \item Study how to 
%  reduce behaviour synthesis to general forms of planning.
%  \item Study how to 
%  reduce general forms of planning to classical planning.
% \end{enumerate}
% 
% Both steps will be done using insights from automated synthesis~\cite{Vard96,KuVa97,DeGiacomoFPS10,DeVa15,DeVa16}, generalised planning~\cite{HuG11,DeGiacomoMRS16,BDGR17}, and reductions of planning with LTL-goals to classical planning~\cite{}. Moreover, the practical aspects of such reductions will be done 
% in collaboration with leading planning experts Hector Geffner and Blai Bonet. \aside{ask Hector/Blai}

% I plan to ground the practical considerations developed in the second phase to real application domains such as cognitive robotics, 
% multiplayer card games such as poker, analysis of multi-party computation.
% \aside{these can be done with people at UNSW}

 
%  \item We need to discover new ways of dealing with the state-explosion problem for systems with imperfect-information.
 
\BOX{3. We need to define, analyse, and tackle the problem of synthesising \emph{optimal strategies} in systems of agents with \emph{quantitative objectives}.}

 Previous work in planning focused on optimal strategies for MDPs and POMDPs, as well as optimal plans~\cite{GeffnerBo13,Penna15,TorralbaAKE17}.
 However, reasoning about multiple agents with possibly different but overlapping objectives requires richer solution concepts from \textbf{Game Theory}, e.g., Nash Equilibrium and Pareto optimality. The candidate recently introduced expressive temporal-strategic-epistemic logics that can be used to reason about equilibria in cases agents have Boolean objectives~\DBLPconfijcaiBelardinelliLMR17. This, together with recent insights from quantitative verification~\cite{DBLP:journals/jacm/AlmagorBK16}, lays the foundation for designing useful logics that can reason about systems in which agents have {quantitative objectives}.
 
 
 
%  \item We need to discover new parameters based on the recently introduced ``width'', that measure the ``complexity'' of the synthesis problem, and prove bounds on the computational complexity wrt these parameters, of synthesis problem.



\paragraph{FUTURE FELLOWSHIP CANDIDATE}

% For Candidates applying for Future Fellowship Level 1:
%  Describe the Future Fellowship Candidate’s research opportunity and performance
% evidence (ROPE).
%  Provide evidence that the Future Fellowship Candidate has the capacity and
% leadership to undertake the proposed research
%  Provide evidence that the Future Fellowship Candidate has a record of high quality
% Research Outputs appropriate to the discipline/s
%  Provide evidence the Future Fellowship Candidate’s research training, mentoring
% and supervision
%  Provide evidence of the Future Fellowship Candidate’s national research standing.

I have deep and extensive knowledge in logics for temporal, strategic and epistemic reasoning, automata-theory and synthesis. 
I have contributed foundational work on synthesis and graph-games~\AR16,\traps13 as well as 
on the connections between synthesis and general forms of planning~\GMRS16IJCAI,\BDGRICAPS,\BDGR17. 

A first step towards synthesis is usually verification, and I have contributed deep work on verification of multi-agent systems~\DBLPconfvmcaiAminofJKR14,\DBLPconfconcurAminofKRSV14,\DBLPconficalpAminofRZS15,\DBLPconfcadeAminofR16,\AKRSV17, including a book on the topic~\DBLPseriessynthesis2015Bloem,\DBLPjournalssigactBloemJKKRVW16, and with a focus on verification of parameterised systems \DBLPconfatalRubin15,\DBLPconfprimaRubinZMA15,\DBLPconfatalAminofMRZ16.

I already have close connections with international experts with expertise and interest in the topic of this project, i.e., 
including Giuseppe De Giacomo (knowledge representation, artificial intelligence, verification, synthesis), Hector Geffner and Blai Bonet (Planning), Moshe Vardi and Aniello Murano (logics for strategic reasoning, automata-theory for synthesis and verification). 

\paragraph{PROSPOSED PROJECT AND QUALITY INNOVATION}

%  Explain how the research addresses a significant problem.
%  Outline the conceptual/theoretical framework, and demonstrate that these are
% adequately developed, well integrated, innovative and original.
%  Explain how the aims, concepts, methods and results advance knowledge.
%  Describe how the design and methods are appropriate for the proposed research.
%  Describe how the proposed research may result in maximising economic,
% environmental, social, and/or cultural benefits to Australia. This statement should
% align with the Impact Statement.
%  If the research has been nominated as focussing on a topic or outcome that falls
% within one of the Science and Research Priorities, explain how it addresses one or
% more of the associated Practical Research Challenges (as selected in question B1 of
% this Proposal form).
%  Describe how the proposed Project involves interdisciplinary research, if appropriate.
%  Describe how the proposed Project will push the boundaries of research and open up
% new research opportunities.
%  Explain how the proposed Project will contribute to public policy formulation and
% debate.

The challenge of building trustworthy agents cannot be met without having some formal guarantees on their 
behaviour. The holy-grail is to automatically synthesise agent behaviour, or part of their behaviour, from 
specifications. This project will advance the state of the art of the mathematical foundations and computational techniques 
for building trustworthy agents from temporal-strategic-epistemic specifications. \todo{talk about specific outcomes}






% % Key Sentence 6, the Project Overview sentence
% % 	The proposed project will [general description of research activity] to [specific description of research outcome] in order to [weak statement indicating partial progress towards solution of huge important problem].
% % 
\paragraph{Objectives and methodology.}
The objectives of the project are to generate new mathematics, algorithms, and pragmatic techniques for describing, reasoning-about and building trustworthy  agents. This will be done using methods and insights from Logic and Formal Methods (and program synthesis in particular), and 
Game Theory (and its development in multi-agent systems). 

% This will be done by building on recent breakthroughs in reasoning about multi-agent systems~\cite{}, 
% 
% In particular, I propose to systematically study how to reduce temporal-strategic-epistemic reasoning 
% of multi-agent systems to these planning problems (e.g., I propose a careful study of the effects that structure of the 
% agent goals have on the resulting translations). This will be done using insights from distributed synthesis~\cite{Vard96,KuVa97,DeGiacomoFPS10,DeVa15,DeVa16} 
% and generalised planning~\cite{HuG11,DeGiacomoMRS16,BDGR17}. 
% 

% % Key Sentences 7-9: Sub-project Overview sentences
% % 	The research objectives are as follows:-
% % 	1) to [do the relevant research activity] in order to discover [the thing that we said we needed to know in the corresponding Aims sentence].
% % 	2) to [do the relevant research activity] in order to discover [the thing that we said we needed to know in the corresponding Aims sentence].
% % 	3) to [do the relevant research activity] in order to discover [the thing that we said we needed to know in the corresponding Aims sentence].

For instance: 
\begin{enumerate}
 \item in order to get new and richer decidable classes, I propose to \emph{generalise} systems in which all moves are public (which we recently explored~\DBLPconfatalBelardinelliLMR17,\DBLPconfijcaiBelardinelliLMR17), e.g., by incorporating stochastic initial states, by allowing a bounded number of private moves. 
 
 \item in order to achieve tractable classes of agents, I propose to restrict systems in which all moves are public to homogenous initial conditions~\cite{DBLP:journals/tocl/LomuscioMR00} and/or bounded-epistemic states.
 
%  \item explore the middle-ground between belief-space (which is exponentially large but accurate) and observation-space (which is linear but coarse) using trajectory constraints (which we recently pioneered~\cite{}), in order to find new ways of dealing with the state-explosion problem for systems of agents with imperfect-information.
%  
 \item in order to reason about optimal strategies, I propose to enrich the models with costs/rewards and analyse these with new measures of strategy-quality.
 \end{enumerate}


% % 
% % 	
% % Key sentence 10, the Dissemination sentence.




\paragraph{FEASABILITY AND STRATEGIC ALIGNMENT}

 Describe the extent to which the Future Fellowship Candidate aligns with and/or
complements the core or developing research strengths and staffing profile of the
Administering Organisation.
 Demonstrate that the necessary facilities are available to conduct the proposed
research.
 Outline what resources will be provided by the Administering Organisation to support
the Future Fellowship Candidate during her/his Future Fellowship.
 At the end of the Future Fellowship, explain what capacity exists at the Administering
Organisation to transition the Candidate to a continuing position.


\paragraph{BENEFIT AND COLLABORATION}

%  Describe how the Future Fellowship Candidate will build collaborations across
% research organisations and/or industry and/or with other disciplines both within
% Australia and internationally.
%  Explain how the Host Organisation(s) will be utilised in the proposed Project, if
% relevant.
%  Outline how the completed project will produce significant new knowledge and/or
% innovative economic, commercial, environmental, social and/or cultural benefit to the
% Australian and international community.
%  Describe how the proposed research will be cost-effective and value for money

This project relies on the hypothesis that, in order to synthesise and verify trustworthy agents, one needs to buid a model of the agent. The model-based approach to controller design underlies automated planning in AI~\cite{GeffnerBo13}, reactive synthesis in programming verification~\cite{DBLP:journals/jcss/BloemJPPS12}, generalised game playing~\cite{GGP}, Bayesian networks and decision graphs~\cite{Jensen2002}, etc. This is in contrast to the model-free approach as epitomised by the recent commercial application of neural networks to vision problems. The key issue in AI is not whether we should use the model-based or the model-free approach. Rather, the challenges are to gain \emph{understanding} of the limitations of each technique. 


% Although the alternative model-free approach, epitomised by neural networks and deep-learning, has had impressive success recently, we scientists should not be ``bullied by success'' and repeat the mistakes that led to the AI winter~\cite{}. Indeed, the model-free approach is impressive in the following restricted setting: it has not been stunningly applied to cognitive processes (the famous alphaGO makes use of a sleuth of AI techniques, including model-based and model-free approaches), and it is successful according to the measure that it is useful (indeed, the learned functions are crude approximations which, although useful and practical, do not compare well to human performance)~\cite{}. 


Thus, the potential impact of this project is that it will \emph{advance our scientific understanding of the reach and limitations of the model-based approach} for designing and reasoning-about the behaviour of artificial agents.


\paragraph{COMMUNICATION OF RESULTS}


Outline plans for communicating the research results to other researchers and the
broader community, including but not limited to scholarly and public communication
and dissemination

\paragraph{MANAGEMENT OF DATA}
 Outline plans for the management of data produced as a result of the proposed
research, including but not limited to storage, access and re-use arrangements.
 It is not sufficient to state that the organisation has a data management policy.
Researchers are encouraged to highlight specific plans for the management of their
research data.

\paragraph{REFERENCES}
\bibliographystyle{abbrv}
\bibliography{researchplan,FWF,References,rubin}

% \paragraph{ACKNOWLEDGEMENTS (IF REQUIRED)}



% % 
% %     
% % Key Sentences 3-5, the Aims sentences
% % 	We need to [know or establish or develop]+[your own statement of whatever the sub-project is going to discover or establish or develop].
% % 	We have three (specific) aims:-
% % 	1) to [discover, develop or establish] [your statement of whatever the first sub-project is going to discover, develop or establish]
% % 	

 

\paragraph{Integration at UNSW}

There are clear connections between the topic of this project (behaviour synthesis for agents) and the work 
being done at UNSW. 

% I plan to ground the practical considerations developed in the second phase to real application domains such as cognitive robotics, 
% multiplayer card games such as poker, analysis of multi-party computation.
% \aside{these can be done with people at UNSW}


% 
% Epistemic protocol specifications allow programs, for settings in which multiple agents act with
% incomplete information, to be described in terms of how actions are related to what the agents know.
% They are a variant of the knowledge-based programs of Fagin et al [Distributed Computing, 1997],
% motivated by the complexity of synthesizing implementations in that framework. The paper proposes
% an approach to the synthesis of implementations of epistemic protocol specifications, that reduces the
% problem of finding an implementation to a sequence of model checking problems in approximations
% of the ultimate system being synthesized. A number of ways to construct such approximations is
% considered, and these are studied for the complexity of the associated model checking problems.
% The outcome of the study is the identification of the best approximations with the property of being
% PTIME implementable.
% 
In particular:
\begin{enumerate}
\item The theory of Reasoning about Knowledge, as applied to distributed computing~\cite{FHMV95}, can be used to formally specify, verify and synthesise artificial-agents 
that act with incomplete and imperfect information. \textbf{Van der Meyden} studies synthesis of epistemic protocol specifications~\cite{DBLP:conf/concur/MeydenV98,DBLP:journals/corr/HuangM16} and symbolic implementations of model-checkers for epistemic strategic logics~\cite{DBLP:conf/aaai/HuangM14}.

 \item General game playing, see \textbf{Thieschler~\cite{GGP}}, is a framework in which programs learn to play games given just a description of the rules of the game. Although related to synthesis, it emphasises an online approach in which the solver is given bounded time to suggest its next move (in this sense it is related to online planning, e.g.,~\cite{GeffnerBo13}). This is in contrast to the classic synthesis approaches in Formal Methods which are offline and generate a policy before execution. 
 
 \item Architectures used in robotics to capture mental states  correspond to a first-person view of agents~\cite{reiter2001knowledge}. Recent work at UNSW aims to formalise an architecture so that formal-methods can be applied, see~\textbf{Rajaratnam, Hengst, Pagnucco, Sammut, Thielscher~\cite{Rajaratnam2016}}.
 
 \item The aim in information flow security is to design observable program behaviour that does not reveal, to an adversary, secret information about its' state. Formal methods for the design and verification of such systems is studied by \textbf{Morgan~\cite{McIver2011}} and \textbf{van der Meyden~\cite{DBLP:conf/sp/EggertMSW11,DBLP:journals/tcs/CassezMZ16}}.  
 Definitions of information flow security serve as important specification of multi-agent system behaviour, e.g., in secure multi-party communication. 
%  Moreover, insights from information flow security will be useful in understanding how strategies of different agents signal private information to other agents. 
% Note that the latter is in some sense the dual problem: how can one define strategies that *do* leak enough private information that the distributed players can co-ordinate and achieve a joint objective.

 \item \todo{Walsh? Aziz?}
 
 \end{enumerate}
 
 
 \newpage
 
\end{document}
% 
%  
%  \newpage
%  
%  
% 
% 
% 
% \section{Goal}
% 
% The goal of this work is to advance the mathematical and computational foundations for supplying building 
% agents we can trust. 
% 
% This challenge cannot be met without having some formal guarantees on the 
% behaviour of agents.
% 
% of formal methods that are used 
% to supply formal guarantees on the for  and practical techniques for 
% 
% Systems built on the insights of AI are increasingly deployed in the world as \emph{agents}, 
% e.g., software agents on the Internet, driverless cars, software that play and compete with humans at games, 
% robots exploring new and dangerous environments, etc. There is a need for humans to be able to {trust} the decisions made by such agents, 
% the need for {meaningful interactions} between humans and agents, and the need for {transparent} agents~\cite{ACMStatement07}. 
% In other words, humans must be able to model, control and predict the \emph{behaviour} of agents.
% 
% 
% 
% \section{Objective}
% 
% I propose developing formal methods for the construction and analysis of trustworthy behaviour of 
% agents. 
% To do this, I will focus on logical and algorithmic tools for the specification and synthesis 
% of such agents. 
% 
% There are many factors that need to be accounted for: 
% \begin{enumerate} 
%  \item agents are often deployed with other agents, 
% %  \item agents may not know the structure of the system or environment, 
%  \item agents may have uncertainty about the state of other agents, or of the environment, 
%  \item agent behaviour is complex, and extends into the future.
% \end{enumerate}
% 
% \aside{running illustrative example?}
% 
% \section{State of the art and Methodology}
% 
% Game theory provides an abstract and sound mathematical framework for reasoning about complex situations involving multiple agents. 
% A situation is modeled as a game, the participating agents are called players, and their behaviour is modeled by strategies. Such games 
% are typically multiplayer, %of incomplete information (agents may not know the game-space), 
% of imperfect information (even if agents know the game-space, 
% they may not know the current state at runtime), and agents have temporally extended goals. Each of these factors adds another layer of complexity to 
% the problem of automatically synthesising agents. 
% 
% % strategic interactions between 
% % agents and amongst groups of agents. 
% % Modeling such situations for agents typically requires game models with the following characteristics: 
% % \begin{enumerate} 
% % \item games are multi-player (``no agent is an island''), 
% % \item games may be of incomplete information, i.e., agents may not know the structure of the system (the state-space),
% % \item games are of imperfect information; even if agents have certainty about the structure of the system, they may not know exactly which state the system is in,
% % \item agents have complex goals that extend into the future. 
% % \end{enumerate}
% 
% My starting point in attacking these problems is the \emph{Theory of Games on Graphs}: indeed, the ``arena'' on which the game is played is a graph whose vertices are the states of the system, and whose transitions result from actions of the players.  In the \emph{Formal Methods} community, logical and automata-theoretic techniques are used for modeling games, specifying goals, and synthesising strategies. I will pursue two directions.
% 
% \subsection{Theoretical Foundations} 
% 
% I will use and extend this framework to identify models of games that are both meaningful 
% (in that they describe scenarios involving agents) and tractable (in that there are precise low-complexity algorithms 
% for doing synthesis and analysis). 
% 
% 
% 
% As a starting point, it is known that solving multi-player graph games of imperfect information is undecidable (this has been discovered in the 
% multiple contexts, i.e., decentralised \pomdp s~\cite{DBLP:journals/mor/BernsteinGIZ02}, multiplayer non-cooperative games of imperfect information~\cite{DBLP:conf/focs/PetersonR79,peterson2001lower}, distributed synthesis~\cite{DBLP:conf/focs/PnueliR90}). Since the 1990s researchers 
% have tried to find meaningful decidable fragments. The standard approach is to assume some sort of hierarchy on the information or observation sets, e.g., ~\cite{DBLP:conf/lics/KupfermanV01,DBLP:conf/atva/BerwangerMB15,BMMRV17}. The applicability 
% of such assumptions is not very high. However, we recently discovered a class \pubact of games in which all agent moves are public, and proved that 
% one can do analysis, i.e., the model-checking problem for games in \pubact against temporal-strategic-epistemic logics is decidable~\cite{DBLP:conf/atal/BelardinelliLMR17,DBLP:conf/ijcai/BelardinelliLMR17} (previously it was only known that one can do synthesis~\cite{vanderMeyden2005}). The importance of this result 
% is that it lays the algorithmic and theoretical foundations for analysing complex strategic properties of agents. Indeed, \pubact can model 
% \begin{itemize}
%  \item various multi-player games such as poker, stratego and bridge, 
%  \item various epistemic puzzles such as the muddy-children problem and the problem of russian cards, 
%  \item various models of distributed computing such as broadcast communication, and thus also idealisations of \emph{twitter}~\cite{DeNicola2015}. 
% \end{itemize}
% I plan to make the following theoretical studies:
% \begin{enumerate}
%  \item Extend the class \pubact, e.g., by allowing some initial non-public actions, and by allowing probabilistic moves.
%  \item Find meaningful ways to restrict \pubact to get better complexity, e.g., if the size of the information 
%  sets is bounded, as in simplified versions of the co-operative card game Hanabi, or if there is some homogeneity in the initial information sets~\cite{DBLP:journals/tocl/LomuscioMR00}.
%  \item Find other classes with similar properties, i.e., that can model meaningful agents and are tractable.
% \end{enumerate}
% 
% % Artificial agents may have dynamic goals, i.e., goals are produced continuously as the agent operates. I plan to incorporate 
% % such dynamicity in game-theoretic models.
% % as the agent operates. However, while in Planning a plan for the next goal must be built only after
% % the current goal is achieved, mechanisms have the option to drop the current goal or combine it with
% % the next one, thus adding a complication that cannot be dealt with straightforwardly by standard
% % Planning techniques.
% % 
% 
% \subsection{Practical considerations}
% 
% In this phase I will explore practical methods for doing synthesis and analysis of systems consisting of agents. 
% 
% The closest techniques that have proven to be scalable come from the \emph{Planning in AI} community. Planning is a branch of 
% AI that addresses the problem of generating a course of action to achieve
% a desired goal, given a description of the domain of interest and its initial state. The area is central
% to the development of agents. Besides theoretical insights, Planning provides practical tools 
% based on heuristic search and symbolic methods~\cite{GeffnerBo13}.The most successful of this technology 
% is for ``classical planning'', i.e., single agent, deterministic environment, with perfect information, 
% and simple reachability goals. In contrast, general forms of planning can be used to capture multiple agents, 
% imperfect information, incomplete information, and temporally extended goals.
% 
% I propose to systematically study how to reduce behaviour synthesis to classical planning. This will be done in two steps:
% \begin{enumerate}
%  \item Study how to 
%  reduce behaviour synthesis to general forms of planning.
%  \item Study how to 
%  reduce general forms of planning to classical planning.
% \end{enumerate}
% 
% Both steps will be done using insights from automated synthesis~\cite{Vard96,KuVa97,DeGiacomoFPS10,DeVa15,DeVa16}, generalised planning~\cite{HuG11,DeGiacomoMRS16,BDGR17}, and reductions of planning with LTL-goals to classical planning~\cite{}. Moreover, the practical aspects of such reductions will be done 
% in collaboration with leading planning experts Hector Geffner and Blai Bonet. \aside{ask Hector/Blai}
% 
% I plan to ground the practical considerations developed in the second phase to real application domains such as cognitive robotics, 
% multiplayer card games such as poker, analysis of multi-party computation.
% \aside{these can be done with people at UNSW}
% 
% \section{Feasibility}
% 
% It has been noticed that synthesis in verification and planning in AI are similar: they are both about model-based controller design, both 
% use succinct and expressive representations (STRIPS and PDDL in planning, and LTL in synthesis).
% However, there are important differences which need to be understood in order to make deep connections between these areas, i.e., 
% \begin{enumerate}
%  \item Planning typically deals with reachability goals on compact domains, while synthesis deals with LTL goals on explicit (or LTL-represented) domains.
%  \item The computational-complexity of doing synthesis is PSPACE-complete in planning and 2EXPTIME-complete in synthesis. This is due to the succinctness of LTL. The Planning community responded to the high complexity by finding ways (e.g., heuristic search~\cite{GeffnerBo13}) to treat the ``easy``, but large, cases. In contrast, the synthesis community 
%  responded to the high complexity by finding fragments of LTL (e.g., GR1~\cite{BloemJPPS12}) for which synthesis has lower computational complexity, and providing algorithms (mainly symbolic~\cite{FiliotJR11,Vardi-symbolic17}) for ameliorating the high-complexity.
% \end{enumerate}
% 
% Some work~\cite{DBLP:conf/aaai/BaierM06,DBLP:conf/ijcai/TorresB15,Camacho17} has studied some translations between planning and synthesis, e.g., LTL reactive synthesis is reduced to classical or non-deterministic planning via automata. \aside{say more} The closeness of the topics and the state of the art shows that translations are possible, and that a systematic study is feasible.
% 
% \subsection{Suitability of candidate}
% 
% I have deep and extensive knowledge in logics for knowledge and strategic reasoning, automata-theory and synthesis. I have contributed foundational work on synthesis and 
% graph-games~\cite{AR16,traps13}, on the connections between synthesis and general forms of planning~\cite{GMRS16IJCAI,BDGR17ICAPS,BDGR17}. A first step towards synthesis is usually verification, and I have contributed deep work on verification of multi-agent systems~\cite{DBLP:conf/vmcai/AminofJKR14,DBLP:conf/concur/AminofKRSV14,DBLP:conf/icalp/AminofRZS15,DBLP:conf/cade/AminofR16,AKRSV17}, including a book on the topic~\cite{DBLP:series/synthesis/2015Bloem,DBLP:journals/sigact/BloemJKKRVW16}, and with a focus on incomplete information~\cite{DBLP:conf/atal/Rubin15,DBLP:conf/prima/RubinZMA15,DBLP:conf/atal/AminofMRZ16,DBLP:conf/prima/MuranoPR15}.
% 
% I already have close connections with international experts with expertise and interest in the topic of this project, i.e., 
% including Giuseppe De Giacomo (knowledge representation, artificial intelligence, verification, synthesis), Hector Geffner and Blai Bonet (Planning), Moshe Vardi and Aniello Murano (logics for strategic reasoning, automata-theory for synthesis and verification).
% 
% 
% % 
% % 
% % Second, the agents may be operating in a \emph{partially-known environment}. For instance, the agents may know they are in a ring, but may not know the size of the ring. I launched the application of automata theory for the verification of high-level properties of light-weight mobile agents in partially-known environments~\cite{DBLP:conf/atal/Rubin15}. In follow-up work I explored this theme further, including finding ways to model agents on grids --- the most common abstraction of 2D and 3D space~\cite{DBLP:conf/prima/RubinZMA15,DBLP:conf/atal/AminofMRZ16,DBLP:conf/prima/MuranoPR15}.
% % \newline  
% 
% 
% % 
% %  
% % \section{Foundations of Automated Planning}
% % Planning in AI can be viewed as the problem of finding strategies in one- or two-player graph-games. In this model vertices represent states, edges represent transitions, and the players represent the agents. I have contributed foundational work to such games. Concretely, I recently extended the classic belief-space construction for games of imperfect-information from finite arenas to infinite-arenas~\cite{GMRS16IJCAI} (infinite arenas often arise in the study of MAS with incomplete information, see above). I have also used these ideas to elucidate the role of observation-projections in generalised planning problems~\cite{BDGR17ICAPS,BDGR17}.
% % I have generalised classic results about certain games with quantitative objectives (i.e., Ehrenfeucht and J. Mycielski. Positional strategies for mean payoff games. International Journal of Game Theory, 8:109--113, 1979) to so-called first-cycle games, i.e., games in which play stops the moment a vertex is repeated~\cite{AR16}. 
% % \newline
% 
% % Although logical and automata-theoretic techniques are standard in reactive-program synthesis, they are not yet standard in 
% % automated planning and multi-agent systems in AI. On the other hand, the most successful area in AI that does automated synthesis of agent 
% % behaviour is \emph{planning}~\cite{BG}. This area focuses on having human-comprehensible descriptions of the scenario 
% % (environment, agent capabilities, agent goals) and has developed principles, techniques and tools for the automatic synthesis 
% % of agents.  I propose to exploit insights from both reactive-program synthesis and planning in AI.
% % 
% 
% % My work consists of developing and applying formal methods to modeling and reasoning about computational systems, often involving strategic behaviour of agents. 
% % This is motivated by ``explainable AI'', the need to ensure that the systems being built using, e.g.,  machine learning, can explain their decisions and actions to human users. 
% % 
% % I approach these questions by extending classic techniques in formal methods, such as the use of games on graphs and reductions to automata, and applying theoretical insights to foundational problems in AI. One notable example of this convergence is my recent IJCAI17 paper~\cite{BDGR17} which is at the intersection of formal methods and automated planning  in AI. I am also pursuing more speculative questions such as ``What is synthesis and how should it be formalised?".
% 
% % 
% % \section{Connections between planning (in AI) and synthesis (in FM)}
% % 
% % \emph{Synthesis} is a cornerstone of FM that addresses the problem of automatically producing systems that 
% % meet a given specification. 
% % 
% % There are many similarities between these two areas, but also important differences.
% % 
% % \section{Similarities}
% % Both Planning and Synthesis are model-based controller design. That is, ...
% % 
% % 
% % \section{Differences}
% % 
% % 
% % goals are dynamic, that is, goals are produced continuously,
% % as the agent operates. However, while in Planning a plan for the next goal must be built only after
% % the current goal is achieved, mechanisms have the option to drop the current goal or combine it with
% % the next one, thus adding a complication that cannot be dealt with straightforwardly by standard
% % Planning techniques.
% % 
% % \emph{planning} in AI and \emph{reactive synthesis} in formal methods 
% % \en
% % \- same basic idea... model-based controller design
% % \- central to both are succinct representations of the systems: STRIPS, PDDL, in planning and LTL/LTLf in synthesis.
% % but for different reasons! planning typically deals with reachability goals on compact domains; synthesis with LTL goals on explicit domains.
% % \- both understood early on the computational complexity of the problem PSPACE-complete ... 2EXPTIME-complete
% % \- planning responded by finding ways to treat the "easy but large cases", e.g., heuristic search
% % \- reactive synthesis responded by studying the theory of the problem (e.g., fragments, extensions), and providing algorithms based on logic and automata theory (antichain).
% % \- some work (Sheilah...) has studied translations between these, i.e., LTL --> AFW/NFW --> planning domain
% % \ne
% 
% 
% %(an active topic of investigation with Giuseppe De Giacomo).
% 
% 
% % 
% % \emph{Formal methods (FM)} is an umbrella-term that describes principles and techniques for reasoning 
% % about systems with some digital component, such as software, hardware, cyber-physical systems, etc.
% % \begin{framed}
% % \noindent FM is built on three pillars: modeling, verification, and synthesis.  
% % \end{framed}
% % All three pillars are founded 
% % on techniques from mathematical logic. Logic is used to formally represent aspects of the system so that both expert humans and computers can 
% % manipulate them and reason about them.
% 
% 
% 
% % 
% % 
% % 
% % \section{Current Research --- Formal methods for multi-agent systems}
% % Multi-agent systems (MAS) involve multiple individual agents (these may be people, software, robots) each with their own goals. Such systems can be viewed as multi-player games, and thus notions from game-theory (e.g., strategies, knowledge, and equilibria) are used to reason about them. Agents in realistic MAS often lack information about other agents and the environment, and this is often categorised in one of two ways: a) \emph{incomplete information} and b) \emph{imperfect information}.
% % \newline
% % 
% % \section{a) MAS with incomplete information}
% % Incomplete-information refers to uncertainty about the environment (i.e., the structure of the game). I have considered two sources of incomplete information for MAS.
% % \newline
% 
% % 
% % First, the \emph{number of agents} may not be known, or may not be bounded a priori.
% % In a series of papers, I have contributed to a generalisation of a cornerstone paper on verification of such systems (``Reasoning about Rings'', E.A. Emerson, K.S. Namjoshi, 
% % \textsc{POPL}, 1995) from ring topologies to arbitrary topologies \cite{DBLP:conf/vmcai/AminofJKR14,DBLP:conf/concur/AminofKRSV14,DBLP:conf/cade/AminofR16,AKRSV17}. Other work on this topic 
% % studied the relative power of standard communication-primitives assuming an unknown number of agents~\cite{DBLP:conf/lpar/AminofRZ15}, as well as the complexity of model-checking timed systems assuming an unknown number of agents~\cite{DBLP:conf/icalp/AminofRZS15}. I also contributed to a book on this topic published by Morgan\&Claypool in 2015~\cite{DBLP:series/synthesis/2015Bloem,DBLP:journals/sigact/BloemJKKRVW16}.
% % \newline
% 
% % 
% % 
% % Second, the agents may be operating in a \emph{partially-known environment}. For instance, the agents may know they are in a ring, but may not know the size of the ring. I launched the application of automata theory for the verification of high-level properties of light-weight mobile agents in partially-known environments~\cite{DBLP:conf/atal/Rubin15}. In follow-up work I explored this theme further, including finding ways to model agents on grids --- the most common abstraction of 2D and 3D space~\cite{DBLP:conf/prima/RubinZMA15,DBLP:conf/atal/AminofMRZ16,DBLP:conf/prima/MuranoPR15}.
% % \newline  
% % 
% % \section{b) MAS with imperfect information}
% % Even if agents have certainty about the structure of the system, they may not know exactly which state the system is in. This is called imperfect information and the associated logic for reasoning about such cases are called \emph{epistemic}. I have studied strategic-epistemic logics in a number of works, namely, with a prompt modality (thus allowing one to express that a property holds ``promptly'' rather than simply ``eventually'')~\cite{DBLP:conf/kr/AminofMRZ16}, and on systems with public-actions (such as certain card games, including a hand of Poker or a round of Bridge)~\cite{BLMR17,BLMR17IJCAI}. The importance of these last works is that they give the first decidability (and sometimes optimal complexity) results for strategic reasoning about games of imperfect information in which the agents may have arbitrary observations. In contrast, following classical restrictions on the observations or information of agents, I have also shown how to extend strategy logic by epistemic operators and identified a decidable fragment in which one can express equilibria concepts~\cite{BMMRV17}.
% % \newline
% 
% % % \section{Logics with Counting Quantifiers}
% % % % Many logics in computer science do not have the ability to count. However, besides being a basic operation, counting allows one to describe finer details of a system.
% % % I have a long-standing interest in logics with quantifiers that count. E.g., the usual first-order quantifier $\exists x$ can be generalised to the counting quantifier $\exists^{\geq k} x$ which says that ``there are at least $k$ many $x$''. Concretely, I have studied logics that count strategies~\cite{AMMR16-SR,DBLP:conf/atal/AminofMMR16}, paths~\cite{DBLP:conf/lpar/AminofMR15}, strings and sets~\cite{DBLP:journals/bsl/Rubin08,DBLP:conf/stacs/KaiserRB08}. With a PhD student of Erich Gr\"adel's (Tobias Ganzow) I
% % % solved a 12 year-old conjecture of Courcelle's on the relationship between order and counting on graphs~\cite{DBLP:conf/stacs/GanzowR08}. I recently established and studied a logical formalism, called ``graded strategy-logic'', that is rich enough to count equilibria \cite{AMMR16-SR,DBLP:conf/atal/AminofMMR16}. The importance of this result to equilibrium selection is that it gives a computational way to decide if a given game has, e.g., a unique Nash equilibrium. 
% 
% % 
% %  
% % \section{Foundations of Automated Planning}
% % Planning in AI can be viewed as the problem of finding strategies in one- or two-player graph-games. In this model vertices represent states, edges represent transitions, and the players represent the agents. I have contributed foundational work to such games. Concretely, I recently extended the classic belief-space construction for games of imperfect-information from finite arenas to infinite-arenas~\cite{GMRS16IJCAI} (infinite arenas often arise in the study of MAS with incomplete information, see above). I have also used these ideas to elucidate the role of observation-projections in generalised planning problems~\cite{BDGR17ICAPS,BDGR17}.
% % I have generalised classic results about certain games with quantitative objectives (i.e., Ehrenfeucht and J. Mycielski. Positional strategies for mean payoff games. International Journal of Game Theory, 8:109--113, 1979) to so-called first-cycle games, i.e., games in which play stops the moment a vertex is repeated~\cite{AR16}. 
% % \newline
% 
% % \end{abstract}
% 
% % % http://www.cs.rice.edu/~vardi/comp409/history.pdf
% % A running theme in my work is the development of logical formalisms for describing and reasoning about objects of interest to computer scientists, from the abstract (e.g., graphs, algebras, orders) to the concrete (e.g., multiplayer games). It is often said that ``logic is the calculus of computer science''~\cite{}. Moshe Vardi has said, of computer science, that ``description is our business''~\cite{}. Seen in this light, my work is of a foundational nature: it sheds light on 
% 
% 
% % 
% % \section{Past Research --- Algorithmic Model Theory}
% % My early work contributed to a research program called ``Algorithmic  Model Theory" whose aim is to develop and extend the success of Finite Model Theory to infinite structures that can be reasoned about algorithmically. 
% % \newline
% % 
% % Specifically, my PhD work pioneered the development of ``automatic structures'': this is a generalisation of the regular languages from sets to mathematical objects with structure, such as graphs, arithmetics, algebras, etc.  The fundamental property of automatic structures is that one can automatically answer logic-based queries about them (precisely, their first-order theory is decidable). I gave techniques for proving that structures are or are not automatic (similar to, but vastly more complicated than, pumping lemmas for regular languages), I studied the computational complexity of deciding when two automatic structures are the same (isomorphic), and I found extensions of the fundamental property, thus enriching the query language \cite{BGR11,DBLP:conf/lics/IshiharaKR02,DBLP:conf/lics/KhoussainovNRS04,DBLP:journals/lmcs/KhoussainovNRS07,DBLP:conf/lics/KhoussainovRS03,DBLP:conf/stacs/KhoussainovRS04,DBLP:journals/tocl/KhoussainovRS05,DBLP:journals/bsl/Rubin08}. I have also worked on extensions of automatic structures to include oracle computation \cite{DBLP:journals/corr/abs-1210-2462,DBLP:conf/lics/RabinovichR12}.
% % \newline
% 
% 
% % 
% % \section{Short-term trajectory}
% % 
% % I recently organised the first workshop on formal methods in artificial intelligence (FMAI) 2017. In the next few years I plan to further integrate into the AI community, and the MAS community specifically. Concretely, I plan to study more richer \emph{models of systems} (rather than richer logics), including finer representations of time, bounded-memory strategies, and probabilistic arenas and strategies.
% 
% % traps: \cite{DBLP:journals/tcs/GrinshpunPRT14}
% 
% % planning: \cite{DBLP:conf/prima/MuranoPR15}
% 
% % \section{Misc}
% % PROMPT: \cite{DBLP:conf/kr/AminofMRZ16}
% 
% 
% 
% % 
% % Probabilistic: \cite{DBLP:conf/cav/BustanRV04}
% %  A fundamental problem in computer science is that of ensuring that a system
% %  satisfies a particular property. Moshe Vardi, Doron Bustan and I \cite{BRV04}
% %  considered the complexity of checking that a probabilistic system (modeled by a
% %  finite-state discrete-time Markov chain) satisfies properties expressed by
% %  automata operating on infinite words. The sorts of properties that can be
% %  expressed extend those of linear temporal logic, a typical example is `Does the
% %  Markov chain almost surely enter this state infinitely often'? We presented an
% %  optimal algorithm that checks whether a given Markov chain satisfies a
% %  specification given by an alternating B\"uchi automaton, thus extending known
% %  work on linear temporal logic \cite{CoYa90}.
%  
% % \small
%  
% %  \pagebreak
% %  
% % \bibliographystyle{plain}
% % \bibliography{/home/sr/svn/forsyte-publications/trunk/rubin.bib, otherbib}
% % 
% % \end{document}
% 
% 
% 
% % 
% % There are many connections between AI and FM. Som
% % \section{Planning in AI and Synthesis in FM}
% % 
% % \emph{planning} in AI and \emph{reactive synthesis} in formal methods 
% % \en
% % \- same basic idea... model-based controller design
% % \- central to both are succinct representations of the systems: STRIPS, PDDL, in planning and LTL/LTLf in synthesis.
% % but for different reasons! planning typically deals with reachability goals on compact domains; synthesis with LTL goals on explicit domains.
% % \- both understood early on the computational complexity of the problem PSPACE-complete ... 2EXPTIME-complete
% % \- planning responded by finding ways to treat the "easy but large cases", e.g., heuristic search
% % \- reactive synthesis responded by studying the theory of the problem (e.g., fragments, extensions), and providing algorithms based on logic and automata theory (antichain).
% % \- some work (Sheilah...) has studied translations between these, i.e., LTL --> AFW/NFW --> planning domain
% % \ne
% % 
% % What's missing from within this picture?
% 
% % \en
% % \-  formal connections between the two fields: e.g., reducing synthesis to planning (cf Sheilah's work).
% % \- formal connections within planning problems: e.g., reduce LTLf planning to reachability planning...
% % \- clear idea of how one can exploit modern planners and heuristic methods to solve problems in automata in practice, e.g., do domain-independent 
% % heuristics work LTLf/LDLf/LTL...? perhaps "LTL-dependent heuristics" should be studied...
% % \- 
% % \ne
% 
% 
% 
% % 
% % How can I complement research at UNSW.
% % 1. Theoretical foundations of strategic epistemic reasoning in complex environments.
% % MIT: Strategic Reasoning and Planning for General Game-Playing Robots (Australia-Germany Joint Research Cooperation Scheme 2016-2017)
% % MIT: Universal Game-Playing Systems for Randomised and Imperfect-Information Games (ARC-DP 2012-2015)
% % 
% % 2. Theory of distributed synthesis (information forks, automata theory for controller synthesis) 
% % could be used to analyse the cognitive meta-hierarchy of 
% % David Rajaratnam
% % Bernhard Hengst
% % Maurice Pagnucco
% % Claude Sammut
% % Michael Thielscher
% 
% 
% 
% % How my work can be complemented by work at UNSW:
% % 1. *unstructured* and incomplete environments are central to many robotic applications (e.g., rescue robots). Adapting definitions and results in reactive synthesis to such a setting is a clear and present challenge.
% 
% 
% 
% 
% % Project Offers
% % ==============
% % 
% % 
% % 
% % ???
% % When faced with a dynamical sys-
% % tem that you want to simulate, control, analyze, or otherwise investigate, first axiomatize
% % it in a suitable logic. Through logical entailment, all else will follow, including system
% % control, simulation, and analysis.
% % 
% % 
% % 
% % - games of incomplete information, imperfect information
% % - epistemic planning
% % 
% % concretely:
% % probabilistic DEL with public announcements
% % PATL* on broadcast iCGS
% % 
% % Quantitative SL: add weights to the arena (e.g., to actions or to
% % states), and add atomic formulas to the logic of the form "the
% % mean-payoff for player i is at least c".
% % 
% % Question: is model-checking decidable if we add these to ATL? ATL*? SL?
% % 
% % --
% % controller manages a collection of programmable mechanisms
% % - monitors and responds to events
% % (e.g., shifts in load, certain specifications fail, ...)
% % 
% % - reprogram mechanisms on the fly
% % (e.g., change ???
% % 
% % 
% % controller is centralised (1 agent vs 1 environment)
% % 
% % FSMs (1) intuitively and concisely capture control dynamics in response to network events; and (2) their structure makes them amenable to verification.
% % 
% % what are the external events?
% % timing, 
% % 
% % what is the current way to solve the problems that whitemech would solve.
% 
% 
% 
