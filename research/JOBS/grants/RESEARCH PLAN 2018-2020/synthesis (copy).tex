\documentclass[10pt,a4paper,sans]{moderncv}       
% possible options include font size ('10pt', '11pt' and '12pt'), paper size ('a4paper', 'letterpaper', 'a5paper', 'legalpaper', 'executivepaper' and 'landscape') and font family ('sans' and 'roman')
% moderncv themes
\moderncvstyle{classic}                             % style options are 'casual' (default), 'classic', 'banking', 'oldstyle' and 'fancy'
\moderncvcolor{blue}                               % color options 'black', 'blue' (default), 'burgundy', 'green', 'grey', 'orange', 'purple' and 'red'
%\renewcommand{\familydefault}{\sfdefault}         % to set the default font; use '\sfdefault' for the default sans serif font, '\rmdefault' for the default roman one, or any tex font name
%\nopagenumbers{}                                  % uncomment to suppress automatic page numbering for CVs longer than one page
% character encoding
%\usepackage[utf8]{inputenc}                       % if you are not using xelatex ou lualatex, replace by the encoding you are using
%\usepackage{CJKutf8}                              % if you need to use CJK to typeset your resume in Chinese, Japanese or Korean
% adjust the page margins
\usepackage[scale=0.75]{geometry}
%\setlength{\hintscolumnwidth}{3cm}                % if you want to change the width of the column with the dates
%\setlength{\makecvtitlenamewidth}{10cm}           % for the 'classic' style, if you want to force the width allocated to your name and avoid line breaks. be careful though, the length is normally calculated to avoid any overlap with your personal info; use this at your own typographical risks...
% personal data
\name{Sasha}{Rubin}
\title{Research plan: Formal Methods in Artificial Intelligence} 
\address{August 2017}{}
% optional, remove / comment the line if not wanted
% \address{University of Naples ``Federico II''}{}% optional, remove / comment the line if not wanted; the "postcode city" and "country" arguments can be omitted or provided empty
% \phone[mobile]{+1~(234)~567~890}                   % optional, remove / comment the line if not wanted; the optional "type" of the phone can be "mobile" (default), "fixed" or "fax"
% \phone[fixed]{+2~(345)~678~901}
% \phone[fax]{+3~(456)~789~012}
% \email{rubin@unina.it}                               % optional, remove / comment the line if not wanted
% \homepage{forsyte.at/alumni/rubin/}                         % optional, remove / comment the line if not wanted
% \social[linkedin]{john.doe}                        % optional, remove / comment the line if not wanted
% \social[twitter]{jdoe}                             % optional, remove / comment the line if not wanted
% \social[github]{jdoe}                              % optional, remove / comment the line if not wanted
% \extrainfo{additional information}                 % optional, remove / comment the line if not wanted
% \photo[70pt][0.4pt]{RUBIN_Sasha.jpg}                       % optional, remove / comment the line if not wanted; '64pt' is the height the picture must be resized to, 0.4pt is the thickness of the frame around it (put
% bibliography adjustements (only useful if you make citations in your resume, or print a list of publications using BibTeX)
%   to show numerical labels in the bibliography (default is to show no labels)
\makeatletter\renewcommand*{\bibliographyitemlabel}{\@biblabel{\arabic{enumiv}}}\makeatother
%   to redefine the bibliography heading string ("Publications")
\renewcommand{\refname}{References}
\usepackage{framed}

%% PACKAGES %%
%\usepackage{latexsym}
\usepackage{amsmath}
\usepackage{amssymb}
%\usepackage{amsthm}


\usepackage{xspace,color,hyperref,upgreek} 
\usepackage{graphicx}
\usepackage{tikz,pgf}
\usetikzlibrary{arrows,shapes,automata,positioning,matrix,calc,petri}

%% ENVIRONMENTS %%
%\theoremstyle{plain}
%\newtheorem{theorem}{Theorem}
%\newtheorem{lemma}{Lemma}
%\newtheorem{fact}{Fact}
%\newtheorem{example}{Example}
%\newtheorem{definition}{Definition}
%\newtheorem{corollary}{Corollary}
%\newtheorem{proposition}{Proposition}
%\newtheorem*{proof*}{Proof}

%% COMMENTS %%

%\newcommand\note[1]{{\color{red}{#1}}}

\newcommand\sr[1]{{\color{red}{(sr:#1)}}}
% \renewcommand{\sr}[1]{}
\newcommand\ba[1]{{\color{purple}{(ba:#1)}}}
\newcommand\fz[1]{{\color{blue}{(fz:#1)}}}

\newcommand{\todo}[1]{{{\color{blue} #1}}}
%\renewcommand{\todo}[1]{}

%% LATEX SHORTCUTS %%

\def\it{\begin{itemize} }
\def\-{\item }
\def\ti{\end{itemize} }
\def\en{\begin{enumerate} }
\def\ne{\end{enumerate} }

%% COMPLEXITY CLASSES %%


\def\exptime{\textsc{exptime}\xspace}
\def\exptimeC{\exptime-complete}

\def\expspace{\textsc{expspace}\xspace}

\def\pspace{\textsc{pspace}\xspace}
\def\pspaceC{\pspace-complete}

\def\logspace{\textsc{logspace}\xspace}
\def\nlogspace{\textsc{nlogspace}\xspace}

\def\ptime{\textsc{ptime}}
\def\np{\textsc{np}}



%% LOGIC %%
\def\fol{\mathsf{FOL}}
\def\msol{\ensuremath{\mathsf{MSOL}}\xspace}
\def\fotc{\mathsf{FOL+TC}}

%% TEMPORAL LOGIC %%
\newcommand{\sqsq}[1]{\ensuremath{[\negthinspace[#1]\negthinspace]}}

\DeclareMathOperator{\ctlE}{{\mathsf{E}}}
\DeclareMathOperator{\ctlA}{\mathsf{A}}

\newcommand{\atlE}[1]{\ensuremath{\langle\!\langle{\text{#1}}\rangle\!\rangle}}
\newcommand{\atlA}[1]{\ensuremath{[\:\!\![\text{#1}]\:\!\!]}}

\DeclareMathOperator{\nextX}{\mathsf{X}}
\DeclareMathOperator{\yesterday}{\mathsf{Y}}
\DeclareMathOperator{\until}{\mathbin{\mathsf{U}}}
\DeclareMathOperator{\weakuntil}{\mathbin{\mathsf{W}}}
\DeclareMathOperator{\since}{\mathbin{\mathsf{S}}}
\DeclareMathOperator{\releases}{\mathbin{\mathsf{R}}}
\DeclareMathOperator{\always}{\mathsf{G}}
\DeclareMathOperator{\hitherto}{\mathsf{H}}
\DeclareMathOperator{\eventually}{\ensuremath{\mathsf{F}}\xspace}
\DeclareMathOperator{\previously}{\mathsf{P}}
\newcommand{\true}{\mathsf{true}}
\newcommand{\false}{\mathsf{false}}


\newcommand{\LTL}{\ensuremath{\mathsf{LTL}}\xspace}
\newcommand{\PLTL}{\textsf{PROMPT-}\LTL}

\newcommand{\CTL}{\ensuremath{\mathsf{CTL}}\xspace}
\newcommand{\CTLS}{\ensuremath{\mathsf{CTL}^*}\xspace}
\newcommand{\PCTLS}{\textsf{PROMPT-}\CTLS}
\newcommand{\PCTL}{\textsf{PROMPT-}\CTL}
\newcommand{\CLTL}{\ensuremath{\textsf{C-}\LTL}\xspace}
\newcommand{\PCLTL}{\ensuremath{\textsf{PROMPT-C}\LTL}\xspace}

\newcommand{\ATL}{\ensuremath{\mathsf{ATL}}\xspace}
\newcommand{\ATLS}{\ensuremath{\mathsf{ATL}^*}\xspace}
\newcommand{\PATLS}{\textsf{PROMPT-}\ATLS}
\newcommand{\PATL}{\textsf{PROMPT-}\ATL}


\def\alt{\, | \,}


\DeclareMathOperator{\Fp}{\eventually_\mathsf{P}}

\newcommand{\AP}{\mathrm{A\!P}}

%% MATH OPERATIONS %%
\newcommand{\tpl}[1]{\langle {#1} \rangle }
\newcommand{\tup}[1]{\overline{#1}}


%% STRUCTURES %%
\newcommand{\LTS}{\mathsf{S}}
\newcommand{\Paths}{\mathsf{pth}}
\newcommand{\nat}{\mathbb{N}}
\def\int{\mathbb{Z}}
\newcommand{\natzero}{\mathbb{N}_0}

\newcommand{\trans}[3]{#1 \stackrel{{#3}}{\rightarrow} #2}


%% HEADINGS ETC %%
\newcommand{\head}[1]{\noindent{\bf #1}}

%% COUNTER MACHINES %%
\newcommand{\cm}{\ensuremath{M}}
\newcommand{\CMinc}{\mathsf{inc}}
\newcommand{\CMdec}{\mathsf{dec}}
\newcommand{\CMzero}{\mathsf{ifzero}}
\newcommand{\CMnonzero}{\mathsf{nzero}}
\newcommand{\CMcommit}{\mathsf{end}}


%% CLTL %%
\def\var{{\sf var}}
\def\ovar{{\sf ovar}}
\def\avar{{\sf avar}}
\def\svar{{\sf svar}}
\def\bvar{{\sf bvar}}

\def\MOD{\equiv}

%% PVP %%
\def\PVP{\mathsf{PVP}}




\usepackage{parskip}
\setlength{\parindent}{10pt}


% bibliography with mutiple entries
%\usepackage{multibib}
%\newcites{book,misc}{{Books},{Others}}
%----------------------------------------------------------------------------------
%            content
%----------------------------------------------------------------------------------
\begin{document}


\makecvtitle


%%%%%%%%%%%%%%%%%%%%%%%%%%%%%%%%%%%%%%%%%%%%%%%%%%%%%%%%%%%%%55
%%%%%%%%%%%%%%%%%%%%%%%%%%%%%%%%%%%%%%%%%%%%%%%%%%%%%%%%%%%%%55
% Why is my research important?
% How will I approach it?
% What are my long-term research goals?
% What are my career goals?
%%%%%%%%%%%%%%%%%%%%%%%%%%%%%%%%%%%%%%%%%%%%%%%%%%%%%%%%%%%%%55
%%%%%%%%%%%%%%%%%%%%%%%%%%%%%%%%%%%%%%%%%%%%%%%%%%%%%%%%%%%%%55
% 
% - SHOW OFF YOUR KNOWLEDGE IN FM 
% - I see lots of connections between logic based AI and FM.
% - I see that AI is looking at very crucial problems and these problems are not 
% the same, but they resonate with FM. these two fields developed separately, I think I am 
% the person that can show these connections.
% - show I want to collaborate with everybody. as with GDG, HG, ...
% - think how i interact with GDG not NM (somehow lucky that FM to MAS).
% - show why DEEP knowledge is needed; not just to know that there are connections. show there 
% are differences. superficially looks the same. 
%%%%%%%%%%%%%%%%%%%%%%%%%%%%%%%%%%%%%%%%%%%%%%%%%%%%%%%%%%%%%55
%%%%%%%%%%%%%%%%%%%%%%%%%%%%%%%%%%%%%%%%%%%%%%%%%%%%%%%%%%%%%55
% RESEARCH PLAN
% - start talking about AI
% - FM can help AI
% - show you are working FOR AI.
% - show how FM can help AI
% e.g., planning for TEG is based on FM
% e.g., generalised planning/synthesis based on FM
%%%%%%%%%%%%%%%%%%%%%%%%%%%%%%%%%%%%%%%%%%%%%%%%%%%%%%%%%%%%%55
%%%%%%%%%%%%%%%%%%%%%%%%%%%%%%%%%%%%%%%%%%%%%%%%%%%%%%%%%%%%%55

This document describes my research plan for 2018 -- 2020 whose objective is to develop \textbf{Formal Methods} to describe and reason about \textbf{Artificial Agents}.

\section{Overview}


Artificial Intelligence (AI) studies the principles behind thinking and reasoning, and it does so 
in a mathematical and algorithmic way. 
Systems built on the insights of AI are called \emph{artificial agents (AA)}. 
As artificial agents are increasingly deployed in the world (e.g., 
software agents on the Internet, driverless cars, software that play and compete with humans at games, 
robots exploring new and dangerous environments, etc.) a clear challenge has materalised: 
there is a need for humans to be able to \emph{trust} 
the decisions made by AA, the need for \emph{meaningful interactions} between humans and AA, 
and the need for \emph{transparent} AA~\cite{ACMStatement07}. One path to meeting these needs is 
to create \emph{explainable AI (XAI)}, i.e., to enable humans to understand, trust and manage AA~\cite{DARPA}.

This is a grand challenge that involves many facets of computer science: psychology (to understand what is a good explanation), 
knowledge representation and logic (to formalise this in a way accessible to humans and machines), 
algorithms (to produce good explanations), NLP (to interface 
with human users), software engineering (to produce reliable and efficient programs), etc.

I take as an hypothesis that this challenge cannot be met without having some formal guarantees on the behaviour 
of AA.  It is in this context that Formal Methods (FM) will play a role. 
\begin{framed}
\noindent FM is built on three pillars: 
\it
\- ``Modeling'' problems at the correct levels of abstraction to be able 
to reason about them,
\- ``Verification'' for deducing that systems do indeed have specified properties,
\- ``Synthesis'' for automatically producing correct-by-design systems. 
\ti
\end{framed}
All three pillars are founded on techniques from mathematical logic. Logic is used to formally represent aspects of the system so that both expert humans and computers can 
manipulate them and reason about them. 

Although FM were pioneered in the areas of hardware and software verification, there are many connections between FM and AI. 

% My work consists of developing and applying formal methods to modeling and reasoning about computational systems, often involving strategic behaviour of agents. 
% This is motivated by ``explainable AI'', the need to ensure that the systems being built using, e.g.,  machine learning, can explain their decisions and actions to human users. 
% 
% I approach these questions by extending classic techniques in formal methods, such as the use of games on graphs and reductions to automata, and applying theoretical insights to foundational problems in AI. One notable example of this convergence is my recent IJCAI17 paper~\cite{BDGR17} which is at the intersection of formal methods and automated planning  in AI. I am also pursuing more speculative questions such as ``What is synthesis and how should it be formalised?".


\section{Connections between planning (in AI) and synthesis (in FM)}

\emph{Planning} is a branch of AI that addresses the problem of generating a course of action to achieve
a desired goal, given a description of the domain of interest and its initial state. The area is central
to the development of artificial agents. Besides theoretical insights, Planning provides practical tools 
based on heuristic search and symbolic methods~\cite{GeBo}. 

\emph{Synthesis} is a cornerstone of FM that addresses the problem of automatically producing systems that 
meet a given specification. 

There are many similarities between these two areas, but also important differences.

\subsection{Similarities}
Both Planning and Synthesis are model-based controller design. That is, ...


\subsection{Differences}


goals are dynamic, that is, goals are produced continuously,
as the agent operates. However, while in Planning a plan for the next goal must be built only after
the current goal is achieved, mechanisms have the option to drop the current goal or combine it with
the next one, thus adding a complication that cannot be dealt with straightforwardly by standard
Planning techniques.

\emph{planning} in AI and \emph{reactive synthesis} in formal methods 
\en
\- same basic idea... model-based controller design
\- central to both are succinct representations of the systems: STRIPS, PDDL, in planning and LTL/LTLf in synthesis.
but for different reasons! planning typically deals with reachability goals on compact domains; synthesis with LTL goals on explicit domains.
\- both understood early on the computational complexity of the problem PSPACE-complete ... 2EXPTIME-complete
\- planning responded by finding ways to treat the "easy but large cases", e.g., heuristic search
\- reactive synthesis responded by studying the theory of the problem (e.g., fragments, extensions), and providing algorithms based on logic and automata theory (antichain).
\- some work (Sheilah...) has studied translations between these, i.e., LTL --> AFW/NFW --> planning domain
\ne


%(an active topic of investigation with Giuseppe De Giacomo).



\emph{Formal methods (FM)} is an umbrella-term that describes principles and techniques for reasoning 
about systems with some digital component, such as software, hardware, cyber-physical systems, etc.
\begin{framed}
\noindent FM is built on three pillars: modeling, verification, and synthesis.  
\end{framed}
All three pillars are founded 
on techniques from mathematical logic. Logic is used to formally represent aspects of the system so that both expert humans and computers can 
manipulate them and reason about them.






\section{Current Research --- Formal methods for multi-agent systems}
Multi-agent systems (MAS) involve multiple individual agents (these may be people, software, robots) each with their own goals. Such systems can be viewed as multi-player games, and thus notions from game-theory (e.g., strategies, knowledge, and equilibria) are used to reason about them. Agents in realistic MAS often lack information about other agents and the environment, and this is often categorised in one of two ways: a) \emph{incomplete information} and b) \emph{imperfect information}.
\newline

\subsection{a) MAS with incomplete information}
Incomplete-information refers to uncertainty about the environment (i.e., the structure of the game). I have considered two sources of incomplete information for MAS.
\newline


First, the \emph{number of agents} may not be known, or may not be bounded a priori.
In a series of papers, I have contributed to a generalisation of a cornerstone paper on verification of such systems (``Reasoning about Rings'', E.A. Emerson, K.S. Namjoshi, 
\textsc{POPL}, 1995) from ring topologies to arbitrary topologies \cite{DBLP:conf/vmcai/AminofJKR14,DBLP:conf/concur/AminofKRSV14,DBLP:conf/cade/AminofR16,AKRSV17}. Other work on this topic 
studied the relative power of standard communication-primitives assuming an unknown number of agents~\cite{DBLP:conf/lpar/AminofRZ15}, as well as the complexity of model-checking timed systems assuming an unknown number of agents~\cite{DBLP:conf/icalp/AminofRZS15}. I also contributed to a book on this topic published by Morgan\&Claypool in 2015~\cite{DBLP:series/synthesis/2015Bloem,DBLP:journals/sigact/BloemJKKRVW16}.
\newline



Second, the agents may be operating in a \emph{partially-known environment}. For instance, the agents may know they are in a ring, but may not know the size of the ring. I launched the application of automata theory for the verification of high-level properties of light-weight mobile agents in partially-known environments~\cite{DBLP:conf/atal/Rubin15}. In follow-up work I explored this theme further, including finding ways to model agents on grids --- the most common abstraction of 2D and 3D space~\cite{DBLP:conf/prima/RubinZMA15,DBLP:conf/atal/AminofMRZ16,DBLP:conf/prima/MuranoPR15}.
\newline  

\subsection{b) MAS with imperfect information}
Even if agents have certainty about the structure of the system, they may not know exactly which state the system is in. This is called imperfect information and the associated logic for reasoning about such cases are called \emph{epistemic}. I have studied strategic-epistemic logics in a number of works, namely, with a prompt modality (thus allowing one to express that a property holds ``promptly'' rather than simply ``eventually'')~\cite{DBLP:conf/kr/AminofMRZ16}, and on systems with public-actions (such as certain card games, including a hand of Poker or a round of Bridge)~\cite{BLMR17,BLMR17IJCAI}. The importance of these last works is that they give the first decidability (and sometimes optimal complexity) results for strategic reasoning about games of imperfect information in which the agents may have arbitrary observations. In contrast, following classical restrictions on the observations or information of agents, I have also shown how to extend strategy logic by epistemic operators and identified a decidable fragment in which one can express equilibria concepts~\cite{BMMRV17}.
\newline

% % \subsection{Logics with Counting Quantifiers}
% % % Many logics in computer science do not have the ability to count. However, besides being a basic operation, counting allows one to describe finer details of a system.
% % I have a long-standing interest in logics with quantifiers that count. E.g., the usual first-order quantifier $\exists x$ can be generalised to the counting quantifier $\exists^{\geq k} x$ which says that ``there are at least $k$ many $x$''. Concretely, I have studied logics that count strategies~\cite{AMMR16-SR,DBLP:conf/atal/AminofMMR16}, paths~\cite{DBLP:conf/lpar/AminofMR15}, strings and sets~\cite{DBLP:journals/bsl/Rubin08,DBLP:conf/stacs/KaiserRB08}. With a PhD student of Erich Gr\"adel's (Tobias Ganzow) I
% % solved a 12 year-old conjecture of Courcelle's on the relationship between order and counting on graphs~\cite{DBLP:conf/stacs/GanzowR08}. I recently established and studied a logical formalism, called ``graded strategy-logic'', that is rich enough to count equilibria \cite{AMMR16-SR,DBLP:conf/atal/AminofMMR16}. The importance of this result to equilibrium selection is that it gives a computational way to decide if a given game has, e.g., a unique Nash equilibrium. 


 
\subsection{Foundations of Automated Planning}
Planning in AI can be viewed as the problem of finding strategies in one- or two-player graph-games. In this model vertices represent states, edges represent transitions, and the players represent the agents. I have contributed foundational work to such games. Concretely, I recently extended the classic belief-space construction for games of imperfect-information from finite arenas to infinite-arenas~\cite{GMRS16IJCAI} (infinite arenas often arise in the study of MAS with incomplete information, see above). I have also used these ideas to elucidate the role of observation-projections in generalised planning problems~\cite{BDGR17ICAPS,BDGR17}.
I have generalised classic results about certain games with quantitative objectives (i.e., Ehrenfeucht and J. Mycielski. Positional strategies for mean payoff games. International Journal of Game Theory, 8:109--113, 1979) to so-called first-cycle games, i.e., games in which play stops the moment a vertex is repeated~\cite{AR16}. 
\newline

% \end{abstract}

% % http://www.cs.rice.edu/~vardi/comp409/history.pdf
% A running theme in my work is the development of logical formalisms for describing and reasoning about objects of interest to computer scientists, from the abstract (e.g., graphs, algebras, orders) to the concrete (e.g., multiplayer games). It is often said that ``logic is the calculus of computer science''~\cite{}. Moshe Vardi has said, of computer science, that ``description is our business''~\cite{}. Seen in this light, my work is of a foundational nature: it sheds light on 



\section{Past Research --- Algorithmic Model Theory}
My early work contributed to a research program called ``Algorithmic  Model Theory" whose aim is to develop and extend the success of Finite Model Theory to infinite structures that can be reasoned about algorithmically. 
\newline

Specifically, my PhD work pioneered the development of ``automatic structures'': this is a generalisation of the regular languages from sets to mathematical objects with structure, such as graphs, arithmetics, algebras, etc.  The fundamental property of automatic structures is that one can automatically answer logic-based queries about them (precisely, their first-order theory is decidable). I gave techniques for proving that structures are or are not automatic (similar to, but vastly more complicated than, pumping lemmas for regular languages), I studied the computational complexity of deciding when two automatic structures are the same (isomorphic), and I found extensions of the fundamental property, thus enriching the query language \cite{BGR11,DBLP:conf/lics/IshiharaKR02,DBLP:conf/lics/KhoussainovNRS04,DBLP:journals/lmcs/KhoussainovNRS07,DBLP:conf/lics/KhoussainovRS03,DBLP:conf/stacs/KhoussainovRS04,DBLP:journals/tocl/KhoussainovRS05,DBLP:journals/bsl/Rubin08}. I have also worked on extensions of automatic structures to include oracle computation \cite{DBLP:journals/corr/abs-1210-2462,DBLP:conf/lics/RabinovichR12}.
\newline


% 
% \section{Short-term trajectory}
% 
% I recently organised the first workshop on formal methods in artificial intelligence (FMAI) 2017. In the next few years I plan to further integrate into the AI community, and the MAS community specifically. Concretely, I plan to study more richer \emph{models of systems} (rather than richer logics), including finer representations of time, bounded-memory strategies, and probabilistic arenas and strategies.

% traps: \cite{DBLP:journals/tcs/GrinshpunPRT14}

% planning: \cite{DBLP:conf/prima/MuranoPR15}

% \section{Misc}
% PROMPT: \cite{DBLP:conf/kr/AminofMRZ16}



% 
% Probabilistic: \cite{DBLP:conf/cav/BustanRV04}
%  A fundamental problem in computer science is that of ensuring that a system
%  satisfies a particular property. Moshe Vardi, Doron Bustan and I \cite{BRV04}
%  considered the complexity of checking that a probabilistic system (modeled by a
%  finite-state discrete-time Markov chain) satisfies properties expressed by
%  automata operating on infinite words. The sorts of properties that can be
%  expressed extend those of linear temporal logic, a typical example is `Does the
%  Markov chain almost surely enter this state infinitely often'? We presented an
%  optimal algorithm that checks whether a given Markov chain satisfies a
%  specification given by an alternating B\"uchi automaton, thus extending known
%  work on linear temporal logic \cite{CoYa90}.
 
% \small
 
%  \pagebreak
%  
% \bibliographystyle{plain}
% \bibliography{/home/sr/svn/forsyte-publications/trunk/rubin.bib, otherbib}
% 
% \end{document}




There are many connections between AI and FM. Som
\subsection{Planning in AI and Synthesis in FM}

\emph{planning} in AI and \emph{reactive synthesis} in formal methods 
\en
\- same basic idea... model-based controller design
\- central to both are succinct representations of the systems: STRIPS, PDDL, in planning and LTL/LTLf in synthesis.
but for different reasons! planning typically deals with reachability goals on compact domains; synthesis with LTL goals on explicit domains.
\- both understood early on the computational complexity of the problem PSPACE-complete ... 2EXPTIME-complete
\- planning responded by finding ways to treat the "easy but large cases", e.g., heuristic search
\- reactive synthesis responded by studying the theory of the problem (e.g., fragments, extensions), and providing algorithms based on logic and automata theory (antichain).
\- some work (Sheilah...) has studied translations between these, i.e., LTL --> AFW/NFW --> planning domain
\ne

What's missing from within this picture?

\en
\-  formal connections between the two fields: e.g., reducing synthesis to planning (cf Sheilah's work).
\- formal connections within planning problems: e.g., reduce LTLf planning to reachability planning...
\- clear idea of how one can exploit modern planners and heuristic methods to solve problems in automata in practice, e.g., do domain-independent 
heuristics work LTLf/LDLf/LTL...? perhaps "LTL-dependent heuristics" should be studied...
\- clear connections between DEL/epistemic programs/GDLIII (Thieschler) and synthesis framework/algorithms.
\ne

Relation with other work
- In fact a first-person view of agents has long be advocated in reasoning
about action in Knowledge Representation [McHa69, Reit01].
- many architectures used in robotics to capture mental states are not formal enough to admit to formal methods \cite{PeterDunney}, 
although some recent work at UNSW aims to address this.
- to deal with the social component I plan to use notions of "contexts" that defined expected behaviours to other agents that is used to reason 
about them. 

- situation awareness, incomplete information,
information classification and actions ontologies, reasoning about others’ expected behaviors and violations,
strategic action deliberation, and synthesis and refinement of execution plans.

Potential future directions
- E. Asarin, R. Chane-Yack-Fa, and D. Varacca, “Fair adversaries and
randomization in two-player games,” in Proceedings of FoSSaCS 2010,
ser. LNCS, vol. 6014. Springer, 2010, pp. 64–78.

- How Good Is a Strategy in a Game With Nature?
- Paying cost for resolving observation (Natasha,Blaise).

How can I complement research at UNSW.
1. Theoretical foundations of strategic epistemic reasoning in complex environments.
MIT: Strategic Reasoning and Planning for General Game-Playing Robots (Australia-Germany Joint Research Cooperation Scheme 2016-2017)
MIT: Universal Game-Playing Systems for Randomised and Imperfect-Information Games (ARC-DP 2012-2015)

2. Theory of distributed synthesis (information forks, automata theory for controller synthesis) 
could be used to analyse the cognitive meta-hierarchy of 
David Rajaratnam
Bernhard Hengst
Maurice Pagnucco
Claude Sammut
Michael Thielscher



How my work can be complemented by work at UNSW:
1. *unstructured* and incomplete environments are central to many robotic applications (e.g., rescue robots). Adapting definitions and results in reactive synthesis to such a setting is a clear and present challenge.

2. Insights from information flow in security of distributed system (CC Morgan) could yield insights into how strategies of different agents in the distributed synthesis problem signal private information to other agents. Note that the latter is in some sense the dual problem: how can one define strategies that *do* leak enough private information that the distributed players can co-ordinate and achieve a joint objective.

3. Toby Walsh, Haris Aziz?



Project Offers
==============



???
When faced with a dynamical sys-
tem that you want to simulate, control, analyze, or otherwise investigate, first axiomatize
it in a suitable logic. Through logical entailment, all else will follow, including system
control, simulation, and analysis.



- games of incomplete information, imperfect information
- epistemic planning

concretely:
probabilistic DEL with public announcements
PATL* on broadcast iCGS

Quantitative SL: add weights to the arena (e.g., to actions or to
states), and add atomic formulas to the logic of the form "the
mean-payoff for player i is at least c".

Question: is model-checking decidable if we add these to ATL? ATL*? SL?

--
controller manages a collection of programmable mechanisms
- monitors and responds to events
(e.g., shifts in load, certain specifications fail, ...)

- reprogram mechanisms on the fly
(e.g., change ???


controller is centralised (1 agent vs 1 environment)

FSMs (1) intuitively and concisely capture control dynamics in response to network events; and (2) their structure makes them amenable to verification.

what are the external events?
timing, 

what is the current way to solve the problems that whitemech would solve.


\bibliographystyle{plain}
\bibliography{researchplan,FWF,References,rubin}
\end{document}

