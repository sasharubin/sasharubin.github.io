\documentclass[10pt,a4paper,sans]{moderncv}       
% possible options include font size ('10pt', '11pt' and '12pt'), paper size ('a4paper', 'letterpaper', 'a5paper', 'legalpaper', 'executivepaper' and 'landscape') and font family ('sans' and 'roman')
% moderncv themes
\moderncvstyle{classic}                             % style options are 'casual' (default), 'classic', 'banking', 'oldstyle' and 'fancy'
\moderncvcolor{blue}                               % color options 'black', 'blue' (default), 'burgundy', 'green', 'grey', 'orange', 'purple' and 'red'
%\renewcommand{\familydefault}{\sfdefault}         % to set the default font; use '\sfdefault' for the default sans serif font, '\rmdefault' for the default roman one, or any tex font name
%\nopagenumbers{}                                  % uncomment to suppress automatic page numbering for CVs longer than one page
% character encoding
%\usepackage[utf8]{inputenc}                       % if you are not using xelatex ou lualatex, replace by the encoding you are using
%\usepackage{CJKutf8}                              % if you need to use CJK to typeset your resume in Chinese, Japanese or Korean
% adjust the page margins
\usepackage[scale=0.75]{geometry}
%\setlength{\hintscolumnwidth}{3cm}                % if you want to change the width of the column with the dates
%\setlength{\makecvtitlenamewidth}{10cm}           % for the 'classic' style, if you want to force the width allocated to your name and avoid line breaks. be careful though, the length is normally calculated to avoid any overlap with your personal info; use this at your own typographical risks...
% personal data
\name{Sasha}{Rubin}
\title{Research Statement} 
\address{June 2017}{}
% optional, remove / comment the line if not wanted
% \address{University of Naples ``Federico II''}{}% optional, remove / comment the line if not wanted; the "postcode city" and "country" arguments can be omitted or provided empty
% \phone[mobile]{+1~(234)~567~890}                   % optional, remove / comment the line if not wanted; the optional "type" of the phone can be "mobile" (default), "fixed" or "fax"
% \phone[fixed]{+2~(345)~678~901}
% \phone[fax]{+3~(456)~789~012}
% \email{rubin@unina.it}                               % optional, remove / comment the line if not wanted
% \homepage{forsyte.at/alumni/rubin/}                         % optional, remove / comment the line if not wanted
% \social[linkedin]{john.doe}                        % optional, remove / comment the line if not wanted
% \social[twitter]{jdoe}                             % optional, remove / comment the line if not wanted
% \social[github]{jdoe}                              % optional, remove / comment the line if not wanted
% \extrainfo{additional information}                 % optional, remove / comment the line if not wanted
% \photo[70pt][0.4pt]{RUBIN_Sasha.jpg}                       % optional, remove / comment the line if not wanted; '64pt' is the height the picture must be resized to, 0.4pt is the thickness of the frame around it (put
% bibliography adjustements (only useful if you make citations in your resume, or print a list of publications using BibTeX)
%   to show numerical labels in the bibliography (default is to show no labels)
\makeatletter\renewcommand*{\bibliographyitemlabel}{\@biblabel{\arabic{enumiv}}}\makeatother
%   to redefine the bibliography heading string ("Publications")
\renewcommand{\refname}{References}


% bibliography with mutiple entries
%\usepackage{multibib}
%\newcites{book,misc}{{Books},{Others}}
%----------------------------------------------------------------------------------
%            content
%----------------------------------------------------------------------------------
\begin{document}


\makecvtitle


%%%%%%%%%%%%%%%%%%%%%%%%%%%%%%%%%%%%%%%%%%%%%%%%%%%%%%%%%%%%%55
%%%%%%%%%%%%%%%%%%%%%%%%%%%%%%%%%%%%%%%%%%%%%%%%%%%%%%%%%%%%%55
% Why is my research important?
% How will I approach it?
% What are my long-term research goals?
% What are my career goals?
%%%%%%%%%%%%%%%%%%%%%%%%%%%%%%%%%%%%%%%%%%%%%%%%%%%%%%%%%%%%%55
%%%%%%%%%%%%%%%%%%%%%%%%%%%%%%%%%%%%%%%%%%%%%%%%%%%%%%%%%%%%%55

\section{Overview}
My work consists of developing and applying formal and logical methods to modeling and reasoning about computational systems, often involving strategic behaviour of agents. 
This is motivated by ``explainable AI'', the need to ensure that the systems being built using, e.g.,  machine learning, can explain their decisions and actions to human users. 
\newline

I approach these questions by extending classic techniques in formal methods, such as the use of games on graphs and reductions to automata, and applying theoretical insights to foundational problems in AI. One notable example of this convergence is my recent IJCAI17 paper~\cite{BDGR17} which is at the intersection of formal methods and automated planning in AI. I am also pursuing more speculative questions such as ``What is synthesis and how should it be formalised?".
%(an active topic of investigation with Giuseppe De Giacomo).
\newline


\section{Current Research --- Formal methods for multi-agent systems}
Multi-agent systems (MAS) involve multiple individual agents (these may be people, software, robots) each with their own goals. Such systems can be viewed as multi-player games, and thus notions from game-theory (e.g., strategies, knowledge, and equilibria) are used to reason about them. Agents in realistic MAS often lack information about other agents and the environment, and this is often categorised in one of two ways: a) \emph{incomplete information} and b) \emph{imperfect information}.
\newline

\subsection{a) MAS with incomplete information}
Incomplete-information refers to uncertainty about the environment (i.e., the structure of the game). I have considered two sources of incomplete information for MAS.
\newline


First, the \emph{number of agents} may not be known, or may not be bounded a priori.
In a series of papers, I have contributed to a generalisation of a cornerstone paper on verification of such systems (``Reasoning about Rings'', E.A. Emerson, K.S. Namjoshi, 
\textsc{POPL}, 1995) from ring topologies to arbitrary topologies \cite{DBLP:conf/vmcai/AminofJKR14,DBLP:conf/concur/AminofKRSV14,DBLP:conf/cade/AminofR16,AKRSV17}. Other work on this topic 
studied the relative power of standard communication-primitives assuming an unknown number of agents~\cite{DBLP:conf/lpar/AminofRZ15}, as well as the complexity of model-checking timed systems assuming an unknown number of agents~\cite{DBLP:conf/icalp/AminofRZS15}. I also contributed to a book on this topic published by Morgan\&Claypool in 2015~\cite{DBLP:series/synthesis/2015Bloem,DBLP:journals/sigact/BloemJKKRVW16}.
\newline



Second, the agents may be operating in a \emph{partially-known environment}. For instance, the agents may know they are in a ring, but may not know the size of the ring. I launched the application of automata theory for the verification of high-level properties of light-weight mobile agents in partially-known environments~\cite{DBLP:conf/atal/Rubin15}. In follow-up work I explored this theme further, including finding ways to model agents on grids --- the most common abstraction of 2D and 3D space~\cite{DBLP:conf/prima/RubinZMA15,DBLP:conf/atal/AminofMRZ16,DBLP:conf/prima/MuranoPR15}.
\newline  

\subsection{b) MAS with imperfect information}
Even if agents have certainty about the structure of the system, they may not know exactly which state the system is in. This is called imperfect information and the associated logic for reasoning about such cases are called \emph{epistemic}. I have studied strategic-epistemic logics in a number of works, namely, with a prompt modality (thus allowing one to express that a property holds ``promptly'' rather than simply ``eventually'')~\cite{DBLP:conf/kr/AminofMRZ16}, and on systems with public-actions (such as certain card games, including a hand of Poker or a round of Bridge)~\cite{BLMR17,BLMR17IJCAI}. The importance of these last works is that they give the first decidability (and sometimes optimal complexity) results for strategic reasoning about games of imperfect information in which the agents may have arbitrary observations. In contrast, following classical restrictions on the observations or information of agents, I have also shown how to extend strategy logic by epistemic operators and identified a decidable fragment in which one can express equilibria concepts~\cite{BMMRV17}.
\newline

% % \subsection{Logics with Counting Quantifiers}
% % % Many logics in computer science do not have the ability to count. However, besides being a basic operation, counting allows one to describe finer details of a system.
% % I have a long-standing interest in logics with quantifiers that count. E.g., the usual first-order quantifier $\exists x$ can be generalised to the counting quantifier $\exists^{\geq k} x$ which says that ``there are at least $k$ many $x$''. Concretely, I have studied logics that count strategies~\cite{AMMR16-SR,DBLP:conf/atal/AminofMMR16}, paths~\cite{DBLP:conf/lpar/AminofMR15}, strings and sets~\cite{DBLP:journals/bsl/Rubin08,DBLP:conf/stacs/KaiserRB08}. With a PhD student of Erich Gr\"adel's (Tobias Ganzow) I
% % solved a 12 year-old conjecture of Courcelle's on the relationship between order and counting on graphs~\cite{DBLP:conf/stacs/GanzowR08}. I recently established and studied a logical formalism, called ``graded strategy-logic'', that is rich enough to count equilibria \cite{AMMR16-SR,DBLP:conf/atal/AminofMMR16}. The importance of this result to equilibrium selection is that it gives a computational way to decide if a given game has, e.g., a unique Nash equilibrium. 


 
\subsection{Foundations of Automated Planning}
Planning in AI can be viewed as the problem of finding strategies in one- or two-player graph-games. In this model vertices represent states, edges represent transitions, and the players represent the agents. I have contributed foundational work to such games. Concretely, I recently extended the classic belief-space construction for games of imperfect-information from finite arenas to infinite-arenas~\cite{GMRS16IJCAI} (infinite arenas often arise in the study of MAS with incomplete information, see above). I have also used these ideas to elucidate the role of observation-projections in generalised planning problems~\cite{BDGR17ICAPS,BDGR17}.
I have generalised classic results about certain games with quantitative objectives (i.e., Ehrenfeucht and J. Mycielski. Positional strategies for mean payoff games. International Journal of Game Theory, 8:109--113, 1979) to so-called first-cycle games, i.e., games in which play stops the moment a vertex is repeated~\cite{AR16}. 
\newline

% \end{abstract}

% % http://www.cs.rice.edu/~vardi/comp409/history.pdf
% A running theme in my work is the development of logical formalisms for describing and reasoning about objects of interest to computer scientists, from the abstract (e.g., graphs, algebras, orders) to the concrete (e.g., multiplayer games). It is often said that ``logic is the calculus of computer science''~\cite{}. Moshe Vardi has said, of computer science, that ``description is our business''~\cite{}. Seen in this light, my work is of a foundational nature: it sheds light on 



\section{Past Research --- Algorithmic Model Theory}
My early work contributed to a research program called ``Algorithmic  Model Theory" whose aim is to develop and extend the success of Finite Model Theory to infinite structures that can be reasoned about algorithmically. 
\newline

Specifically, my PhD work pioneered the development of ``automatic structures'': this is a generalisation of the regular languages from sets to mathematical objects with structure, such as graphs, arithmetics, algebras, etc.  The fundamental property of automatic structures is that one can automatically answer logic-based queries about them (precisely, their first-order theory is decidable). I gave techniques for proving that structures are or are not automatic (similar to, but vastly more complicated than, pumping lemmas for regular languages), I studied the computational complexity of deciding when two automatic structures are the same (isomorphic), and I found extensions of the fundamental property, thus enriching the query language \cite{BGR11,DBLP:conf/lics/IshiharaKR02,DBLP:conf/lics/KhoussainovNRS04,DBLP:journals/lmcs/KhoussainovNRS07,DBLP:conf/lics/KhoussainovRS03,DBLP:conf/stacs/KhoussainovRS04,DBLP:journals/tocl/KhoussainovRS05,DBLP:journals/bsl/Rubin08}. I have also worked on extensions of automatic structures to include oracle computation \cite{DBLP:journals/corr/abs-1210-2462,DBLP:conf/lics/RabinovichR12}.
\newline


% 
% \section{Short-term trajectory}
% 
% I recently organised the first workshop on formal methods in artificial intelligence (FMAI) 2017. In the next few years I plan to further integrate into the AI community, and the MAS community specifically. Concretely, I plan to study more richer \emph{models of systems} (rather than richer logics), including finer representations of time, bounded-memory strategies, and probabilistic arenas and strategies.

% traps: \cite{DBLP:journals/tcs/GrinshpunPRT14}

% planning: \cite{DBLP:conf/prima/MuranoPR15}

% \section{Misc}
% PROMPT: \cite{DBLP:conf/kr/AminofMRZ16}



% 
% Probabilistic: \cite{DBLP:conf/cav/BustanRV04}
%  A fundamental problem in computer science is that of ensuring that a system
%  satisfies a particular property. Moshe Vardi, Doron Bustan and I \cite{BRV04}
%  considered the complexity of checking that a probabilistic system (modeled by a
%  finite-state discrete-time Markov chain) satisfies properties expressed by
%  automata operating on infinite words. The sorts of properties that can be
%  expressed extend those of linear temporal logic, a typical example is `Does the
%  Markov chain almost surely enter this state infinitely often'? We presented an
%  optimal algorithm that checks whether a given Markov chain satisfies a
%  specification given by an alternating B\"uchi automaton, thus extending known
%  work on linear temporal logic \cite{CoYa90}.
 
% \small
 
 \pagebreak
 
\bibliographystyle{plain}
\bibliography{/home/sr/svn/forsyte-publications/trunk/rubin.bib, otherbib}

\end{document}

