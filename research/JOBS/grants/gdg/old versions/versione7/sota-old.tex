This part of the proposal should be read in conjunction with Part B1,
which introduces the project providing its vision. 


\section{State-of-the-art and objectives}

\vspace{-1ex}

%\subsection*{MOTIVATION AND LONG TERM VISION}

\subsection{STATE OF THE ART} % AND OBJECTIVES}

%
\vspace{-1ex}




We are witnessing an increasing availability of \textbf{mechanisms} that offer
some form of programmability. 
%%
These obviously comprise software in our computers and mobile devices,
%. These are indeed where programming has developed first and we expect
%programming effort to be dedicated to it. 
%%
%Moreover 
as well as
intelligent machines, such as cognitive robots, self-driving cars, flying drones, etc., which are becoming a reality. 

Beside these,  there are less obvious systems, currently considered of pivotal importance in Business, that can be considered programmable mechanisms. For
example in \textbf{Manufacturing} significant research efforts are focusing on
improving flexibility, agility and productivity of manufacturing
systems, under the umbrella term \emph{Industry 4.0}, or \emph{4th industrial
revolution}. These efforts are shifting interests towards
cyber-physical systems and cloud computing
with the ultimate vision of \textbf{manufacturing as-a-service} 
\cite{WuRWS15,Lorenz15}. 

Similarly, the rising of the \textbf{Internet of Things} as a
virtual fabric that connects ``things''
equipped with chips, sensors and actuators allows for building
programmable mechanisms acting in the physical world.  The originally
inanimate objects and spaces can develop into \textbf{smart objects}
and \textbf{smart spaces} with a level of awareness of the environment
they are immersed in, which together with their steadily increasing
computation power, makes them able to interact with human occupants
in engaging ways \cite{MacGillivray16,Sailer16}. 


Switching to a different context, \textbf{Business Process Management}
\cite{BPMN-book} advocates explicit conceptual descriptions of a
process to be enacted within an organization or possibly across
organizations. Business processes are explicitly modelled and analyzed
through IT technologies nowadays, and executed through a process
management system. According to Gartner, business processes
improvement is the top business strategy of CIOs in enterprises
\cite{Gartner11}. Such (formalized) \textbf{business processes} can be
considered programmable mechanisms too.

In particular there are compelling reasons to introduce \textbf{self-programming
abilities} in these systems.
%%
For example, it is advocated that cyber-physical systems in
\textbf{Manufacturing} or \textbf{Internet of Things} should be able
to \textbf{adapt} themselves to the current users and environment by
exploiting information gathered at runtime. This would allow for
evolving into so-called \emph{smart environments}, e.g., smart homes,
smart offices, smart public spaces, and smart factories. However it is
considered impossible to determine apriori all possible adaptations
that may be needed at runtime. Self-programming abilities are required
\cite{Seiger2016}.
% \footnote{Ronny Seiger, Steffen Huber, Peter
%   Heisig, Uwe Assmann: Enabling Self-adaptive Workflows for
%   Cyber-physical Systems. BMMDS/EMMSAD 2016: 3-17} \footnote{Ronny
%   Seiger, Christine Keller, Florian Niebling, Thomas Schlegel:
%   Modelling complex and flexible processes for smart cyber-physical
%   environments. J. Comput. Science 10: 137-148 (2015)}

In \textbf{Business Processes} it is considered important for the next generation
of process management systems to allow processes to automatically
\textbf{adapt executions} when unanticipated exceptions occur, without
explicitly defining apriori \textbf{recovery policies}, and without the
intervention of domain experts at runtime. These self-programming
abilities would reduce costly and error-prone manual ad-hoc changes, and would relieve
software engineers from mundane adaptation
tasks \cite{MarrellaMS17}.  
%%
Note that such concerns have been shared by
\textbf{autonomic computing}, which has advocated self-configuration, self-healing, self-optimization, and self-protection, though by using policies provided by
IT professionals \cite{ibm2005autonomic}.

Although the interest is clearly apparent, currently these
self-programing abilities are missing in actual mechanisms, and
science is focussing on limited forms  of self-programming, e.g., for
exception handling and recovery, or forms of composition and 
autonomic reconfiguration.

\subsection{OBJECTIVES}

\vspace{-1ex}

The overarching objective of \project is the following:

\begin{quote}\textit{
\project aims at laying the theoretical foundations and practical
methodologies of a science and engineering of \textbf{white-box self-programming mechanisms}. 
}
\end{quote}
\project intends to consider self-programmable mechanisms, as forms
of \textbf{Agents} studied in \textbf{Artificial Intelligence}
\cite{Reiter01,Wooldridge09}. \footnote{We stress that \project does
  not aim at general AI, but envisions self-programming mechanisms
  that act intelligently within the specific domain of interest in
  which they operate.}

Since ``\emph{with great power comes great responsibility}'',
introducing advanced forms of self-programming calls for the ability
to make the behavior automatically synthesized by the mechanism
\textbf{understandable} to human supervisors.
%%
%must be checked to be \textbf{harmless}. 
% In line with a large part of the AI community, 
% \project considers this point essential \cite{RussellDT15}.
%%
So it is indeed crucial to develop self-programming mechanisms that are \textbf{white-box}: in every moment the
mechanism can be queried for its specifications, its behavior and how
it relates to the specifications. Ultimately it is the possibility of \textbf{explain in human terms}
the resulting behavior that will make white-box self-programming mechanisms  \textbf{trustworthy} \cite{CaDa10,Neumann17}.
%
Being white-box contrasts with most current approaches, which consider
acceptable synthesized solutions that remain opaque to humans, as long
as they work \cite{MnihKSGAWR13,SilverHMGSDSAPL16}.

We further stress that the need to move towards \textbf{white-box}
approaches is advocated by a large part of the \textbf{AI community}
\cite{RussellDT15}, and has been recently taken up by DARPA within the
context of machine learning, through the DARPA-BAA-16-53 ``Explainable
Artificial Intelligence (XAI)''
program\footnote{\url{http://www.darpa.mil/program/explainable-artificial-intelligence}}

\begin{quote}{\it
Knowledge representation, the primary field of the PI, will be central for realizing the shift towards a white-box approach.}
\end{quote}
%\vspace{-1ex}



Towards the goal of building \textbf{white-box self-programming mechanisms}, \project will address the following objectives. %challenges.
\begin{enumerate}

\item \textbf{Equip mechanisms with general self-programming abilities.}

\item \textbf{Make self-programming abilities available while in operation.} 

\item \textbf{Make white-box self-programming mechanisms verifiable.}

\item 
\textbf{Allow learning and stochastic decisions, while remaining within safe bounds.}

\item \textbf{Make white-box self-programming mechanisms comprehensible to
    humans}. 
\item \textbf{Make self-programming mechanisms data-aware.}

\item \textbf{Favor component-based approaches.}  
\end{enumerate}




Technically, \project intends to make a quantum leap in
mechanisms' self-programming abilities while keeping them white-box. To do so,  \project will push forward the emerging cross-fertilization
among from \textbf{Reasoning About Action} in \textbf{Knowledge
  Representation}, \textbf{Data-aware Processes} in \textbf{Databases}, \textbf{Verification and Synthesis} in
\textbf{Formal Methods} and \textbf{Planning} in \textbf{Artificial
  Intelligence}. 

\begin{quote}{\it
The PI has profoundly contributed to all these areas, and he is %in a unique position to thrive this cross-fertilization.}
one of the most prominent AI scientist leading this cross-fertilization.}
\end{quote}


As a result, \project is \textbf{very timely} and of \textbf{greatest significance for European science}.

\project is a \textbf{high risk, high gain project}: if successful, it
will result in a radically more useful automated mechanisms than what
we have today, namely \textbf{white-box self-programming mechanisms},
unleashing full potential of \textbf{self-programmability} and
removing the main barriers to the uptake of automated mechanisms in
real business context, namely \textit{predefined forms of automation}, and
difficulties in \textit{formally analyzing their automated behavior in human terms}.

Given the crucial role that automated mechanisms plays in our modern economy and society and that \textbf{white-box self-programming mechanisms} have the potential to resolve a number of central hurdles, % (see part B1 for both), 
this implies a potentially very high impact on computer science as well as in crucial  business contexts, facilitating the already ongoing uptake of automated mechanisms in industry eventually also on economy and society. 

The \project project involves \textbf{high risk} because the ultimately we need to merge explicit representation, with advanced form of process synthesis/coordination and refinement and  identify novel and useful islands of effective feasibility that \project aims at is technically extremely challenging, much more so than the related Verification and  Planning both of which have made tremendous progresses in the last decade, touching upon several notorious open problems in computer science. 
%%
On top of that, the path from theoretical analysis to practically efficient algorithm, which we plan to fully explore within \project, is a huge challenge given that we aim realizing real automated mechanisms to be deployed in real business scenarios. 

Despite the challenges that we will face in \project, the general feasibility of combining Reasoning about Actions with Synthesis, Planning and Service Compositions required in \project has already been demonstrated in our original, exploratory publications \cite{IJCAI13,IJCAI15,IJCAI16,BPM,AIJ13,AIJ14}, and follow ups \cite{IJCAI15,AAAI17,}. 



%%% Local Variables:
%%% mode: latex
%%% TeX-master: "PartB2"
%%% TeX-PDF-mode: t
%%% End:
