\section{Extended Synopsis of the scientific proposal}

% \note{5 pages\\
% The Extended Synopsis should give a concise presentation of the scientific proposal, with particular attention to the ground-breaking nature of the research project, which will allow evaluation panels to assess, in Step 1 of the evaluation, the feasibility of the outlined scientific approach. Describe the proposed work in the context of the state of the art of the field. References to literature should also be included.

% Please respect the following formatting constraints: Times New Roman, Arial or similar, at least font size 11, margins (2.0 cm side and 1. 5cm top and bottom), single line spacing.
% }



\vspace{-1ex}

%\subsection*{MOTIVATION AND LONG TERM VISION}

\subsection*{LONG TERM VISION} % AND OBJECTIVES}

%
\vspace{-1ex}

Consider the following scenario.
\begin{quote}\it
  After a long week-end, the human supervisor inspects the
  manufacturing system and notices that a production line has
   slowed down significantly, though it is still producing.
%%
  She queries the system on what it is doing.  The system exposes the
  revised process, which is avoiding the use of the production island
  176-671, by repurposing the tools in island 176-716 and sending
  items there.
%%
  She then queries why the system has reprogrammed itself to do so.
  The system answers by showing that on Sunday 11:43pm the island
  176-671 started to produce an unacceptable percentage of defective
  items, based on tests performed during production. 
%%
  So, the system restructured the process to achieve the specified
  objectives (quality and throughput) as well as it could in presence
  of the faulty island, instead of shutting down the production line:
%%
the system analyzed the available capabilities and 
%%
reprogrammed itself to perform
  the current revised process moving the fabrication of the items to
  island 176-716 and reconfiguring at best the tools there to do so.
%%
  Upon request, the %high-level 
  data about the defective process are displayed to the
  supervisor and stored for planning maintenance. Moreover the system
  provides formal evidence that the reconfigured line meets both the
  product specifications and the system constraints.


\end{quote}

\noindent
In the scenario above we have a \textbf{mechanism} (the manufacturing
system) with components, possibly ML-based,  that provide it the sensing and acting capabilities (detecting defective items, reconfiguring the tools of an island), which can be organized suitably to enact a \emph{dynamic behaviour} (the
production process)
%%. 
to meet its \emph{specifications} (constraints on the product and production model).
% The dynamic behavior can be reprogrammed if
% needed.
%%
% The specifications are expressed in a declarative, system-independent form 
% which does not prescribe how it should be realised (here, an equipement-independent process model),
%%
The mechanism has the ability to monitor and detect the faulty part of the process, and crucially it has  \textbf{self-programming}  abilities that it can use to modify its 
current behavior %(the production process) 
\emph{without human intervention}.  
%%
Notably, the mechanism can be \emph{queried} to expose, in terms
understandable to humans, its \emph{self-synthesized program}, the
\emph{specifications} used and the \emph{relationship} between them. In a slogan, the
mechanism is \textbf{white-box}.

The overarching objective of \project is to make this vision a reality:

\begin{quote}\textit{
\project aims at laying the theoretical foundations and developing practical 
methodologies of a science and engineering of \textbf{white-box self-programming mechanisms}. 
}
\end{quote}


To make apparent the significance of this enterprise, \project will
ground its research in three driving applications  considered of pivotal
importance in the current socio-economic context, namely:
\begin{enumerate}
\item  \textbf{Smart Manufacturing}, where significant research efforts are focusing on
improving flexibility, agility and productivity of manufacturing
systems, under the umbrella term \emph{Industry 4.0}, or \emph{4th industrial
revolution}.% \cite{Lorenz15}. 
\footnote{M. Lorenz et al. \textit{Man and Machine in Industry 4.0: How Will Technology Transform the Industrial Workforce Through 2025? }The Boston Consulting Group. 2015.}

\item \textbf{Internet of Things}, which is rising as a
virtual fabric that connects ``things''
equipped with chips, sensors and actuators and allows for building
\textbf{smart objects}
and \textbf{smart spaces} with high levels of awareness of the environment and its human occupants.% \cite{MacGillivray16}. 
\footnote{C. MacGillivray et al. \textit{Worldwide Internet of Things Forecast Update, 2016-2020}. IDC. Doc \# US40755516. 2016.}

\item \textbf{Business Process Management}, which advocates explicit
  conceptual descriptions of a process to be enacted within an
  organization or possibly across organizations, and which is
  instrumental to business processes improvement, the top business
  strategy of CIOs in organizations according to
  Gartner. % \cite{Gartner11}.
\footnote{Gartner Group. \textit{BPM Survey Insights}. Gartner Report. \url{http://www.gartner.com/it/page.jsp?id=1740414}.}
\end{enumerate}
%%


Interestingly forms of self-programmability have been advocated in all
the above contexts.
%%
For example, it is advocated that cyber-physical systems in
Manufacturing or Internet of Things should be able
to \textbf{adapt} themselves to current users and environment by
\textbf{exploiting information gathered at runtime}.  However it is
considered \textbf{impossible to determine a priori all possible adaptations}
that may be needed; thus self-programming abilities would be highly disederable
\cite{Seiger2016}.
%%
In Business Processes, it is considered important for the next
generation of process management systems to allow processes to
automatically \textbf{recover} when unanticipated exceptions
occur, without explicitly defining a priori \textbf{recovery policies},
and \textbf{without the intervention of domain experts} at runtime. These
self-programming abilities would reduce costly and error-prone manual
ad-hoc changes, and would relieve software engineers from mundane
adaptation tasks \cite{MarrellaMS17}.
%%
Note that some of these concerns have been shared by \textbf{autonomic
  computing}, which has promoted self-configuration, self-healing,
self-optimization, and self-protection, though by using policies
provided by IT professionals \cite{ibm2005autonomic}. 
%%
Sophisticated languages and
methodologies for streamlining 
the development of
adaptation and exception
handling recovery procedures have been developed, however  IT
professionals still need to write
all code by hand in the end \cite{XXX}.


% %%
% As a result, in spite of the progresses in the organization of the
% software development process, this traditional way of tacking
% automated reactions in mechanisms is showing serious limitations.


Although the interest is clearly apparent, currently these
self-programing abilities are missing in actual mechanisms, and
science is focussing on limited forms  of self-programming, e.g., for
exception handling and recovery, or forms of composition and 
autonomic reconfiguration \cite{???}.

% \project instead will consider self-programmable mechanisms, as forms
% of \textbf{Agents} studied in \textbf{Artificial Intelligence}
% \cite{Reiter01,Wooldridge09}. \footnote{We stress that \project does
%   not aim at general AI, but envisions self-programming mechanisms
%   that act intelligently within the specific domain of interest in
%   which they operate.}

%More precisely, 
\project intends to \textbf{make a quantum leap in
mechanisms' self-programming abilities, while keeping them white-box}. To do so,  \project will bring together and cross-fertilize four distinct areas with overlapping interests:
\textbf{Knowledge
  Representation} in AI, \textbf{Data-aware Processes} in Databases,  \textbf{Verification and Synthesis} in
Formal Methods, and  \textbf{Automated Planning} in AI. 

\begin{quote}{\it
The PI has profoundly contributed to all these areas and is in a unique position to lead this cross-fertilization.}
%one of the most prominent AI scientist leading this cross-fertilization.}
\end{quote}

 Through enhanced self-programming abilities
 such mechanisms can, e.g.:
%%
\begin{itemize}
\item \emph{Achieve desired goals}, that is guarantee that a certain
  desired state of affairs is eventually reached.  In the above example
  a manufacturing system automatically reconfigures the fabrication
  process if a some workstation is producing too many defective items, by
  changing the sequencing of processing units so as to momentarily
  cut-out the defective tool from the process.
  % As another example, the manufacturing process may refine itself so
  % that every time a defective item arrive to a manufacturing island
  % it is eventually classified and processed accordingly. Note that
  % the notion of defecting, classification and desired reaction can
  % change through self-programming while the tool is in execution.
%%
\item \emph{Maintain themselves within a safe boundary} in the
  changing environment in which they operate.  For example a smart
  space system may keep the desired temperature and humidity in a
  museum room at some desired level, even in presence of a
  particularly large crowd of visitors, possibly by momentarily
  repurposing other actuators, such as the secondary air
  conditioning systems typically used only as a back up of the main
  one.
\item \emph{Keep following rules, regulations and
conventions} that evolve over time while enacting their behavior.  
For example, to answer a new privacy regulation, a business process may refine its behavior to guarantee that the sensible data
  are  erased from the system before the completion of each process instance.
\end{itemize}
%These are just some examples. 
More generally, \project wants to enable
mechanisms to act in an informed and intelligent way in their
environment, by changing the way they behave as a consequence of the
information they acquire from the external world and exchange
with the humans operating therein.
%% 
% That is, we want to introduce to such mechanisms some aspects typical
% of \textbf{Artificial Intelligence} (although limited to their domain of
% interest) so as to able to act in their environment in an informed
% way, changing the way they behave as a consequence of the information
% they acquire from the external world.

Since ``\emph{with great power comes great responsibility}'',
introducing advanced forms of self-programming calls for the ability
to make the behavior automatically synthesized by the mechanism
\textbf{comprehensible} to human supervisors, who are thus able to control and guide it.
%%
%must be checked to be \textbf{harmless}. 
% In line with a large part of the AI community, 
% \project considers this point essential \cite{RussellDT15}.
%%
So it is indeed crucial to develop self-programming mechanisms that are \textbf{white-box}: in every moment the
mechanism can be queried for its specifications, its behavior and how
this relates to the specifications. Ultimately it is the fact that 
the resulting behavior is \textbf{comprehensible in human terms} that will make white-box self-programming mechanisms  \textbf{trustworthy} \cite{CaDa10,Neumann17}.
%


In the first example above, both the reconfiguration goal (cutting out
a defective tool) and how the fabrication process has been modified
need to be explicitly understandable by the humans analyzing the
manufacturing system.
%%
In the second example, the sudden repurposing of the secondary air conditioning
system also needs to be understandable to humans as a reaction to
avoid violating certain safety requirements.
%%
Similarly, in the third example, the goal of erasing sensible data
from the system, and even more importantly how this is achieved, must
be understandable.



We observe that there is a well justified enthusiasm for using ML-techniques
to develop smarter systems.  
%
While typically these ML-components are
black-boxes, in the sense that how they work remains opaque to humans
\cite{MnihKSGAWR13,SilverHMGSDSAPL16},
%
there is currently much work ongoing on incorporating some forms of
human-control into such ML-systems to provide safety guarantees
\cite{ML1,ML2,ML3}. 
%
\project  intends to fully support the integration
of such ML-components with suitable safety guarantees into white-box
self-programming mechanisms.



We further stress that the need to move towards \textbf{white-box}
approaches is advocated by a large part of the \textbf{AI community}
\cite{RussellDT15} as well as the \textbf{CS community} \cite{ACMStatement07}, and has been recently taken up by DARPA within the
context of machine learning, through the DARPA-BAA-16-53 ``Explainable
Artificial Intelligence (XAI)''
program\footnote{\url{http://www.darpa.mil/program/explainable-artificial-intelligence}}

\begin{quote}{\it
Knowledge representation, the primary field of the PI, will be central for realizing the shift towards a white-box approach.}
\end{quote}
%\vspace{-1ex}


\project, exploiting the cross-fertilization among the above mentioned areas, has the potential to yield a breakthrough in engineering self-programming mechanisms that are human-comprehensible and safe by design. This makes it \textbf{very timely} and of the \textbf{greatest significance for European science}.


\vspace{-1ex}

\subsection*{OBJECTIVES}

\vspace{-1ex}

Towards the goal of building \textbf{white-box self-programming mechanisms}, \project will address the following specific objectives. %challenges.
\begin{enumerate}

\item \textbf{Equip mechanisms with general self-programming abilities.}
Mechanisms need general self-programming abilities, not
  restricted to a particular task, such as exception recovery, but ready
  to refine and modify the behavior of the mechanisms as new
  opportunities or constraints arise. 
Self-programming abilities are needed \textbf{while mechanisms
    are in operation}, that is while the mechanisms are executing, not
   at design time.
\project aims at                           
  advanced forms of \textbf{process synthesis} as those studied in
  \textbf{reactive synthesis} by Verification and Synthesis
  community in Formal Methods \cite{PnRo89,Vardi17}, but
%%
  \textbf{sidestepping} the notorious difficulties
     by focusing on non-traditional forms of
  specification formalisms, such as LTL and LDL on finite
    traces, recently proposed in
  AI \cite{TorresB15,DeVa15,DeVa16,CamachoTMBM17} 
and exploiting algorithmic insights form the \textbf{Automated Planning} in AI
  \cite{GeffnerBo13,GNT2016,NauGT15}.


% \item \textbf{Make white-box self-programming mechanisms verifiable.}
% %\textbf{Self-programming mechanisms need to be verifiable.}
%   \project aims at building self-programming mechanisms whose replanned behaviors
%   are \emph{verifiable} agains their specifications, e.g., by \textbf{model checking}\cite{},
%   possibly \textbf{modulo theories}  \cite{EiterGS10}.
%   This is a crucial step towards the understandability required by a
%   \textbf{white-box approach}.  In this way mechanisms can be checked, e.g.,  to understand if important safety conditions are
%   satisfied. In tackling this aspect \project will also leverage on
%   recent advances of model checking of autonomous
%   agents \cite{Wooldridge09,LomuscioQR17}.

\item \textbf{Make white-box self-programming mechanisms comprehensible to
    humans and verifiable}. 
% \textbf{Self-programming mechanisms need to be comprehensible to
%     humans}. 
\project requires specifications, solutions (synthesized programs)
  and the relationship between them to be
  \textbf{comprehensible to humans}.   
% Neural networks, can be
%   effectively used for finding a specific solution, but they cannot be
%   used as a human comprehensible representation of the solution space,
%   since we do not have control on the abstraction/compression they
%   perform \cite{MnihKSRVBGRFOPB15}.
%%
  This means that specifications and solution spaces must be
  \textbf{semantically} described at high-level using predicates that
  are understandable to humans, as advocated by \textbf{Knowledge
    Representation} in AI (that is, it is fine to say
  \texttt{Island named ``176-671'' is under stress}, but not to say \texttt{Flag123456=on})
  \cite{Baral10,EiterEFS10,BrewkaEP14,Shoham16,Levesque14,Levesque17}. 
%%
Clearly replanned behaviors
  must be  \emph{verifiable} agains the specifications, e.g., by \textbf{model checking} \cite{}.
  This is a crucial step towards the understandability required by a
  \textbf{white-box approach} \cite{}.  

\item \textbf{Make self-programming mechanisms data-aware.}
%\textbf{Self-programming mechanisms need to be information-aware.}
  During the execution, new facts about the world are observed,
  learned, or received as input. This calls for a representation that
  distinguishes \textbf{intensional information} such as that provided
  by knowledge of the domain, from \textbf{extensional information}
  provided by actual data. Self-programming mechanisms leverage on the
  intensional information to be able to interpret new data
  (extensional information) acquired, observed, learned.  Notice that
  this calls for a \textbf{relational (first-order) representation of
    the state}.  New results on verifiability of \textbf{data-aware
    processes}, based on faithfull abstraction fo finite state
  transition systems, are
  crucial for this \cite{ClassenL08,HaririCGDM13,ClassenLLZ14,HullSV13,BelardinelliLP14,DeGLP16,CDMP17}.


\item \textbf{Support component-based approaches.}  
% \textbf{Self-programming mechanisms favor component-based
%     systems.}  
  By no means should we consider programmable mechanisms to be
  composed of a single unit.  Indeed, understandability calls for
  building \textbf{high-level components} that are relatively simple
  to understand, verify and combine. Then, it is crucial to study how
  \textbf{composing} ``correct'' components leads to an overall
  ``correct'' behavior. 
%%
\project will leverage on work on composition and customization in SOC and more recently in AI
 \cite{wsf2014,SohrabiPM09,BertoliPT10,DePS13,DeGGPSS16}. %NissimB14,DeGGPSS16}.
%%
as well as in execution \textbf{monitoring} to detect 
failure, identify responsible components, and synthesize recovery \cite{DeGMGMM14,MarrellaMS17}.


\item 
\textbf{Allow learning and stochastic decisions, while remaining within safe bounds.}
% \textbf{Self-programming mechanisms need to keeping satisfying
%     their specifications, while they learn and make stochastic
%     decisions}.  
We want to allow mechanisms to have forms of decision making that
resist formal analysis (at least in human terms), because we want to
make use of the possibilities that advancements in deep learning,
MDP's, and reinforcement learning bring about. Though, while the actual
execution could be chosen stochastically, we do want to have
guarantees on \textbf{all possible generated executions}  \cite{ML1,ML2,ML3}.  In this
way, it is the \textbf{entire space of solutions} that has
\textbf{formal guarantees}, and the specific solution chosen by the
learning algorithm or the stochastic decision maker will also satisfy
them. %Hence, however chosen, the actual solution will satisfy the desired guarantees.



\end{enumerate}




\subsection*{METHODOLOGY}

As mentioned, the scientific work integrates elements from 4 distinct areas, namely 
\textbf{Knowledge Representation}, \textbf{Data-aware Processes},
\textbf{Verification and Synthesis } and \textbf{Automated Planning}.

Specifically the work within \project will be organized in 5 workpackages roughly
corresponding to the 5 objectives above, plus 3 workpackage corresponding to
the 3 driving applications. 

\textbf{WP1: Supplying core self-programming abilities.}
\project  will make use of the mathematically elegant theory of \textbf{Reactive
  Synthesis} \cite{PnRo89} developed in formal methods in the last 30 years \cite{RecentSynthesis}, which
however has not found diffused practical application because of the
\textbf{intrinsic difficulties} of certain algorithms and constructions \cite{TsaiFVT14,DFogartyKVW13}.
%
We aim at \textbf{sidestepping these difficulties all-together}, by
focusing on non-traditional kinds of specification
formalisms. Examples of these are LTL and
LDL on finite traces, recently proposed by the PI together with Moshe Vardi (Rice U, Huston)
\cite{DeVa13,DeVa15,DeVa16} and adopted in advanced forms of planning  \cite{TorresB15,CamachoTMBM17} and in \textbf{declarative business processes} in BPM \cite{AalstPS09,DeGMGMM14,AAAI17}, as well as safety/co-safety LTL/LDL formulas, which have been shown to be more expressive than expected while well-behaved \cite{}
%%
On this theme, the PI team will collaborate with Moshe Vardi (Rice U.) on automata-based verification and synthesis, and Nello Murano (U.\ Napoli, Italy), Sasha Rubin (U.\ Napoli, Italy) and Benjamin Aminof (TU Wien, Austria) on game-based verification and synthesis.
%%
%%
 While \project will consider symbolic techniques
adopted in synthesis by model checking \cite{BloemJPPS12}, it
aims at leveraging on the exceptional scalability improvements of
current algorithms in \textbf{Automated Planning} in AI, to devise radically
different techniques to effectively tackle reactive synthesis in
practice \cite{GeffnerBo13}.  
%%
In particular like agents in planning, we expect mechanisms to be able
to handle quickly and efficiently most cases, i.e., those cases that
do not require to solve difficult, ``puzzle-like'', situations. Indeed
while the Planning community has concentrated on simpler forms of
process synthesis, it has developed a sort of \textbf{science of
  search algorithms for planning}, which has brought about
improvements by orders of magnitude in the last decade
\cite{PommereningHB17,SteinmetzH17,LipovetzkyG17,DeGMMP17}. \project
will exploit this knowledge and extend it to generalized forms of
planning and to reactive synthesis.
%%
The PI has been pioneering cross-fertilization of planning and
synthesis since the end of the '90
\cite{DeGiacomoV99,CalvaneseGV02,SardinaGLL06,DeGiacomoFPS10,PatriziLGG11,DeGMMP17}. More
recently the PI has established \textbf{tight connections betweens synthesis
and generalized forms planning}
\cite{HuG11,HuG13,DeGiacomoMRS16,BDGR17} as well as between planning
and behavior compositions
\cite{SardinaG08,DePS13,DeGGPSS16,CalvaneseGLV16}.  On this theme, the
PI team will collaborate with Hector Geffner (UPF, Barcelona), Blai
Bonet (U. Simon Bolivar, Caracas), Alfonso Gerevini (U. Brescia,
Italy) and Malte Helmert (U.\ Basil).



\textbf{WP2: Achieving core white-box characteristics.}
 \project will leverage on the large
body of work on actions theoreis developed in \textbf{Knowledge
Representation}, to which the PI has contributed significantly in the
years
\cite{DeGiacomoRS98,DeGiacomoLL00,DeGiacomoLS01,SardinaGLL04,SardinaG09,DeGiacomoLP10,DeGiacomoLM12,DeGiacomoLPV14,DeGiacomoLPV16,BanihashemiGL17}. From
such work, \project will draw key ideas on how to represent
mechanisms, the domain in which they are operating, and the properties
of interest in a high-level human-comprehensible fashion. However
particular attention will be given to computational effectiveness, in
line with some recent exploratory work by the PI
\cite{DeGLP16,DeGiacomoLPS16,CDMP17}.  
%%
%%
Notice that on the other hand \project does not aim at defining new concrete representation languages. Instead it intends to use well-known formalisms such as BPMN, UML, OWL, etc. as concrete languages, see e.g., \cite{CAISE17}.
%%
Clearly replanned behaviors
  must be  \emph{verifiable} agains the specifications, e.g., by \textbf{model checking},
  possibly \textbf{modulo theories} \cite{EiterGS10}.
  This is a crucial step towards the understandability required by a
  \textbf{white-box approach}.  In this way mechanisms can be checked, e.g.,  to understand if important safety conditions are
  satisfied. In tackling this aspect, \project will also leverage on
  recent advances of model checking of autonomous
  agents \cite{Wooldridge09,LomuscioQR17}.
%%
On this theme, the PI team will
collaborate with Yves Lesperance (York U., Toronto, Canada), Hector
Levesque (U. Toronto, Canada), Sebastian Sardina (RMIT, Melbourne,
Australia), and Yongmei Liu (Sun Yat-sen U., Guangzhou, Cina).

\textbf{WP3: Making white-box self-programming mechanisms data-aware} 
\project will consider mechanisms that
deal with data. It is known that verification and even more
synthesis of \textbf{data-aware processes} are in general problematic, since
processes generate infinite-state transition systems. However the PI has
already shown, within the EU FP7-ICT-257593 ACSI: Artifact-Centric
Service Interoperation, that such difficulties can be overcome in
notable cases
\cite{BerardiCGHM05,CalvaneseGHS09,HaririCGDM13,CalvaneseGMP13}. Since
then important advancements in undestanding how to deal with such
complexity have been established as well as relationships with action theories
\cite{HaririCMGMF13,BelardinelliLP14,CalvaneseGS15,CDMP17}.  On this
theme, the PI and his team will collaborate with Rick Hull (IBM Research,
USA), Jianwen Su (UCSB, USA), Diego Calvanese and
Marco Montali (U.\ Bolzano, Italy), and with Alessio Lomuscio
(Imperial College, London, UK).





\textbf{WP4: Supporting component-based mechanisms.}
% \paragraph{Composition and Customization.}
% Other examples are the form of behavior specification adopted to
% specify target collective behavior when composing conversational
% services or component behaviors, adopted in SOC or in \textbf{behavior
%   composition} in AI \cite{DePS13,DeGGPSS16}.
%%
  By no means should we consider programmable mechanisms to be
  composed of a single unit. \textbf{Service-Oriented Computing} and
  \textbf{Open APIs} frameworks have long been pushing for
  \textbf{component-based systems}, in which a set of components are
  \textbf{customized} and \textbf{orchestrated} to deliver a required
  service \cite{wsf2014}.  Indeed, understandability calls for
  building \textbf{high-level components} that are relatively simple
  to understand, verify and combine. Then, it is crucial to study how
  \textbf{composing} ``correct'' components leads to an overall
  ``correct'' behavior.
%%
\project will leverage on the body of work on composition and customization developed in SOC and more recently in AI
 \cite{SohrabiPM09,BertoliPT10,DePS13,DeGGPSS16}.%NissimB14,DeGGPSS16}.
%%
Moreover, execution should be \textbf{monitored} so that in case of
failure it is possible to identify the responsible component, and
recover the situation by reprogramming the mechanism for alternative
solutions that circumvent the failing
component \cite{DeGMGMM14,MarrellaMS17}.



\textbf{WP5: Supporting learning and stochastic components.}
We want to allow mechanisms to have forms of decision making that
resist formal analysis (at least in human terms), because we want to
make use of the possibilities that advancements in deep learning,
MDPs, and reinforcement learning bring about. Though, while the actual
execution could be chosen stochastically, we do want to have
guarantees on \textbf{all possible generated executions}.  In this
way, it is the \textbf{entire space of solutions} that has
\textbf{formal guarantees}, and the specific solution chosen by the
learning algorithm or the stochastic decision maker will also satisfy
them. %Hence, however chosen, the actual solution will satisfy the desired guarantees.
%%
  This calls for allowing \textbf{coexistence of logical constraints
    with stochastic solutions}, a theme that has only be scratched by
  the scientific community so far \cite{BeckL12,SprauelKT14,BDMS17}.
%%
   We also observe that synthesis against constraints has been used to bound the
  possible solutions in several context, most notably in supervisory
  control \cite{Wo14,BanihashemiGL16}.
%%
In this theme the PI will collaborate  with
Ronen Brafman (Ben-Gurion U.) to explore
temporally-extended goal planing and synthesis in non-Markovian MDPs
and reinforcement learning (first ideas in \cite{BDMS17}).



\textbf{Application and Evaluation.}
\project will ground its scientific results in
diverse real \textbf{driving application} to demonstrate the actual utilization of the scientific
achievements within the project: \textbf{WP6: Smart manufacturing (Industry 4.0)},
\textbf{WP7: Smart spaces (IoT)},
\textbf{WP8: Business Processes Management Systems (BPM)},
%%
The PI and his group at Sapienza has
contributed to all these fields, see e.g.,
\cite{DeGiacomoCFHM12,DeGiacomoDMM15,SilvaFCLSR17}.  Moreover the PI
has shown to be able to apply advanced science to real-cases in the
area Semantic Data Integration where he has been as one of the main
proposers of the Ontology-Based Data Access paradigm, possibly the
most successful approach for Semantic Data Integration \cite{PoggiLCGLR08,SequedaM17,Statoil17}; he
contributed to W3C recommendation of OWL 2 Web Ontology Language
Profiles (\url{https://www.w3.org/TR/owl2-profiles/}); and he has
founded a Sapienza Start-Up \textbf{OBDA Systems}
(\url{http://www.obdasystems.com}) to  commercially exploit Ontology-Based Data Access  in real data integration scenarios. 


%DeGiacomoDMM15 DeGiacomoCFHM12


% textbf{Project structure.}
% The scientific work in \project will be divided into 3 research streams of the duration of the  project.
% \begin{itemize}
% \item \textbf{Stream 1: Foundations.} This stream will deal with the
% scientific foundations of white-box self-programming mechanisms. 

% \item \textbf{Stream 2: Algorithms and Tools.}  This stream  will deal with the
% development of practical algorithms, optimizations and tools for realizing
% white-box self-programming mechanisms. 

% \item \textbf{Stream 3: Applications and Evaluation.}  This stream  will evaluate white-box self programming mechanisms in the three business critical application contexts mentioned above.
% \end{itemize}


\vspace{-1ex}
\subsection*{HIGH RISK, HIGH GAIN}

\vspace{-1ex}

\project is a \textbf{high risk, high gain project}: if successful, it
will result in radically more useful automated mechanisms than what
we have today, namely \textbf{white-box self-programming mechanisms},
unleashing the full potential of \textbf{self-programmability} and
removing the main barriers to the uptake of automated mechanisms in
real business context, namely: \textit{predefined and restricted forms of automation}, and
difficulties in \textit{formally analyzing their automated behavior in human terms}.

Despite the challenges that \project will face, 
%%
% the general feasibility of combining Reasoning about Actions with Synthesis, Planning and Service Compositions required in \project has already been demonstrated in our original, exploratory publications \cite{IJCAI13,IJCAI15,IJCAI16,BPM,AIJ13,AIJ14}, and follow ups \cite{IJCAI15,AAAI17,}. 
%%
% We are seeing the need of self-programming abilities stemming out in
% several fields of CS and AI.
%%
%At the same time, 
%we are witnessing a convergence between research in
it can take advantage of an emerging convergence between research on
\textbf{Knowledge Representation}, \textbf{Data-aware
  Processes},\textbf{Verification and Synthesis } and
\textbf{Automated Planning}, which are the most prominent areas developing
methodologies, algorithms and tools related to \textbf{white-box
  self-programming mechanisms}. This convergence has already allowed
the PI to show decidability of data-aware processes in spite of them
being infinite state due to data \cite{HaririCGDM13,DeGLP16,CDMP17}, and to bringing about new
feasibility results, which sidestep some intrinsic difficulties of
reactive synthesis algorithms and constructions (e.g.,
determinization) making synthesis practically feasible in notable
cases \cite{DeVa13,DeVa15,DeVa16}.

% \begin{quote}{\it
% The PI is one of the most prominent AI scientist leading this cross-fertilization.}
% \end{quote}

% We further stress that the need to move towards \textbf{white-box} approaches is advocated by a large part of the \textbf{AI community} \cite{RussellDT15}, and has been recently taken up by DARPA within the context of machine learning, through the DARPA-BAA-16-53 ``Explainable Artificial Intelligence (XAI)'' program\footnote{\url{http://www.darpa.mil/program/explainable-artificial-intelligence}}

% \begin{quote}{\it
% Knowledge representation, the primary field of the PI, will be central for realizing the shift towards a white-box approach.}
% \end{quote}

% These two aspects make the \project very timely and of greatest significance for European science.



% \subsection*{PROJECT STRUCTURE}

% %\vspace{-1ex}

% The scientific work in \project will be divided into 3 research streams of the duration of the  project.

% \textbf{Stream 1: Foundations.} This stream will deal with the
% scientific foundations of white-box self-programming mechanisms. The
% research work will span from reasoning about actions, planning, and
% reactive synthesis to non-Markovian MDPs and reinforcement
% learning. But it will also concern work on semantic technologies for
% representing mechanisms states in first-order terms.

% \textbf{Stream 2: Algorithms and Tools.}  This stream  will deal with the
% development of practical algorithms, optimizations and tools for realizing
% white-box self-programming mechanisms. It will concern work in synthesis, including
% synthesis by model checking, and especially work in planning, which is
% delving deeply in the algorithmic aspects of (simplified forms of)
% synthesis.

% \textbf{Stream 3: Applications and Evaluation.}  This stream will
% ground its scientific results in diverse real \textbf{application
%   contexts}, including manufacturing systems
% (Industry 4.0),  smart spaces (IoT), and business processes (BPM) to demonstrate the
% actual utilization of the scientific achievements within the project.


% \textbf{Collaborations.}
% The scientific work for will take advantage of  collaboration with several top research groups working in various aspects related to \project, including
% %Specifically:
% %\begin{inparaenum}[\it (i)]
% % \begin{itemize}
% % \item 
% Moshe Vardi (Rice U.) for Reactive Synthesis;
% %
% %\item
% Nello Murano (U.\ Napoli), Sasha Rubin (U.\ Napoli), Benjamin Aminof (TU Wien), for Game-Theoretic Verification; 
% %
% %\item 
% Hector Geffner (UPF, Barcelona)
%  and Malte Helmert (U.\ Basil) for Planning;
% %
% %\item 
% Ronen Brafman (Ben-Gurion U.) for MDPs and Reinforcement Learning;
% %
% %\item 
% Alessio Lomuscio (IC London) for Agents;
% %
% %\item 
% Diego Calvanese and Marco Montali (U.\ Bolzano) for Data-Aware Processes;
% %
% %\item 
% Yves Lesperance (York U.) and Hector Levesque (U. Toronto) for Reasoning about Actions.
% %
% %\item 
% Paolo Felli (Nottingham U.) for Manufacturing;
% %
% %\item 
% Maurizio Lenzerini and other colleagues at Sapienza for Ontologies and Semantic Technologies;
% %
% %\item 
% Massimo Mecella and other colleagues at Sapienza for applications in Manufacturing, IoT and Business Processes.
% %\end{inparaenum}
% %\end{itemize}
% % The collaboration will allow for reciprocal visits for short and long
% % periods to write papers and develop tools and application together, as
% % well as to allow for organizing joint scientific workshops on the themes
% % of the project.

% %%
% %%
% % \begin{tabular}{p{2cm}p{14cm}}
% % %& Benjamin Aminof, Technische Universität Wien, Vienna, Austria\\
% % %& Blai Bonet, Universidad Simón Bolívar, Caracas, Venezuela\\
% % & Ronen Brafman, Ben-Gurion University, Beer-Sheva, Israel \\
% % & Diego Calvanese, Free University of Bozen-Bolzano, Italy\\
% % %& Alin Deutsch, University of California, San Diego, CA, USA\\
% % & Rick Hull, IBM Research, Yorktown Heights, NY, USA\\
% % & Hector Geffner, Universitat Pompeu Fabra, Barcelona, Spain\\
% % %& Alfonso Emilio Gerevini, Universit\`a di Brescia, Italy\\
% % & Alessio Lomuscio, Imperial College London, UK\\
% % & Maurizio Lenzerini, Sapienza Universit\`a di Roma, Italy\\
% % & Yves Lesperance, York University, Toronto, ON, Canada\\
% % & Hector Levesque, University of Toronto, Toronto, ON, Canada\\
% % & Yongmei Liu, University in Guangzhou, China\\
% % %& Aniello Murano, Universit\`a di Napoli, Italy\\
% % & Adrian Pearce, University of Melbourne, Melbourne, SW, Australia\\
% % & Ray Reiter, University of Toronto, Toronto, ON, Canada\\
% % %& Riccardo Rosati, Sapienza Universit\`a di Roma, Italy\\
% % %& Sasha Rubin, Universit\`a di Napoli, Italy\\
% % & Sebastian Sardina, RMIT, Melbourne, SW, Australia\\
% % %& Jianwen Su, University of California, Santa Barbara, CA, USA\\
% % & Moshe Vardi, Rice University, Huston, TX, USA, Topic
% % \end{tabular}


% % Yves' comments:

% % issues:
% % - Feasibility?
% % - Human vs Machine Understandable?
% % - Detail relation to machine learning approaches
% %    - Safe regions vs goals/objective vs optimization criteria
% %    - Dynamism

%%% Local Variables:
%%% mode: latex
%%% TeX-master: "PartB1"
%%% TeX-PDF-mode: t
%%% End:
