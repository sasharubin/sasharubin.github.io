\section{Extended Synopsis of the scientific proposal}

% \note{5 pages\\
% The Extended Synopsis should give a concise presentation of the scientific proposal, with particular attention to the ground-breaking nature of the research project, which will allow evaluation panels to assess, in Step 1 of the evaluation, the feasibility of the outlined scientific approach. Describe the proposed work in the context of the state of the art of the field. References to literature should also be included.

% Please respect the following formatting constraints: Times New Roman, Arial or similar, at least font size 11, margins (2.0 cm side and 1. 5cm top and bottom), single line spacing.
% }



\vspace{-3ex}

%\subsection*{MOTIVATION AND LONG TERM VISION}

\subsection*{LONG TERM VISION} % AND OBJECTIVES}

%
\vspace{-3ex}

Consider the following scenario.
\begin{quote}\it
  After a long week-end, the human supervisor inspects the
  manufacturing system and notices that a production line has
   slowed down significantly, though it is still producing.
%%
  She queries the system on what it is doing.  The system reveals to her  the
  revised process, which is avoiding the use of the production island
  176-176, by repurposing the tools in island 176-671 and sending
  items there.
%%
  She then queries why the system has reprogrammed itself to do so.
  The system answers by showing that on Sunday 11:43pm the island
  176-176 started to produce an unacceptable percentage of defective
  items, based on tests performed during production. 
%%
  So, the system restructured the process to achieve the specified
  objectives (quality and throughput) as well as it could in the presence
  of the faulty island, instead of shutting down the production line:
%%
  the  system analyzed  the  available  capabilities and  reprogrammed
  itself resulting in the current revised process.
 % moving the fabrication of the items to
 %  island 176-671 and reconfiguring the tools there to do so.
%%
  % Upon request, the %high-level 
  % data about the defective process are displayed to the
  % supervisor and stored for planning maintenance. Moreover the system
  % provides formal evidence, through verification, on whether  the reconfigured process meets dynamic properties queried by the supervisor. % both the product specifications and the system constraints.


\end{quote}

\noindent
In the scenario above we have a \textbf{mechanism} (the manufacturing
system) with multiple components,  possibly using Machine Learning (ML) to provide sensing and acting capabilities (detecting defective items, reconfiguring tools parameters), which can be suitably organized  to enact a \emph{dynamic behaviour} (the
production process)
%%. 
to meet its \emph{specifications} (constraints on the product and production model).
% The dynamic behavior can be reprogrammed if
% needed.
%%
% The specifications are expressed in a declarative, system-independent form 
% which does not prescribe how it should be realised (here, an equipement-independent process model),
%%
The mechanism has the ability to monitor and detect faulty parts of the process, and crucially it has  \textbf{self-programming}  abilities that it can use to modify its 
current behavior %(the production process) 
\emph{without human intervention}.  
%%
Notably, the mechanism can be \emph{queried} to display, in terms
understandable to humans, under which circumstances and for fulfilling which \emph{specifications}, it reprogrammed itself. Moreover it can be queried on whether  its \emph{self-synthesized program} meets any dynamic property of interest to the human supervisor (e.g., for ``what-if'' analysis).
%its \emph{self-synthesized program}, the
%\emph{specifications} used and the \emph{relationship} between them. 
In a slogan, the
mechanism is \textbf{white-box}.

The overarching objective of \project 
 % (pronounced ``wisemech''\footnote{From Urban Dictionary: ``Whise is an uber form of wise. Has higher epic factor than dumb old wise. Whenever used, you must say whise-with-an-H to avoid confusion with wise. Coolness guaranteed whenever used.''}) 
is to make this vision a reality: %%Yves: drop this sentence

\begin{framed}\textit{
\project aims at laying the theoretical foundations and developing practical 
methodologies of a science and engineering of \textbf{white-box self-programming mechanisms}. 
}
\end{framed}


To make apparent the significance of this enterprise, \project will
ground its research in three driving applications  considered of pivotal
importance in the current socio-economic context, namely:
\begin{enumerate}
\item  \textbf{Smart Manufacturing}, where significant research efforts are focusing on
improving flexibility, agility and productivity of manufacturing
systems, under the umbrella term \emph{Industry 4.0}, or \emph{4th industrial
revolution}.% \cite{Lorenz15}. 
\footnote{M. Lorenz et al. \textit{Man and Machine in Industry 4.0: How Will Technology Transform the Industrial Workforce Through 2025? }The Boston Consulting Group. 2015.}

\item \textbf{Internet of Things}, which is rising as a
virtual fabric that connects ``things''
equipped with chips, sensors and actuators and allows for building
\textbf{smart objects}
and \textbf{smart spaces} with high levels of awareness of the environment and its human occupants.% \cite{MacGillivray16}. 
\footnote{C. MacGillivray et al. \textit{Worldwide Internet of Things Forecast Update, 2016-2020}. IDC. Doc \# US40755516. 2016.}

\item \textbf{Business Process Management}, which advocates explicit
  conceptual descriptions of a process to be enacted within an
  organization or possibly across organizations, and which is
  instrumental to business processes improvement, the top business
  strategy of CIOs in organizations according to
  Gartner. % \cite{Gartner11}.
\footnote{Gartner Group. \textit{BPM Survey Insights}. Gartner Report. \url{http://www.gartner.com/it/page.jsp?id=1740414}.}
\end{enumerate}
%%


Interestingly forms of self-programmability have been advocated in all
the above contexts.
%%
For example, it is advocated that cyber-physical systems in
Manufacturing or Internet of Things should be able
to \textbf{adapt} themselves to current users and environment by
\textbf{exploiting information gathered at runtime}.  However it is
considered \textbf{impossible to determine a priori all possible adaptations}
that may be needed; thus self-programming abilities would be highly desirable
\cite{Seiger2016}.
%%
In Business Processes, it is considered important for the next
generation of process management systems to allow processes to
automatically \textbf{recover} when unanticipated exceptions
occur, without explicitly defining a priori \textbf{recovery policies},
and \textbf{without the intervention of domain experts} at runtime. These
self-programming abilities would reduce costly and error-prone manual
ad-hoc changes, and would relieve software engineers from mundane
adaptation tasks \cite{MarrellaMS17}.
%%
Note that some of these concerns have been shared by \emph{autonomic
  computing}, which has promoted self-configuration, self-healing,
self-optimization, and self-protection, though by using policies
provided by IT professionals \cite{ibm2005autonomic}. 
%%
Sophisticated languages and
methodologies for streamlining 
the development of
adaptation and exception
handling recovery procedures have been developed, however  IT
professionals still need to write
all code by hand in the end \cite{Cerf15,MarronAEGKLMSSW16}.


% %%
% As a result, in spite of the progresses in the organization of the
% software development process, this traditional way of tacking
% automated reactions in mechanisms is showing serious limitations.


Although the interest is clearly apparent, \textbf{currently these
self-programing abilities are missing} in actual mechanisms, and
science is focusing on limited forms of self-programming, e.g., for
exception handling and recovery, or forms of composition and 
autonomic reconfiguration to be applied at design time \cite{HarelKMM16}.

% \project instead will consider self-programmable mechanisms, as forms
% of \textbf{Agents} studied in \textbf{Artificial Intelligence}
% \cite{Reiter01,Wooldridge09}. \footnote{We stress that \project does
%   not aim at general AI, but envisions self-programming mechanisms
%   that act intelligently within the specific domain of interest in
%   which they operate.}

%More precisely, 


\project intends to \textbf{make a quantum leap in
mechanisms' self-programming abilities}. %, while keeping them white-box}. 
% To do so,  \project will bring together and cross-fertilize four distinct areas with overlapping interests:
% \textbf{Knowledge
%   Representation} in AI, \textbf{Data-aware Processes} in Databases,  \textbf{Verification and Synthesis} in
% Formal Methods, and  \textbf{Planning} in AI. 
%%
% \begin{quote}{\it
% The PI has profoundly contributed to all these areas and is in a unique position to lead this cross-fertilization.}
% %one of the most prominent AI scientist leading this cross-fertilization.}
% \end{quote}
%%
 Through enhanced self-programming abilities
 such mechanisms can, e.g.:
%%
\begin{itemize}
\item \emph{Achieve desired goals}, that is guarantee that a certain
  desired state of affairs is eventually reached.  In the above example
  a manufacturing system automatically reconfigures the fabrication
  process if some workstation is producing too many defective items, by
  changing the sequencing of processing units so as to temporarily
  cut-out the defective tool from the process, so as to achieve an acceptable error rate.
  % As another example, the manufacturing process may refine itself so
  % that every time a defective item arrive to a manufacturing island
  % it is eventually classified and processed accordingly. Note that
  % the notion of defecting, classification and desired reaction can
  % change through self-programming while the tool is in execution.
%%
\item \emph{Maintain themselves within a safe boundary} against unanticipated
  changes in the environment in which they operate.  For example a smart
  space system may keep the desired temperature and humidity in a
  museum room at some desired level, even in presence of an unforeseen
  large crowd of visitors, possibly by momentarily repurposing other
  actuators, such as a secondary air conditioning systems typically
  used only in case of failure of the main one.
\item \emph{Keep following regulations and conventions} that evolve
  over time while enacting their behavior.  For example, to answer a
  new privacy regulation, a business process may refine its behavior
  to guarantee that the sensitive data are erased from the system
  before the completion of each process instance.
\end{itemize}
%These are just some examples. 
More generally, \project wants to enable
mechanisms to act in an informed and intelligent way in their
environment, by changing the way they behave as a consequence of the
information they acquire from the external world and exchange
with the humans operating therein.
%% 
% That is, we want to introduce to such mechanisms some aspects typical
% of \textbf{Artificial Intelligence} (although limited to their domain of
% interest) so as to able to act in their environment in an informed
% way, changing the way they behave as a consequence of the information
% they acquire from the external world.

Since ``\emph{with great power comes great responsibility}'',
introducing advanced forms of self-programming calls for the ability
to make the behavior automatically synthesized by the mechanism
\textbf{comprehensible} to human supervisors, who are thus able to control and guide it.
%%
%must be checked to be \textbf{harmless}. 
% In line with a large part of the AI community, 
% \project considers this point essential \cite{RussellDT15}.
%%
So it is indeed crucial to develop self-programming mechanisms that
\textbf{can be queried}, i.e.,  that are \textbf{white-box}: in every moment
the mechanism can be queried for the status of its specifications, 
%on how such specifications relate to the synthesized behaviour, 
and on whether its behavior meets any dynamic property of interest to
the human supervisors.
%%
Ultimately it is the fact that the resulting behavior is
\textbf{comprehensible in human terms} that will make white-box
self-programming mechanisms \textbf{trustworthy}
\cite{CaDa10,Neumann17}.
%


In the first example above, both the reconfiguration goal (cutting out
a defective tool) and how the fabrication process has been modified
need to be explicitly understandable by the humans analyzing the
manufacturing system.
%%
In the second example, the sudden repurposing of the secondary air conditioning
system also needs to be understandable to humans as a reaction to
avoid violating certain safety requirements.
%%
Similarly, in the third example, the goal of erasing sensitive data
from the system, and even more importantly how this is achieved, must
be understandable.


We  stress that the need to move towards \textbf{white-box}
approaches is advocated by a large part of the \textbf{AI community}
\cite{RussellDT15} as well as the \textbf{CS community} \cite{ACMStatement07}, and has been recently taken up by DARPA within the
context of machine learning, through the DARPA-BAA-16-53 ``Explainable
Artificial Intelligence (XAI)''
program.\footnote{\url{http://www.darpa.mil/program/explainable-artificial-intelligence}}
%%
%\begin{framed}{\it
\emph{Knowledge representation, the primary field of the PI, will be central for realizing the shift towards a white-box approach.}
%}
%\end{framed}
%\vspace{-3ex}


We observe that there is a well justified enthusiasm for using ML-techniques
to develop smarter systems.  
%
While typically these ML-components are
black-boxes, in the sense that how they work remains opaque to humans
\cite{MnihKSGAWR13,SilverHMGSDSAPL16},
%
there is currently much work ongoing on incorporating some forms of
human-control into such ML-systems to provide safety guarantees
\cite{AmodeiOSCSM16,TYLC16,Hadfield-Menell16a}. 
%
\project  intends to fully support the integration
of such ML-components with suitable safety guarantees into white-box
self-programming mechanisms.



\vspace{-3ex}

\subsection*{OBJECTIVES}

\vspace{-3ex}

Towards the goal of building \textbf{white-box self-programming mechanisms}, \project will address 5 specific objectives, described below together with the area involved in achieving them. 

\begin{enumerate}
\item \textbf{Equip mechanisms with general self-programming abilities.}
Mechanisms need general self-programming abilities, not
  restricted to a particular task, such as exception recovery, but ready
  to refine and modify the behavior of the mechanisms as new
  opportunities or constraints arise. 
Self-programming abilities are needed \textbf{while mechanisms
    are in operation}, that is while the mechanisms are executing, not
   at design time.
\project aims at                           
  advanced forms of \textbf{process synthesis} as those studied in
  \textbf{Verification and Synthesis} in Formal Methods \cite{PnRo89,EhlersLTV17} and especially in \textbf{Generalized Planning} in AI \cite{TorresB15,DeVa15,DeVa16,CamachoTMBM17}.
\begin{comment}
, which is proposing techniques to  
%
  \textbf{sidestep} the notorious difficulties in synthesis
     by focusing on \textbf{non-traditional forms of
  specification formalisms}, such as LTL and LDL on finite
    traces \cite{TorresB15,DeVa15,DeVa16,CamachoTMBM17}. Moroever we intend to
 exploite \textbf{algorithmic insights} form the \textbf{AI Planning}
  \cite{GeffnerBo13,GNT2016,NauGT15}.
\end{comment}


\item \textbf{Make self-programming mechanisms comprehensible and
    verifiable by humans.}  \project requires specifications,
  solutions (synthesized programs) and the relationship between them
  to be \textbf{comprehensible to humans}.  This means that
  specifications and solution spaces must be \textbf{semantically}
  described at high-level using predicates that are understandable to
  humans, as advocated by \textbf{Knowledge Representation}  (KR)
  % (that is, it is fine to say \texttt{Island named ``176-671'' is
  %   under stress}, but not to say \texttt{Flag123456=on})
  \cite{Baral10,EiterEFS10,EiterGS10,BrewkaEP14,Shoham16,Levesque14,Levesque17}.
%%
Clearly,  a crucial step towards the understandability, is to be able to 
\textbf{verify} replanned behaviors against their specifications as advocated by \textbf{Verification and Synthesis}.

\item \textbf{Make self-programming mechanisms data-aware.}
  During the execution, new facts about the world are observed,
  learned, or received as input. This calls for a representation that
  distinguishes \textbf{intensional information} such as that provided
  by knowledge of the domain, from \textbf{extensional information}
  provided by actual data. Self-programming mechanisms leverage on the
  intensional information to be able to interpret new data
  (extensional information) acquired, observed, and learned.  Notice that
  this calls for a \textbf{relational (first-order) representation of
    the state}.  New results on verifiability of \textbf{Data-Aware
    Processes}, based on faithful abstraction fo finite-state
  transition systems, are
  crucial for this \cite{ClassenL08,HaririCGDM13,ClassenLLZ14,HullSV13,BelardinelliLP14,DeGLP16,CDMP17}.


\item \textbf{Support component-based approaches.}  
% \textbf{Self-programming mechanisms favor component-based
%     systems.}  
  By no means should we consider programmable mechanisms to be
  composed of a single unit.  Indeed, understandability calls for
  building \textbf{high-level components} that are relatively simple
  to understand, verify and combine. Then, it is crucial to study how
  \textbf{composing} ``correct'' components leads to an overall
  ``correct'' behavior. 
%%
\project will leverage  work on composition and customization in Service Oriented Computing and more recently in \textbf{Reasoning about Action} in KR and \textbf{Generalized Planning} in AI
 \cite{wsf2014,SohrabiPM09,BertoliPT10,DePS13,DeGGPSS16}, %NissimB14,DeGGPSS16}.
%%
also looking at execution \textbf{monitoring} to detect 
failure, identify responsible components, and synthesize recovery \cite{DeGMGMM14,MarrellaMS17}.



\item \textbf{Allow learning and stochastic decisions, while remaining within safe bounds.}
% \textbf{Self-programming mechanisms need to keeping satisfying
%     their specifications, while they learn and make stochastic
%     decisions}.  
We want to allow mechanisms to have forms of decision making that
resist formal analysis (at least in human terms), because we want to
make use of the possibilities that advancements brought about by deep learning,
MDP's, and reinforcement learning. However, while the
actual execution could be chosen stochastically, we do want to have
safety guarantees on \textbf{all possible generated executions}
\cite{AmodeiOSCSM16,TYLC16,Hadfield-Menell16a}.  In this way, it is
the \textbf{entire space of solutions} that has \textbf{formal
  guarantees}, and the specific solution chosen by the learning
algorithm or the stochastic decision maker will also satisfy them. We
will study this objective in the context of \textbf{Planning
under stochastic uncertainty} \cite{GeffnerBo13,GNT2016}.

\end{enumerate}


\begin{framed}\it
Recent  foundational results by the PI chart a novel path that within \project will revolutionize \textbf{Reasoning about Action}  in KR and \textbf{Generalized Planning} in AI by introducing rich objectives, data, and componentization in order to produce a \textbf{breakthrough in engineering self-programming mechanisms that are human-comprehensible and safe by design.}
\end{framed}

\vspace{-3ex}
\subsection*{METHODOLOGY}
\vspace{-3ex}

The scientific work in \project will be methodologically structured into 3 broad research streams:
\begin{itemize}
\item \textbf{Principles and Foundations} that will deal with the
scientific foundations of white-box self-programming mechanisms. 

\item \textbf{Algorithms and Tools}  that will deal with the
development of practical algorithms, optimizations and tools for realizing
%white-box self-programming 
such mechanisms. 

\item \textbf{Applications and Evaluation}  that will evaluate white-box self programming mechanisms in the three business critical driving applications  mentioned above.
\end{itemize}
The \textbf{Principles and Foundations} and \textbf{Algorithms and Tools} streams
will cut across 5 workpackages (WPs) roughly corresponding to the 5
objectives above.  The \textbf{Applications and Evaluation} stream
will be further refined into 3 separate WPs for the 3 driving
applications. WPs are described in Part B2.
%%

Below we sketch \project's scientific approach, focusing on \textbf{novelty} and \textbf{feasibility}. 






%The main elements of the  approach are detailed below.
%%
\project will take from \textbf{Planning} in AI \cite{GeffnerBo13} (which in turn
coming from \textbf{Reasoning about Action} in KR \cite{Reiter01}) the idea of having
\textbf{human-comprehensible} descriptions of the domain (the mechanism and its
capabilities) and of the goals (the task specifications). 
%%
However
instead of considering simple domain specifications, as e.g., in PDDL, \project
will look at extensions that use \textbf{rich semantic descriptions} from
Knowledge Representation and \textbf{componentization} from behavior
compositions in Reasoning about Action,
%%
to which the PI has contributed significantly in the
years
\cite{DeGiacomoRS98,DeGiacomoLL00,DeGiacomoLS01,SardinaGLL04,SardinaG09,DeGiacomoLP10,DeGiacomoLM12,DeGiacomoLPV14,DeGiacomoLPV16,BanihashemiGL17}. 
% From
% such work, \project will draw key ideas on how to represent
% mechanisms, the domain in which they are operating, and the properties
% of interest in a high-level human-comprehensible fashion. 
%%
Particular attention will be given to computational effectiveness, in
line with some recent exploratory work by the PI
\cite{DeGLP16,DeGiacomoLPS16,CDMP17}.  
%%
%%
Notice that on the other hand \project does not aim at defining new
concrete representation languages. Instead it intends to use
well-known formalisms such as BPMN, UML, OWL, etc. as concrete
languages, though with a precise formal semantics to allow for
automated reasoning, verification and synthesis, see e.g.,
\cite{BerardiCG05,DeGiacomoOET17}.
%%
% Clearly specifications and replanned behaviors must be formally represented so as to be
%   \textbf{verifiable}, e.g., by model checking,
%   possibly modulo theories \cite{EiterGS10}.
%   This is a crucial step towards the understandability required by a
%   \textbf{white-box approach}.  In this way mechanisms can be checked, e.g.,  to \textbf{understand} if important safety conditions are
%   satisfied. In tackling this aspect, \project will also leverage on
%   recent advances of model checking of autonomous
%   agents \cite{Wooldridge09,LomuscioQR17}.

Another crucial difference wrt Planning is that, instead of
considering simple planning tasks, \project will make use of the
mathematically elegant theory of Reactive Synthesis \cite{PnRo89}
developed in \textbf{Verification and Synthesis} in the last 30 years
\cite{EhlersLTV17}, which however has not found widespread practical
application because of the \textbf{intrinsic difficulties} of certain
algorithms and constructions \cite{TsaiFVT14,DFogartyKVW13}.
%
We aim at \textbf{sidestepping these difficulties all-together}, by
focusing on non-traditional kinds of specification formalisms proposed
recently in, e.g., Reasoning about Action and Generalized Planning,
such as LTL and LDL on finite traces, recently proposed by the PI
together with Moshe Vardi (Rice U, Huston) \cite{DeVa13,DeVa15,DeVa16}
and adopted in generalized forms of Planning in AI
\cite{TorresB15,CamachoTMBM17} and in declarative business processes
in BPM \cite{AalstPS09,DeGMGMM14,DeGMMP17}, as well as safe/co-safe
LTL/LDL formulas, which have been shown to be more expressive than
expected while remaning \textbf{well-behaved}
\cite{FinkbeinerS13,FiliotJR11,Lacerda0H15,FaymonvilleFRT17}. Solvers
for these are substantially simpler that for general reactive
synthesis, being based on reachability and safety games, which are
amenable to efficient implementations.

% will adopt
% rich temporally extended objectives such as those adopted in
% mathematically elegant theory of \textbf{Reactive Synthesis}
% \cite{PnRo89} developed in Verification and Synthesis in the last 30
% years \cite{EhlersLTV17}, which however has not found diffused
% practical application because of the \textbf{intrinsic difficulties}
% of certain algorithms and constructions
% \cite{TsaiFVT14,DFogartyKVW13}.  We aim at \textbf{sidestepping these
%   difficulties all-together}, by focusing on non-traditional kinds of
% specification formalisms proposed recently in, e.g., Reasoning about
% Action, that reduces to reachability and safety games which are
% amenable to efficient implementations.
% %%




Moreover, like in Planning, but differently form Reactive Synthesis,
we expect mechanisms to be able to handle quickly and efficiently most
cases, i.e., those cases that do not require to solve difficult,
``puzzle-like'', situations. Indeed while the Planning community has
concentrated on simpler forms of process synthesis (reachability of a
state of affairs), it has developed a sort of \textbf{science of search
  algorithms for Planning}, which has brought about exceptional
scalability improvements (orders of magnitude) in the last decade
\cite{PommereningHB17,SteinmetzH17,LipovetzkyG17,DeGMMP17}. \project
intends to devise new algorithms for solving reachability and safety
games that are based on heuristic search as adopted in Planning, but
also considering symbolic techniques adopted in synthesis by model
checking \cite{BloemJPPS12}.
%%
The PI has been pioneering cross-fertilization of Planning and
Synthesis since the end of the '90
\cite{DeGiacomoV99,CalvaneseGV02,SardinaGLL06,DeGiacomoFPS10,PatriziLGG11,DeGMMP17}. More
recently the PI has established \textbf{tight connections between
  synthesis and generalized forms Planning}
\cite{HuG11,HuG13,DeGiacomoMRS16,BDGR17} as well as between Planning
and Behavior Composition
\cite{SardinaG08,DePS13,DeGGPSS16,CalvaneseGLV16}.  

Crucially, differently from current work in Planning and in
Verification and Synthesis, which operate with a propositional
representation of the state, \project does not want to discard data, and hence
it will need to consider a first-order or relational representation of
the state. This gives rise to infinite transition systems which are generally problematic to analyze.
%%
However the PI
has already shown within the EU FP7-ICT-257593 ACSI: Artifact-Centric
Service Interoperation, that such difficulties can be overcome in
notable cases
\cite{BerardiCGHM05,CalvaneseGHS09,HaririCGDM13,CalvaneseGMP13} in the context of \textbf{Data-Aware Processes} in Databases. A key
decovery is that under natural assumptions these infinite-state
transitions systems admit faithful \textbf{abstractions} to
finite-state ones, hence enabling the possibility of using the large
body of techniques developed within  Verification and Synthesis in
Formal Methods. Moreover important advancements in undestanding how to
deal with such complexity have been established as well as
relationships with Reasoning about Action in KR
\cite{HaririCMGMF13,BelardinelliLP14,CalvaneseGS15,CDMP17,BanihashemiGL17}.
We will leverage on these ideas to lift our results so
as to handle data.


% however recent results on \textbf{Data-aware Process} in databases. A
% key decovery is that under natural assumptions these infinite-state
% transitions systems admit faithful \textbf{abstractions} in
% finite-state ones, hence enabling the possibility of using the large
% body of techniques developed with the Verification as Synthesis in
% Formal Methods. We will leverage on these ideas to lift our results so
% as to handle data.

%From the above it can be seen that:
%The project aims at impacting the research agenda of four communities:
\begin{framed}{\it
 \project will, in addition, be the spark that will bring together and cross-fertilize four distinct research areas, namely \textbf{Reasoning about Action} in \emph{Knowledge Representation}, 
\textbf{Data-aware Processes} in \emph{Databases}, \textbf{Verification and Synthesis} in \emph{Formal Methods},  and \textbf{Generalized Planning} in \emph{AI}.}
\end{framed}

%\begin{framed}{\it
\emph{The PI has profoundly contributed to all these areas and is in a unique position to lead this cross-fertilization.}
%}
%\end{framed}

To foster such a cross-fertilization \project  will establish research exchanges with several research groups in the above mentioned areas.
 \begin{itemize}
\item \textbf{Reasoning about Action.}
%  \project will leverage on the large
% body of work on actions theoreis developed in Knowledge
% Representation, to which the PI has contributed significantly in the
% years
% \cite{DeGiacomoRS98,DeGiacomoLL00,DeGiacomoLS01,SardinaGLL04,SardinaG09,DeGiacomoLP10,DeGiacomoLM12,DeGiacomoLPV14,DeGiacomoLPV16,BanihashemiGL17}. From
% such work, \project will draw key ideas on how to represent
% mechanisms, the domain in which they are operating, and the properties
% of interest in a high-level human-comprehensible fashion. 
% %%
% Particular attention will be given to computational effectiveness, in
% line with some recent exploratory work by the PI
% \cite{DeGLP16,DeGiacomoLPS16,CDMP17}.  
% %%
% %%
% Notice that on the other hand \project does not aim at defining new
% concrete representation languages. Instead it intends to use
% well-known formalisms such as BPMN, UML, OWL, etc. as concrete
% languages, though with a precise formal semantics to allow for
% automated reasoning, verification and synthesis, see e.g.,
% \cite{BerardiCG05,DeGiacomoOET17}.
% %%
% Clearly specifications and synthesized behaviors must be formally represented so as to be
%   \textbf{verifiable}, e.g., by model checking,
%   possibly modulo theories \cite{EiterGS10}.
%   This is a crucial step towards the understandability required by a
%   \textbf{white-box approach}.  In this way mechanisms can be checked, e.g.,  to \textbf{understand} if important safety conditions are
%   satisfied. In tackling this aspect, \project will also leverage on
%   recent advances of model checking of autonomous
%   agents \cite{Wooldridge09,LomuscioQR17}.
% %%
%On this theme, the PI will collaborate
Collaborations with Yves Lesperance (York U., Toronto, Canada), Hector
Levesque (U. Toronto, Canada), Sebastian Sardina (RMIT, Melbourne,
Australia), and Yongmei Liu (Sun Yat-sen U., Guangzhou, China).

\item \textbf{Data-Aware Processes.}  
% \project will consider mechanisms that
% deal with data. It is known that verification and even more synthesis
% of \textbf{data-aware processes} are in general problematic, since
% processes generate infinite-state transition systems. However the PI
% has already shown, within the EU FP7-ICT-257593 ACSI: Artifact-Centric
% Service Interoperation, that such difficulties can be overcome in
% notable cases
% \cite{BerardiCGHM05,CalvaneseGHS09,HaririCGDM13,CalvaneseGMP13}. A key
% decovery is that under natural assumptions these infinite-state
% transitions systems admit faithful \textbf{abstractions} in
% finite-state ones, hence enabling the possibility of using the large
% body of techniques developed with the Verification as Synthesis in
% Formal Methods. Moreover important advancements in undestanding how to
% deal with such complexity have been established as well as
% relationships with action theories
% \cite{HaririCMGMF13,BelardinelliLP14,CalvaneseGS15,CDMP17,BanihashemiGL17}.
%On this theme, the PI will collaborate 
Collaborations with Rick Hull
(IBM Research, USA), Jianwen Su (UCSB, USA), Diego Calvanese and Marco
Montali (U.\ Bolzano, Italy), and with Alessio Lomuscio (Imperial
College, London, UK).

\item \textbf{Verification and Synthesis.}
% \project  will make use of the mathematically elegant theory of \textbf{Reactive
%   Synthesis} \cite{PnRo89} developed in Verification and Synthesis in the last 30 years \cite{EhlersLTV17}, which
% however has not found diffused practical application because of the
% \textbf{intrinsic difficulties} of certain algorithms and constructions \cite{TsaiFVT14,DFogartyKVW13}.
% %
% We aim at \textbf{sidestepping these difficulties all-together}, by
% focusing on non-traditional kinds of specification
% formalisms. Examples of these are LTL and
% LDL on finite traces, recently proposed by the PI together with Moshe Vardi (Rice U, Huston)
% \cite{DeVa13,DeVa15,DeVa16} and adopted in advanced forms of Planning in AI \cite{TorresB15,CamachoTMBM17} and in declarative business processes in BPM \cite{AalstPS09,DeGMGMM14,DeGMMP17}, as well as safe/co-safe LTL/LDL formulas, which have been shown to be more expressive than expected while \textbf{well-behaved} \cite{FinkbeinerS13,FiliotJR11,Lacerda0H15,FaymonvilleFRT17}.
%%
%On this theme the PI  will collaborate 
Collaborations with Moshe Vardi (Rice U.)
on automata-based verification and synthesis, Sasha Rubin (U.\ Napoli,
Italy), and Benjamin Aminof (TU Wien, Austria) on game-based
verification and synthesis, and with Nello Murano (U.\ Napoli, Italy)
on devising parallel algorithms to run on multicore GPUs \cite{ArcucciMMS17,ArcucciMPS17}.
%%
%%

\item \textbf{Generalized Planning.}
%  While \project will consider symbolic techniques
% adopted in synthesis by model checking \cite{BloemJPPS12}, it
% aims at leveraging on the exceptional scalability improvements of
% current algorithms in AI Planning, to devise radically
% different techniques to effectively tackle reactive synthesis in
% practice \cite{GeffnerBo13}.  
% %%
% In particular like agents in Planning, we expect mechanisms to be able
% to handle quickly and efficiently most cases, i.e., those cases that
% do not require to solve difficult, ``puzzle-like'', situations. Indeed
% while the Planning community has concentrated on simpler forms of
% process synthesis, it has developed a sort of \textbf{science of
%   search algorithms for Planning}, which has brought about
% improvements by orders of magnitude in the last decade
% \cite{PommereningHB17,SteinmetzH17,LipovetzkyG17,DeGMMP17}. \project
% will exploit this knowledge and extend it to generalized forms of
% Planning and to reactive Synthesis.
%%
% The PI has been pioneering cross-fertilization of Planning and
% Synthesis since the end of the '90
% \cite{DeGiacomoV99,CalvaneseGV02,SardinaGLL06,DeGiacomoFPS10,PatriziLGG11,DeGMMP17}. More
% recently the PI has established \textbf{tight connections between
%   synthesis and generalized forms Planning}
% \cite{HuG11,HuG13,DeGiacomoMRS16,BDGR17} as well as between Planning
% and Behavior Compositions
% \cite{SardinaG08,DePS13,DeGGPSS16,CalvaneseGLV16}.  
%On this theme, the PI will collaborate 
Collaborations with Hector Geffner (UPF, Barcelona), Blai
Bonet (U. Simon Bolivar, Caracas) and Alfonso Gerevini (U. Brescia,
Italy), and Malte Helmert (U.\ Basil).
%%


\end{itemize}

Moreover the PI will collaborate 
with Ronen Brafman (Ben-Gurion U.) to
explore how to \textbf{incorporate ML-components} into mechanisms, and study
Planning and Synthesis for temporally-extended goals in \textbf{non-Markovian} MDPs
and reinforcement learning (first ideas in
\cite{BeckL12,Lacerda0H15,BDMS17}).


% \textbf{WP4: Supporting component-based mechanisms.}
% % \paragraph{Composition and Customization.}
% % Other examples are the form of behavior specification adopted to
% % specify target collective behavior when composing conversational
% % services or component behaviors, adopted in SOC or in \textbf{behavior
% %   composition} in AI \cite{DePS13,DeGGPSS16}.
% %%
%   By no means should we consider programmable mechanisms to be
%   composed of a single unit. \textbf{Service-Oriented Computing} and
%   \textbf{Open APIs} frameworks have long been pushing for
%   \textbf{component-based systems}, in which a set of components are
%   \textbf{customized} and \textbf{orchestrated} to deliver a required
%   service \cite{wsf2014}.  Indeed, understandability calls for
%   building \textbf{high-level components} that are relatively simple
%   to understand, verify and combine. Then, it is crucial to study how
%   \textbf{composing} ``correct'' components leads to an overall
%   ``correct'' behavior.
% %%
% \project will leverage on the body of work on composition and customization developed in SOC and more recently in AI
%  \cite{SohrabiPM09,BertoliPT10,DePS13,DeGGPSS16}.%NissimB14,DeGGPSS16}.
% %%
% Moreover, execution should be \textbf{monitored} so that in case of
% failure it is possible to identify the responsible component, and
% recover the situation by reprogramming the mechanism for alternative
% solutions that circumvent the failing
% component \cite{DeGMGMM14,MarrellaMS17}.



% \textbf{WP5: Supporting learning and stochastic components.}
% We want to allow mechanisms to have forms of decision making that
% resist formal analysis (at least in human terms), because we want to
% make use of the possibilities that advancements in deep learning,
% MDPs, and reinforcement learning bring about. Though, while the actual
% execution could be chosen stochastically, we do want to have
% guarantees on \textbf{all possible generated executions}.  In this
% way, it is the \textbf{entire space of solutions} that has
% \textbf{formal guarantees}, and the specific solution chosen by the
% learning algorithm or the stochastic decision maker will also satisfy
% them. %Hence, however chosen, the actual solution will satisfy the desired guarantees.
% %%
%   This calls for allowing \textbf{coexistence of logical constraints
%     with stochastic solutions}, a theme that has only be scratched by
%   the scientific community so far \cite{BeckL12,SprauelKT14,BDMS17}.
% %%
%    We also observe that synthesis against constraints has been used to bound the
%   possible solutions in several context, most notably in supervisory
%   control \cite{Wo14,BanihashemiGL16}.
% %%
% In this theme the PI will collaborate  with
% Ronen Brafman (Ben-Gurion U.) to explore
% temporally-extended goal planing and synthesis in non-Markovian MDPs
% and reinforcement learning (first ideas in \cite{BDMS17}).



\textbf{Application and Evaluation}
\project will ground its scientific results in three
diverse real \textbf{application} to demonstrate the actual utilization of the scientific
achievements within the project: Smart manufacturing (Industry 4.0),
 Smart spaces (IoT), and
Business Processes Management Systems (BPM).
%%
The PI and his group at Sapienza have
contributed to all these fields, see e.g.,
\cite{DeGiacomoCFHM12,DeGiacomoDMM15,SilvaFCLSR17}.  Moreover the PI
has applied advanced science to real-cases in the
area Semantic Data Integration where he and his group have invented the Ontology-Based Data Access paradigm, possibly the
most successful approach for Semantic Data Integration \cite{PoggiLCGLR08,SequedaM17,Statoil17}.
%he contributed to W3C recommendation of OWL 2 Web Ontology Language
%Profiles (\url{https://www.w3.org/TR/owl2-profiles/}); 
Such an approach has matured to the point that  the PI
founded a Sapienza Start-Up \textbf{OBDA Systems}
(\url{http://www.obdasystems.com}) to commercially exploit it in real data integration scenarios. 


%DeGiacomoDMM15 DeGiacomoCFHM12


% textbf{Project structure.}
% The scientific work in \project will be divided into 3 research streams of the duration of the  project.
% \begin{itemize}
% \item \textbf{Stream 1: Foundations.} This stream will deal with the
% scientific foundations of white-box self-programming mechanisms. 

% \item \textbf{Stream 2: Algorithms and Tools.}  This stream  will deal with the
% development of practical algorithms, optimizations and tools for realizing
% white-box self-programming mechanisms. 

% \item \textbf{Stream 3: Applications and Evaluation.}  This stream  will evaluate white-box self programming mechanisms in the three business critical application contexts mentioned above.
% \end{itemize}


\vspace{-3ex}
\subsection*{HIGH RISK, HIGH GAIN}

\vspace{-3ex}

\project boldly aims at
\textbf{bringing together and cross-fertilizing the above cited four distinct research
  areas} 
%, Reasoning about Action, Data-aware Processes, Verification and Synthesis and Generalized AI Planning, 
with overlapping interests but developed by different communities with different view-points, to
produce a \textbf{breakthrough} that will make white-box self programming mechanisms a reality. 
% in engineering self-programming
%   mechanisms that are human-comprehensible and safe by design
%%
The \textbf{high risk} of this enterprise is \textbf{mitigated} by the PI expertise, who (by cross-fertilize some of these areas) has already obtained initial foundational results, such as  effective techniques for verification of data-aware processes in spite of them being
infinite state in nature due to data \cite{HaririCGDM13,DeGLP16,CDMP17}, and feasibility results for synthesis, which sidestep some intrinsic
difficulties of reactive synthesis algorithms and constructions (e.g.,
determinization) \cite{DeVa13,DeVa15,DeVa16}.
%%
\project will act as a \textbf{catalyst} for these areas and, together
with current ML advancements, to bring about a \textbf{novel AI framework for self-programmability}
that puts \textbf{human-comprehensibility at the center of the stage}.



% \begin{quote}{\it
% The PI is one of the most prominent AI scientist leading this cross-fertilization.}
% \end{quote}

% We further stress that the need to move towards \textbf{white-box} approaches is advocated by a large part of the \textbf{AI community} \cite{RussellDT15}, and has been recently taken up by DARPA within the context of machine learning, through the DARPA-BAA-16-53 ``Explainable Artificial Intelligence (XAI)'' program\footnote{\url{http://www.darpa.mil/program/explainable-artificial-intelligence}}

% \begin{quote}{\it
% Knowledge representation, the primary field of the PI, will be central for realizing the shift towards a white-box approach.}
% \end{quote}

% These two aspects make the \project very timely and of greatest significance for European science.



% \subsection*{PROJECT STRUCTURE}

% %\vspace{-3ex}

% The scientific work in \project will be divided into 3 research streams of the duration of the  project.

% \textbf{Stream 1: Foundations.} This stream will deal with the
% scientific foundations of white-box self-programming mechanisms. The
% research work will span from Reasoning about Actions, Planning, and
% reactive synthesis to non-Markovian MDPs and reinforcement
% learning. But it will also concern work on semantic technologies for
% representing mechanisms states in first-order terms.

% \textbf{Stream 2: Algorithms and Tools.}  This stream  will deal with the
% development of practical algorithms, optimizations and tools for realizing
% white-box self-programming mechanisms. It will concern work in synthesis, including
% synthesis by model checking, and especially work in Planning, which is
% delving deeply in the algorithmic aspects of (simplified forms of)
% synthesis.

% \textbf{Stream 3: Applications and Evaluation.}  This stream will
% ground its scientific results in diverse real \textbf{application
%   contexts}, including manufacturing systems
% (Industry 4.0),  smart spaces (IoT), and business processes (BPM) to demonstrate the
% actual utilization of the scientific achievements within the project.


% \textbf{Collaborations.}
% The scientific work for will take advantage of  collaboration with several top research groups working in various aspects related to \project, including
% %Specifically:
% %\begin{inparaenum}[\it (i)]
% % \begin{itemize}
% % \item 
% Moshe Vardi (Rice U.) for Reactive Synthesis;
% %
% %\item
% Nello Murano (U.\ Napoli), Sasha Rubin (U.\ Napoli), Benjamin Aminof (TU Wien), for Game-Theoretic Verification; 
% %
% %\item 
% Hector Geffner (UPF, Barcelona)
%  and Malte Helmert (U.\ Basil) for Planning;
% %
% %\item 
% Ronen Brafman (Ben-Gurion U.) for MDPs and Reinforcement Learning;
% %
% %\item 
% Alessio Lomuscio (IC London) for Agents;
% %
% %\item 
% Diego Calvanese and Marco Montali (U.\ Bolzano) for Data-Aware Processes;
% %
% %\item 
% Yves Lesperance (York U.) and Hector Levesque (U. Toronto) for Reasoning about Actions.
% %
% %\item 
% Paolo Felli (Nottingham U.) for Manufacturing;
% %
% %\item 
% Maurizio Lenzerini and other colleagues at Sapienza for Ontologies and Semantic Technologies;
% %
% %\item 
% Massimo Mecella and other colleagues at Sapienza for applications in Manufacturing, IoT and Business Processes.
% %\end{inparaenum}
% %\end{itemize}
% % The collaboration will allow for reciprocal visits for short and long
% % periods to write papers and develop tools and application together, as
% % well as to allow for organizing joint scientific workshops on the themes
% % of the project.

% %%
% %%
% % \begin{tabular}{p{2cm}p{14cm}}
% % %& Benjamin Aminof, Technische Universität Wien, Vienna, Austria\\
% % %& Blai Bonet, Universidad Simón Bolívar, Caracas, Venezuela\\
% % & Ronen Brafman, Ben-Gurion University, Beer-Sheva, Israel \\
% % & Diego Calvanese, Free University of Bozen-Bolzano, Italy\\
% % %& Alin Deutsch, University of California, San Diego, CA, USA\\
% % & Rick Hull, IBM Research, Yorktown Heights, NY, USA\\
% % & Hector Geffner, Universitat Pompeu Fabra, Barcelona, Spain\\
% % %& Alfonso Emilio Gerevini, Universit\`a di Brescia, Italy\\
% % & Alessio Lomuscio, Imperial College London, UK\\
% % & Maurizio Lenzerini, Sapienza Universit\`a di Roma, Italy\\
% % & Yves Lesperance, York University, Toronto, ON, Canada\\
% % & Hector Levesque, University of Toronto, Toronto, ON, Canada\\
% % & Yongmei Liu, University in Guangzhou, China\\
% % %& Aniello Murano, Universit\`a di Napoli, Italy\\
% % & Adrian Pearce, University of Melbourne, Melbourne, SW, Australia\\
% % & Ray Reiter, University of Toronto, Toronto, ON, Canada\\
% % %& Riccardo Rosati, Sapienza Universit\`a di Roma, Italy\\
% % %& Sasha Rubin, Universit\`a di Napoli, Italy\\
% % & Sebastian Sardina, RMIT, Melbourne, SW, Australia\\
% % %& Jianwen Su, University of California, Santa Barbara, CA, USA\\
% % & Moshe Vardi, Rice University, Huston, TX, USA, Topic
% % \end{tabular}


% % Yves' comments:

% % issues:
% % - Feasibility?
% % - Human vs Machine Understandable?
% % - Detail relation to machine learning approaches
% %    - Safe regions vs goals/objective vs optimization criteria
% %    - Dynamism

%%% Local Variables:
%%% mode: latex
%%% TeX-master: "PartB1"
%%% TeX-PDF-mode: t
%%% End:
