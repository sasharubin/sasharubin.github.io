% This is the abstract.  Max 2000 characters.\\

% \note{Proposal summary (identical to the abstract from the online proposal submission forms, section 1). 

% The abstract (summary) should, at a glance, provide the reader with a clear understanding of the objectives of the research proposal and how they will be achieved. The abstract will be used as the short description of your research proposal in the evaluation process and in communications to contact in particular the potential remote referees and/or inform the Commission and/or the programme management committees and/or relevant national funding agencies (provided you give permission to do so where requested in the online proposal submission forms, section 1). It must therefore be short and precise and should not contain confidential information. 

% Please use plain typed text, avoiding formulae and other special characters. The abstract must be written in English. There is a limit of 2000 characters (spaces and line breaks included).}

We are witnessing an increasing availability of \textbf{mechanisms} that offer
some form of programmability.
%%
These include
software, % (CS), 
manufacturing devices, % (Industry 4.0), 
smart objects and smart spaces, % (IoT), 
intelligent robots, % (Robotics),
%autonomous agents (MAS)
business process management systems, % (BPM),
%component-based systems, % (SOC and open APIs), 
and many others.
%%
All these mechanisms are being currently revolutionized by means of
advances in Machine Learning (ML).  In particular central sensing
capabilities (vision, language undestanding) and actuation
capabilities (robot-arm movement, camera twisting) are envisioned to be handled
by next generation ML-components.
%%
However, irrespective of the technological nature of the components, the connections between them are still
organized, monitored, and coordinated through standard programming.

\project aims at developing the \textbf{science} and the \textbf{tools} for a new
generation of mechanisms to emerge: machanisms that are able to
\textbf{program themselves} without human intervention, and
automatically tailor their behavior so as to
%%
achieve desired goals,  maintain themselves within safe boundaries in a
changing environment, and keep following rules,
regulations and conventions that evolve over time. 
%%

%Since with ``\emph{with great power comes great responsibility}'',
Unlike ML approaches which are typically black-box, 
\project intends to
exploite \textbf{Knowledge Representation} (KR) for realizing self-programming mechanisms that are
%to organize, monitor and coordinate ML-components in a \textbf{white-box} in a
\textbf{white-box}: specifications and automatically synthesized
programs must be human comprehensible. In
other words, possibly on top of suitably characterized black-box 
ML-components, the behavior enacted by \project self-programming
mechanisms intends to be fully \textbf{explainable in human terms by design}.


Scientifically, \project aims at \textbf{repurposing KR} so as to bring about, together with current ML advancements, a
\textbf{new AI framework} that merges key ideas
%1
from \textbf{traditional KR}  on how to represent the domain of interest, the
system, and their properties in a high-level human
comprehensible fashion,
with ideas from \textbf{Data-aware Processes} in \emph{Databases} on how to build data-aware dynamic behaviors,
%2
and from \textbf{Verification and Synthesis} in \emph{Formal Methods}
which provide mathematically elegant foundations for synthesis.  For
 effectiveness \project will focus on synthesis agains
computationally well-behaved temporal specification formalisms
recently proposed in KR
%3
and will exploit recent advancements in \textbf{Automated Planning} in AI to gain
algorithmic insights to the synthesis process.
% from \textbf{Service Composition} in Service-Oriented Computing and
% Open APIs and more recently in AI on how to compose, customize and
% orchestrate modules and behaviors;
%4

\project grounds its scientific results upon diverse real
\textbf{application contexts}, including manufacturing systems
(Industry 4.0), smart spaces (IoT) and business process management
(BPM).%, and components-based systems (SOC and Open APIs).

 


%%% Local Variables:
%%% mode: latex
%%% TeX-master: "PartB1"
%%% TeX-PDF-mode: t
%%% End:
