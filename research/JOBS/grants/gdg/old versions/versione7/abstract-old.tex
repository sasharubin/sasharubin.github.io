% This is the abstract.  Max 2000 characters.\\

% \note{Proposal summary (identical to the abstract from the online proposal submission forms, section 1). 

% The abstract (summary) should, at a glance, provide the reader with a clear understanding of the objectives of the research proposal and how they will be achieved. The abstract will be used as the short description of your research proposal in the evaluation process and in communications to contact in particular the potential remote referees and/or inform the Commission and/or the programme management committees and/or relevant national funding agencies (provided you give permission to do so where requested in the online proposal submission forms, section 1). It must therefore be short and precise and should not contain confidential information. 

% Please use plain typed text, avoiding formulae and other special characters. The abstract must be written in English. There is a limit of 2000 characters (spaces and line breaks included).}

We are witnessing an increasing availability of \textbf{mechanisms} that offer
some form of programmability.
%%
These include
software, % (CS), 
manufacturing devices, % (Industry 4.0), 
smart objects and smart spaces, % (IoT), 
intelligent robots, % (Robotics),
%autonomous agents (MAS)
business process models, % (BPM),
component-based systems, % (SOC and open APIs), 
and many others.
%%
All these mechanisms have some built-in capabilities that give them a
dynamic behavior, which can be organized, refined and repurposed through
programming.

\project aims at allowing such mechanisms to \textbf{program
  themselves}, without human intervention.
%%
Through this self-programming ability such mechanisms can tailor their
behavior so as to
%%
achieve desired goals,  maintain themselves within a safe boundary in a
changing environment, and keep following rules,
regulations and conventions that evolve over time. 
%%

%Since with ``\emph{with great power comes great responsibility}'',
Though, unlike some machine learning approaches, \project aims at the self-programming
mechanisms that are \textbf{white-box}:
specifications and automatically synthesized programs must be human
comprehensible. In other words, \project aims at obtaining self-programming
mechanisms whose behavior is fully \textbf{explainable in human terms}
by design.

\project's scientific work will merge key ideas 
%1
from \textbf{Reasoning about Action} in \emph{Knowledge Representation} on how to represent the domain of interest, the
system, and their properties in a high-level human
comprehensible fashion;
from \textbf{Data-aware Processes} in \emph{Databases}, on how to build data-aware dynamic behaviors;
%2
from \textbf{Verification and Synthesis} in \emph{Formal Methods} on providing
mathematically elegant foundations for synthesis, though focusing on
computationally more tractable formalisms recently proposed in
Reasoning about Action;
%3
and from \textbf{Planning} in \textbf{Artificial Intelligence} to gain
algorithmic insight to the synthesis process.
% from \textbf{Service Composition} in Service-Oriented Computing and
% Open APIs and more recently in AI on how to compose, customize and
% orchestrate modules and behaviors;
%4

\project grounds its scientific results on diverse real
\textbf{application contexts}, including manufacturing systems
(Industry 4.0), smart spaces (IoT) and business processes
(BPM).%, and components-based systems (SOC and Open APIs).

 


%%% Local Variables:
%%% mode: latex
%%% TeX-master: "PartB1"
%%% TeX-PDF-mode: t
%%% End:
