\section{Ten-year track record}

% \note{2 pages\\
% The Principal Investigator must provide a list of achievements in the last 10 years. 

% The PI should list his/her activity as regards: 

% 1. Ten representative publications, as senior author (or in those fields where alphabetic order of authorship is the norm, joint author) in major international peer-reviewed multi-disciplinary scientific journals and/or in the leading international peer-reviewed journals and peer-reviewed conferences proceedings of their respective research fields, also indicating the number of citations (excluding self-citations) they have attracted (if applicable). 

% 2. Research monographs and any translations thereof (if applicable). 

% 3. Granted patents (if applicable). 

% 4. Invited presentations to peer-reviewed, internationally established conferences and/or international advanced schools (if applicable). 

% 5. Research expeditions that the applicant Principal Investigator has led (if applicable). 

% 6. Organisation of international conferences in the field of the applicant (membership in the steering and/or organising committee) (if applicable). 

% 7. International Prizes/ Awards/ Academy memberships (if applicable). 

% 8. Major contributions to the early careers of excellent researchers (if applicable) 

% 9. Examples of leadership in industrial innovation or design (if applicable). 
% }

Giuseppe De Giacomo's research activity has concerned theoretical, methodological and applicative aspects in different areas of CS and AI. Below the main accomplishments of the  last 10 years are listed. 

\subsection*{REPRESENTATIVE PUBLICATIONS}


\begin{enumerate}

% \item \textit{\textbf{TBox and ABox reasoning in expressive description logics.}
% G. De Giacomo, M. Lenzerini
% KR 96 (316-327), 1996} (184 citations) 

% % \textit{Boosting the correspondence between description logics and propositional dynamic logics
% % G De Giacomo, M Lenzerini
% % AAAI 94, 205-212, 1994,} (179 citations)


% % \textit{ A Uniform Framework for Concept Definitions in Description Logics.
% % G De Giacomo, M Lenzerini
% % J. Artif. Intell. Res.(JAIR) 6, 87-110, 1997} (67 citations 


% % \textbf{Expressive Description Logics.} 

% This paper is the best representative of a series of papers including AAAI'94 (179 citations), ECAI'94 (79 citations), JAIR (67 citations), IJCAI'95 (82 citations) DOOD'95 (96 citations), and other, on expressive description logics based on the correspondence with propositional dynamic logics and mu-calculus. This is essentially the work of his PhD thesis ``Decidability of class-based knowledge representation formalisms''  (127 citations). This research has been at the base of the revolution in the field happened at the end of ’90.  In particular, it shaped the theoretical landscape of such logics over which enterprises such as OWL (Ontology Web Language) could flourish in the 2000’s. Notably this series of papers also closed several open questions on computational aspects of introducing in modal logics unusual constructs such as “graded modalities” and “nominals”. 


\item
\textit{\textbf{Conjunctive query containment and answering under description logic constraints.} Diego Calvanese, Giuseppe De Giacomo, Maurizio Lenzerini ACM Trans. Comput. Log. 9(3): 22:1-22:31 (2008)} (87 citations)

% \item \textit{\textbf{On the decidability of query containment under constraints.}
% D. Calvanese, G. De Giacomo, M. Lenzerini
% PODS 1998} (419 citations)

%\textbf{Query Answering in Description Logics.} 

This paper studies the decidability of answering conjunctive queries over description logics knowledge bases. Together with its conference version PODS'98 (419 citations) it had a notable impact on the description logic community, making query answering in description logics one of the main research themes of the field. 

\item \textit{\textbf{Tractable reasoning and efficient query answering in description logics: The DL-Lite family.}. D. Calvanese, G. De Giacomo, D. Lembo, M. Lenzerini, R. Rosati.
Journal of Automated Reasoning 39 (3), 385-429, 2007} (1131 citations) \\[1ex]
\textit{\textbf{Data complexity of query answering in description logics.}. D. Calvanese, G. De Giacomo, D. Lembo, M. Lenzerini, R. Rosati.
Artif. Intell. 195: 335-360 (2013)} (400 citations, including KR'06)


% \textit{DL-Lite: Tractable description logics for ontologies
% D Calvanese, G De Giacomo, D Lembo, M Lenzerini, R Rosati
% AAAI 5, 602-607, 2005} (411 citations)

%\textbf{DL-lite and Ontology Based Data Access.} 

Together with the group in Rome, the PI introduced the lightweight
description logic DL-lite (AAAI'05 411 citations), which allow for
query conjunctive answering over ontologies t in $\mathit{AC}^0$ in
data complexity, as plain relational databases, while capturing most
aspects of UML class diagrams and Entity Relationship Diagrams (studied in AIJ'05 550 citations). This
study refocused the interest on data complexity in description logics,
and together wit the work by Baader et al. on $\cal{EL}$ has generated
a new wave of interest in description logics. DL-Lite Such directly
influenced the OWL 2 QL profile in the OWL 2 W3C standard. Initial
ideas were based on the PI work on partial evaluation for logic
programs done in his master thesis as an ERASMUS student under the
joint supervision of John Lloyd at University of Bristol and Maurizio
Lenzerini at Sapienza. The original . The second paper above has beeinvited in the journal track of IJCAI'15.

% \item \textit{\textbf{Reasoning on UML class diagrams.} D. Berardi, D. Calvanese,
%     G .De Giacomo. Artificial Intelligence 168 (1), 70-118, 2005.} (547
%     citations)

% %\textbf{Reasoning on UML Class Diagrams.} 
% He worked on the formalization in logic of a variety of class-based conceptual models. The most prominent result on this line of research is a sound, complete and computationally optimal technique for reasoning on UML class diagrams. Interestingly, expressive description logics could be used to show both EXPTIME-hardness and EXPTIME-membership of reasoning on UML class diagram. The AIJ 2005 paper (540 citations according to Google Scholar) on this research has been one of the most downloaded AIJ papers for 2005.

\item \textit{\textbf{Linking data to ontologies.}
A.\ Poggi, D.\ Lembo, D.\ Calvanese, G.\ De Giacomo, M.\ Lenzerini, R.\ Rosati.
Journal on data semantics X, 133-173, 2008} (546 citations)\\[1ex]
%%
\textit{\textbf{View-based query answering in Description Logics: Semantics and complexity.}
D.\ Calvanese, G.\ De Giacomo, M.\ Lenzerini, R.\ Rosati. J. Comput. Syst. Sci. 78(1): 26-46 (2012)} (24 citations)

%\textbf{Data Integration.} 
The PI internationally renowned for his contributions to data Integration, an area that is highly relevant, from both the scientific and the industrial point of view, and is often cited as one of the biggest challenges in IT. The PI contributions in the field are characterized by a combination of elements coming from database theory, data management, knowledge representation and AI. Of particular relevance are his contributions on the analysis of peer-to-peer systems in terms of logics of knowledge, and, most prominently, the introduction of  the Ontology-Based Data Access (OBDA) paradigm, which is considered one of the major advancement in this field.

\item \textit{\textbf{View-based query processing: On the relationship between rewriting, answering and losslessness.} D.\ Calvanese, G.\ De Giacomo, M.\ Lenzerini, M.\ Vardi.  Theor. Comput. Sci. 371(3): 169-182 (2007)} (40 citations)

% \item \textit{\textbf{On simplification of schema mappings.} D.\ Calvanese, G.\ De Giacomo, M.\ Lenzerini, M.\ Vardi.  J. Comput. Syst. Sci. 79(6): 816-834 (2013)} %(4 citations)

% \item \textit{\textbf{Rewriting of regular expressions and regular path queries.}
% D Calvanese, G De Giacomo, M Lenzerini, MY Vardi
% Proceedings of the eighteenth ACM SIGMOD-SIGACT-SIGART symposium on ... 1999} (198 citations)

% \textit{Containment of conjunctive regular path queries with inverse
% D Calvanese, G De Giacomo, M Lenzerini, MY Vardi
% KR, 176-185, 2000} (148 citations)


% \textit{Answering regular path queries using views
% D Calvanese, G De Giacomo, M Lenzerini, MY Vardi
% Data Engineering, 2000. Proceedings. 16th International Conference on, 389-398, 200} (143 citations)


% LICS'00 (92) SIGRECORD 2003 (75) PODS'00 (69), TCS'05 (55), ISDPL (33), losslessness ICDS (30), VLDB 2012 (16) ICDT'11 (14)

% On simplification of schema mappings
% D Calvanese, G De Giacomo, M Lenzerini, MY Vardi
% Journal of Computer and System Sciences 79 (6), 816-834


%\textbf{Graph Databases.} 
The PI  together with Diego Calvanese, Maurizio Lenzerini and Moshe Vardi developed a rich theory of regular path query processing over graph databases, which merges elements coming from automata theory with elements coming from Constraint Satisfaction PODS'99 (198 citations), LICS'00 (92), KR'00(148), ICDE'00(143), SIGRECORD 2003 (75) PODS'00 (69), TCS'05 (55), ISDPL (33), PODS'02 (35), ICDT'05 (30), DBLP'08 (33) VLDB 2012 (16), AAAI'10 (11) ICDT'11 (14), JCSS'13 (4). 
% Through such techniques problems such as query answering using views, query rewriting, view-based containment, losslessness, and perfectness of rewritings have been addressed for (several extension of) regular path queries. 
% It is worth noting that regular path queries in the context of graph databases play a basic role analogous to that of conjunctive queries for relational databases. Indeed they are the basic component of all query languages over graph databases including the ones for structured, semistructured and XML databases. 


 \item \textit{\textbf{Bounded situation calculus action theories.} G.\ De Giacomo, Y.\ Lesperance, F.\ Patrizi:
 Artif. Intell. 237: 172-203 (2016)}
% \item \textit{\textbf{ConGolog, a concurrent programming language based on the situation calculus.} G. De Giacomo, Y. Lesperance, H. Levesque. Artificial Intelligence 121 (1-2), 109-169, 2000} (652 citations)

% \textit{Reasoning about concurrent execution, prioritized interrupts, and exogenous actions in the situation calculus
% G De Giacomo, Y Lespérance, HJ Levesque
% IJCAI 97, 1221-1226, 1997} (138 citations)

% \textit{An incremental interpreter for high-level programs with sensing
% G De Giacomo, HJ Levesque
% Logical foundations for cognitive agents, 86-102, 1999}  (176)

% \textit{On the semantics of deliberation in IndiGolog—from theory to implementation
% S Sardina, G De Giacomo, Y Lespérance, HJ Levesque
% Annals of Mathematics and Artificial Intelligence 41 (2-4), 259-299, 2004} (74)

% \textit{Projection using regression and sensors
% G De Giacomo, HJ Levesque
% IJCAI 99, 160-165, 1999} (54 citatios)

% \textbf{ConGolog and Reasoning about Actions.} 

The PI has deeply contributed to Reasoning About Actions 
% since when he visited Yoav Shoham in Stanford for 8 months in 1993/94 during his PhD. Such interest flourished when he visited Hector Levesque and Ray Reiter at University of Toronto in 1996 and then again in 1997 and in 1999. 
He contributed to development of what can be called the Toronto School in Situation Calculus. In particular his work on the semantics of the concurrent version (ConGolog) of the Golog high-level logic programming language based on the Situation Calculus played a crucial role in later work in Situation Calculus and in languages for Cognitive Robotics (AIJ'00 652 citations). 
% It is also worth mentioning that he was the organizer of the first Cognitive Robotics Workshop held as a 1998 AAAI Fall Symposium and which then became the first of an established series. He is continuing to contribute to Situation Calculus and Reasoning about Actions, and
Recently together with Fabio Patrizi and Yves Lesperance (York University, Toronto) he devised situation-bounded action theories. These are standard basic action theories with the additional constraints that the size of the extension of fluents in every situation must be bounded, though such an extension changes from situation to situation.  Such action theories give rise to infinite transition systems that can be faithfully abstracted into finite ones, making verification decidable. The work on this issue includes KR’12, IJCAI’13, AAMAS’14, ECAI’14, AIJ’16, AAAI’16, KR’16, IJCAI’16, AIJ’16, AAAI’17.

\item \textit{\textbf{Linear temporal logic and linear dynamic logic on finite traces.}
G. De Giacomo, M. Vardi/
IJCAI'13 Proceedings of the Twenty-Third international joint conference on ..., 2013} (75 citations)
% \textit{Synthesis for LTL and LDL on Finite Traces.
% G De Giacomo, MY Vardi
% IJCAI, 1558-1564, 2015} (11 citations)
% \textit{Reasoning on LTL on Finite Traces: Insensitivity to Infiniteness.
% G De Giacomo, R De Masellis, M Montali
% AAAI, 1027-1033, 2014} (38 citations)

% \textit{Monitoring business metaconstraints based on LTL and LDL for finite traces
% G De Giacomo, R De Masellis, M Grasso, FM Maggi, M Montali
% International Conference on Business Process Management, 1-17} (18 citations)

% \textit{LTLf and LDLf Synthesis under Partial Observability
% G De Giacomo, MY Vardi
% Proceedings of the 25th International Joint Conference on Artificial ...2016} (1 citations)

% \textit{Automata-theoretic approach to planning for temporally extended goals
% G De Giacomo, MY Vardi
% European Conference on Planning, 226-238, 1999} (113 citations)

% \textit{Reasoning about actions and planning in LTL action theories
% D Calvanese, G De Giacomo, MY Vardi
% KR 2, 593-602 2002} (67 citations)

%\textbf{AI Planning.} 

He is well recognized for is work on nonstandard form of AI planning. In particular he has done pioneering work of plan synthesis from LTL specifications (ECP’99, KR’02), and recently on forms of generalized plan synthesis (IJCAI’11, ICAPS’13), i.e., synthesis of plans that work simultaneously in several distinct domains. Also he has studied planning for mu-calculus specifications via model checking of game structures (AAAI’10), and he has drawn connections between forms of planning starting from available components and service compositions (IJCAI’07, AAAI’07, KR’08), and lately between certain forms of planning and discrete event control synthesis (AAMAS’12, IJCAI’13). Finally together with Moshe Vardi and others he is studying system specification, verification and synthesis in LTL and its extension LDL on finite traces, IJCAI’13, AAAI’14, BPM’14, CAiSE’15, IJCAI’15, AIJ’16, IJCAI’16, AAAI’17.


\item \textit{\textbf{Agent planning programs.} G.\ De Giacomo, A.\ Gerevini, F.\ Patrizi, A.\, Saetti, S.\ Sardina. Artif. Intell. 231: 64-106 (2016)}\\[1ex]
\textit{\textbf{Automatic behavior composition synthesis}. G.\ De Giacomo,  F.\ Patrizi, S. Sardina. Artif. Intell. 196: 106-142 (2013)}
% \item \textit{\textbf{Automatic composition of e-services that export their behavior.}
% D. Berardi, D. Calvanese, G. De Giacomo, M. Lenzerini, M. Mecella.
% International Conference on Service-Oriented Computing, 43-58, 2003} (516 citations)
% + (227 IJCIS05)
% \textit{
% Automatic composition of transition-based semantic web services with messaging
% D Berardi, D Calvanese, G De Giacomo, R Hull, M Mecella
% Proceedings of the 31st international conference on Very large data bases ... 2005} (331 citations) 

% \item \textit{Automatic behavior composition synthesis
% G De Giacomo, F Patrizi, S Sardina
% Artificial Intelligence 196, 106-142, 2013} (46 citations)

%\textbf{Service composition.} 
Together with Massimo Mecella and others he has developed one of the most prominent techniques for composition of stateful services, which has been later named the “Roman Model” after the name used for it by Rick Hull in several tutorials. The main ideas behind the Roman Model is reusing and repurposing fragments of the computation performed by available services. Such an idea has attracted interest also in AI where it has laid the bases for a novel form of agent-behavior automated synthesis starting from available components (see connection with AI planning). The original paper on the Roman Model appeared at ICSOC’03 (the first of a now established series of conferences) and is one of the most cited papers on foundations of service compositions (with 505 citations according to Google Scholar). The most recent journal papers on the subject appeared on AIJ’13, and AIJ’16. 

\item \textit{\textbf{Verification of relational data-centric dynamic systems with external services.}
B.\ Bagheri Hariri, D.\ Calvanese, G.\ De Giacomo, A.\ Deutsch, M.\ Montali
Proceedings of the 32nd ACM SIGMOD-SIGACT-SIGAI symposium on Principles of .. 2013} (112 citations)\\[1ex]
%
\textit{\textbf{Description Logic Knowledge and Action Bases.}
B.\ Bagheri Hariri, D.\ Calvanese, M.\ Montali, G.\ De Giacomo, R.\ De Masellis, P.\ Felli:
J. Artif. Intell. Res. (JAIR) 46: 651-686 (2013)}

% \textit{Bounded Situation Calculus Action Theories and Decidable Verification.
% G De Giacomo, Y Lespérance, F Patrizi
% KR, 2012,} (47 citations)

% \textbf{Process Verification and Synthesis in Presence of Data.} 

He has recently focused on studying process verification and synthesis from temporal specification in presence of data. The presence of data, which may grow unboundedly, makes such processes infinite state. However most previous research infinite state systems does not apply, since it is mostly concern with recursive control rather than data, which are either ignored or finitely abstracted. In this area he has proposed novel techniques that guarantees the possibility under certain assumptions of faithful abstracting such infinite state systems into a finite one, including a quite promising one based on a connection with the theory of conjunctive queries and data exchange. Such a research is having a practical impact on the formal analysis, verification and synthesis in so called “artifact-based business processes”. This work has been published in several papers including ICSOC’10, BPM’11, KR’12, ECAI’12, JAIR’13, PODS’13, RR’13, AAMAS’14, IJCAI’15, KR’16.
\end{enumerate}

\subsection*{IMPACT MEASURES}
He is the author of more than 200 publications in international journals, conference and workshop proceedings. Many of these papers are widely cited in the scientific literature. According to Google Scholar, April 2017, his h-index is 67 and his i10-index is 187. These values are among the highest in Europe in CS and AI. According to a study (based again Google Scholar) on the top CS scientists working in Italy available at http://via-academy.org/, Giuseppe De Giacomo ranked 3rd among the most cited researchers working in Italy. 


\subsection*{INDUSTRIAL IMPACT}

ConGolog

OWL2QL

OBDA Systems


\subsection*{INVITED TALKS}

\begin{itemize}
% \item	``AI Foundations for Data-Aware Business Processes'' Distinguished Talk at York University's Centre for Innovation in Computing at Lassonde (IC@L), Toronto, ON, Canada, April 2017. 
\item	``AI Foundations for Data-Aware Business Processes'' Distinguished Talk at University of Toronto, Toronto, ON, Canada, April 2017. 
\item	``First-Order mu-Calculus over Generic Transition Systems and Applications to the Situation Calculus'', 1st Workshop on Formal Methods in AI, University of Naples "Federico II", Naples, Italy, February 2017 
\item	``LTL and LDL on Finite Traces: Reasoning, Verification, and Synthesis'' invited talk at the 4th International Workshop on Strategic Reasoning (SR’16), New York, USA, July 2016.
\item	``Synthesis in Linear-time Dynamic Logic on Finite Traces'' keynote at Highlights of Logic, Games and Automata 2015 (Highlights 2015), Prague, Czech Republic, September 2015.
\item	``Temporal Reasoning in Bounded Situation Calculus'' keynote at 22nd International Symposium on Temporal Representation and Reasoning (TIME), Kassel, Germany, September 2015.
\item	``On Bounded Situation Calculus'' invited talk at HYBRIS Workshop, Potsdam, Germany June 2015.
\item	``Verification of Data-Aware Processes'', keynote at 11th International Workshop on Web Services and Formal Methods: Formal Aspects of Service-Oriented and Cloud Computing (WS-FM:FASOCC 2014), Eindhoven, September 2014.
\item	 ``Reasoning about data and knowledge-aware processes'', Invited talk for Frontiers of Artificial Intelligence at 21st European Conference on Artificial Intelligence (ECAI 2014), Prague, August 2014.
\item	 ``Automatic Composition of E-services That Export Their Behavior'', talk for the prize as the most influential ICSOC paper in the last 10 years, ICSOC 2013, Berlin, December 2013.
\item	 ``Cognitive Robotics: The science of building intelligent autonomous robots and software agents'', Miegunyah Fellow Public Lecture, Melbourne, Australia August 2013 
\item	``Actions, Processes, and Ontologies'' keynote at 26th International Workshop on Description Logics (DL 2013), Ulm, Germany, July 2013.
\item	 ``Linear Temporal Logics on Finite Traces: Reasoning, Verification, and Synthesis'' keynote at 23rd International Conference on Automated Planning and Scheduling (ICAPS’13), Rome, Italy, June 2013.
% \item	 ``Conjunctive Queries: Evaluation and Containment'' at University of Toronto, Canada, November 2010. 
% \item	``Automatic Service Composition and Synthesis: the Roman Model'' at York University, Toronto, Canada, October 2010;
% \item	``Towards Systems for Ontology-based Data Access and Integration using Relational Technology'' at University of Toronto, Canada, October 2010;
% \item	``Automatic Service Composition and Synthesis: the Roman Model'' at University of Toronto Canada, October 2010;
% \item	 ``QUONTO: ontology-based data access and integration using relational technology'', Semantic Days 2009 Conference Stavanger, Norway, May 2009;
% \item	  ``The Roman model for Service Composition'' at INFINT’09 Bertinoro Workshop on Data and Service Integration, Bertinoro, Italia, March 2009;
% \item	 ``Ontology Based Data Integration'' at IBM Research Center Watson, Hawthorne, NY, USA, August 2008;
% \item	 ``Process integration: Look at how you behave!'' at INFINT’07 Bertinoro Workshop on Information Integration, Bertinoro, Italia, October 2007;
% \item	 ``Automatic composition synthesis of web services: a conceptual perspective'' at MSI’05, Caen, France, 2005.
\end{itemize}

\subsection*{PUBLICATIONS}
A comprehensive list of publication can be found at
\url{http://dblp.uni-trier.de/pers/hd/g/Giacomo:Giuseppe_De.html} and
\url{https://scholar.google.it/citations?user=Sfo4K0oAAAAJ&hl=en}.



%%% Local Variables:
%%% mode: latex
%%% TeX-master: "PartB1"
%%% TeX-PDF-mode: t
%%% End:
