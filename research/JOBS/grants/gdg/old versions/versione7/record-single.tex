
\renewcommand{\baselinestretch}{1}
\section{TEN-YEAR TRACK RECORD}
\vspace{-1ex}

% \note{2 pages\\
% The Principal Investigator must provide a list of achievements in the last 10 years. 

% The PI should list his/her activity as regards: 

% 1. Ten representative publications, as senior author (or in those fields where alphabetic order of authorship is the norm, joint author) in major international peer-reviewed multi-disciplinary scientific journals and/or in the leading international peer-reviewed journals and peer-reviewed conferences proceedings of their respective research fields, also indicating the number of citations (excluding self-citations) they have attracted (if applicable). 

% 2. Research monographs and any translations thereof (if applicable). 

% 3. Granted patents (if applicable). 

% 4. Invited presentations to peer-reviewed, internationally established conferences and/or international advanced schools (if applicable). 

% 5. Research expeditions that the applicant Principal Investigator has led (if applicable). 

% 6. Organisation of international conferences in the field of the applicant (membership in the steering and/or organising committee) (if applicable). 

% 7. International Prizes/ Awards/ Academy memberships (if applicable). 

% 8. Major contributions to the early careers of excellent researchers (if applicable) 

% 9. Examples of leadership in industrial innovation or design (if applicable). 
% }

% PI's research activity has concerned theoretical, methodological and realization aspects in different areas of CS and AI, most prominently: Knowledge Representation and Reasoning; Reasoning about Actions and Planning; Autonomous Agents; Service Composition and Orchestration; Process Modeling, Verification and Synthesis; Semantic Technologies and Ontologies; Description Logics; and Data Management and Integration. 

% The PI has deep impact on several areas of CS and AI. 
% \begin{itemize}
% \item  Mid 90's, \textbf{Correspondence between description logics and logics of programs}, his PhD thesis, deeply influenced expressive description logics ultimately leading to the definition of the W3C Standards OWL and OWL2. 
% % (AAAI94,KR96, JAIR96). 

% \item Late 90's \textbf{Conjunctive query answering in description logics}, 
%  with D.\ Calvanese, and M.\ Lenzerini, opened up the research on enabling the use of description logics for query databases. 
% % (PODS'98 - 419 cit, TOCL - 87 cit.).

% \item Early of 00's, \textbf{ConGolog programming language based on Situation Calculus},  with Y.\ Lesperance and H.\ Levesque, introduced the now standard transition semantics based programming languages in reasoning about action. 
% %(AIJ'00 - 652 cit).

% \item Early of 00's, \textbf{Planning using automata-theoretic
%     techniques from Verification}, together with M.\ Vardi, pioneered
%   studying advanced form of planning using techniques used from
%   Verification and Synthesis. % (ECP'99).

% \item Early 00's \textbf{Rich theory on regular path queries for views based query answering in
%      graph databases}, with D.\ Calvanese, M.\ Lenzerini, M.\ Vardi, a corner stone of the research in the field.

% \item Mid 00's  \textbf{Composition of stateful services}
%  with D.\ Berardi, M.\ Mecella and M.\ Lenzerini, on of the most influential framework for service composition, original paper (ICSOC'03) won the most influential paper of the decade at ICSOC'13.

% \item Mid 00's \textbf{Automated reasoning on UML Class Diagrams},
%  with D.\ Berardi, D.\ Calvanese,  EXPTIME-complete characterization reasoning on UML diagrams, deeply influential in formal analysis of conceptual object models.% (AIJ'05).

% \item Late 00's \textbf{DL-lite family: description logics with tractable data complexity},
%  with D.\ Calvanese,  M.\ Lenzerini, R.\ Rosati,  developed during EU TONES, made it possible to use description logic ontologies with dealing with big data (see EU OPTIQUE) % (JODS07 - 1131 cit.)

% \item Late 00's \textbf{Ontology Based Data Access} 
%  with D.\ Calvanese,  D.\ Lembo, M.\ Lenzerini, A.\ Poggi, R.\ Rosati, the most used framework using semantic technologies for data integration (Lutz's ERC consolidator grant CODA is based on it). 


% \item Early 10's \textbf{Advanced planning by model checking game structures},
% with F. Patrizi and S. Sardina and others,  cross-fertilization among planning, service-composition and syntesis by model checking game structures led to some of the most influential forms of generalized planning.

% \item Mid 10's \textbf{Decidability of data-aware processes}
%  with D.\ Calvanese, A.\ Deutsch, M.\ Montali,  developed during EU ACSI, cross-fertilization among databases, verification and knowledge representation, led to fundations for decidable verifiability of data-aware processes. % (PODS'13).

% \item Late 10's \textbf{Bounded situation calculus}
%  with Y.\ Lesperance and F. Patrizi, defined general condition for decidable verification of situation calculus.

% \item Late 10's \textbf{Verification, synthesis, planning using LTL and LDL over finite traces}
%  with M.\ Vardi, of interest in AI and declarative processes, promises to be a revolution in synthesis since sidesteps notorious constructions required for infinite traces.  
% \end{itemize}

PI's research concerns theoretical, methodological and applicative aspects in different areas of AI and CS, including:
%%
LTL and LDL over finite traces;
%%
Bounded situation calculus;
%%
Decidability of data-aware processes;
%%
Generalized planning by model checking and automata-theoretic
techniques from Verification;
%%
ConGolog programming language based on Situation Calculus;
%%
Ontology Based Data Access (OBDA);
%%
DL-lite family: description logics with tractable data complexity;
%%
Reasoning on UML Class Diagrams;
%%
Service composition and synthesis;
%%
Regular path queries for view-based query answering in graph databases;
%%
Data integration with description logics constraints;
%%
Conjunctive query answering in description logics;
%%
Correspondence between description logics and logics of programs.
%%
This work has deeply impacted several areas of AI and CS.

 % theoretical, methodological and realization aspects in different areas of AI and CS. , most prominently: Knowledge Representation and Reasoning; Reasoning about Actions and Planning; Autonomous Agents; Service Composition and Orchestration; Process Modeling, Verification and Synthesis; Semantic Technologies and Ontologies; Description Logics; and Data Management and Integration. 

\vspace{-1ex}
\subsection*{IMPACT MEASURES}
\vspace{-1ex}
The PI is the author of more than 250 publications in top journals and conferences, including the following CORE A*/A conferences and journals (since 2007 in bold, career in italic):
IJCAI (\textbf{17}, \textit{22}, A*), 
AAAI (\textbf{11}, \textit{16}, A*), 
KR (\textbf{10}, \textit{20}, A*), 
AAMAS (\textbf{7}, \textit{8}, A*), 
ICAPS (\textbf{5}, \textit{5}, A*), 
PODS (\textbf{3}, \textit{9}, A*), 
%LICS (\textbf{0}, \textit{2}, A*),
VLDB (\textbf{2}, \textit{3}, A*),
ISWC  (\textbf{2}, \textit{2}, A),
CAiSE (\textbf{2}, \textit{4}, A), 
ECAI (\textbf{3}, \textit{4}, A),
BPM (\textbf{3}, \textit{3}, A),
ICSOC (\textbf{1}, \textit{4}, A),  
ICDT (\textbf{1}, \textit{3}, A), 
Artif.\ Intell.\ (\textbf{4}, \textit{7}, A*),
J.\ Comput.\ Syst.\ Sci.\ (\textbf{2}, \textit{3}, A*), 
Inf.\ Syst.\ (\textbf{1}, \textit{2}, A*),
J.\ Log.\ Comput.\ (\textbf{1}, \textit{4}, A),
J.\ Artif.\ Intell.\ Res.\ (\textbf{1}, \textit{2}, A),
J.\ Autom.\ Reasoning (\textbf{1}, \textit{1}, A),
ACM Trans.\ Comput.\ Log.\ (\textbf{1}, \textit{2}, A),
Theor.\ Comput.\ Sci.\ (\textbf{1}, \textit{2}, A).


%%
A comprehensive list of publication can be found on DBLP \url{http://dblp.uni-trier.de}.
%\url{http://dblp.uni-trier.de/pers/hd/g/Giacomo:Giuseppe_De.html}.
%and\url{https://scholar.google.it/citations?user=Sfo4K0oAAAAJ&hl=en}.
%%
Currently the PI's 5 most cited papers are (citations from google scholar):
%\begin{tabular}{p{1cm}p{15cm}}
\begin{enumerate}[topsep=0pt,itemsep=-1ex,partopsep=1ex,parsep=1ex]
\item 
%1. & 
\textit{\textbf{Tractable reasoning and efficient query answering in description logics: The DL-Lite family.} D. Calvanese, G. De Giacomo, D. Lembo, M. Lenzerini, R. Rosati.  Journal of Automated Reasoning 39 (3), 385-429, 2007} - \textbf{1146} cit. %\\

\item 
%2. & 
\textit{\textbf{ConGolog, a concurrent programming language based on the situation calculus.} G.\ De Giacomo, Y.\ Lesperance, H.\ Levesque: Artif. Intell. 121(1-2): 109-169 (2000)} -  \textbf{654} cit. %\\


\item 
%3. &
\textit{\textbf{Linking data to ontologies.}
A.\ Poggi, D.\ Lembo, D.\ Calvanese, G.\ De Giacomo, M.\ Lenzerini, R.\ Rosati.
Journal on data semantics X, 133-173, 2008} - \textbf{560} cit. %\\


\item 
%4. & 
\textit{\textbf{Reasoning on UML class diagrams.} D.\ Berardi, D.\ Calvanese, G.\ De Giacomo. Artif. Intell. 168(1-2): 70-118 (2005)} - \textbf{547} cit. %\\

\item 
%5. &
\textit{\textbf{Automatic Composition of E-services That Export Their Behavior.} D.\ Berardi, D.\ Calvanese, G.\ De Giacomo, M.\ Lenzerini, M.\ Mecella. ICSOC 2003: 43-58} - \textbf{519} cit. Awarded as ``\textbf{the most influential ICSOC paper in the last 10 years}'' at ICSOC 2013.

\end{enumerate}
%\end{tabular}
According to Google Scholar, July 2017, the PI's h-index is 68, with 16952 citations and his i10-index is 191. These values are among the highest in Europe in AI and CS and make the PI the 3rd most cited CS author in Italy, according to a study available at \url{http://via-academy.org}. 

\vspace{-1ex}
\subsection*{REPRESENTATIVE PUBLICATIONS (SELECTION)}
\vspace{-1ex}

\begin{enumerate}[topsep=0pt,itemsep=-1ex,partopsep=1ex,parsep=1ex]

% \item \textit{\textbf{First-Order $\mu$-Calculus over Generic Transition Systems and Applications to the Situation Calculus.} D.\ Calvanese, G.\ De Giacomo, M.\ Montali, F.\ Patrizi, Information and Computation (2017). To appear.} - CORE  B.

\item \textit{\textbf{Generalized Planning: Non-Deterministic Abstractions and Trajectory Constraints}. B.\ Bonet, G.\ De Giacomo, H.\ Geffner, S.\ Rubin. IJCAI 2017} - CORE  A*. %Cit. --

% \item \textit{\textbf{On the Disruptive Effectiveness of Automated Planning for LTL$_f$-based Trace Alignment.} G.\ De Giacomo, F.\ Maggi, A.\ Marrella, F.\ Patrizi. AAAI 2017: 3555-3561} - CORE  A*.

\item \textit{\textbf{Agent planning programs.} G.\ De Giacomo, A.\ Gerevini, F.\ Patrizi, A.\, Saetti, S.\ Sardina. Artif. Intell. 231: 64-106 (2016)} - CORE  A*. % Cit.\ 2

 \item \textit{\textbf{Bounded situation calculus action theories.} G.\ De Giacomo, Y.\ Lesperance, F.\ Patrizi.
 Artif. Intell. 237: 172-203 (2016)} - CORE  A*. %Cit.\ 5

\item \textit{\textbf{Regular Open APIs.}
D.\ Calvanese, G.\ De Giacomo, M.\ Lenzerini, M.\ Vardi. KR 2016: 329-338.}  - CORE  A*. %Cit.\ 0.

\item \textit{\textbf{Synthesis for LTL and LDL on Finite Traces.} G.\ De Giacomo, M.\ Vardi: IJCAI 2015: 1558-1564} - CORE  A*. %Cit.\ 10

% \item \textit{\textbf{Adding DL-Lite TBoxes to Proper Knowledge Bases.} G.\ De Giacomo, H.\ Levesque. ISWC 2015: 305-321} - CORE  A.

% \item \textit{\textbf{Declarative Process Modeling in BPMN.} G.\ De Giacomo, M.\ Dumas, F.\ Maggi, M.\ Montali. CAiSE 2015: 84-100} - CORE  A.

\item \textit{\textbf{Data complexity of query answering in description logics.}. D. Calvanese, G. De Giacomo, D. Lembo, M. Lenzerini, R. Rosati.
Artif. Intell. 195: 335-360 (2013)} - CORE  A*. %Cit. 402  including IJCAI15


% \item \textit{\textbf{Reasoning on LTL on Finite Traces: Insensitivity to Infiniteness. .} G.\ De Giacomo, R.\ De Masellis, M.\ Montali. 
% AAAI 2014: 1027-1033}

\item
\textit{\textbf{Automatic behavior composition synthesis.} G.\ De Giacomo,  F.\ Patrizi, S. Sardina. Artif. Intell. 196: 106-142 (2013)} - CORE  A*. %Cit.\ 46


\item \textit{\textbf{Verification of relational data-centric dynamic systems with external services.}
B.\ Bagheri Hariri, D.\ Calvanese, G.\ De Giacomo, A.\ Deutsch, M.\ Montali. PODS 2013: 163-174} - CORE  A*. %Cit. 113

% \item \textit{\textbf{Automatic composition of e-services that export their behavior.}
% D. Berardi, D. Calvanese, G. De Giacomo, M. Lenzerini, M. Mecella.
% International Conference on Service-Oriented Computing, 43-58, 2003} (516 citations)
% + (227 IJCIS05)
% \textit{
% Automatic composition of transition-based semantic web services with messaging
% D Berardi, D Calvanese, G De Giacomo, R Hull, M Mecella
% Proceedings of the 31st international conference on Very large data bases ... 2005} (331 citations) 

% \item \textit{Automatic behavior composition synthesis
% G De Giacomo, F Patrizi, S Sardina
% Artificial Intelligence 196, 106-142, 2013} (46 citations)

% \item
% \textit{\textbf{Description Logic Knowledge and Action Bases.}
% B.\ Bagheri Hariri, D.\ Calvanese, M.\ Montali, G.\ De Giacomo, R.\ De Masellis, P.\ Felli:
J. Artif. Intell. Res. (JAIR) 46: 651-686 (2013)} - CORE  A.

\item \textit{\textbf{On simplification of schema mappings.} D.\ Calvanese, G.\ De Giacomo, M.\ Lenzerini, M.\ Vardi.  J. Comput. Syst. Sci. 79(6): 816-834 (2013)} - CORE  A*. %Cit. 4

% \item \textit{\textbf{Linear temporal logic and linear dynamic logic on finite traces.}
% G. De Giacomo, M. Vardi/
% IJCAI'13 Proceedings of the Twenty-Third international joint conference on ..., 2013} (75 citations)

\item 
\textit{\textbf{View-based query answering in Description Logics: Semantics and complexity.}
D.\ Calvanese, G.\ De Giacomo, M.\ Lenzerini, R.\ Rosati. J. Comput. Syst. Sci. 78(1): 26-46 (2012)} - CORE  A*. %Cit. 24

% \item
\textit{\textbf{Conjunctive query containment and answering under description logic constraints.} Diego Calvanese, Giuseppe De Giacomo, Maurizio Lenzerini ACM Trans. Comput. Log. 9(3): 22:1-22:31 (2008)} (87 citations)




% \item \textit{\textbf{Linking data to ontologies.}
% A.\ Poggi, D.\ Lembo, D.\ Calvanese, G.\ De Giacomo, M.\ Lenzerini, R.\ Rosati.
% Journal on data semantics X, 133-173, 2008} (546 citations)



% \item \textit{\textbf{Tractable reasoning and efficient query answering in description logics: The DL-Lite family.}. D. Calvanese, G. De Giacomo, D. Lembo, M. Lenzerini, R. Rosati.
% Journal of Automated Reasoning 39 (3), 385-429, 2007} (1131 citations) 



% \item \textit{\textbf{View-based query processing: On the relationship between rewriting, answering and losslessness.} D.\ Calvanese, G.\ De Giacomo, M.\ Lenzerini, M.\ Vardi.  Theor. Comput. Sci. 371(3): 169-182 (2007)} (40 citations)




% % \textit{Synthesis for LTL and LDL on Finite Traces.
% % G De Giacomo, MY Vardi
% % IJCAI, 1558-1564, 2015} (11 citations)
% % \textit{Reasoning on LTL on Finite Traces: Insensitivity to Infiniteness.
% % G De Giacomo, R De Masellis, M Montali
% % AAAI, 1027-1033, 2014} (38 citations)

% % \textit{Monitoring business metaconstraints based on LTL and LDL for finite traces
% % G De Giacomo, R De Masellis, M Grasso, FM Maggi, M Montali
% % International Conference on Business Process Management, 1-17} (18 citations)

% % \textit{LTLf and LDLf Synthesis under Partial Observability
% % G De Giacomo, MY Vardi
% % Proceedings of the 25th International Joint Conference on Artificial ...2016} (1 citations)



% \item \textit{\textbf{Verification of relational data-centric dynamic systems with external services.}
% B.\ Bagheri Hariri, D.\ Calvanese, G.\ De Giacomo, A.\ Deutsch, M.\ Montali
% Proceedings of the 32nd ACM SIGMOD-SIGACT-SIGAI symposium on Principles of .. 2013} (112 citations)



% % \textit{Bounded Situation Calculus Action Theories and Decidable Verification.
% % G De Giacomo, Y Lespérance, F Patrizi
% % KR, 2012,} (47 citations)
\end{enumerate}

\vspace{-1ex}
\subsection*{INVITED TALKS (SELECTION)}
\vspace{-1ex}
%\begin{tabular}{p{1cm}p{15cm}}

\begin{itemize}[topsep=0pt,itemsep=-1ex,partopsep=1ex,parsep=1ex]

% \item\textbf{AI Foundations for Data-Aware Business Processes}. Invited Talk at University of Toronto, Toronto, ON, Canada, April 2017. %\\

\item 
On \textbf{LTL and LDL on Finite Traces: Reasoning, Verification, and Synthesis}:
GenPlan@ICAPS 2017 (Keynote), Pittsburgh, USA; SR 2016 (Keynote), New York, USA; Highlights of Logic, Games and Automata 2015 (Keynote), Prague, CZ; ICAPS 2013 (Keynote), Rome, IT.

% On \textbf{LTL and LDL on Finite Traces: Reasoning, Verification, and Synthesis}. 
% Invited talk at the GenPlan17@ICAPS, Pittsburgh, USA, June 2017; Invited talk at the 4th Int.\ Work.\ on Strategic Reasoning (SR 2016), New York, USA, July 2016; Keynote at Highlights of Logic, Games and Automata 2015 (Highlights 2015), Prague, Czech Republic, September 2015. Keynote at 23rd International Conference on Automated Planning and Scheduling (ICAPS 2013), Rome, Italy, June 2013.%\\

\item 
On \textbf{Data-Aware Business Processes}:  York U., (Distinguished Lassonde Lecture), Canada, 2017 ; U.\ of Toronto, Canada, 2017; WS-FM:FASOCC@BPM 2014, Eindhoven NL;  ECAI 2014 (Keynote for Frontiers of AI), Prague, CZ. 

% Distinguished Lecture at York University's Centre for Innovation in Computing at Lassonde (IC@L), Toronto, April 2017; University of Toronto, April 2017; Keynote at 11th Inter.\ Work.\ on Web Services and Formal Methods: Formal Aspects of Service-Oriented and Cloud Computing (WS-FM:FASOCC 2014), Eindhoven, September 2014;  Invited talk for Frontiers of Artificial Intelligence at 21st European Conf.\ on Artificial Intelligence (ECAI 2014), Prague, August 2014. 


\item 
On \textbf{Bounded Situation Calculus}: WS.\ Formal Methods in AI 2017 (Keynote), U.\  ``Federico II'', Naples, IT; TIME 2015 (Keynote), Kassel, GE; WS.\ HYBRIS 2015 (Keynote), Potsdam, GE.  

% On \textbf{Bounded Situation Calculus}. Invited talk 1st Workshop on Formal Methods in AI, University  ``Federico II'', Naples, Italy, February 2017; Keynote at 22nd Int.\ Symp.\ on Temporal Representation and Reasoning (TIME 2015), Kassel, Germany, September 2015; Keynote at HYBRIS Work.\, Potsdam, Germany, June 2015.  %\\

\item 
On \textbf{Service Composition and Synthesis}: ICSOC 2013 (Invited talk for prize as  most influential paper of  decade), Berlin, GE; U. Brescia, IT 2012; U.\ of Toronto Canada, 2010; York University, Toronto, Canada, 2010; INFINT WS 2009, Bertinoro, IT; MSI 2005 Berlin, GE , Caen, FR.


% Invited talk for the prize as the most influential ICSOC paper in the last 10 years, ICSOC 2013, Berlin, December 2013; University of Toronto Canada, October 2010; York University, Toronto, Canada, October 2010; INFINT’09 Bertinoro Workshop on Data and Service Integration, Bertinoro, Italia, March 2009; Keynote at MSI’05, Caen, France, 2005%\\
% \item 
% On \textbf{Cognitive Robotics: The science of building intelligent autonomous robots and software agents}. Miegunyah Fellow Public Lecture, Melbourne, Australia August 2013. %\\

\item 
On \textbf{Ontology-based Data Access and Integration}: DL 2013  (Keynote), Ulm, Germany; U.
 of Toronto, Canada, 2010; Semantic Days Conference 2009, Stavanger, NO; IBM Research Center Watson, Hawthorne, NY, USA 2008.

% On \textbf{Ontology-based Data Access and Integration}: Keynote at 26th Int.\ Work.\ on Description Logics (DL 2013), Ulm, Germany, July 2013; University of Toronto, Canada, October 2010; Semantic Days 2009 Conference Stavanger, Norway, May 2009; IBM Research Center Watson, Hawthorne, NY, USA, August 2008.
\end{itemize}
%\end{tabular}

% \begin{itemize}[topsep=0pt,itemsep=-1ex,partopsep=1ex,parsep=1ex]

% % \item\textbf{AI Foundations for Data-Aware Business Processes}. Invited Talk at University of Toronto, Toronto, ON, Canada, April 2017. %\\


% \item 
% %2017 & 
% \textbf{AI Foundations for Data-Aware Business Processes}. Distinguished Talk at York University's Centre for Innovation in Computing at Lassonde (IC@L), Toronto, ON, Canada, April 2017. %\\

% \item 
% %2017 &
% \textbf{First-Order mu-Calculus over Generic Transition Systems and Applications to the Situation Calculus}. Invited talk 1st Workshop on Formal Methods in AI, University  ``Federico II'', Naples, Italy, February 2017. %\\

% \item 
% %2016 & 
% \textbf{LTL and LDL on Finite Traces: Reasoning, Verification, and Synthesis}. Invited talk at the 4th Int.\ Work.\ on Strategic Reasoning (SR 2016), New York, USA, July 2016. %\\

% \item 
% %2015 & 
% \textbf{Synthesis in Linear-time Dynamic Logic on Finite Traces}. Keynote at Highlights of Logic, Games and Automata 2015 (Highlights 2015), Prague, Czech Republic, September 2015. %\\

% \item 
% %2015 & 
% \textbf{Temporal Reasoning in Bounded Situation Calculus}. Keynote at 22nd Int.\ Symp.\ on Temporal Representation and Reasoning (TIME 2015), Kassel, Germany, September 2015. %\\

% \item 
% %2015 & 
% \textbf{On Bounded Situation Calculus}. Keynote at HYBRIS Work.\, Potsdam, Germany, June 2015. %\\

% \item 
% %2014 & 
% \textbf{Verification of Data-Aware Processes}. Keynote at 11th Inter.\ Work.\ on Web Services and Formal Methods: Formal Aspects of Service-Oriented and Cloud Computing (WS-FM:FASOCC 2014), Eindhoven, September 2014. %\\

% \item 
% %2014 & 
% \textbf{Reasoning about data and knowledge-aware processes}. Invited talk for Frontiers of Artificial Intelligence at 21st European Conf.\ on Artificial Intelligence (ECAI 2014), Prague, August 2014. %\\

% \item 
% %2013 & 
% \textbf{Automatic Composition of E-services That Export Their Behavior}. Invited talk for the prize as the most influential ICSOC paper in the last 10 years, ICSOC 2013, Berlin, December 2013. %\\

% \item 
% %2013 & 
% \textbf{Cognitive Robotics: The science of building intelligent autonomous robots and software agents}. Miegunyah Fellow Public Lecture, Melbourne, Australia August 2013. %\\

% \item 
% %2013 & 
% \textbf{Actions, Processes, and Ontologies}. Keynote at 26th Int.\ Work.\ on Description Logics (DL 2013), Ulm, Germany, July 2013. %\\

% \item 
% %2013 & 
% \textbf{Linear Temporal Logics on Finite Traces: Reasoning, Verification, and Synthesis}. Keynote at 23rd International Conference on Automated Planning and Scheduling (ICAPS 2013), Rome, Italy, June 2013.

% % \item	 ``Conjunctive Queries: Evaluation and Containment'' at University of Toronto, Canada, November 2010. 
% % \item	``Automatic Service Composition and Synthesis: the Roman Model'' at York University, Toronto, Canada, October 2010;
% % \item	``Towards Systems for Ontology-based Data Access and Integration using Relational Technology'' at University of Toronto, Canada, October 2010;
% % \item	``Automatic Service Composition and Synthesis: the Roman Model'' at University of Toronto Canada, October 2010;
% % \item	 ``QUONTO: ontology-based data access and integration using relational technology'', Semantic Days 2009 Conference Stavanger, Norway, May 2009;
% % \item	  ``The Roman model for Service Composition'' at INFINT’09 Bertinoro Workshop on Data and Service Integration, Bertinoro, Italia, March 2009;
% % \item	 ``Ontology Based Data Integration'' at IBM Research Center Watson, Hawthorne, NY, USA, August 2008;
% % \item	 ``Process integration: Look at how you behave!'' at INFINT’07 Bertinoro Workshop on Information Integration, Bertinoro, Italia, October 2007;
% % \item	 ``Automatic composition synthesis of web services: a conceptual perspective'' at MSI’05, Caen, France, 2005.

% \end{itemize}


\vspace{-1ex}
\subsection*{RESEARCH PROJECTS (SELECTION)}
\vspace{-1ex}
The PI has leaded several projects, including:
%
%\begin{itemize}[topsep=0pt,itemsep=-1ex,partopsep=1ex,parsep=1ex]
%\begin{tabular}{p{2cm}p{14cm}}
%\item 
(2005--2008) EU FP6-7603 \textbf{TONES: Thinking ONtologiES}, PI for Sapienza, value: EUR 264,000  (total value: EUR 1,438,910), from final review:  \emph{The TONES project can be considered as a success story of a FET-project in terms of scientific achievements};
%\item 
(2010--2013) EU FP7-ICT-257593 \textbf{ACSI: Artifact-Centric Service
  Interoperation}, PI for Sapienza, and Scientific Coordinator for the wole project, value: EUR 435,000 (total value:
EUR 3,243,937), from final review: \emph{The scientific productivity
of ACSI has been extraordinary, leading to a great impact on BPM, DB
and AI research, as indicated by invited talks and tutorials on ACSI
results in many high profile conferences, and by the addition of ACSI
topics to the list of conference and workshop topics in these
areas.};
%\item 
(2012--2016) EU FP7-IST-IP-318338 \textbf{OPTIQUE: Scalable End-user Access to Big Data}, key personel of Sapienza, value: EUR 802 488  (total  value: EUR 9,838,329);
%\item 
(2009--2014) Open Collaboration Research %Agreement 
W0954341 with Rick Hull of IBM T. J. Watson Research Center, NY, on \textbf{Data Aware Business Processes and Operation, through An Artifact-Centric Approach}, PI, value: USD 45,000;
%\item 
(2010--2012) UK Royal Society International Joint Project 2009/R2 on \textbf{Web Services Automatic Synthesis through ATL Symbolic Model Checking}, with Alessio Lomuscio, Imperial College London, total value: GBP 12,000;
%\item 
(2012--2014) Australian Research Council  (ARC) %Competitive Research Grant - 
Discovery Project DP120100332 \textbf{Optimisation of Embedded Virtual Complex Systems by Re-using a Library of available component}, with Sebastian Sardina of RMIT and Maurice Pagnucco of Univ.\ of Sidney, PI for Sapienza, total  value: AUD 406,278;
%\item 
(2013--2015) Sapienza Ateneo Project \textbf{Spiritles: Spiritlet-based Smart Spaces}, PI, value: EUR 60,000.
%Currently active project are listed in the CV.
%\item 
(2013--2018) US NSF Award Number: 1319459 \textbf{SHF: Small: Pushing the Frontier of Linear-Time Model-Checking Technology}, with Moshe Vardi, KP, total  value: USD 304,582;
% %\item 
(2015--2018)  Sapienza Ateneo Project \textbf{ICE: Immersive Cognitive Environments}, PI, value: EUR 40.000;
% %\item 
(2016--2019) China NFS Project %Grant 
No. 61572535 \textbf{Theory and Techniques for Reasoning about Actions and High-level Agent Control in Multi- agent Domains}, with Yongmei Liu of Sun Yat-sen U.\ in Guangzhou, PI for Sapienza, total value: CNY 804,000.
%\end{tabular}
%\end{itemize}

% He has been involved in several contracts within the department to do data integration, preparation and discovery activities for private and public organizations, including Monte dei Paschi di Siena, Italian Minstery of Finance aand Economics, Telecom Italia, Bloomberg, Italian Automobil Club (current). This industrial success has led to founding the Sapienza Start Up ``OBDA Systems'' dedicated to the application of semantic technologies to data integration in February 2017.



% He has been involved in several National and European projects, including:
% %
% PI of the Sapienza unit of  EU FP6-7603 TONES – Thinking ONtologiES (2005-2008);
% %
% PI of the Sapienza unit of EU FP7-ICT-257593 ACSI Artifact-Centric Service Interoperation (2010-2013);
% %
% key personel of   EU FP7-IST-IP-3183382012-2016 Scalable End-user Access to Big Data (Optique);
% %
% PI  of Open Collaboration Research Agreement W0954341 with Rick Hull of IBM T. J. Watson Research Center, NY, on ``data aware business processes and operation, through an artifact-centric approach'' (2009-2014); PI of UK Royal Society International Joint Project 2009/R2 on ``web services automatic synthesis through ATL symbolic model checking'', joint with Alessio Lomuscio, Imperial College London; PI of Australian Research Council (ARC) Competitive Research Grant - Discovery Project DP120100332 ``Optimisation of embedded virtual complex systems by re-using a library of available component'' with Sebastian Sardina of RMIT and Maurice Pagnucco of Univ.\ of Sidney (2012-2016); PI of Project Natural Science Foundation of China project Grant No. 61572535 ``Theory and Techniques for Reasoning about Actions and High-level Agent Control in Multi-agent Domains'' with Yongmei Liu of Sun Yat-sen Univ.\ in Guangzhou (2016-2019); PI of Project Award Sapienza 2013 ``Spriritles: Spiritlet-based Smart Spaces'' (2013-2015); PI of Project Award Sapienza 2015 ``ICE: Immersive Cognitive Environments'' (2015-2018).

\vspace{-1ex}
\subsection*{INDUSTRIAL LEADERSHIP}
\vspace{-1ex}

A pioneer project in 2003 with IBM Tivoli Lab highlighted several fundamental limitations in using semantic technologies available at that time for knowledge management. This gave rise to a novel research line that led to a new kind of Description Logic, called DL-Lite, and a new paradigm for accessing data through semantic technologies, called Ontology-based Data Access (OBDA). Since then, the PI led several industrial explorations promoting OBDA for data integration, preparation and discovery, which involved private and public organizations, such as the Monte dei Paschi di Siena Bank, Italian Minstery of Finance and Economics, Telecom Italia, Bloomberg, Italian Automobil Club (current). The success of such explorations has led the PI to found the Sapienza Start Up \textbf{OBDA Systems} (\url{http://www.obdasystems.com}) in Feb.\ 2017. The PI has been also a Contributor to W3C  recommendation of  OWL 2 Web Ontology Language 
Profiles (\url{https://www.w3.org/TR/owl2-profiles/}).
% dedicated to the application of semantic technologies to data integration in February 2017.

 % In a notable project with the Monte dei Paschi di Siena bank, during the acquisition of Banca Antonveneta, an ontology has been built regarding all customer data and the associated risk management tasks, and mappings between the ontology and the existing data sources have been defined. Similar projects have been carried out with Selex SI about data on satellite equipments, Telecom Italia about the data on the Italian Telecommunication Network,  the Italian Ministry of Economy and Finance about the data regarding the Italian public debt, and the Autom

 % In all the above contexts, the combination of front- and back-end scalability problems had led to an inflexible and slow regime in which only a fixed set of predefined queries, optimized and tuned for their purpose, was available to the domain experts. This severely hampered the ability to explore and analyze the data, and prevented from fully exploiting and creating value out of it. The OBDA approach enabled from the one hand the establishment of new data governance functions, and, on the other hand, with the use of MASTRO, a new way to deal with data access. Queries about the domain are now posed over the ontology, and the reasoning system derives the answers by rewriting it over the real data sources. Among the various advantages of the new framework, non-IT experts can now access relevant data in much more easy and flexible ways. In all the above contexts, the success of the project led to extend the collaboration with the University of Roma La Sapienza with the aim of replicating the approach in order to cover the whole domain of the organization, as well as the whole set of data sources within the organization.

% \begin{tabular}{p{1cm}p{15cm}}
% 2017 & Co-founder and scientific councillor of Sapienza Start Up
%   \textbf{OBDA Systems} (\url{http://www.obdasystems.com}) for the
%   semantic integration and governance of data.\\
% 2012 & Contributor to W3C  recommendation of  OWL 2 Web Ontology Language 
% Profiles (\url{https://www.w3.org/TR/owl2-profiles/}).
%\end{tabular}

% 2001 & Dall’ottobre 1998 all’ottobre 2001 ha sviluppato con i Prof. Hector Levesque e Prof. Ray Reiter della University of Toronto e il Prof. Yves Lesperance della York University, Toronto, la specifica dei linguaggi di programmazione ConGolog ed IndiGolog basati sul Situation Calculus e degli interpreti per la loro esecuzione. Tali prodotti sono stati depositati presso le autorit`a competenti dell’Ontario, Canada sotto il nome ``Congolog and Indigolog agent/robot programming languages and their implementation'', e il Governing Council of the University of Toronto `e stato investito del compito di verificare un loro utilizzo in ambito commerciale.



%\subsection*{PUBLICATIONS}
% A comprehensive list of publication can be found at
% \url{http://dblp.uni-trier.de/pers/hd/g/Giacomo:Giuseppe_De.html} and
% \url{https://scholar.google.it/citations?user=Sfo4K0oAAAAJ&hl=en}.


\renewcommand{\baselinestretch}{1}

%%% Local Variables:
%%% mode: latex
%%% TeX-master: "PartB1"
%%% TeX-PDF-mode: t
%%% End:
