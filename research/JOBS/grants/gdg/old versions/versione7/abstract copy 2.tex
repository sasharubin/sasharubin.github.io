% This is the abstract.  Max 2000 characters.\\

% \note{Proposal summary (identical to the abstract from the online proposal submission forms, section 1). 

% The abstract (summary) should, at a glance, provide the reader with a clear understanding of the objectives of the research proposal and how they will be achieved. The abstract will be used as the short description of your research proposal in the evaluation process and in communications to contact in particular the potential remote referees and/or inform the Commission and/or the programme management committees and/or relevant national funding agencies (provided you give permission to do so where requested in the online proposal submission forms, section 1). It must therefore be short and precise and should not contain confidential information. 

% Please use plain typed text, avoiding formulae and other special characters. The abstract must be written in English. There is a limit of 2000 characters (spaces and line breaks included).}

We are witnessing an increasing availability of \textbf{mechanisms}
that offer some form of programmability.
%%
These include software, manufacturing devices, smart objects and smart
spaces, intelligent robots, business process management systems, and
many others.
%%
All these mechanisms are being currently revolutionized by means of
advances in Machine Learning (ML).  In particular central sensing
capabilities (vision, language undestanding) and actuation
capabilities (robot-arm movement, camera twisting) are increasingly
 handled by ML-components.
%%
However, irrespective of the technological nature of the components, the connections between them are still
organized, monitored, and coordinated through standard programming.

\project (pronounced ``wisemech'') aims at developing the \textbf{science} and the \textbf{tools} for a new
generation of mechanisms to emerge: mechanisms that are able to
\textbf{program themselves} without human intervention, and
automatically tailor their behavior so as to
%%
achieve desired goals,  maintain themselves within safe boundaries in a
changing environment, and keep following rules,
regulations and conventions that evolve over time. 
%%
Crucially, self-programming requires \textbf{balancing power with safety}.
For this reason, \project intends 
%to exploit \textbf{Knowledge Representation} (KR) for realizing 
to realize self-programming mechanisms that are \textbf{white-box},
that is, 
that are \textbf{guided and guarded by %human guided
specifications}, which together with the synthesized behavior, 
are fully \textbf{comprehensible in human terms}.
Note that such mechanisms might incorporate black-box
ML-components, though with suitable safety guarantees.


% \project will deliver a novel synthesis paradigm setting off
% the ability of planning to deal efficiently with a finite horizon,
% through the richness of classical synthesis in dealing with
% sophisticated temporal specifications.

% \project will deliver a novel synthesis paradigm that
% will boost the ability of planning to efficiently deal with a finite horizon,
% building on the richness of classical synthesis to express temporal specifications.

% \project will deliver a novel synthesis paradigm that incorporates
% ability of Automated Planning in AI to efficiently deal with finite
% horizons, and the richness of classical synthesis to express
% sophisticated temporal specifications, but
% without the complexity at the infinitum that made it hardly applicable in
% practice.

% \project will deliver a novel synthesis paradigm tailored towards handling the richness of finite horizons with the efficiency of  
% Automated Planning in AI but with the richness reactive synthesis in Verification.

%  richness of classical synthesis to express
% sophisticated temporal specifications, but
% without the complexity at the infinitum that made it hardly applicable in
% practice.

% \project will deliver a novel paradigm for self-programming that features the efficiency of planning while offering the richness of classical synthesis. To this end, \project
% will combine the ability of planning to efficiently deal with a finite horizon, with the
% expressiveness of classical synthesis to capture sophisticated temporal specifications.




Scientifically, \project aims at integrating insights from four areas:
%1
\textbf{Knowledge Representation} on how to represent the domain
of interest and the mechanisms' sensing and acting capabilities in a
high-level human comprehensible fashion,
%2
\textbf{Data-aware Processes} in \emph{Databases} on how to
build data-aware dynamic behaviors,
%3
\textbf{Verification and Synthesis} in \emph{Formal
  Methods}, which provides mathematically elegant foundations for
synthesis,
%4
and \textbf{Automated Planning} in \emph{AI}, which yields
algorithmic insights into the search process required by synthesis.
%
In this way \project will \textbf{bring together and cross-fertilize
  four distinct research areas} with overlapping interests to produce a \textbf{breakthrough in engineering self-programming mechanisms that are human-comprehensible and safe by design}.

% Note that while \project is not about ML, it intends to incorporate
% into white-box self-programming mechanisms black-box ML-components,
% though with suitable safety guarantees.

\project will ground its scientific results upon diverse
\textbf{driving applications}, including smart manufacturing
(Industry 4.0), smart spaces (IoT) and business process management systems
(BPM).

 


%%% Local Variables:
%%% mode: latex
%%% TeX-master: "PartB1"
%%% TeX-PDF-mode: t
%%% End:
