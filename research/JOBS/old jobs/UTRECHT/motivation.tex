\documentclass[10.5pt,a4paper,sans]{moderncv}       
% possible options include font size ('10pt', '11pt' and '12pt'), paper size ('a4paper', 'letterpaper', 'a5paper', 'legalpaper', 'executivepaper' and 'landscape') and font family ('sans' and 'roman')
% moderncv themes
\moderncvstyle{classic}                             % style options are 'casual' (default), 'classic', 'banking', 'oldstyle' and 'fancy'
\moderncvcolor{blue}                               % color options 'black', 'blue' (default), 'burgundy', 'green', 'grey', 'orange', 'purple' and 'red'
%\renewcommand{\familydefault}{\sfdefault}         % to set the default font; use '\sfdefault' for the default sans serif font, '\rmdefault' for the default roman one, or any tex font name
%\nopagenumbers{}                                  % uncomment to suppress automatic page numbering for CVs longer than one page
% character encoding
%\usepackage[utf8]{inputenc}                       % if you are not using xelatex ou lualatex, replace by the encoding you are using
%\usepackage{CJKutf8}                              % if you need to use CJK to typeset your resume in Chinese, Japanese or Korean
% adjust the page margins
\usepackage[scale=0.75]{geometry}
%\setlength{\hintscolumnwidth}{3cm}                % if you want to change the width of the column with the dates
%\setlength{\makecvtitlenamewidth}{10cm}           % for the 'classic' style, if you want to force the width allocated to your name and avoid line breaks. be careful though, the length is normally calculated to avoid any overlap with your personal info; use this at your own typographical risks...
% personal data
\name{Sasha}{Rubin}
\title{Curriculum Vitae, April 2017}                               % optional, remove / comment the line if not wanted
\address{University of Naples ``Federico II''}{Naples, Italy}% optional, remove / comment the line if not wanted; the "postcode city" and "country" arguments can be omitted or provided empty
% \phone[mobile]{+1~(234)~567~890}                   % optional, remove / comment the line if not wanted; the optional "type" of the phone can be "mobile" (default), "fixed" or "fax"
% \phone[fixed]{+2~(345)~678~901}
% \phone[fax]{+3~(456)~789~012}
\email{rubin@unina.it}                               % optional, remove / comment the line if not wanted
\homepage{forsyte.at/alumni/rubin/}                         % optional, remove / comment the line if not wanted
% \social[linkedin]{john.doe}                        % optional, remove / comment the line if not wanted
% \social[twitter]{jdoe}                             % optional, remove / comment the line if not wanted
% \social[github]{jdoe}                              % optional, remove / comment the line if not wanted
% \extrainfo{additional information}                 % optional, remove / comment the line if not wanted
% \photo[70pt][0.4pt]{RUBIN_Sasha.jpg}                       % optional, remove / comment the line if not wanted; '64pt' is the height the picture must be resized to, 0.4pt is the thickness of the frame around it (put
% bibliography adjustements (only useful if you make citations in your resume, or print a list of publications using BibTeX)
%   to show numerical labels in the bibliography (default is to show no labels)
\makeatletter\renewcommand*{\bibliographyitemlabel}{\@biblabel{\arabic{enumiv}}}\makeatother
%   to redefine the bibliography heading string ("Publications")
%\renewcommand{\refname}{Articles}

% bibliography with mutiple entries
%\usepackage{multibib}
%\newcites{book,misc}{{Books},{Others}}
%----------------------------------------------------------------------------------
%            content
%----------------------------------------------------------------------------------
\begin{document}

%-----       letter       ---------------------------------------------------------
% recipient data
\recipient{Prof M. van Kreveld, Research Director}
{Department of Information and Computing Science\\
Utrecht University}
\date{May 13, 2017}
\opening{Dear Prof van Kreveld,}
\closing{Sincerely,\vspace{-1cm}}
% \enclosure[Attached]{curriculum vit\ae{}}          % use an optional argument to use a string other than "Enclosure", or redefine \enclname
\makelettertitle




I am applying for the position of assistant professor.

I am a computer scientist interested in formal methods in artificial intelligence, and since 2015 I have been a Post-doc at the University of Naples ``Federico II'', mentored by Aniello Murano, with a focus on logics for temporal, epistemic and strategic reasoning in artificial intelligence including  multi-agent systems. As this letter will show, formal methods in artificial intelligence informs my research, supervision and teaching since 2014.

\textbf{Background}
My PhD thesis, titled ``Automatic Structures'' (2004), was supervised by Bakhadyr Khoussainov at the University of Auckland, and won the best-doctoral thesis in the faculty of computer science. It was about using automata and logic to describe and reason about infinite 
mathematical structures. From 2004-2007 I held a prestigious individual postdoctoral fellowship funded by the New Zealand government on the same topic. I then held various postdoctoral and teaching positions until 2014. Feeling the need to work in an area with more relevance to computer scientists, since 2014 I started shifting my research to formal methods in AI. From 2015-2016 I held another individual fellowship, a COFUND Marie Curie fellowship, 
jointly funded by the European Commission and the Institute for Higher Mathematics (INdAM ``F. Severi''). The topic of this fellowship was verification of lightweight multi-agent systems; one of the publications from this work resulted in a best-paper award at PRIMA15. 

\textbf{Integration in the international community}
Since 2013, I am chair or organiser of 5 events (workshops and conferences), including the First Workshop on Formal Methods in Artificial Intelligence in 2017 with keynote speakers including Hector Geffner and Michael Fisher. I have served 
as a PC member for AI conferences such as IJCAI17, AAAI17, and ECAI16. Since 2014 I have collaborated with leading experts in Knowledge Representation (Giuseppe De Giacomo), Automated Planning (Hector Geffner and Blai Bonet), Logic in Computer Science (Moshe Vardi and Helmut Veith), and Formal Methods in Multi-agent systems (Michael Wooldridge and Alessio Lomuscio). These collaborations demonstrate my expanding breadth of interest and research.


\textbf{Teaching}
I am very interested in good pedagogy. Notably, while at Cornell University 2008-2009 I sought a number of teaching mentors, including Maria Terrell (Department of Mathematics) and David Way (associate director of the Cornell University Centre for Teaching Excellence) to discuss successful teaching strategies, both philosophical and concrete. As a result, according to my student evaluations, I was clear, organised, proactively willing to help, and motivating.

% in Naples.
% on the topic of ``Graphical Games'' at the University of Naples.

Regarding my approach to curriculum, I believe that computer science curricula should provide students with a rigorous foundation in the reasoning required to design and develop computational systems. Thus, even if there are not many dedicated logic courses, I believe that one can and should inject computational aspects of logic into existing topics, e.g., Boolean logic and satisfiability in courses on discrete mathematics, unique-readability of logical formulas into courses on parsers, first-order logic in courses on databases, etc. Just as important as learning foundations, is developing creativity in students that will be useful in real-world scenarios. To this end I like to motivate and illustrate with examples from robot and mobile-agent control, as well as to provide concrete algorithms, the simplest of which can be simulated by hand.

I have a strong record of undergraduate supervision: I have supervised 7 undergraduates (all resulting in publications), and I am currently supervising an undergraduate on ``Graphical Games''.

% While in Naples, I worked closely with two PhD students, resulting in three publications, and I am supervising s an undergraduate thesis.

Regarding graduate-level teaching and supervision, I have co-taught a 10 hour PhD mini-course on ``Games on Graphs'' at the University of Naples, 
a 1 semester PhD course on ``Logical Definability and Random Graphs'' at Cornell University in 2009, and 
a 5 day advanced course on ``Logic and Computation in Finitely Presentable Infinite Structures'' at ESSLLI in 2006. I have also worked closely with two PhD students of Prof. Murano at the University of Naples, also resulting in publications.

All this demonstrates I would make a good fit for teaching and supervising students in the Masters in Artificial Intelligence, especially on the topics of Logic and Computation, Multi-agent Systems, Intelligent Agents, and Methods in AI Research. Once my Dutch improves (I have some Afrikaans to use as a foundation) I am also keen to contribute to the undergraduate teaching programme.

\textbf{Research}
The quality of my research can be roughly gauged from the venues in which I publish. These include 5 papers in LICS, 4 in IJCAI, 4 in AAMAS, 1 in KR, 1 in CAV, 1 in IJCAR (all of these are ranked A* by the CORE ranking). The researchers in the Department of Information and Computing Science whose interests are closest to mine are 
John-Jules Ch. Meyer (autonomous-, intelligent-, and multi-agent systems), Mehdi Dastani (verification of normative systems), and Henry Prakken (formal aspects of argumentation). My work on verification of parameterised mobile-systems is based on Courcelle's Theorem, which also forms the basis of much work of Hans Bodlaender (algorithms and complexity). 
% My work on verification of parameterised systems is close in spirit, but not in techniques, to the work of Gerard Barkema (complex systems). 

% Besides my ``bread-and-butter'' work, which consists of formal and logical methods for understanding multi-agent systems, I am also pursuing more speculative questions such as ``What is synthesis and how should it be formalised?" (an active topic of debate, and work in progress with Giuseppe De Giacomo).


My ``bread-and-butter'' work consists of developing and applying formal and logical methods. As indicated, for the past few years I have been motivated by applications in AI. One notable example of this convergence is my recent IJCAI17 work with Blai Bonet, Giuseppe De Giacomo, and Hector Geffner which is at the intersection of formal methods and automated planning in AI. I am also pursuing more speculative questions such as ``What is synthesis and how should it be formalised?" (an active topic of investigation with Giuseppe De Giacomo).

 \textbf{Vision for future research}
My research trajectory is to help bridge two communities, the formal-methods community and the AI community. To that end, I plan to continue my integration in the AI community and bring formal and logical methods to bear on questions of importance to computer scientists and society, e.g., ''explainable AI``. Concretely, I plan to organise future editions of the ``Formal Methods in Artificial Intelligence'' workshop, and to expand my work to cover other central aspects of AI such as formal methods for Neural Networks.

\textbf{Names and emails of two references}
Giuseppe De Giacomo (degiacomo@dis.uniroma1.it) and Alessio Lomuscio (a.lomuscio@imperial.ac.uk).
% or any of the names listed in my CV.
% Finally, I will remark that there is a non-trivial overlap between formal-methods and AI, both on the level of
% problems and gross techniques. For instance, ``synthesis'' in formal-methods is called ``planning`` 
% in AI; many influential languages for modeling and reasoning in all three fields are 
% explicitly based on mathematical logic (e.g., situation calculus, alternating-time temporal
% logic). That said, there are important differences, e.g., synthesis usually involves exact algorithms on explicit game-graphs while planning typically involves 
% heuristic algorithms on symbolically presented arenas. 
% These communities have a I am currently working to bridge communities, not just technical fields.
% 
% In particular, I have a number of ongoing collaborations that bring logical foundations to bear on a 
% broad variety of issues and problems in 
% \MAS and \AI, including strategic reasoning for data-aware systems, automatic decomposition of 
% business processes into human-understandable representations, and foundations of synthesis 
% (including synthesis under assumptions, rational synthesis, strategic-epistemic logics). 
% 


\makeletterclosing


\end{document}



I have recently contributed a number of works that import and adapt methodologies and techniques from \FM to
\MAS/\AI [2,3,4,8,10,14,15]. I stress that besides novel technical content, 
my key contributions were to identify which models and techniques from \FM, 
amongst many possibilities, are suitable for \MAS/\AI. 




% I am applying for posn X seen at Y.
% 
% Summary of stations of career. Leading project + topic.
% 
% Summary of main research interests. 
% 
% Organisation. How am integrated into community.
% 
% Future research interests/possible topics/possible collaborations at university.
% 
% Teaching experience: what have taught, number of students, supervision.




% My main tools come from mathematical logic and related fields such as automata theory. 
% The \emph{power of logic} in computer science is that it provides a foundation for designing and formally reasoning about 
% computational systems. The \emph{importance} of this is that as computational systems take more responsibilities in the world 
% we should train developers of these systems to build safe, secure, and robust systems. Consequently, 
% 
% 
% 
% 
% My research is closest to the 
% 
% 
%     
%     Teach in Msc on AI: Methods in AI research, MAS, Intelligent Agents (BDI), Logic and Computation
%     People:  	
%     
% I am writing to apply for 
% ntent
% • State why you are writing and for what position you are applying.
% • Demonstrate energy and enthusiasm for the position.
% • Highlight or expand on key information from your resume, but do not simply repeat
% what is listed.
% • Actively sell your unique qualities and tell the reader why he or she should choose you.
% • Target your skills, interests and experience to the needs of the organization.
% • Show you have done your homework; emphasize why you want to work for that
% particular organization.
% • Encourage the reader to take a closer look at your resume

% My research is in Logic has been called the ``calculus of computer science'' since it provides tools and techniques for formally modeling and reasoning about various discrete systems. My work exploits the power and elegance of logic applied to various aspects multi-agent systems, including temporal, strategic and epistemic properties.
% 
invigorate and enrich the scope of expertise of our Department and can enhance its involvement in interdisciplinary research projects within and outside the university.

demonstrable motivation to teach
The preferred candidate has teaching experience, and is actively interested in improving her or his teaching, the courses and the teaching programme. 


    PhD in Computer Science, Information Science or another relevant discipline;
    track record of international publications in leading conferences and journals;
    experience with or good prospects for acquiring external research funds;
    vision on future research directions in own area of expertise;
    experience with or readiness to supervise PhD projects;
    active role in international scientific communities.

    
    enthusiasm for teaching and student supervision;
    ability to teach in departmental BSc and MSc programmes;
    vision on teaching and your own contribution to teaching.


    play an active and cooperative role in the Department and the University;
    willingness to organize scientific events, such as research seminars or teaching seminars;
    willingness to partake in departmental committees.

    

