\documentclass[10pt]{article}


%% PACKAGES %%
\usepackage{latexsym}
\usepackage{amsmath}
\usepackage{amssymb}
%\usepackage{amsthm} %


\usepackage{color}
\usepackage{graphicx}
\usepackage{tikz,pgf}
  \usetikzlibrary{automata,positioning,matrix,calc,petri,arrows}

%% ENVIRONMENTS %%
%\theoremstyle{plain}
%\newtheorem{theorem}{Theorem}
%\newtheorem{lemma}{Lemma}
%\newtheorem{fact}{Fact}
%\newtheorem{example}{Example}
%\newtheorem{definition}{Definition}
%%\newtheorem{corollary}{Corollary}
%\newtheorem{proposition}{Proposition}
%%\newtheorem*{proof*}{Proof}

%% COMMENTS %%

\newcommand\note[1]{{\color{red}{#1}}}
\newcommand{\todo}[1]{{{\color{blue} #1}}}
%\renewcommand{\todo}[1]{}

%% LATEX SHORTCUTS %%
% cause problems with aamas style file
%\def\it{\begin{itemize} }
%\def\-{\item[-] }
%\def\ti{\end{itemize} }
%\def\en{\begin{enumerate} }
%\def\ne{\end{enumerate} }

%% COMPLEXITY CLASSES %%

\def\UPTime{\textsc{up}\xspace}
\def\CoUPTime{\textsc{coup}\xspace}


\def\exptime{\textsc{exptime}\xspace}
\def\exptimeC{\exptime-complete}

\def\pspace{\textsc{pspace}\xspace}
\def\pspaceC{\pspace-complete}

\def\logspace{\textsc{logspace}\xspace}
\def\nlogspace{\textsc{nlogspace}\xspace}

\def\ptime{\textsc{ptime}}
\def\np{\textsc{np}}



%% LOGIC %%
\def\fol{\mathsf{FOL}}
\def\SL{\textsf{SL}}

\def\msol{\mathsf{MSOL}}
\def\fotc{\mathsf{FOL+TC}}

\def\know{\mathbb{K}}
\def\dknow{\mathbb{D}}
\def\cknow{\mathbb{C}}
\def\eknow{\mathbb{E}}

\def\dualknow{\widetilde{\mathbb{K}}}
\def\dualdknow{\widetilde{\mathbb{D}}}
\def\dualcknow{\widetilde{\mathbb{C}}}
\def\dualeknow{\widetilde{\mathbb{E}}}

\renewcommand\implies{\rightarrow}

%% TEMPORAL LOGIC %%
\newcommand{\sqsq}[1]{\ensuremath{[\negthinspace[#1]\negthinspace]}}

\DeclareMathOperator{\ctlE}{{\mathsf{E}}}
\DeclareMathOperator{\ctlA}{\mathsf{A}}


\newcommand{\atlE}[1][A]{\ensuremath{\langle\!\langle{#1}\rangle\!\rangle}}
\newcommand{\atlA}[1][A]{\ensuremath{[[{#1}]]}}

\DeclareMathOperator{\nextX}{\mathsf{X}}
\DeclareMathOperator{\yesterday}{\mathsf{Y}}
\DeclareMathOperator{\until}{\mathbin{\mathsf{U}}}
\DeclareMathOperator{\weakuntil}{\mathbin{\mathsf{W}}}
\DeclareMathOperator{\since}{\mathbin{\mathsf{S}}}
\DeclareMathOperator{\releases}{\mathbin{\mathsf{R}}}
\DeclareMathOperator{\always}{\mathsf{G}}
\DeclareMathOperator{\hitherto}{\mathsf{H}}
\DeclareMathOperator{\eventually}{\ensuremath{\mathsf{F}}\xspace}
\DeclareMathOperator{\previously}{\mathsf{P}}
\newcommand{\true}{\mathsf{true}}
\newcommand{\false}{\mathsf{false}}


\newcommand{\LTL}{\ensuremath{\mathsf{LTL}}\xspace}
\newcommand{\PLTL}{\textsf{PROMPT-}\LTL}

\newcommand{\CTL}{\ensuremath{\mathsf{CTL}}\xspace}
\newcommand{\CTLS}{\ensuremath{\mathsf{CTL}^*}\xspace}
\newcommand{\PCTLS}{\textsf{PROMPT-}\CTLS}
\newcommand{\PCTL}{\textsf{PROMPT-}\CTL}
\newcommand{\CLTL}{\ensuremath{\textsf{C-}\LTL}\xspace}
\newcommand{\PCLTL}{\ensuremath{\textsf{PROMPT-C}\LTL}\xspace}

\newcommand{\ATL}{\ensuremath{\mathsf{ATL}}\xspace}
\newcommand{\ATLS}{\ensuremath{\mathsf{ATL}^*}\xspace}
\newcommand{\PATLS}{\textsf{PROMPT-}\ATLS}
\newcommand{\PATL}{\textsf{PROMPT-}\ATL}

\newcommand{\KATL}{\ensuremath{\mathsf{KATL}}\xspace}
\newcommand{\KATLS}{\ensuremath{\mathsf{KATL}^*}\xspace}
\newcommand{\PKATLS}{\textsf{PROMPT-}\KATLS}
\newcommand{\PKATL}{\textsf{PROMPT-}\KATL}


\def\red{{red}}
\def\col{{col}}
\def\alt{\, | \,}

%% PROMPT 
\def\kmodels{\models^k}
\def\twokmodels{\models^{2k}}
\DeclareMathOperator{\Fp}{\eventually_\mathsf{P}}
\DeclareMathOperator{\Gp}{\always_\mathsf{P}}
\DeclareMathOperator{\within}{\mathsf{within}}

\newcommand{\AP}{{AP}}
\def\Ag{{Ag}}
\def\Act{{Act}}

%% MATH OPERATIONS %%
\newcommand{\tpl}[1]{\langle {#1} \rangle }
\newcommand{\tup}[1]{\overline{#1}}
\def\proj{\mathsf{proj}}
\newcommand{\defeq}{\ensuremath{\triangleq}}

%% STRUCTURES and STRATEGIES %%
\newcommand{\cgs}{\ensuremath{\mathsf{S}}}
\newcommand{\LTS}{\mathsf{S}}
\newcommand{\Comp}{\mathsf{cmp}}
\newcommand{\Hist}{\mathsf{hist}}
\newcommand{\out}{{out}}

\newcommand{\Paths}{\mathsf{pth}}

\newcommand{\nat}{\mathbb{N}}
\def\int{\mathbb{Z}}
\newcommand{\natzero}{\mathbb{N}_0}

\newcommand{\trans}[3]{#1 \stackrel{\mathsf{#3}}{\rightarrow} #2}


%% HEADINGS ETC %%
\newcommand{\head}[1]{\noindent {\bf #1}.}

%% COUNTER MACHINES %%
\newcommand{\cm}{M}
\newcommand{\CMinc}{\mathsf{inc}}
\newcommand{\CMdec}{\mathsf{dec}}
\newcommand{\CMzero}{\mathsf{ifzero}}
\newcommand{\CMnonzero}{\mathsf{nzero}}
\newcommand{\CMcommit}{\mathsf{end}}


%% CLTL %%
\def\var{{\sf var}}
\def\ovar{{\sf ovar}}
\def\avar{{\sf avar}}
\def\svar{{\sf svar}}
\def\bvar{{\sf bvar}}

\def\MOD{\equiv}

%% PVP %%
\def\PVP{\mathsf{PVP}}



% CV REFS
\def\BMMRV17{[C3]}
\def\DBLPconfatalBelardinelliLMR17{[C2]}
\def\DBLPconfijcaiBelardinelliLMR17{[C1]}
\def\DBLPconfvmcaiAminofJKR14{[C17]}
\def\DBLPconfconcurAminofKRSV14{[C18]}
\def\DBLPconficalpAminofRZS15{[C13]}
\def\DBLPconfcadeAminofR16{[C9]}
\def\AKRSV17{[J1]}
\def\DBLPseriessynthesis2015Bloem{[B1]}
\def\DBLPjournalssigactBloemJKKRVW16{[J6]}
\def\DBLPconfatalRubin15{[C16]}
\def\DBLPconfprimaRubinZMA15{[C12]}
\def\DBLPconfatalAminofMRZ16{[C7]}
% \def\DBLPconfprimaMuranoPR15{[C]}

\def\AR16{[J5]}
\def\traps13{[J7]}
\def\GMRS16IJCAI{[C10]}
\def\BDGRICAPS{[W1]}
\def\BDGR17{[C4]}

\def\GMPRW17{[C5]}


%%% PACKAGES 

\usepackage[margin=3.3cm]{geometry}
\usepackage{framed}
\usepackage{parskip}
% \setlength{\parindent}{10pt}

\usepackage{enumitem}
\setlist[itemize]{parsep=0pt,itemsep=0pt}
\setlist[enumerate]{parsep=0pt,itemsep=0pt}
\renewcommand{\labelitemi}{--}

\setlength{\fboxrule}{1pt}
\newcommand{\BOX}[1]{\noindent\fbox{\parbox{\textwidth}{#1}}}




%% MACROS 


\def\TITLE{Synthesis of Trustworthy Behaviour of Artificial Agents (SYBA)} 
\title{\TITLE}
\author{Sasha Rubin (Candidate XXXX)}

\def\TSE{Temporal-Strategic-Epistemic\xspace}
\newcommand\aside[1]{\textcolor{red}{#1}}
\newcommand\pubact{\textsf{PUBACT}\xspace}
\newcommand\pomdp{\textsf{POMDP}}

\begin{document}


\section{Summary}



% EXEC SUMMARY
% • What is the nature of the challenge?
% • Which part is the focus of the project and why is it important to address it?
% • How are others trying to address it?
% • How do you propose to address it and why is your approach different, better and more exciting/innovative
% • Why you /your Team are particularly suited?


\emph{Nature of the challenge.} Systems built on the insights of Artificial Intelligence are increasingly deployed in the world as \emph{agents}, 
e.g., software agents negotiating on our behalf on the internet, driverless cars,  
robots exploring new and dangerous environments, bots playing games with humans. There is an obvious need for humans to 
be able to \emph{trust} the decisions made by artificial agents, the need for {meaningful interactions} between humans and agents, 
and the need for {transparent} agents~\cite{ACMStatement17,Neumann17}. This grand challenge will require integrating research from a variety of fields 
besides computer science, including psychology and economics.
	
One contribution that computer scientists must make towards this challenge is to be able to model, control and predict the behaviour of agents. 
This is made 
all the more complicated since: 1) agents are often deployed with \emph{other} agents leading to \emph{multi-agent systems}, 2) agent behaviour extends into the future, leading to the need for reasoning about \emph{time}, 3)  agents are often ``self-interested'', leading to the need to reason about \emph{strategies}, 4)
agents may have uncertainty about the state, or even the structure, of other agents and the environment, leading to the need to reason about \emph{knowledge}.
 

Broadly speaking, there are two existing approaches to building agents: model-based and function-based. In the model-based approach one \emph{represents and reasons} about the domain of interest, e.g., state-based models of dynamical systems in automated planning (the main tools are logic and probability). 
In the function-based approach one \emph{fits functions to data} (an important tool today is neural networks).  
The long-term goal for scientists is to reconcile these two approaches, i.e., a) to clearly understand the scope limitations of each approach, and b) to integrate the approaches~\cite{Darwiche17}.


% As Judea Pearl writes~\cite{}:``There is only one way a thinking entity (computer or human)
% can work out what would happen in multiple scenarios, including some that it has never
% experienced before. It must possess, consult, and manipulate a mental causal model of
% that reality.''

% (is required to model objects of interest Logic-based techniques are a standard approach to modeling, building and analysing computational systems. Indeed, simply {formalising} the reasoning tasks 
% unambiguously requires a formal language. Other approaches include 

\emph{Goals.} My long-term goal is to contribute to bridging formal-methods and artificial intelligence. 

In the next 5-15 years I aim to expand our understanding of the limitations and techniques of the model-based approach to building trustworthy agents. This will involve tackling foundational mathematical problems, integrating various approaches (i.e., logic, probability, connectionist), as well as tackling issues in Knowledge Representation (e.g., what language should be used to express domains and specifications). 

In the next 5 years I have three specific goals: 1) discover new classes of systems, with applications in mind, for which synthesis and analysis is computationally tractable, 
2) develop the theory of reasoning about optimal strategies and socially optimal equilibria, and 3) establish scalable algorithms and tools for solving these computational problems. This rests on the following:
\BOX{Hypothesis: The model-based approach is the best approach we have for understanding the behaviour of systems, and thus for building trustworthy systems.} 


\emph{Candidate.} I have a background in formal methods and am open to ideas coming from other fields, including Artificial Intelligence. In particular, my 
background in mathematical logic and formal methods has enabled me to devise effective conceptual frameworks to address problems in Artificial Intelligence and Multi-agent systems~\cite{DBLPconfatalRubin15,DBLPconfkrAminofMRZ16,DeGiacomoMRS16,DBLPconfatalAminofMMR16,BDGR17,GMPRW17,BDGR17,BLMRIJCAI17,BLMR17,BMMRV17}. 

My standing is reflected in the fact that I serve as PC member of top conferences in Artificial Intelligence and Multi-agent systems (IJCAI 2017, AAAI 2017, AAAI 2018, AAMAS 2018); 
I have chaired one national conference on theoretical computer science (ICTCS 2017, Italy), one international workshop on strategic reasoning (SR 2017), and one international workshop on 
Formal Methods in Artificial Intelligence (FMAI 2017) that attracted leaders in Formal Methods and Artificial Intelligence including Giuseppe De Giacomo, Michael Wooldridge, Michael Fischer, and Hector Geffner; I have served as an external reviewer for the Icelandic Research Fund (IRF 2017). 

\section{Short-term}

The short-term aims are to generate new mathematics, algorithms, and tools for describing, reasoning-about and building trustworthy agents. 

I now describe three challenges that should be met in order to achieve these aims.

\subsection{Challenges}

\BOX{1. We need to discover meaningful \emph{new classes} of multi-agent systems for which synthesis and reasoning is decidable and tractable.}

Not surprisingly, reasoning about the behaviour of multi-agent systems is computationally hard. In fact, it is \emph{undecidable} when it involves agents with 
private knowledge, a fact known since the late 1970s and rediscovered in multiple contexts, i.e., decentralised \pomdp s~\cite{DBLP:journals/mor/BernsteinGIZ02}, multiplayer non-cooperative games of imperfect information~\cite{DBLP:conf/focs/PetersonR79}, distributed synthesis~\cite{DBLP:conf/focs/PnueliR90}. 
The classic approach to ameliorate this 
is to restrict to classes of multi-agent systems in which agents' private knowledge is hierarchical (typically, one assumes some sort of hierarchy on agent observation or information~\cite{DBLP:conf/focs/PetersonR79,DBLP:conf/focs/PnueliR90,DBLP:conf/lics/KupfermanV01, DBLP:conf/atva/BerwangerMB15,BMMRV17}). 
Although {mathematically elegant} and well-explored, the \emph{applicability of such assumptions is not very high} since in almost all meaningful scenarios, agents' private knowledge are not hierarchical.
An orthogonal approach that does not suffer from this long-standing limitation is to limit the way agents communicate/interact. One such assumption is that \emph{agent actions are fully observable}. 
In such a setting, multi-agent epistemic planning is tractable~\cite{DBLP:conf/aips/KominisG15}, synthesis of epistemic extensions of linear-temporal logic is decidable~\cite{vanderMeyden2005}, and 
the model-checking problem against an epistemic extension of strategy logic is decidable \emph{and no harder than the case of agents with no private knowledge}~\cite{BLMRIJCAI17,BLMR17}. Many scenarios already fall into this class, e.g., distributed computing and multi-party computation based on broadcast communication~\cite{DBLP:books/mk/Lynch96, ADGH06}, multi-player games with public play such as poker~\cite{Bowling145}, and e-auctions with public bidding~\cite{EasleyK10}.
 
% One central challenge is to extend these models to capture scenarios, i.e., \todo{???}

% \begin{enumerate}
%  \item various models of collaborative robot exploration in controlled but dynamic environments~\cite{amazon},
%  \item various models of cloud manufacturing~\cite{DBLP:conf/ijcai/FelliSLR17},
%  \item various models of collusion in e-auctions and auction-based mechanisms~\cite{EasleyK10},
%  \item various models of social networks that use broadcast communication, and thus also formalisations of \emph{twitter}~\cite{DeNicola2015,DBLP:journals/jlp/MaggiPST17},
%  \item various models of secure cloud-storage that use data-dispersal~\cite{DBLP:journals/internet/LiQLL16} and secret-sharing protocols~\cite{ADGH06},
%  \item parts of multi-player games in which bidding and play is public, such as poker~\cite{Bowling145}.
% \end{enumerate}

%  \item various epistemic puzzles of interest to computer scientists~\cite{}.
%  such as the  muddy-children problem and the problem of russian cards, %% NOT STRONG POINT FOR PROPOSAL!


%  \todo{another use case that people not in area of agents say ``yes, i like it!'' think big and crazy. then say something meaningful. be bold and concrete. car, hacker attack, something concrete that we could do in principle}


% In contrast, general forms of planning can be used to capture multiple agents, 
% imperfect information, incomplete information, and temporally extended goals.

% I propose to systematically study how to reduce behaviour synthesis to classical planning. This will be done in two steps:
% \begin{enumerate}
%  \item Study how to 
%  reduce behaviour synthesis to general forms of planning.
%  \item Study how to 
%  reduce general forms of planning to classical planning.
% \end{enumerate}
% 
% Both steps will be done using insights from automated synthesis~\cite{Vard96,KuVa97,DeGiacomoFPS10,DeVa15,DeVa16}, generalised planning~\cite{HuG11,DeGiacomoMRS16,BDGR17}, and reductions of planning with LTL-goals to classical planning~\cite{}. Moreover, the practical aspects of such reductions will be done 
% in collaboration with leading planning experts Hector Geffner and Blai Bonet. \aside{ask Hector/Blai}

% I plan to ground the practical considerations developed in the second phase to real application domains such as cognitive robotics, 
% multiplayer card games such as poker, analysis of multi-party computation.
% \aside{these can be done with people at UNSW}

 
%  \item We need to discover new ways of dealing with the state-explosion problem for systems with imperfect-information.
 
\BOX{2. We need to define, analyse, and tackle the problem of reasoning about {optimal strategies} and \emph{socially optimal equilibria}.}
% in systems of agents that also have \emph{quantitative objectives}.}

In order to have evidence that one agent's behaviour is better or worse than another, or whether a collection of agents are acting in the good of society, 
we need to be able to measure the quality of agent strategies against each other.
This, in turn, would be facilitated by endowing agent objectives with a quantitative component. Although most work in verification deals with qualitative objectives, there has been a recent focus on verification of quantitative models of programs~\cite{DBLP:journals/ife/Henzinger13,ABK16}. However, this has yet to be systematically generalised to complex reasoning for multi-agent systems. 

Building on more classic work~\cite{DBLP:journals/jacm/AlurHK02,MogaveroMPV14}, I have recently introduced expressive logics that can be used to reason about socially optimal equilibria in cases agents have qualitative objectives~\cite{DBLP:conf/atal/AminofMMR16,BLMRIJCAI17,BMMRV17}. For instance, in games that model security, robustness to deviations from the behaviour of attackers is a critical issue~\cite{}. This, together with recent insights from quantitative verification \cite{DBLP:conf/concur/UmmelsW11,DBLP:journals/ife/Henzinger13,DBLP:journals/tocl/MarchioniW15,ABK16}, lays the foundation for \emph{designing useful logics and measures of strategy-quality} for reasoning about socially optimal equilibria. 

% 
% a measure of quality of agent strategy.
%  Most existing work in Automated Planning and verification deals with agents with qualitative objectives.  This, in turn, likely requires that objectives be formalised with a quantitative component. However, most work in Automated Planning focuses on finding optimal strategies or optimal plans for qualitative (i.e., reachability) objectives~\cite{GeffnerBo13,Penna15,TorralbaAKE17}. Some work deals with multiple agents with possibly different but overlapping objectives, which revolves around finding stable solutions (e.g., Nash equilibria)~\cite{DBLP:journals/amai/KupfermanPV16,DBLP:journals/ai/GutierrezHW17}. Techniques for synthesising and reasoning about equilibria are all the more needed when agents have a mix of qualitative and quantitative objectives. 
 
 
 
%  \item We need to discover new parameters based on the recently introduced ``width'', that measure the ``complexity'' of the synthesis problem, and prove bounds on the computational complexity wrt these parameters, of synthesis problem.


\BOX{3. We need to establish \emph{scalable algorithms and tools} for reasoning about multi-agent systems.}

To automatically reason about multi-agent systems, including reasoning about social equilibria, we need 
scalable algorithms and tools. Since the worst-case complexity of such reasoning is typically very high, 
we need tools that can deal with large but ``easy'' cases. This grand challenge is being met by a number 
of branches of computer science, notably the Automated Planning community in Artificial Intelligence.

Automated Planning is a \emph{form of synthesis} that is central to the development of agents. 
It is a branch of Artificial Intelligence that addresses the problem of generating a course of actions to achieve
a specified goal, given a description of the domain of interest and its initial state. 
The Automated Planning community has developed a ``science of search'', based on heuristic-search and symbolic methods, 
that efficiently plans for most problems of practical interest~\cite{GeffnerBo13,DBLP:conf/aaai/LipovetzkyG17}. 
The most successful of this technology  is for ``classical planning'', i.e., single agent, deterministic environment, with perfect information, 
and simple reachability goals, and ``fully observable non-deterministic planning'' (which amounts to the case of one agent in an adversarial environment).

Previous work has reduced planning with temporal goals or epistemic goals to classical and fully-observable 
nondeterministic planning~\cite{DBLP:conf/aaai/BaierM06,TorresB15,DBLP:conf/aips/KominisG15,Camacho17}. This lays the foundation for refining and extending the translations to handle full \TSE reasoning for multi-agent systems.



\subsection{Work plan}  

The challenges will be met using methods and insights from Logic and Formal Methods (including program synthesis), and Game Theory (including its development in multi-agent systems). 


\BOX{1a. In order to discover richer decidable classes, I propose to \emph{generalise} systems in which all actions are fully observable.}

The class of systems in which all actions are fully observable holds promise since a) reasoning in it is no harder than the non-epistemic case, b) it can be used to formalise many scenarios. I now outline the first directions I will pursue in order to expand this theory to encompass even more scenarios:
\begin{enumerate}
 \item Incorporate stochastic initial states, which is \emph{widely applicable}. Indeed, not only do finite horizon stochastic systems 
 fall into this setting~\cite{DBLP:conf/uai/LittmanDK95}, but so do probabilistic multi-agent systems, called decentralised partially observable Markov decision processes~\cite{DBLP:series/sbis/OliehoekA16}, which are a framework for modeling uncertainty with respect to outcomes, environmental information and communication.  That is, this extension will addresses the problem of ensuring \emph{agents behave well in unknown environments}.
 
 \item Incorporate symmetric and asymmetric encryption, which is applicable to online \emph{privacy and security}. Indeed, private-keys can be stored in an agent's private state, and thus private-key encryption can already be simulated by fully observable actions. Furthermore, public-key encryption consists of public keys that can be widely disseminated, and thus encrypting with a public key can be modeled as a public action. 
 
 \item Limiting the number of non-public actions, which is applicable to design and analysis of \emph{collusion analysis} in e-auctions~\cite{EasleyK10}.
%  is a type of sealed-bid auction, in which each bidder submits a written bid without knowing the bids of the other bidders. The highest bid wins but the price paid is the second-highest bid. Vickrey-Clarke-Grove (VCG) auctions are generalisations of Vickrey's
% auctions to multiple items. It is known that Vickrey-Clarke-Grove auctions provide bidders with an incentive to bid their true value. It is also known that these auctions are vulnerable to \emph{collusion}: if all bidders reveal their true values to each other, using a \emph{limited number of non-public actions}, they can lower some or all of these values, while preserving who wins the auction. Thus, collusion analysis gives the auctioneer, and thus the market, confidence that bidders cannot game the system.

 \item Tuning the amount of observability of actions. Indeed, although systems with hidden actions are undecidable and fully observable actions are decidable, there is likely a measure of ``action observability'' that can be tuned so that one can incorporate systems in which certain actions are partially observable (but not completely hidden). Whatever this measure will look like, the result will be a \emph{deeper understanding} of the borders between decidability and undecidability for various systems.
\end{enumerate}

Here are  some scenarios that could be handled by such extensions:
\begin{itemize}
 \item various models of collaborative robot exploration in controlled but dynamic environments~\cite{amazon,DBLP:journals/trob/Kress-GazitFP09},
 \item various models of cloud manufacturing~\cite{DBLP:conf/ijcai/FelliSLR17},
 \item various models of collusion in e-auctions and auction-based mechanisms~\cite{EasleyK10},
 \item various models of social networks that use broadcast communication, and thus also formalisations of \emph{twitter}~\cite{DeNicola2015,DBLP:journals/jlp/MaggiPST17},
 \item various models of secure cloud-storage that use data-dispersal~\cite{DBLP:journals/internet/LiQLL16} and secret-sharing protocols~\cite{ADGH06},
 \item various models of multi-player games in which bidding and play is public, such as poker~\cite{Bowling145}.
\end{itemize}

\BOX{1b. In order to discover tractable classes of agents, I propose to \emph{restrict} to sub-systems of those in 1a.}

The complexity of reasoning in multi-agent systems identified in 1a is expected to be high. To achieve better computational complexity I propose to restrict them to sub-systems, 
while still maintaining the features in 1a that allow one to model systems from a wide variety of fields (e.g., that agent's observations need not be hierarchical). 
In particular, I will start by restricting to classes in which:
\begin{enumerate}
 \item the set of initial states is homogenous~\cite{DBLP:journals/tocl/LomuscioMR00}. This is applicable to situations in which agents are initially ignorant of each others local states;
 \item the size of the epistemic states is bounded, which is applicable to situations in which each agent has full observation except of its \emph{own} finite state;
 \item the strategies considered do not depend on the full history, but on a bounded summary of the history. Besides lowering the complexity, this assumption reflects the assumption of \emph{bounded rationality}~\cite{simon1982models}.
 \end{enumerate}
 
 

%  \item explore the middle-ground between belief-space (which is exponentially large but accurate) and observation-space (which is linear but coarse) using trajectory constraints (which we recently pioneered~\cite{}), in order to find new ways of dealing with the state-explosion problem for systems of agents with imperfect-information.



\BOX{2. In order to reason about socially optimal equilibria, I propose to {enrich} the models {and} specification 
languages with costs/rewards and analyse these with measures of \emph{strategy quality}.}

Agents are typically ``self-interested'', and thus they may not act in a way that is socially optimal. Moreover, it is often not possible to ascribe agent behaviour as simply being good or bad. Thus, I will explore measures of strategy quality and algorithms for synthesising socially optimal strategies. Although many game-theoretic solution concepts, such as Nash equilibria, can be expressed in recently introduced strategic logics~\cite{MogaveroMPV14}, and  their epistemic extensions~\cite{BMMRV17,BLMR17}, these logics can only express qualitative agent objectives. 
Thus, I will define and explore logics that can reason about quality of agent behaviour. In particular I will extend and evaluate state-of-the-art proposals for measuring quality of strategies to 
reasoning about multi-agent systems, i.e.: the logic $\LTL[\mathcal{F}]$, and extension of \LTL with a set of quality operators $\mathcal{F}$~\cite{ABK16}, that was designed to reason about the quality of programs and can be used to reason about the \emph{quality of agent behaviour};  the logic $\LTL_f$ with costs~\cite{DBLP:journals/corr/BrafmanGP17}, that allows one to reason about 
non-Markovian objectives; logics that combine qualitative behaviour (expressed for instance in $\LTL$ or $\LTL_f$) and quantitative, expressed for instance as long-term average of the cost of some resource~\cite{GMPRW17} or as the total cost of some resource~\cite{DBLP:journals/acta/BrihayeGHM17}.
% \item min-cost reachability games~\cite{DBLP:journals/acta/BrihayeGHM17} that can be used to compute a measure of strategy-quality in fully-observable non-deterministic planning;
% % by a preference over a mix of qualitative  and quantitative. Recent work by the candidate~\cite{GMPRW17} paves the way for reasoning about multi-agent systems in which agent objectives are a mix between the % 
 
 \BOX{3. In order to establish scalable tools and algorithms, I propose to \emph{translate} reasoning tasks to Automated Planning.}

I will extend and refine the translations that handle temporal goals and epistemic goals~\cite{DBLP:conf/aaai/BaierM06,TorresB15,Camacho17} to full \TSE reasoning for the multi-agent setting.
I will do this by leveraging the automata-theoretic approach to 
model-checking of strategic epistemic logics~\cite{BLMR17,BMMRV17}, as well as search through strategy-space~\cite{DBLP:conf/aips/BonetPG09}. 
One parallel direction to meet this objective is to explore generalisations of specification formalisms over finite traces~\cite{DeVa13,DeVa15,DeVa16}, adopted in automated planning~\cite{DBLP:conf/aaai/BaierM06,TorresB15,DBLP:conf/ijcai/AguasCJ16,Camacho17}.
Indeed, specifications on finite traces allow one to avoid notorious difficulties of infinite-traces~\cite{Vardi-symbolic17}, namely complementation of B\"uchi-automata~\cite{TsaiFVT14}. In particular, I will define and study ``epistemic strategy logic over finite traces'', and extend the mentioned translations to this logic. 

% This objective will be achieved with \emph{tight co-ordination} with four world-experts in automated planning, namely Hector Geffner (Pompeu Fabra University), Blai Bonet (Sim\'on Bol\'ivar University), Sebastian Sardina (Associate Professor at RMIT University), and Nir Lipovetzky (Lecturer at the University of Melbourne). The budget includes one international visit to Hector Geffner in Spain, as well as multiple international visits to Europe where the candidate will meet both H. Geffner and B. Bonet, as well as yearly national visits between the candidate and his students and S. Sardina and N. Lipovetzky in Melbourne. One Msc student student will be hired for 2 years to help with refining and implementing the translations. 

\bibliographystyle{abbrv}
\bibliography{researchplan,FWF,References,/home/sr/svn/research/CV/data/rubin-core}


\end{document}


\paragraph{Project Innovation}
% Make sure you confront the innovation, not discuss.
% The way in which you believe the project is innovative (how/why) rather than describing the innovation itself (what). The innovation may be conceptual, technological, or methodological – use subheadings of these aspects of the innovation as flags for assessors. Novel concepts and/or original thought relevant. E.g. This project is innovative as for the first time it will…

The scientific innovation of the project is demonstrated in three ways.

\BOX{1. The project will develop a new mathematical and algorithmic theory for the design and analysis of meaningful multi-agent systems.}

\BOX{2. The project will advance the study of automatically finding socially optimal equilibria in which agents have a mix of qualitative and quantitative objectives.}

\BOX{3. The project will leverage the successful theory and technology of Automated Planning for the design and analysis of multi-agent systems.}  

Thus, the project has conceptual innovation by supplying viable frameworks for computer scientists and engineers to think about the behaviour of the agents they build, 
as well as technological innovation in leveraging Automated Planning to automated reasoning about artificial agents. 

% 
% The applications in mind include: models of collaborative robot exploration in controlled but dynamic environments~\cite{amazon};  models of cloud manufacturing~\cite{DBLP:conf/ijcai/FelliSLR17};  models of collusion in e-auctions and auction-based mechanisms~\cite{EasleyK10};  models of social networks that use broadcast communication, and thus also formalisations of \emph{twitter}~\cite{DeNicola2015,DBLP:journals/jlp/MaggiPST17}; models of multi-player games in which bidding and play is public, such as poker~\cite{Bowling145}; models of secure cloud-storage that use data-dispersal~\cite{DBLP:journals/internet/LiQLL16} and secret-sharing protocols~\cite{ADGH06};  \todo{models of security}.
% 

% His individual expertise is complemented by his close integration with world experts in logic and automata-based verification and synthesis (M. Y. Vardi), multi-agent systems (M. Wooldridge), knowledge representation (G. De Giacomo), and automated planning (H. Geffner).





\paragraph{Overall Aim} The aim of this project is to develop the mathematical foundations and computational techniques for building and analysing 
trustworthy artificial agents, by leveraging the insights from recent results developed by the candidate on automated reasoning for single and multi-agent systems.


% We need to establish meaningful classes of agents that are amenable to automatic computational analysis.
\paragraph{Specific Aims}

We state and justify $3$ specific aims.



% For Candidates applying for Future Fellowship Level 1:
%  Describe the Future Fellowship Candidate’s research opportunity and performance
% evidence (ROPE).
%  Provide evidence that the Future Fellowship Candidate has the capacity and
% leadership to undertake the proposed research
%  Provide evidence that the Future Fellowship Candidate has a record of high quality
% Research Outputs appropriate to the discipline/s
%  Provide evidence the Future Fellowship Candidate’s research training, mentoring
% and supervision
%  Provide evidence of the Future Fellowship Candidate’s national research standing.

\paragraph{Leadership}

% Reflect on your suitability to undertake the Future Fellowship in relation to the level of fellowship you are requesting.
% In reflecting your leadership present evidence with reference to the previous projects that you have led or on which you have played a significant role. 
% These do not have to be ARC projects only and, if appropriate, can include projects with industry. 
% In reflecting your leadership you reflect your research leadership as well as your project management abilities. As a Future Fellow you are wholly responsible for the project management over the entire 4yrs of funding – demonstrating these skills will also be important.

I have previously held two individual fellowships: a 3-year New Zealand Science and Technology Postdoctoral Fellowship (NZ\$ 224532), and a 2-year Marie-Curie Postdoctoral Fellowship confunded by the National Institute of Higher Mathematics (Eu 107000). I was project co-ordinator for the \emph{Handbook of Model Checking}, edited by Ed Clarke et. al., and published by Springer (Dec 2017).

\paragraph{High quality research outputs}


% The ROPE section offers you the opportunity to list publications so this section is not about restating this information.
% Provide a narrative of the resonance of your outputs – detail how they have been received, the standard of journal or other outlet or forum you regularly use for dissemination.
% If you have received positive comments and endorsements from reviewers or key people in the field you can quote them here - a few words only.
% Discuss how your outputs have been picked up by other researchers within or beyond your discipline.

In theoretical computer science and Artificial Intelligence, top conferences are highly prestigious venues for dissemination. I regularly publish in conferences of the highest level, e.g., I published 5 CORE A* papers in 2017 and 5 CORE A* papers in 2016. According to google scholar: I have 785 citations (391 since 2012), my H-index is 15 (12 since 2012), and my Phd Thesis has 72 citations. In my field it is expected that authors are ordered alphabetically. That said, my contribution to my published papers always meets and often far exceeds an equal division of labour. 
% \todo{More detailed contributions for each paper are given in the References section of the accompanying CV.}


% \todo{add endorsements from key people in field}

% I have deep and extensive knowledge in logics for temporal, strategic and epistemic reasoning, automata-theory and synthesis. 
% I have contributed foundational work on synthesis and graph-games~\AR16,\traps13 as well as 
% on the connections between synthesis and general forms of planning~\GMRS16IJCAI,\BDGRICAPS,\BDGR17. 
% 
% A first step towards synthesis is usually verification, and I have contributed deep work on verification of multi-agent systems~\DBLPconfvmcaiAminofJKR14,\DBLPconfconcurAminofKRSV14,\DBLPconficalpAminofRZS15,\DBLPconfcadeAminofR16,\AKRSV17, including a book on the topic~\DBLPseriessynthesis2015Bloem,\DBLPjournalssigactBloemJKKRVW16, and with a focus on verification of parameterised systems \DBLPconfatalRubin15,\DBLPconfprimaRubinZMA15,\DBLPconfatalAminofMRZ16.


\paragraph{Research training, mentoring and supervision}

% Detail the numbers of HDR students you have supervised and their successes.
% If any students have established careers in academia or industry of note mention these.
% Discuss any innovative or successful supervision/mentoring strategies you have developed and/or used.
% Indicate instances where your mentorship has been sought out and why.

I worked closely with a PhD student of Erich Gradel's (Tobias Ganzow) and solved a 12-year open problem in finite model-theory~\cite{DBLP:conf/stacs/GanzowR08}.
I have mentored 1 Msc Internship (2017) on logics for epistemic reasoning, 1 Undergraduate thesis (2017) on graphical games, and 7 undergraduate 
students doing research on models of computation extending finite automata, solving games on graphs, and edit-distance between languages (2012, 2009). 
\todo{detail relationship with this project}

\paragraph{National and International standing}

% Detail invitations for collaborations, expert opinion and engagement.
% Reflect on growing/established reputation (dependent on level) within both academia and industry, government policy etc.
% If you have received awards prizes these will be noted in D6 however explain how these reflect your reputational value here.


I serve as PC member of top conferences in Artificial Intelligence and Multi-agent systems (IJCAI 2017, AAAI 2017, AAAI 2018, AAMAS 2018). I have chaired one national conference on theoretical computer science (ICTCS 2017, Italy) and one international workshop on strategic reasoning (SR 2017). I conceived and organised a Workshop on Formal Methods in Artificial Intelligence (FMAI 2017). 
I have served as an external reviewer for the Icelandic Research Fund (IRF 2017). I have been invited to talk at various universities, including UNSW (Australia), IMT Lucca (Italy), Sapienza University of Rome (Italy), Universit\'e Paris-Diderot (France), and Oxford University (UK).

% I already have close connections with international experts with expertise and interest in the topic of this project, all of whom have agreed to collaborate on the project: Giuseppe De Giacomo (knowledge representation, artificial intelligence, verification, synthesis), Hector Geffner and Blai Bonet (Planning), Moshe Vardi and Aniello Murano (logics for strategic reasoning, automata-theory for synthesis and verification), and Michael Wooldridge and Alessio Lomuscio (multi-agent systems).



\section{PROSPOSED PROJECT QUALITY AND INNOVATION}

Every society that has embraced the digital world faces the issue of whether, or to what extent, it can trust the behaviour 
of artificial agents. The challenge of building trustworthy artificial agents cannot be met without having some formal guarantees on their 
behaviour. There are two fundamental parts to this problem: automatically synthesise agent behaviour, or part of their behaviour, from unambigious declarative 
specifications; analyse behaviour of built or existing agents. This project will advance the state of the art of the mathematical foundations and computational techniques 
for building and analysing trustworthy agents from \TSE specifications. In particular, it will provide extend the use of logical methods to the analysis of multi-agent systems coming 
from a variety of fields, including high-level robot control, trustworthy social-media bots, collusion-free e-auctions, and safe and secure cloud storage. 
Safer and securer interactions with artificial agents are clearly in the interests of society. 

% You may choose to start from a ‘big picture’ perspective (i.e. global and national importance and potential benefits in the context of the broader discipline/area) and then focus on the specific outcomes of your project.
% Be realistic – assessors have complained of exaggerated or grandiose claims. Try to come up with several different dimensions. These can include the following areas, in whatever order is best for your case:
% Training – students and post-docs gain specific knowledge and important generic skills, potential workforce capability for Australia in new smart/innovation economy).
% Social or cultural impact/benefit particularly if beyond the immediate discipline.
% Creating intellectual linkages and leadership – national, regional and international.
% Direct economic benefits (e.g. wealth creation through commercialisation)
% Contribution to cutting-edge national and/or international research knowledge creation.
% 

%  Explain how the research addresses a significant problem.
%  Outline the conceptual/theoretical framework, and demonstrate that these are
% adequately developed, well integrated, innovative and original.
%  Explain how the aims, concepts, methods and results advance knowledge.
%  Describe how the design and methods are appropriate for the proposed research.
%  Describe how the proposed research may result in maximising economic,
% environmental, social, and/or cultural benefits to Australia. This statement should
% align with the Impact Statement.
%  If the research has been nominated as focussing on a topic or outcome that falls
% within one of the Science and Research Priorities, explain how it addresses one or
% more of the associated Practical Research Challenges (as selected in question B1 of
% this Proposal form).
%  Describe how the proposed Project involves interdisciplinary research, if appropriate.
%  Describe how the proposed Project will push the boundaries of research and open up
% new research opportunities.
%  Explain how the proposed Project will contribute to public policy formulation and
% debate.

\paragraph{Project Quality}
% In this section show HOW you will undertake the project. Not what, but HOW? 
% It needs energy and therefore must be future tense and preferably active voice throughout.  
% Check that the start of each paragraph is strong and convincing (you know exactly what you will do)
% Break up the text as appropriate with relevant figures and diagrams, bullet points, selective use of bold/italics to highlight key statements.
% Provide a brief overview/introduction of the different aspects of your project – which may be theoretical, experimental, numerical/computational, quantitative and qualitative.
% Write for an expert audience - you must include sufficient technical detail to demonstrate your credibility and suitability to successfully achieve your aims. Make sure you explain how the aims, concepts, methods and results advance knowledge.
% If there are risks to the research acknowledge them and detail your contingency plans.
% In detailing what you will do remember this is a fellowship – a 1 person project – ‘I’ should figure predominantly in your articulation of activities.



% The project will train one Phd student and two Msc students by research in cutting edge research in methods for building and analysing artificial agents. 
% The project will further integrate the Administering Organisation with world-leaders in a number of fields of computer science, namely multi-agent systems (Michael Wooldridge and Alessio Lomuscio), knowledge representation (Giuseppe De Giacomo), logic and automata theory (Moshe Y. Vardi), and automated planning (Hector Geffner). 




% This will be done by building on recent breakthroughs in reasoning about multi-agent systems~\cite{}, 
% 
% In particular, I propose to systematically study how to reduce \TSE reasoning 
% of multi-agent systems to these planning problems (e.g., I propose a careful study of the effects that structure of the 
% agent goals have on the resulting translations). This will be done using insights from distributed synthesis~\cite{Vard96,KuVa97,DeGiacomoFPS10,DeVa15,DeVa16} 
% and generalised planning~\cite{HuG11,DeGiacomoMRS16,BDGR17}. 
% 



% 
% \subsection{5. FEASIBILITY AND STRATEGIC ALIGNMENT}
% 
% %  Describe the extent to which the Future Fellowship Candidate aligns with and/or
% % complements the core or developing research strengths and staffing profile of the
% % Administering Organisation.
% %  Demonstrate that the necessary facilities are available to conduct the proposed
% % research.
% %  Outline what resources will be provided by the Administering Organisation to support
% % the Future Fellowship Candidate during her/his Future Fellowship.
% %  At the end of the Future Fellowship, explain what capacity exists at the Administering
% % Organisation to transition the Candidate to a continuing position.
% 
% % Reflect on the qualities at UNSW that make it the ideal location for your project
% % Include information in access to essential equipment and resources.
% % Discuss the research cohort that you will have access to and with who you will collaborate.
% % Discuss any supports on offer from your School/Faculty.
% 
% \todo{Morri: 1. What qualities at UNSW make it an ideal location for the project? 2. What support is on offer from school/faculty?}
% 
% UNSW is an \emph{ideal} location for the project. Indeed, there are clear connections between the topic of this project (behaviour synthesis for agents) and the work being done, both theoretical and practical, at the School of Computer Science and Engineering. Notably:
% \begin{enumerate}
% \item The theory of Reasoning about Knowledge, as applied to distributed computing~\cite{FHMV95}, can be used to formally specify, verify and synthesise artificial agents 
% that act with incomplete and imperfect information. \emph{Van der Meyden} studies synthesis of epistemic protocol specifications~\cite{DBLP:conf/concur/MeydenV98,DBLP:journals/corr/HuangM16} and symbolic implementations of model-checkers for epistemic strategic logics~\cite{DBLP:conf/aaai/HuangM14}.
% 
%  \item General game playing, see \emph{Thielscher~\cite{GGP}}, is a framework in which programs learn to play games given just a description of the rules of the game. Although related to synthesis, it emphasises an online approach in which the solver is given bounded time to suggest its next move (in this sense it is related to online planning, e.g.,~\cite{GeffnerBo13}). This is in contrast to the classic synthesis approaches in Formal Methods which are offline and generate a policy before execution. 
%  
%  \item Architectures used in robotics to capture mental states  correspond to a first-person view of agents~\cite{reiter2001knowledge}. Recent work at UNSW aims to formalise an architecture so that formal-methods can be applied, see~\emph{Rajaratnam, Hengst, Pagnucco, Sammut, Thielscher~\cite{Rajaratnam2016}}.
%  
%  \item The aim in information flow security is to design observable program behaviour that does not reveal, to an adversary, secret information about its' state. Formal methods for the design and verification of such systems is studied by \emph{Morgan~\cite{McIver2011}} and \emph{van der Meyden~\cite{DBLP:conf/sp/EggertMSW11,DBLP:journals/tcs/CassezMZ16}}.  
%  Definitions of information flow security serve as important specification of multi-agent system behaviour, e.g., in secure multi-party communication. 
% %  Moreover, insights from information flow security will be useful in understanding how strategies of different agents signal private information to other agents. 
% % Note that the latter is in some sense the dual problem: how can one define strategies that *do* leak enough private information that the distributed players can co-ordinate and achieve a joint objective.
% 
% %  \item Walsh? Aziz?
%  
%  \end{enumerate}
% 
% The faculty of engineering is offering \$40,000 per year for the duration of the project (4 years) to assist the candidate with research activities. Some of this money will be used to hire 2 Msc students by research who will support the theoretical and implementation aspects of the research.
% 
% \subsection{6. BENEFIT AND COLLABORATION}
% 
% 
% \todo{Morri, Ron: can you suggest some other research organisations within Australia? And how UNSW will utilise the project?}
% 
% % \todo{
% % - Describe how the Future Fellowship Candidate will build collaborations across
% % research organisations and/or industry and/or with other disciplines both within
% % Australia and internationally.
% % 
% % - Explain how the Host Organisation(s) will be utilised in the proposed Project, if
% % relevant.
% % 
% % - Outline how the completed project will produce significant new knowledge and/or
% % innovative economic, commercial, environmental, social and/or cultural benefit to the
% % Australian and international community.
% % 
% % - Describe how the proposed research will be cost-effective and value for money
% % 
% % 
% % - Provide 2 subheadings Collaboration and Benefit
% % 
% % - For Benefit make sure you discuss connecting with your overarching research aim/ambition.
% % This section is important if you are claiming resonance of your project beyond the academy (see Significance and National Benefit)
% % If you are claiming Benefits beyond the academy (e.g. policy change, industry benefits) ensure you build in your methods, timeline, collaboration, budget and outputs vehicles to realise these benefits. For example, if you are intending to impact policy your project may include a stakeholder advisory group, enduser seminars and non-academic outputs such as white papers. These will need to be reflected in timelines, budgets and methods. Assessors will immediately question claims of non-academic impact if these strategies and bridges are not considered and detailed in your proposal. 
% % If you have specified that your proposal falls within one of the Science and Research Priorities in Part B1, you should state here how the expected outcomes from the project would make a contribution to the selected Practical Research Challenge(s)
% % Where possible with Benefits quantify the problems you are addressing and how your project may help reduce these problems. For example, ``It currently costs xxxx per annum for each young offender caught within the juvenile justice system…’ you can then detail how your project outcomes might help reduce these costs. Quantifying the problem can cover a range of aspects – incidence of failure, numbers of persons effected, international rankings etc.
% % }
%  
% 
% \subparagraph{Collaboration}
% 
% The project will hire 1 Phd student and 2 Msc students (by research). The applicant will train all these students in research, presentation and academic publication.  The project will also provide travel-funding for the Phd student to present his/her work at national and international conferences.
% 
% The following national experts in their fields have agreed to collaborate on this project: Sebastian Sardina (Associate Professor at RMIT University), Nir Lipovetzky (Lecturer at the University of Melbourne). The budget includes multiple 1 week visits between Sydney and Melbourne.
% 
% The following international leaders in their fields have agreed to collaborate on this project: Moshe Vardi (Rice University), Michael Wooldridge (Oxford University), Giuseppe De Giacomo (Sapienza University of Rome), Hector Geffner (Pompeu Fabra University), Blai Bonet (Sim\'on Bol\'ivar University) and Alessio Lomuscio (Imperial College London). The candidate will make one 4 week visit per year (ideally overlapping with an international conference) to visit an international collaborator (in the first year this will be to visit De Giacomo and Wooldridge).
% 
% 
% 
% \subparagraph{Benefit}
%  
% The anticipated benefit of this project to science is that it will advance the state-of-the-art of the verification and synthesis of artificial agents. This includes techniques, useable by computer scientists and engineers, for building trustworthy agents in realms such as high-level robot control, trustworthy social-media bots, collusion-free e-auctions, and safe and secure cloud storage.
% 
% The potential impact to UNSW's Strategic Theme ``Future Intelligence'' will be tighter integration with world-renowned experts in Artificial Intelligence and Autonomous Systems, including attracting short- and long-term leaders in Automated Planning and Knowledge Representation. The budget includes money for long-term and short-term visits by international and national collaborators to the Administering Organisation.
% 
% Finally, but most importantly, it is in Australia's interest to have safer and securer interactions with artificial agents.
% 
% 
% \subsection{7. COMMUNICATION OF RESULTS}
% 
% % 
% % Outline plans for communicating the research results to other researchers and the
% % broader community, including but not limited to scholarly and public communication
% % and dissemination
% 
% Research results arising from the project will be published in CORE A* conferences in artificial intelligence, 
% multi-agent systems, and logic in computer science. The candidate has an established track record of publishing in 
% these venues (e.g., IJCAI (x4), AAMAS (x4), LICS (x5)) and serves on the program committee of IJCAI, AAMAS and AAAI.
% The best of this work will be consolidated and published in SJR Q1 journals such as JAIR and AIJ. 
% 
% All publications will be made available through the institutional repository UNSWorks in compliance with all funding rules 
% including the ARC's Open Access Policy.
% 
% 
% \subsection{8. MANAGEMENT OF DATA}
% %  Outline plans for the management of data produced as a result of the proposed
% % research, including but not limited to storage, access and re-use arrangements.
% %  It is not sufficient to state that the organisation has a data management policy.
% % Researchers are encouraged to highlight specific plans for the management of their
% % research data.
% 
% Project data will be stored and managed in accordance with the Australian Code for the Responsible Conduct of
% Research, as implemented in the UNSW Procedure for Handling Research Material \& Data. 
% UNSW has implemented a data storage solution for every stage in the life cycle of a research
% project. The data management plan for the project will be established using UNSW's ResData portal (https://resdata.unsw.edu.au).
% Data will be archived using UNSW's Data Archive (http://www.dataarchive.unsw.edu.au/). The 
% Data Archive will give the candidate free access to securely archive all research data so that it can be shared with 
% the team. Data will be made discoverable through publication by Research Data Australia in order to facilitate reuse.
% 
% 
% 


  
\end{document}
