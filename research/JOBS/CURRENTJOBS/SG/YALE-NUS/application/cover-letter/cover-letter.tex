\documentclass[10,a4paper,sans]{moderncv}       
% possible options include font size ('10pt', '11pt' and '12pt'), paper size ('a4paper', 'letterpaper', 'a5paper', 'legalpaper', 'executivepaper' and 'landscape') and font family ('sans' and 'roman')
% moderncv themes
\moderncvstyle{classic}                             % style options are 'casual' (default), 'classic', 'banking', 'oldstyle' and 'fancy'
\moderncvcolor{blue}                               % color options 'black', 'blue' (default), 'burgundy', 'green', 'grey', 'orange', 'purple' and 'red'
%\renewcommand{\familydefault}{\sfdefault}         % to set the default font; use '\sfdefault' for the default sans serif font, '\rmdefault' for the default roman one, or any tex font name
%\nopagenumbers{}                                  % uncomment to suppress automatic page numbering for CVs longer than one page
% character encoding
%\usepackage[utf8]{inputenc}                       % if you are not using xelatex ou lualatex, replace by the encoding you are using
%\usepackage{CJKutf8}                              % if you need to use CJK to typeset your resume in Chinese, Japanese or Korean
% adjust the page margins
\usepackage[scale=0.75]{geometry}
%\setlength{\hintscolumnwidth}{3cm}                % if you want to change the width of the column with the dates
%\setlength{\makecvtitlenamewidth}{10cm}           % for the 'classic' style, if you want to force the width allocated to your name and avoid line breaks. be careful though, the length is normally calculated to avoid any overlap with your personal info; use this at your own typographical risks...
% personal data
\name{Sasha}{Rubin}
% \title{}                               % optional, remove / comment the line if not wanted
\address{University of Naples ``Federico II''}{}% optional, remove / comment the line if not wanted; the "postcode city" and "country" arguments can be omitted or provided empty
% \phone[mobile]{+1~(234)~567~890}                   % optional, remove / comment the line if not wanted; the optional "type" of the phone can be "mobile" (default), "fixed" or "fax"
% \phone[fixed]{+2~(345)~678~901}
% \phone[fax]{+3~(456)~789~012}
\email{rubin@unina.it}                               % optional, remove / comment the line if not wanted
% \homepage{forsyte.at/alumni/rubin/}                         % optional, remove / comment the line if not wanted
% \social[linkedin]{john.doe}                        % optional, remove / comment the line if not wanted
% \social[twitter]{jdoe}                             % optional, remove / comment the line if not wanted
% \social[github]{jdoe}                              % optional, remove / comment the line if not wanted
% \extrainfo{additional information}                 % optional, remove / comment the line if not wanted
% \photo[70pt][0.4pt]{RUBIN_Sasha.jpg}                       % optional, remove / comment the line if not wanted; '64pt' is the height the picture must be resized to, 0.4pt is the thickness of the frame around it (put
% bibliography adjustements (only useful if you make citations in your resume, or print a list of publications using BibTeX)
%   to show numerical labels in the bibliography (default is to show no labels)
\makeatletter\renewcommand*{\bibliographyitemlabel}{\@biblabel{\arabic{enumiv}}}\makeatother
%   to redefine the bibliography heading string ("Publications")
%\renewcommand{\refname}{Articles}

% bibliography with mutiple entries
%\usepackage{multibib}
%\newcites{book,misc}{{Books},{Others}}
%----------------------------------------------------------------------------------
%            content
%----------------------------------------------------------------------------------
\begin{document}

%-----       letter       ---------------------------------------------------------
% recipient data
\recipient{Divisional Manager\\
Ms. Aniza A. Wahid}{}
\date{June 20, 2017}
\opening{Dear Ms. Wahid,}
\closing{I look forward to hearing from you,\vspace{-1cm}}
% \enclosure[Attached]{curriculum vit\ae{}}          % use an optional argument to use a string other than "Enclosure", or redefine \enclname

\makelettertitle

% Make connections. Whenever possible, acknowledge how your work would complement the
% research already happening at the institution where you are applying, or benefit from collaborations
% with members of the institution. (This is something you should definitely do in your cover letter as
% well.)


I am applying for the position of Assistant Professor in Computer Science. 

I am a computer scientist interested in formal aspects of artificial intelligence, and since 2015 I have been a Post-doc at the University of Naples ``Federico II'', mentored by Aniello Murano, with a focus on logics for temporal, epistemic and strategic reasoning in artificial intelligence. As this letter will show, formal methods in artificial intelligence informs my research, supervision and teaching since 2014.

\textbf{Background}
My PhD thesis, titled ``Automatic Structures'' (2004) won the best-doctoral thesis in the faculty of computer science at the University of Auckland. It was about using automata and logic to describe and reason about infinite 
mathematical structures. From 2004-2007 I held a prestigious individual postdoctoral fellowship funded by the New Zealand government on the same topic. I then held various postdoctoral and teaching positions until 2014. Feeling the need to work in an area with more relevance to computer scientists, since 2014 I started shifting my research to formal methods in AI. From 2015-2016 I held another individual fellowship, a COFUND Marie Curie fellowship, 
jointly funded by the European Commission and the Institute for Higher Mathematics (INdAM ``F. Severi''). The topic of this fellowship was verification of lightweight multi-agent systems; one of the publications from this work resulted in a best-paper award at PRIMA15. 

\textbf{Integration in the international community}
Since 2013, I am chair or organiser of 5 events (workshops and conferences), including the First Workshop on Formal Methods in Artificial Intelligence in 2017 with keynote speakers including Hector Geffner and Giuseppe De Giacomo. I have served 
as a PC member for AI conferences such as IJCAI17, AAAI17, and ECAI16. Since 2014 I have collaborated with leading experts in Knowledge Representation (Giuseppe De Giacomo), Automated Planning (Hector Geffner and Blai Bonet), Logic in Computer Science (Moshe Vardi and Helmut Veith), and Formal Methods in Multi-agent systems (Michael Wooldridge and Alessio Lomuscio). 


\textbf{Teaching and Supervision}
While teaching at Cornell University 2008-2009 I sought a number of teaching mentors, including Maria Terrell (Department of Mathematics) and David Way (Centre for Teaching Excellence) to discuss successful teaching strategies, both philosophical and concrete. As a result, according to my student evaluations, I was clear, organised, proactively willing to help, and motivating.

I have a strong record of undergraduate supervision: I have supervised 7 undergraduates (all resulting in publications), and I am currently supervising an undergraduate thesis on ``Graphical Games''. 

Although I did not attend a liberal-arts college, while at university I attended many lectures and seminars on various non-scientific topics of interest to me, including poetry, comparative literature, history of music, and history of mathematics (I even had a few summer jobs assisting with the publication of the South African Review of Books, a now defunct periodical). In short, in this age of specialisation, I try to keep my mind open to other ways of thinking. Being a model of clear and transparent thinking is 
probably the most important task of a teacher. 


% in Naples.
% on the topic of ``Graphical Games'' at the University of Naples.

% Regarding my approach to curriculum, I believe that computer science curricula should provide students with a rigorous foundation in computational thinking and the reasoning required to design and develop computational systems. Thus, even if there are not many dedicated logic courses, I believe that one can and should inject computational aspects of logic into existing topics, e.g., Boolean logic and satisfiability in courses on discrete mathematics, unique-readability of logical formulas into courses on parsers, first-order logic in courses on databases, etc. Just as important as learning foundations, is developing creativity in students that will be useful in real-world scenarios. To this end I like to motivate and illustrate with examples from robot and mobile-agent control, as well as to provide concrete algorithms, the simplest of which can be simulated by hand.


% My usual approach to supervision is to discuss possible problems with students and let them pick one to work on. 
% If the student lacks confidence or is unsure about how to proceed, I create a mini-proposal and timeline for them to follow. I meet with students once a week to discuss progress and troubleshoot. I consider undergraduate research successful if the student a) has fun, b) is challenged, and c) produces and publishes novel research.

% While in Naples, I worked closely with two PhD students, resulting in three publications, and I am supervising s an undergraduate thesis.

% Regarding graduate-level teaching and supervision, I have co-taught a 10 hour PhD mini-course on ``Games on Graphs'' at the University of Naples, 
% a 1 semester PhD course on ``Logical Definability and Random Graphs'' at Cornell University in 2009, and 
% a 5 day advanced course on ``Logic and Computation in Finitely Presentable Infinite Structures'' at ESSLLI in 2006. I have also worked closely with two PhD students of Prof. Murano at the University of Naples, also resulting in publications.



% In that sense, content is secondary.
% Content, while essential, can always be learned independently if one knows how to think clearly.

% I am well prepared to incorporate my experience as a learner and a teacher into helping to mold core and elective courses. In particular, I know that one can teach computational ways of thinking to students of all backgrounds (I taught an interactive class 
% to non-mathematics majors at Cornell University titled ``Algorithms and Termination''). I look forward to teaming up with seasoned liberal-arts educators to teach such courses as YSC1212 and YCC1122.
% 
% All this demonstrates I  make a good fit for teaching and supervising students in the MCS major, as well as supervising a Maths, Computational 
% \& Statistical Science Capstone Project. My background in mathematics and computer science means that I can teach a variety of theoretical courses (including pure mathematics courses) as well as programming courses.

\textbf{Research}
The quality of my research can be quickly but roughly gauged from the venues in which I publish (that said, I strongly maintain that the only way to gauge the strength of a paper is for an expert to read it). These include 16 papers in conferences ranked A* by the CORE ranking (\url{portal.core.edu.au/conf-ranks/}), 10 of which were published since 2016, as well as one book published in 2015.

My vision is to bring-formal methods and artificial intelligence closer together. This is motivated by the need to ensure that the systems being built using, e.g.,  machine learning, can explain their decisions and actions to human users, so-called ``explainable AI'' (for instance, in healthcare, a diagnostic and prescription system without such features will likely go unused and untrusted). To this end, my work has some connections with and could benefit from that of Aquinas Hobor (verification, automated theorem proving) and Robby Tan (machine learning, neural networks). I bring personal expertise on logics and formal-methods for temporal, strategic and epistemic reasoning; this is complemented by my deep integration with world-leaders in AI. Concretely, I plan to organise future editions of the ``Formal Methods in Artificial Intelligence'' workshop.
%and to expand my work to cover other central aspects of AI such as formal methods for Neural Networks.

% 
% \section{Vision for future research}
% My research trajectory is to help bridge two communities: formal-methods and AI. To that end, I plan to continue bringing formal- and logical-methods to bear on questions of importance to computer scientists and society. Notably, ``explainable AI'' (\url{www.darpa.mil/program/explainable-artificial-intelligence}) will require an integration of formal methods, knowledge representation, and machine-learning. 



% The researcher at Yale-NUS whose interests are closest to mine are 
% Aquinas Hobor (verification and automated theorem proving). 

% My ``bread-and-butter'' work consists of developing and applying formal and logical methods. As indicated, for the past few years I have been motivated by applications in AI. One notable example of this convergence is my recent IJCAI17 work with Blai Bonet, Giuseppe De Giacomo, and Hector Geffner which is at the intersection of formal methods and automated planning in AI. I am also pursuing more speculative questions such as ``What is synthesis and how should it be formalised?".
% %(an active topic of investigation with Giuseppe De Giacomo).
% 
%  \textbf{Vision for future research}
% My research trajectory is to help bridge two communities: formal-methods and AI. To that end, I plan to continue bringing formal and logical methods to bear on questions of importance to computer scientists and society. Notably, ``explainable AI'' (\url{www.darpa.mil/program/explainable-artificial-intelligence}) will require an integration of formal methods, knowledge representation, and machine-learning.  Concretely, I plan to organise future editions of the ``Formal Methods in Artificial Intelligence'' workshop, and to expand my work to cover other central aspects of AI such as formal methods for Neural Networks.

% or any of the names listed in my CV.
% Finally, I will remark that there is a non-trivial overlap between formal-methods and AI, both on the level of
% problems and gross techniques. For instance, ``synthesis'' in formal-methods is called ``planning`` 
% in AI; many influential languages for modeling and reasoning in all three fields are 
% explicitly based on mathematical logic (e.g., situation calculus, alternating-time temporal
% logic). That said, there are important differences, e.g., synthesis usually involves exact algorithms on explicit game-graphs while planning typically involves 
% heuristic algorithms on symbolically presented arenas. 
% These communities have a I am currently working to bridge communities, not just technical fields.
% 
% In particular, I have a number of ongoing collaborations that bring logical foundations to bear on a 
% broad variety of issues and problems in 
% \MAS and \AI, including strategic reasoning for data-aware systems, automatic decomposition of 
% business processes into human-understandable representations, and foundations of synthesis 
% (including synthesis under assumptions, rational synthesis, strategic-epistemic logics). 
% 

More details can be found in the accompanying documents.

\makeletterclosing


\end{document}



I have recently contributed a number of works that import and adapt methodologies and techniques from \FM to
\MAS/\AI [2,3,4,8,10,14,15]. I stress that besides novel technical content, 
my key contributions were to identify which models and techniques from \FM, 
amongst many possibilities, are suitable for \MAS/\AI. 




% I am applying for posn X seen at Y.
% 
% Summary of stations of career. Leading project + topic.
% 
% Summary of main research interests. 
% 
% Organisation. How am integrated into community.
% 
% Future research interests/possible topics/possible collaborations at university.
% 
% Teaching experience: what have taught, number of students, supervision.




% My main tools come from mathematical logic and related fields such as automata theory. 
% The \emph{power of logic} in computer science is that it provides a foundation for designing and formally reasoning about 
% computational systems. The \emph{importance} of this is that as computational systems take more responsibilities in the world 
% we should train developers of these systems to build safe, secure, and robust systems. Consequently, 
% 
% 
% 
% 
% My research is closest to the 
% 
% 
%     
%     Teach in Msc on AI: Methods in AI research, MAS, Intelligent Agents (BDI), Logic and Computation
%     People:  	
%     
% I am writing to apply for 
% ntent
% • State why you are writing and for what position you are applying.
% • Demonstrate energy and enthusiasm for the position.
% • Highlight or expand on key information from your resume, but do not simply repeat
% what is listed.
% • Actively sell your unique qualities and tell the reader why he or she should choose you.
% • Target your skills, interests and experience to the needs of the organization.
% • Show you have done your homework; emphasize why you want to work for that
% particular organization.
% • Encourage the reader to take a closer look at your resume

% My research is in Logic has been called the ``calculus of computer science'' since it provides tools and techniques for formally modeling and reasoning about various discrete systems. My work exploits the power and elegance of logic applied to various aspects multi-agent systems, including temporal, strategic and epistemic properties.
% 
invigorate and enrich the scope of expertise of our Department and can enhance its involvement in interdisciplinary research projects within and outside the university.

demonstrable motivation to teach
The preferred candidate has teaching experience, and is actively interested in improving her or his teaching, the courses and the teaching programme. 


    PhD in Computer Science, Information Science or another relevant discipline;
    track record of international publications in leading conferences and journals;
    experience with or good prospects for acquiring external research funds;
    vision on future research directions in own area of expertise;
    experience with or readiness to supervise PhD projects;
    active role in international scientific communities.

    
    enthusiasm for teaching and student supervision;
    ability to teach in departmental BSc and MSc programmes;
    vision on teaching and your own contribution to teaching.


    play an active and cooperative role in the Department and the University;
    willingness to organize scientific events, such as research seminars or teaching seminars;
    willingness to partake in departmental committees.

    

