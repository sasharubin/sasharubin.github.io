\documentclass{article}

 \usepackage[margin=3.6cm,noheadfoot]{geometry} 
%\usepackage{multibibliography}
\newcommand{\mysidestyle}{\sc}

\renewcommand{\labelitemi}{$-$}
\pagenumbering{gobble}% Remove page numbers (and reset to 1)

\usepackage[backend=bibtex, sorting=ynt,maxbibnames = 10,citestyle = 
ieee]{biblatex}

% \usepackage[backend=bibtex, sorting=ydnt,maxbibnames = 10,citestyle = ieee]{biblatex}
\bibliography{/home/sr/svn/forsyte-publications/trunk/rubin.bib}
\DefineBibliographyStrings{english}{%
  references = {\normalsize{Refereed publications in chronological order}},
}





%\usepackage{multicol}
%
%\usepackage[latin1]{inputenc}
%\usepackage[english,danish]{babel}
\usepackage[T1]{fontenc}
%\usepackage{graphicx,wrapfig}
\usepackage{url}
\usepackage[colorlinks=true, a4paper=true, pdfstartview=FitV,
linkcolor=blue, citecolor=blue, urlcolor=blue]{hyperref}
\pdfcompresslevel=9
 
% \usepackage{etoolbox}
% \patchcmd{\thebibliography}{\section*{\refname}}{}{}{}
 
\begin{document}
\noindent\textsc{\LARGE {Sasha Rubin} -- Curriculum Vitae \hfill 4.2017} 


\vspace{0.5cm}
\noindent\makebox[\linewidth]{\rule{10cm}{0.4pt}}

%
%
%\newcommand\cc[1]{(#1 Citations)}
%\renewcommand\cc[1]{}
%



\section{\mysidestyle Personal Details}
 Affiliation: University of Naples ``Federico II'', Italy.\\
 Nationality: New Zealand\\
% Date of Birth: $16.02.1976$\\
 Email: \textsf{rubin@unina.it}\\
%\textsf{forsyte.at/people/rubin/}\\


\section{\mysidestyle Education}

\begin{description}


\item[PhD] Department of Mathematics and Department of Computer Science \hfill $2004$\\
University of Auckland, New Zealand.\\
Supervisor: Bakhadyr Khoussainov\\
Title: Automatic Structures\\
Awards:  Prize for the best doctoral thesis in the Faculty of Science, and
Montgomery memorial prize in logic from the Department of Philosophy.

\item[MSc] Department of Mathematics and Department of Computer Science \hfill $1998$\\
University of Auckland, New Zealand.\\
Award: First Class.

\item[BSc] Department of Mathematics and Department of Computer Science \hfill $1996$\\
University of Cape Town, South Africa\\
Award: Dean's Merit List.

\end{description}

\section{\mysidestyle University Positions}

\begin{description}

\item[Fellow] \hfill $03.2017$ -- present\\ Fellow of the ASTREA lab, University of Naples ``Federico II'', Italy.\\

\item[Postdoctoral Researcher] \hfill $03.2015$ -- $03.2017$\\ University of Naples ``Federico II'', Italy.\\
Marie Curie fellow of the National Institute of Higher Mathematics (INdAM ``F. Severi'').\\
INdAM-COFUND-2012, FP7-PEOPLE-2012-COFUND, Proj. ID 600198.

%\item[Postdoctoral Researcher] $3.2014$ -- $2.2015$\\ TU Vienna and TU Graz, Austria.

\item[Postdoctoral Researcher] \hfill $03.2014$ -- $02.2015$\\ Institute for Information Systems, Technical University of Vienna, Austria and Institute of Applied Information Processing and Communications, Technical University of Graz, Austria.

\item[Postdoctoral Researcher] \hfill $03.2012$ -- $03.2014$\\ Institute for Information Systems, Technical University of Vienna, Austria, and IST Austria

\item[Visiting Researcher] \hfill $05.2010$ -- $03.2012$\\ Department of Computer Science, University of Auckland, New Zealand.\\
Including research visits to: Tel Aviv University, Israel; University of Naples, Italy; University of Paris Diderot, France.

%\section{\mysidestyle{Recent research visits}}
%
%Topic: Application of Logic to AI\\
%Host: \L ukasz Kaiser, Universit\'e Paris Diderot, France $10.2011$
%
%Topic: Games of Imperfect Information and Pushdown Automata\\
%Host: Aniello Murano, Universit\`a degli Studi di Napoli Federico II., Italy $08.2011$.
%
%Topic: Logical-Interpretability and Trees\\
%Host: Alexander Rabinovich, Tel Aviv University, Israel $5.2011$ -- $8.2011$

\item[Visiting Lecturer] \hfill $02.2010$ -- $05.2010$\\ Department of Mathematics, University of Cape Town, South Africa.

\item[Visiting Researcher] \hfill $12.2009$ -- $02.2010$\\ Department of Computer Science, University of Auckland, New Zealand.

\item[Visiting Assistant Professor] \hfill $08.2008$ -- $12.2009$\\ Department of Mathematics, Cornell University, USA.


\item[Honorary Research Fellow] \hfill $12.2004$ -- $08.2008$\\ Department of Computer
Science, University of Auckland, New Zealand.\\
New Zealand Science and Technology Postdoctoral Fellowship $\textrm{UOAX}0413$.
 
\end{description}



\section{\mysidestyle {Research}}

I work in formal methods, a branch of theoretical computer science, and study the power of automata theory (broadly construed) and mathematical logic for describing, reasoning and controlling systems.

\begin{description}

	
%\item[Interests]\

\item[Classification]\ ACM Computing Classification System: Theory of Computation [Models of Computation, Logic, Formal Languages and Automata Theory]; Computing Methodologies [Artificial Intelligence: Planning and Scheduling, Knowledge Representation and Reasoning, Distributed Artificial Intelligence]

Mathematics Subject Classification: 03Bxx (General Logic), 03Cxx (Model Theory); 68Qxx (Theory of Computing), 68Txx (Artificial Intelligence)
% 68T30 (Knowledge Representation), 	68T37 (Reasoning under Uncertainty)

% $\mathrm{03Bxx}$ General logic, $\mathrm{03Cxx}$ Mathematical logic and foundations, $\mathrm{68Qxx}$ Theory of computing, $\mathrm{68Txx}$ Artificial Intelligence

% 	03B25   	Decidability of theories and sets of sentences [See also 11U05, 12L05, 20F10]
% 	03B42   	Logic of knowledge and belief
%		03B44   	Temporal logic
%03B45   	Modal logic
% 	03B70   	Logic in computer science
%68Q60  Specification and verification (program logics, model checking, etc.) [See also 03B70]
%68T27   	Logic in artificial intelligence
%		68T30   	Knowledge representation
%68T37   	Reasoning under uncertainty


\item[Main Areas]\

% (including Parameterised Systems, Probabilistic Systems, Distributed Systems, Timed Systems; 
Formal methods (Modeling, Verification, Synthesis) of Multi-agent Systems (including Parameterised Systems, Distributed Systems, Probabilistic Systems, Timed Systems); 
Logics for Games and Strategic Reasoning; Foundations of Planning; Automata Theory; Finite and Algorithmic Model Theory.

\def\citeKNRS04{94} % 94 in Feb 2017, 91 in July 2016, 
\def\citeRubin08{85} % 85 in Feb 2017, 85 in Sept 2016, 84 in July 2016 
\def\citeThesis{67} % 67 in Feb 2017, 64 in August 2016
\def\citeAminofJKR14{34} % 34 in Feb 2017, 39?? in July 2016

\item[Accomplishments]\

%list  their  five  most  important publications since their last review, along with brief explanations  of  why  each  paper  is  significant.

During my PhD I (and my co-authors) pioneered the development of the theory of automatic structures. My most cited publications in this area are:
\cite{DBLP:conf/lics/KhoussainovNRS04} (\citeKNRS04\ citations; all citation counts are as reported by Google Scholar) and 
\cite{DBLP:journals/bsl/Rubin08} (\citeRubin08\ citations). My PhD thesis (\citeThesis\ citations) was awarded  the Vice-chancellor's prize for the best doctoral thesis in the Faculty of Science, and Montgomery memorial prize in logic from the Department of Philosophy. 

I was then awarded a prestigious New Zealand Science and Technology Postdoctoral Fellowship. During this fellowship, I published a survey and extension of the main results in my thesis in the Bulletin of Symbolic Logic \cite{DBLP:journals/bsl/Rubin08}, and I (with a PhD student of Erich Gr\"adel's) solved a 12 year-old conjecture of Courcelle's \cite{DBLP:conf/stacs/GanzowR08}.

In the last few years, I (with my co-authors) generalised a cornerstone paper on verification of parameterised systems ("Reasoning about Rings", E.A. Emerson, K.S. Namjoshi, {\sc POPL}, 1995) from ring topologies to arbitrary topologies (\citeAminofJKR14\ citations) \cite{DBLP:conf/vmcai/AminofJKR14}. We also completed a book, published by Morgan \& Claypool, surveying decidability results in parameterised verification \cite{DBLP:series/synthesis/2015Bloem}.

Recently, I was awarded a two year Marie-Curie fellowship from the Istituto Nazionale di Alta Matematica to work on formal methods for parameterised light-weight mobile agents. I opened this direction with \cite{DBLP:conf/atal/Rubin15}.  Subsequently (with my co-authors) I continued this direction and published in top rated conferences \cite{DBLP:conf/prima/RubinZMA15,DBLP:conf/atal/AminofMRZ16} and won a best-paper award \cite{DBLP:conf/prima/MuranoPR15} (invited to {\sc JAAMAS}.)


\end{description}


\section{\mysidestyle{Awards and Distinctions}}
\begin{itemize}
\item 2 individual fellowships (Marie Curie fellow of INdAM, New Zealand Science and Technology Postdoctoral Fellowship).
\item 2 PhD prizes (best doctoral thesis in the Faculty of Science, Montgomery memorial prize in logic from the Department of Philosophy).

\end{itemize}


\begin{itemize}
\item 6 Invited Workshop-Talks
%
%\subsection*{Recent Seminar Talks}  IST Austria and TU Vienna ($2011,2012$), CNRS Liafa Paris 7 ($2011$), Tel Aviv University ($2011$), University of Cape Town ($2010$) \\
%Cornell ($2007,2008,2009$),
%\ifcut LSV Cachan ($2008$), CNRS LIAFA Paris 7 ($2007$), Heidelberg ($2007$)\\
%\fi
%\subsection*{Invited Talks}
\begin{itemize}
\item {\it  Verification of Multi-Agent Systems with Imperfect Information and Public Actions}, Naples, Italy \hfill $02.2017$
\item  {\it Finite and Algorithmic Model Theory}, Les Houches, France \hfill $05.2012$
 \item {\it Automata theory and Applications}, IMS programme,  Singapore \hfill $09.2011$
\item {\it Computational Model Theory}, CNRS SIG, Bordeaux, France \hfill $06.2008$
 \item {\it Algorithmic-Logical Theory of Infinite Structures}, Dagstuhl, Germany \hfill $10.2007$
\item {\it Finite and Algorithmic Model Theory}, Newton Institute, England \hfill $01.2006$
\item {\it Workshop on Automata, Structures and Logic}, Auckland, New Zealand \hfill $12.2004$
\end{itemize}
%\subsection*{Conference Talks} CiE ($2008$),

%$2008 - 2009$: Logic Seminar, Cornell Talks in the Logic Seminar at Cornell. \\
%Ongoing work with Anil Nerode and Dexter Kozen.\\

%$2.2007$: Attended the 'Model Theory and Computable Model Theory' workshop, part
%of the University of Florida's Special Year in Logic.\\

\item Competed as part of a team of three, in the world finals of the 1998 ACM Programming Contest in Atlanta, Georgia USA, representing the University of Auckland and New Zealand.


\end{itemize}

\section{\mysidestyle{Recent Research visits}}

\begin{itemize}

\item Host: Mike Wooldridge, Oxford University \hfill $03.2016, 01.2017$\\
Topic: Rational Synthesis 

\item  Host: Alessio Lomuscio, Imperial College London \hfill  $03.2016, 01.2017$\\
Topic: Strategic-Epistemic logics for Multi-Agents Systems

\item Host: Diego Calvanese, University of Bolzanno \hfill $07.2016$\\
Topic: Data-aware strategic logics\\
Topic: Knowledge Representation for Business Process Management

\item Host: Frank Stephan and Sanjay Jain, National University of Singapore \hfill $05.2016$\\
Topic: Learning Theory and Verification 


\item Host: Giuseppe De Giacomo, Sapienza, Rome \hfill $12.2015$\\
Topic: Synthesis under Assumptions\\
Topic: Generalised Planning with Partial Observability

\item Host: Helmut Veith, TU Wien \hfill $08.2015$\\
Topic: Logic and Impossibility Results in Distributed Computing\\
Topic: Abstractions for Fault-tolerant Distributed Algorithms

% %\item Host: \L ukasz Kaiser, Universit\'e Paris Diderot, France \hfill  $10.2011$\\
% %Topic: Application of Logic to AI
% 
% \item Host: Aniello Murano, Universit\`a degli Studi di Napoli ``Federico II'' \hfill $08.2011$\\
% Topic: Games of Imperfect Information and Pushdown Automata 
% 
% \item Host: Alexander Rabinovich, Tel Aviv University, Israel \hfill $5.2011$ -- $8.2011$\\
% Topic: Logical-Interpretability and Trees 
% 
% %Visited Vince 20 August? till end of September, 2010 in Warsaw.
% 
% \item Host: Erich Gr\"adel, RWTH Aachen \hfill $08.2006-01.2007$
% 
% 
% \item Host: Moshe Vardi, Rice University \hfill $01.2001-05.2001$
% 
%\item Steffen Lempp at UW Madison \hfill $08.2000-12.2000$
\end{itemize}
%
%
%$1998$: Competed as part of a team of three, in the world finals of the $1998$
%ACM Programming Contest in Atlanta, Georgia USA, representing the University of Auckland
%and New Zealand. 
%%The same team won the Regional Programming Contest in $1997$.\\ 
%\fi

\section{\mysidestyle{Teaching and Supervision}}

%\section{\mysidestyle{Undergraduate Teaching philosophy}}

I have a passion for teaching, and a proactive approach to learning best-practices.
I spent 1.5 years teaching undergraduate calculus at Cornell. I sought out a number of teaching mentors, including Maria Terrell (Department of Mathematics) and David Way (associate director of the Cornell University Centre for Teaching Excellence) to discuss successful teaching strategies, both philosophical and concrete. As a result, according to my student evaluations, I was clear, organised, proactively willing to help, and motivating.

I have a strong record of undergraduate supervision. While at Cornell I mentored six students for three months in a research programme. This resulted in two publications \cite{DBLP:journals/tcs/GrinshpunPRT14,DBLP:journals/corr/abs-1210-2462} and gave students a taste of research to help them decide if they should pursue a PhD. While at IST Austria I co-mentored one intern which resulted in \cite{DBLP:conf/lata/ChatterjeeCR13}. 
While in Naples, I worked closely with two PhD students, resulting in \cite{DBLP:conf/atal/AminofMMR16,AMMR16-SR,GMRS16IJCAI}.\\

%My goal as a teacher of undergraduate courses is to guide students through the material (eg. I point out which ideas are fundamental and which are technicalities), show students how the material is relevant to their degree, and help students think deeply.  I regularly self-evaluate and engage colleagues in order to discover good teaching principles.
%% (eg. is there a better way of breaking the material into chunks to help students follow?).
%%and engage colleagues in discussions
%% in order to improve my presentation %of the material, 
%I employ questions which encourage students to express themselves clearly and internalise the material, eg. `can anyone help A with her answer?', `can you explain B's idea to me?', `what do you mean by X?', `are you sure?'.
%Another technique I have used is administering an easy online quiz that requires students to read the relevant section of the textbook before coming to class; as a result students ask deeper questions than they otherwise would, a sign that they are better prepared to understand the material discussed in class.
%%,   `who will summarise today's class?'.
%%,  `if all I do is teach you skills I'm short-changing you'.




\noindent {\bf Teaching}


\begin{itemize}
\item Logic and Computation (undergraduate) \hfill $2010$\\
{\it University of Cape Town, Department of Mathematics} 

\item Logical Definability and Random Graphs (graduate) \hfill $2009$ \\
{\it Cornell University, Department of Mathematics} 


\item Totally Awesome Mathematics (undergraduate) \hfill  $2009$\\
Two interactive lectures:\\
i) Hilbert's Hotel and Infinite Cardinals\\
ii) Algorithms and Termination\\
{\it Cornell University, Department of Mathematics}

\item Calculus for Engineers (undergraduate) \hfill $2008 - 2009$\\
 {\it Cornell University, Department of Mathematics}  

%Computational Biology (undergraduate, TA)\\
%{\it University of Auckland, Department of Computer Science} $(2008)$\\

\item Discrete Structures in Mathematics and Computer Science (undergraduate) \hfill $2007$\\
Mathematical Foundations of Software Engineering (undergraduate) \\
{\it University of Auckland, Department of Computer Science} 

\item Logic and Computation in Finitely Presentable Infinite Structures (co-taught five day advanced course) \hfill $2006$\\
{\it European Summer School in Logic, Language and Information} 

\item  Introduction to Formal Verification (advanced undergraduate) \hfill $2003$\\
  {\it University of Auckland, Department of Computer Science} 

\item Automata Theory (undergraduate) \hfill $2002$\\
  {\it University of Auckland, Department of Computer Science} 

\item Pre-calculus (undergraduate) \hfill $2001$ \\
 {\it University of Wisconsin, Madison, Department of Mathematics} 
% 
\end{itemize}

\noindent{\bf Supervision}

\begin{itemize}
\item Summer undergraduate project \hfill $2012$\\
Topic: Edit-distance and Formal Languages.\\
{\em IST Austria}  

\item Summer research experience for undergraduates (REU)\hfill $2009$ \\
Topic 1: Parity Games.\\
Topic 2: Automatic Structures with Advice.\\
{\em  Cornell University, Department of Mathematics} 
\end{itemize}

 

%  {\bf Tutoring}\\
%  {\it University of Auckland, Department of Mathematics} \\
%  $1998$ and $1999$: Undergraduate Mathematics\\


\iffalse $1998$ and $1999$: Stages $1$, $2$ and $3$ in the Mathematics \\
   	 %Assistance Room \\
	 $1999$: Assistant tutor for `Discrete Mathematics', Stage $2$
	Mathematics\\
	 $1999$: Demonstrator for `Combinatorial Computing', Stage $3$
	Mathematics\\
\fi

%\pagebreak 

\section{\mysidestyle{Recent service}}

%%%HBMC
%I am a chair of the ...

\begin{itemize}
\item I am co-chair of the Italian Conference on Theoretical Computer Science (ICTCS) 2017 (\textsf{ictcs2017.unina.it}), co-chair of the International Workshop on Strategic reasoning (SR) 2017 (\textsf{http://sr2017.csc.liv.ac.uk/}), and a 
co-organiser of the Italian Conference on Computational Logic (CILC) 2017 (\textsf{http://cilc2017.unina.it/}).

\item I am a PC member of the International Joint Conference on Artificial Intelligence (IJCAI) 2017,
the AAAI Conference on Artificial Intelligence (AAAI) 2017, the International Workshop of Strategic Reasoning (SR) 2016, the International
Symposium on Games, Automata, Logics and Formal Verification (GandALF) 2016,
and the European Conference on Artificial Intelligence (ECAI) 2016.

\item I co-organised the First Workshop on Formal Methods in AI (FMAI) 2017.\\
\textsf{https://sites.google.com/site/fmai2017homepage/home}

\item I am a Program Committee member for IRISA Master Research Internship 2016-2017.

\item Between 2013 and 2016 I was involved in reviewing and assisting with the Handbook of Model Checking, to be published by Springer, and edited by Edmund Clarke, Thomas Henzinger and Helmut Veith.

\item In 2014, I assisted Helmut Veith with writing and editing consortium grant applications and reports.

\item In 2014, I volunteered for the Vienna Summer of Logic, the largest event in the history of logic.\\
\textsf{http://vsl2014.at/}
%One reviewer wrote "We are extremely grateful to this reviewer for his/her careful reading of the paper, and for his/her constructive suggestions."

\item In $2012/2013$ I was one of the organisers of the IST Austria Young Scientist Symposium on the topic `Understanding Shape: {\it in silico} and {\it in vivo}'.\\
\textsf{ist.ac.at/young-scientist-symposium-2013/}

\item In $2012$ I formed and ran the computer science seminar at IST Austria whose goal was to foster collaborations within the institute between computer scientists and, at the time, biologists.\\
\textsf{ist.ac.at/computer-science-seminar/}

\item In $2010$ I briefly volunteered at a secondary school in Accra, Ghana, teaching, observing and commenting on grade $5$ mathematics classes. I also  briefly volunteered in Khayelitsha, South Africa, helping high-school students prepare for their mathematics exams.
%equaleducation.org.za

%\pagebreak

\item I have reviewed for the following:\\
\emph{Journals}: Artificial Intelligence, Journal of Symbolic Logic, Logical Methods in Computer Science, Theory of Computing Systems, Central European Journal of Mathematics, Information and Computation, Journal of Logic and Computation, Annals of Mathematics and Artificial Intelligence, Theory and Practice of Logic Programming.
\\
\emph{Books}: Handbook of Model Checking.
\\
\emph{Conferences}:\sc{IJCAI, KR, AAMAS, AAAI, EUMAS, ECAI, LICS, STACS, 
ICALP, MFCS, CONCUR, CSL, FoSSaCS, FSTTCS, SR, KRR@SAC, CiE, GandALF, RV, LPAR, 
LATA.}
\end{itemize}

% \newpage

\section{\mysidestyle{References}}

%\begin{multicols}{2}
%\subsection*{Primary}

\subsubsection*{\sc{Academic}}


\noindent{\bf Roderick Bloem}\\
Institute for Applied Information Processing and Communication, 
Technische Universit\"at Graz\\
roderick.bloem@iaik.tugraz.at\\

\noindent{\bf Giuseppe De Giacomo} \\
Dipartimento di Ingegneria Informatica, Automatica e Gestionale, 
Sapienza Universit\`a di Roma\\
%Via Ariosto 25\\
%00185 Roma, Italy\\
degiacomo@dis.uniroma1.it\\
%Tel: +39 06 77274010 (int. 35010)
%Fax: +39 06 77274002 

\noindent{\bf Erich Gr\"adel} \\
Mathematische Grundlagen der Informatik, 
RWTH Aachen, Germany\\
%D-52056 Aachen\\
graedel@logic.rwth-aachen.de\\
%	+$49$ $241$ $80$ $21730$


\noindent{\bf Bakhadyr Khoussainov} --- PhD Supervisor\\
Department of Computer Science, 
University of Auckland\\
bmk@cs.auckland.ac.nz\\
%+$64$ $9$ $373$ $7599$ Ext $85120$
%Fax   : +$64$ $9$ $373$ $7453$\\
%- PhD supervisor\\
%- Project supervisor for `Finite Automaton Presentable Unary Structures', 1998


\noindent{\bf Alessio Lomuscio}\\
Faculty of Engineering, Department of Computing, 
Imperial College London\\
a.lomuscio@imperial.ac.uk\\

\noindent{\bf Aniello Murano}\\
 Dipartimento di Ingegneria Elettrica e Tecnologie dell'Informazione, 
Universit� degli Studi di Napoli ``Federico II''\\
murano@na.infn.it\\






%%\columnbreak
%\noindent{\bf Moshe Y. Vardi} --- Host for visit $01.2001-05.2001$\\
%Department of Computer Science\\
%Rice University, USA\\
%vardi@cs.rice.edu\\
%% +$1$ $713$ $348$ $5977$
%%Fax   : +$1$ $713$ $348$ $5930$\\



%\columnbreak
\noindent{\bf Frank Stephan} \\
School of Computing, 
National University of Singpore\\
%3 Science Drive 2, Singapore 117543\\
fstephan@comp.nus.edu.sg\\
%Phone :  +$65$ $6516$ $2759$\\
%Fax   :  +$65$ $7795$ $5452$\\

\noindent{\bf Michael Wooldridge}\\
Department of Computer Science, 
University of Oxford\\
mjw@cs.ox.ac.uk


\iffalse
{\bf Anil Nerode}\\
Goldwin Smith Professor of Mathematics\\
%545 Malott Hall\\
Cornell University\\
	 Ithaca, NY 14853\\
	 anil@math.cornell.edu\\
Phone: +$1$  $607$ $255$ $3577$\\
\fi
% \end{multicols}

% \vspace{-0.2cm}
% \subsubsection*{\sc{Academic Service}}
% 
% {\bf Tom Henzinger}\\
% President\\
% Institute of Science and Technology Austria\\
% tah@ist.ac.at\\
% 
% \vspace{-0.2cm}

\subsubsection*{\sc{Teaching}}

%\begin{multicols}{2}

{\bf Maria Terrell}\\
Director of Teaching Assistant Programs, 
% 	 225 Malott Hall \\
	 Cornell University\\
%	 Ithaca, NY 14853\\
	 maria@math.cornell.edu\\
%	   +$1$  $607$ $255$ $3905$


\noindent{\bf David Way}\\
Associate Director of
Instructional Support\\
Center for Teaching Excellence, 
%420 D CCC\\
Cornell University\\
%Ithaca, NY 14853\\
dgw2@cornell.edu\\
%+$1$ $607$ $255$ $2663$
%www.cte.cornell.edu\\
%fax: (607) 255 1562\\

%\end{multicols}

\subsubsection*{\sc{Supervision}}

{\bf Bob Strichartz}\\
Department of Mathematics, 
 Cornell University\\
%	 Ithaca, NY 14853\\
str@math.cornell.edu\\
%+$1$  $607$ $255$ $3509$




\section{\mysidestyle{Refereed Publications}}


I have 36 refereed publications, most in top conferences and journals, 
including 1 (co-authored) book, 1 (sole authored) book-chapter, 6 journal 
articles (5 of them invited), 5 {\sc LICS} papers, 4 {\sc AAMAS} papers, 3 {\sc 
STACS} papers, 1 {\sc IJCAI} paper, 1 {\sc KR} paper, and a best-paper at {\sc 
PRIMA}. Not listed, are 6 invited journal articles (in preparation or under 
evaluation) and an invited chapter in a handbook on automata theory and 
applications (J.E. Pin (ed.), to be published by EMS).

% \nocite{*}

% \printbibliography


%prima x2, SR special issue, concur, 

%The following conference papers were invited to journals \cite{
%DBLP:journals/jalc/KhoussainovR01,
%DBLP:journals/jalc/KhoussainovR03,
%DBLP:conf/lics/KhoussainovNRS04,
%DBLP:journals/corr/AminofR14,
%DBLP:journals/lmcs/KhoussainovNRS07,
%DBLP:conf/concur/AminofKRSV14}. 
%The journal versions of the following are under evaluation, and not listed below: \cite{DBLP:conf/concur/AminofKRSV14}.





%\begin{description}

%\item[Book]\
%
%{\it Decidability of Parameterized Verification} with R. Bloem, S. Jacobs, A. Khalimov, I.
%    Konnov, H. Veith and J. Widder, in {\it Synthesis Lectures in Distributed Computing Theory}, N. Lynch Ed., September 2015, 170 pages
%    
%    
%\item[Book chapters]\
%
% {\it Automatic Structures} in {\it Automata: From mathematics to applications}, J.E. Pin, Ed., to be published by EMS.
%
% {\it Automata based presentations of infinite structures} with V. B{\'a}r{\'a}ny and E. Gr{\"a}del,
%in {\it Finite and Algorithmic Model Theory}, J. Esparza, C. Michaux, and C. Steinhorn, Eds.,
%Series: London Mathematical Society Lecture Note Series (379), $1-76$, $2011$. \cc{15}
%
%\item[Journals]\
%
%{\it First-Cycle Games} with B. Aminof, Information and Computation, $2016$.
%
%{\it Alternating Traps in Parity Games} with P. Phalitnonkiat, A. Grinshpun, A.Tarfulea, Theoretical Computer Science,  $73-91, 2014$.
%
%{\it Automata presenting structures: A survey of the finite-string case}, The Bulletin of Symbolic Logic, 
%$14 (2), 169-209, 2008$. \cc{66}
%
%{\it Automatic Structures: Richness and Limitations}, with B. Khoussainov, A. Nies and F. Stephan, 
%Logical Methods in Computer Science, Vol $3$, $2007$.  \cc{78}
%
%{\it Automatic linear orders and trees}, with B. Khoussainov and F. Stephan, 
%ACM Transactions on Computational Logic,
%$6 (4), 675-700, 2005$.  \cc{49}
%
%
%%{\it Automatic Linear Orders and Trees: Revised}, CDMTCS Technical Report $208$,
%%Department of Computer Science, University of Auckland, $2003$.   
% 
% %{\it Definability and Regularity in Automatic Presentations of Subsystems of
%%Arithmetic}, CDMTCS Technical Report $209$,
%%Department of Computer Science, University of Auckland, $2003$.   
% 
%{\it Automatic Structures - Overview and Future Directions}, with 
%B. Khoussainov,
%Journal of Automata, Languages and Combinatorics, $8(2), 287-301, 2003$.   \cc{28}
%
%{\it Graphs with Automatic Presentations over a Unary Alphabet}
%Journal of Automata, Languages and Combinatorics, $6(4), 467-480, 2001$. \cc{15}  
%
%{\it Finite Automata and Well Ordered Sets},
%New Zealand Journal of Computing, $7(2), 39-46, 1999$. 
%
%
%\item[IJCAI Proceedings]\
%
%{\it Imperfect-Information Games and Generalized Planning}, with
%G. De Giacomo, A. Di Stasio, A. Murano, $2016$.
%
%\item[KR Proceedings]\
%
%{\it Prompt Alternating-Time Epistemic Logics}, with B. Aminof, A. Murano, F. Zuleger, $2016$.
%
%\item[AAMAS Proceedings]\
%
%{\it Automatic verification of multi-agent systems in parameterised grid-environments}, with
%B. Aminof, A. Murano, F. Zuleger, $2016$.
%
%{\it Graded Strategy Logic: Reasoning about Uniqueness of Nash Equilibria}, with
%B. Aminof, V. Malvone, A. Murano, $2016$.
%
%{\it Parameterised Verification of Autonomous Mobile-Agents in Static but Unknown Environments}, $2015$.
%
%\item[PRIMA Proceedings]\
%
%{\it Multi-Agent Path Planning in Known Dynamic Environments}, with A. Murano, G. Perelli, $2015$. {\sc Awarded best-paper.}
%
%\item[LICS Proceedings]\
%
%{\it Interpretations in trees with countably many branches}, with A. Rabinovich, $551-560$, $2012$. \cc{3}
%
%
%{\it Automatic Structures: Richness and Limitations}, with B. Khoussainov, A. Nies and F. Stephan, 
%$44-53$, $2004$. \cc{78} 
%
%{\it Automatic Partial Orders}, with B. Khoussainov and F. Stephan, $168-177$, $2003$. \cc{33}
%
%{\it Some Results on Automatic Structures}, with B. Khoussainov
%and H. Ishihara,  $235-244$, $2002$. \cc{13}
%
%\item[STACS Proceedings]\
%
%{\it Cardinality and counting quantifiers on omega-automatic structures}, with V.  B{\'a}r{\'a}ny and \L. Kaiser, $385-396$, $2008$.  \cc{22}
%
%{\it Order invariant MSO is stronger than counting MSO}, with T. Ganzow, $313-324$,  
% $2008$.  \cc{9}
% 
%{\it Definability and Regularity in Automatic Structures}, with B. Khoussainov
%and F. Stephan,  $440-451, 2004$.  \cc{23}
%
%%International Workshop on Logic and Computational Complexity $2007$.\\
%
%\item[CONCUR Proceedings]\
%
%{\it  Parameterized model checking of Rendezvous Systems}, with B. Aminof, T. Kotek, F. Spegni and H. Veith, $109-124$, $2014$
%
%\item[CAV Proceedings]\
%
%{\it Verifying $\omega$-regular Properties of Markov Chains}, with D. Bustan and
%M. Vardi, $189-201, 2004$. \cc{18}
%
%\item[ICALP Proceedings]\
%
%{\it Liveness of Parameterized Timed Networks}, with B. Aminof, F. Spegni and F. Zuleger, $375-387, 2015$.
%
%
%\item[VMCAI Proceedings]\
%
%{\it Parameterized Model Checking of Token-Passing Systems}, with B. Aminof, S. Jacobs and A. Khalimov, $262-281, 2014$. \cc{6}
%
%
%\item[Other Refereed Proceedings]\
%
%{\it Model Checking Parameterised Multi-Token Systems via the Composition Method}, with
%B. Aminof, IJCAR $2016$.
%
%
%{\it On CTL* with Graded Path Modalities}, with
%B. Aminof, A. Murano, {\it LPAR}, $2015$.
%
%{\it On the expressive power of communication primitives in parameterised systems},
%B. Aminof and F. Zuleger, {\it LPAR} $2015$.
%
%{\it Cycle Games} with B. Aminof, {\it Strategic Reasoning}, ETAPS workshop, $2014$.
%
%{\it How to Travel Between Languages} with  K. Chatterjee and S. Chaubal, {\it LATA}, $2013$.
%
%{\it A Myhill-Nerode Theorem for Automata with Advice} with A. Kruckman, J. Sheridan and B. Zax, {\it GandALF}, $238-246$, $2012$. \cc{2}
%
%%{\it Quantifiers on Automatic Structures} with V. Goranko, {\it CiE}, $2008$.
%

%\item[Thesis]\
%
%{\it Automatic Structures}, University of Auckland, $2004$ \cc{55}
%
%%{\it Satisfiability of $CTL^*$ with graded path modalities}, with B. Aminof, A. Murano.
%    
%\end{description}







\end{document}



%----------------------------------------------------
%----------------------------------------------------
%----------------------------------------------------
%----------------------------------------------------
%----------------------------------------------------
%----------------------------------------------------


\iffalse
{\bf Dr. Valentin Goranko}\\
School of Mathematics\\
University of Witwatersrand\\ 
Private Bag 3, WITS 2050\\
Johannesburg, South Africa\\
goranko@maths.wits.ac.za\\
Phone : +$27$ $11$ $717$ $6243$ \\
%Fax   : +$27$ $11$ $717$ $6259$ \\
\fi
%- Tutorial co-ordinator for undergraduate paper `Discrete Mathematics', %1999 \\
%- Project and summer scholarship supervisor for `Application of Elementary \\
%\phantom{- }Submodels to Topology', 1998\\ 
%- Lecturer for `Logic and Set Theory', 1998\\ 
% \enlargethispage*{1cm}

%\section*{Conferences Attended}
%\begin{tabular}{@{}ll}
% {\bf 2000} & Computational Group Theory, Sydney\\
% {\bf 1999} & Third New Zealand Computer Science Research Students'
%Conference \\
%	    & April, Waikato\\
% {\bf 1999} & ACSC - DMTCS/CATS \\
%	    & January, Auckland \\
% {\bf 1999} & NZMRI summer workshop, Harmonic Analysis\\
%	    & January, Raglan, New Zealand\\
% {\bf 1998} & Second Japan-New Zealand Workshop on 
%		{\em Logic in Computer Science}\\
%	    & October, Auckland\\
% {\bf 1998} & First International Conference on 
%		{\em Unconventional Models of Computation}\\
%	    & January, Auckland \\
% {\bf 1997} & First Japan-New Zealand Workshop on {\em Logic in Computer Science}\\
%	    & August, Auckland\\
% {\bf 1997} & Fifth Australasian Mathematics Convention\\
%	    & July, Auckland\\
%\end{tabular}


\iffalse
\subsection*{Additional}
{\bf Dr. David McIntyre}\\
Department of Mathematics\\
University of Auckland, New Zealand\\
mcintyre@math.auckland.ac.nz\\
Phone : (+64 9) 373 7599 Ext 8763\\
%- Tutorial co-ordinator for undergraduate paper `Discrete Mathematics',
%1999 \\
%- Project and summer scholarship supervisor for `Application of Elementary \\
%\phantom{- }Submodels to Topology', 1998\\
%- Lecturer for `Logic and Set Theory', 1998\\
\enlargethispage*{1cm}


%\pagebreak
\section*{Non Academic}
\subsection*{University of Auckland}
{\it Membership}\\
  Auckland University Dramatic Society (1998/1999/2000)\\
  Auckland University Comedy and Improvisation Club (1997) Secretary (1998)\\


{\it Performance}\\
  `Morte Accidentale Di Un Anarchico', (Dario Fo) (1999)\\
  Celebration of Performing Arts, Auckland Town Hall (Stoppard, Simon) (1999)\\
  Three comedy shows, Auckland University (1997 and 1998)\\
  A duet in an evening of short plays, Auckland University (Stoppard) (1998)\\
  Cultural Mosaic Festival, Auckland University (Stoppard) (1997)\\
\begin{center}
--------------------
\end{center}
\fi


\section*{Seminars/Talks} 

 
`Techniques to prove non-automaticity', University of Heidelberg Logic Seminar, 2002 \\


`Some Results on Automatic Structures', with Bakhadyr Khoussainov
and Hajime Ishihara, 17th Annual IEEE Symposium on Logic in Computer Science, 2002. \\


`Automata-theoretic approach to verification of probabilistic systems',
Rice University Computer Science Theory Seminar, 2001. \\


`Automatic Structures', University of Notre Dame Logic Seminar, 2001, and \\
University of Madison, Wisconsin, Logic Seminar, 2001.\\


`Finite Automata and Relational Structures', with Bakhadyr Khoussainov, 
DCAGRS, July 2000, London, Ontario \\


 `Finite Automata and Well Ordered Sets', 
3rd New Zealand Computer Science Research Conference, 1999, Waikato, New
Zealand \\


{\em Auckland Department of Computer Science:}\\
 `Finite Model Theory - Ehrenfeucht-Fraisse Theorem', 2000\\
 `Extracting Algebraic Information from Finite State Machines', 1999\\
 `Finite Automata and Regular Languages', 1999\\


{\em Auckland Department of Mathematics:}\\
 `Algebraic Structures and Finite Automata', 1999\\
 `Applications of Elementary Submodels to Topology', 1999\\ 
\fi

\iffalse
		
  {\bf Marking -} {\it University of Auckland} \\
	 $1998$: Stage $3$ Assignments for the Department of Mathematics\\
  	 1997: Stage 1 Assignments for the Departments of 
	 Computer Science and Mathematics\\
\fi


