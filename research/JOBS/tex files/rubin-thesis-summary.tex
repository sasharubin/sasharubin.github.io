\documentclass{article}

\usepackage{amsmath}
\usepackage{amssymb}
\usepackage{amsthm}
\usepackage{amscd}
\usepackage{amsfonts}
\usepackage{graphicx}%
\usepackage{fancyhdr}
\def\buchi{B{\"u}chi}
\def\str{\mbox{\tt str}}
\def\wstr{\mbox{\tt wstr}}
\def\tree{\mbox{\tt tree}}
\def\trees{\mbox{\tt trees}}
\def\FO{\mbox{\rm FO}}
\def\MSO{\mbox{\rm MSO}}
\def\WMSO{\mbox{\rm WMSO}}
%\def\FO{FO}
%\def\MSO{MSO}
%\def\WMSO{WMSO}

%%%%%%%%%USED IN BSL
%\def\waut{\mbox{\tt W-AutStr}}
%\def\taut{\mbox{\tt T-AutStr}}
%\def\baut{\mbox{\tt $\omega$W-AutStr}}
%\def\raut{\mbox{\tt $\omega$T-AutStr}}
%\def\aut{\mbox{\tt AutStr}}
%%%%%%%%%%%%%%

\def\waut{\mbox{\tt S-AutStr}}

\def\taut{\mbox{\tt T-AutStr}}
\def\injtaut{\mbox{\tt inj-T-AutStr}}

\def\baut{\mbox{\tt $\omega$S-AutStr}}
\def\injbaut{\mbox{\tt inj-$\omega$S-AutStr}}

\def\raut{\mbox{\tt $\omega$T-AutStr}}
\def\injraut{\mbox{\tt inj-$\omega$T-AutStr}}

\def\aut{\mbox{\tt AutStr}}

%\newcommand{\aut}{\diamondsuit}

\def\suc{\text{suc}}
\def\edom{\equiv_{\text{dom}}}
\def\el{\text{el}}

\def\qed{\nolinebreak \hfill $\lhd$}
\def\qex{\hfill $\lhd$}


\newcommand{\dom}{\mbox{dom}}
\newcommand{\st}{\ \vert \ }
\def\Fraisse{Fra\"{\i}ss\'{e} }

\newcommand{\U}{{\mathcal U}}
\newcommand{\Power}{{\mathbb P}}
%\renewcommand{\L}{{\mathcal L}}
\newcommand{\eL}{{\mathcal L}}

\newcommand{\A}{{\mathcal A}}
\newcommand{\B}{{\mathcal B}}
\newcommand{\C}{{\mathcal C}}
\newcommand{\D}{{\mathcal D}}
\newcommand{\E}{{\mathcal E}}
\newcommand{\F}{{\mathcal F}}
\newcommand{\G}{{\mathcal G}}
\newcommand{\J}{{\mathcal J}}
\newcommand{\M}{{\mathcal M}}
\newcommand{\Hch}{{\mathcal H}}
\newcommand{\T}{{\mathcal T}}
\newcommand{\Oh}{{\mathcal O}}
\newcommand{\I}{{\mathcal I}}
\newcommand{\R}{{\mathcal R}}
\newcommand{\V}{{\mathcal V}}
\newcommand{\W}{{\mathcal W}}
\newcommand{\eS}{{\mathcal S}}

\newcommand{\frakA}{\mathfrak{A}}
\newcommand{\frakD}{\mathfrak{D}}

\newcommand{\NS}{\mbox{$(\N,\textrm{S})$}}     %successor
\newcommand{\NL}{\mbox{$(\N,\leq)$}}  %ordering
\newcommand{\IR}{\mbox{\rm IR}}

\newcommand{\tup}[1]{\mbox{$\overline {#1}$}}

\newcommand{\K}{{\mathbf K}}
\newcommand{\mb}[1]{{\mathbf {#1}}}

\newcommand{\class}{\mathsf C}
\newcommand{\klass}{\mathsf K}
\newcommand{\klassaut}{{\mathsf K}^{\rm aut}}

\newcommand{\0}{{\bf 0}}
%\newcommand{\1}{{\bf 1}}


%\newcommand{\kb}{<_{kb}}
%\newcommand{\kbeq}{\leq_{kb}}
%\newcommand{\kbg}{>_{kb}}
%\newcommand{\val}{{\rm val}}
%\newcommand{\num}{{\rm num}}
%\newcommand{\base}{{\rm base}}
%\newcommand{\size}{{\rm size}}
%\newcommand{\last}{{\rm last}}
%\newcommand{\first}{{\rm first}}
%\newcommand{\WMSO}{\mbox{${\rm WMSO}$}}

%\newcommand{\modm}{\, (\rm{mod \,} m) \,}
%\newcommand{\modk}{\, (\rm{mod \,} k) \,}
%\newcommand{\modp}{\, (\rm{mod \,} p) \,}
%\newcommand{\mult}{\mbox{\rm mult}}
%\newcommand{\el}{\mbox{\rm el}}
\newcommand{\FOADD}{\mbox{\rm FO} + \exists^\infty + \exists^{\rm mod}}
%\def\true{\mbox{${\rm {\bf true}}$}}
%\def\false{\mbox{${\rm {\bf false}}$}}

\newcommand{\N}{{\mathbb N}}
\newcommand{\Nt}{\mathcal (\N,+,|_2)}
\def\Nn{\mathcal N}
\newcommand{\Np}{{\mathbb N}^{+}}
\newcommand{\Q}{{\mathbb Q}}
\renewcommand{\R}{{\mathbb R}}
\newcommand{\Quan}{{\mathbf Q}}
\newcommand{\Z}{{\mathbb Z}}

\newcommand{\pair}[2]{\langle{#1},{#2}\rangle}
%\renewcommand{\diamond}{{\perp}}
\newcommand{\blank}{{\perp}}
\newcommand{\zz} {{0 \choose 0}}
\newcommand{\oo} {{1 \choose 1}}
\newcommand{\oz} {{1 \choose 0}}
\newcommand{\zo} {{0 \choose 1}}
\newcommand{\zd} {{0 \choose \blank}}
\newcommand{\dz} {{\blank \choose 0}}
\newcommand{\oD} {\mbox{$\pmatrix{1\cr \blank \cr}$}}
\newcommand{\Do} {\mbox{$\pmatrix{\blank \cr 1\cr}$}}

\newcommand{\szz} {\mbox{$ {0 \choose 0} $}}
\newcommand{\soo} {\mbox{$ {1 \choose 1} $}}
\newcommand{\szo} {\mbox{$ {0 \choose 1} $} }
\newcommand{\soz} {\mbox{$ {1 \choose 0} $} }
\newcommand{\szd} {\mbox{$ {0 \choose \diamond} $}}
\newcommand{\sdz} {\mbox{$ {\diamond \choose 0} $}}
\newcommand{\sdo} {\mbox{$ {\diamond \choose 1} $}}
\newcommand{\sdd} {\mbox{$ {\diamond \choose \diamond} $}}
\newcommand{\sod} {\mbox{$ {1 \choose \diamond} $}}
\newcommand{\stack}[2] {\mbox{$ {#1 \choose #2} $} }

\newcommand{\PR}{{\mathbb P}}
\newcommand{\PA}{\mbox{$(\N,+)$}}
\newcommand{\pref}{\prec_{\tt prefix}}
\newcommand{\prefeq}{\preceq_{\tt prefix}}
\newcommand{\subtree}{\preceq_{\tt ext}}
\newcommand{\con}{\otimes}

%\newcommand{\lex}{\preceq_{\tt lex}} %lexeq used in bsl
\newcommand{\lex}{<_{\mathrm{lex}}} %THIS ONE USED FOR DAGSTUHL
\newcommand{\llex}{<_{\text{llex}}} %THIS ONE USED FOR DAGSTUHL

\newcommand{\ext}{\prec_{\tt ext}}
\newcommand{\exteq}{\preceq_{\tt ext}}


%\newcommand{\ll}[1]{ {\, <_{{#1}\text{-llex}}\,} }
%\newcommand{\lleq}[1]{ {\, \leq_{{#1}\text{ -llex}} \,} }
\newcommand{\kllex}{\, <_{k\text{-llex}}\,}
\newcommand{\kllexeq}{\, \leq_{k\text{-llex}}\,}


%\newcommand{\FO}{\mathrm FO}
\newcommand{\existsinfty}{\exists^{\infty}}
\newcommand{\existsmod}{\exists^{\text{mod}}}
\newcommand{\existscount}{\exists^{\leq \aleph_0}}
\newcommand{\existsuncount}{\exists^{> \aleph_0}}

\newcommand{\FOEXT}{\mbox{\rm FO} + \exists^\infty + \exists^{\rm mod} + \existscount + \existsuncount}
\newcommand{\FOinfty}{\FO[\existsinfty]}
\newcommand{\FOmod}{\FO[\existsmod]}
\newcommand{\qreg}{{\mathbf Q}^{\text{reg}}}

\newcommand{\op}{\mathrm{O}}
\newcommand{\one}{\mbox{$(\N,\mathrm{S})$}}
\newcommand{\two}{\mbox{$(\{0,1\}^{\star},\mathrm{s}_0, \mathrm{s}_1)$}}
\newcommand{\twolr}{\mbox{$(\{l,r\}^{\star},\suc_l, \suc_r)$}}
\newcommand{\oracleB}{\mbox{$(\{0,1\}^{\star},\mathrm{s}_0,
\mathrm{s}_1,\tup{B},Y)$}}
\newcommand{\oracle}{\mbox{$(\{0,1\}^{\star},\mathrm{s}_0, \mathrm{s}_1,\op)$}}



\theoremstyle{plain} \numberwithin{equation}{section}
\newtheorem{theorem}{Theorem}[section]
\newtheorem{corollary}[theorem]{Corollary}
\newtheorem{conjecture}{Conjecture}
\newtheorem{lemma}[theorem]{Lemma}
\newtheorem{proposition}[theorem]{Proposition}
\theoremstyle{definition}
\newtheorem{definition}[theorem]{Definition}
\newtheorem{finalremark}[theorem]{Final Remark}
\newtheorem{remark}[theorem]{Remark}
\newtheorem{example}[theorem]{Example}
\newtheorem{question}{Question} \topmargin-2cm

\textwidth6in

\setlength{\topmargin}{0in} \addtolength{\topmargin}{-\headheight}
\addtolength{\topmargin}{-\headsep}

\setlength{\oddsidemargin}{0in}

\oddsidemargin  0.0in \evensidemargin 0.0in %\parindent0em
\pagestyle{plain}

\lhead{Thesis Summary} \rhead{January 2008\\ rubin@cs.auckland.ac.nz}
\chead{{\large{\bf Sasha Rubin}}} \lfoot{} \rfoot{} \cfoot{\thepage}

\begin{document}

%\raisebox{1cm}
\thispagestyle{fancy}

%begin general intro
\noindent

\begin{center} \subsection*{Automatic Structures}
\end{center}
\noindent

I investigated {\it (finite-word) automatic structures}. These are mathematical structures such as graphs, arithmetics, and algebras that can be described by automata. Specifically this means that the elements can be coded by finite words so that the domain and atomic operations are computable by finite automata operating synchronously on their inputs. My thesis work, which includes results from co-authored conference publications\footnote{Coauthors are Bakhadyr Khoussainov (K), Frank Stephan (S), Andr{\'e} Nies (N)},  forms the basis of a survey {\it Automata presenting structures: A survey of the finite-string case} (to appear in The Bulletin of Symbolic Logic $2008$) and is enclosed with this application.

I focused on the problem of classifying classes of automatic structures up to isomorphism. For example, what do the automatic linear orderings look like? or the automatic Boolean algebras? These problems are technically difficult because one is required to show, in essence, that there is no way to code the given structure so that the domain and operations are regular.

%Automatic structures were introduced in \cite{KhNe95}, and picked up in \cite{blum99,blgr00} where basic counting arguments were used to show that the following structures are not automatic: Skolem arithmetic $(\N,\times)$, any pairing function $(\N,\left<\right>)$, any free monoid on more than two generators.

{\bf Improving techniques for proving non-automaticity}: A breakthrough was made in 2001 by Delhomm{\'e} and Stephan who communicated proofs that the least ordinal with no automatic presentation is $\omega^\omega$, and that the countable random-graph is not automatic. Using reasonably straightforward adaptations of these proofs, I established the non-automaticity of other random structures ($K_n$-free random graph, random partial order), and that every automatic linear order has finite VD-rank. A non-trivial adaptation is that every automatic tree (in the signature of partial order) has finite Cantor-Bendixson rank. With KNS, I found novel counting techniques that prove, for instance, that any monoid with $(\N,\times)$ as a substructure is not automatic, and that there is no automatic infinite integral domain. We also showed that the infinite automatic Boolean algebras (in the usual signature) are those that are finite powers of the algebra of finite or co-finite subsets of the natural numbers.

{\bf Extending the fundamental theorem:} Standard closure properties of automata show that FO query evaluation on automatic structures is effective. It is easy to see that
one can extend FO with the quantifier $\existsinfty$ (`there exists infinitely many elements such that ...' is equivalent to the expression 'there are longer and longer strings such that ...). With KS, I established one can also add the quantifier `there exists $k$ modulo $m$ many', collectively written $\existsmod$.

{\bf Measuring the complexity of classes of automatic structures:} 
On the other hand some classes (for instance graphs, trees in the signature of successor, lattices of height 4) have no hope of being reasonably described. This is because the isomorphism problem (whether or not two sequences of automata describe isomorphic automatic structures) for the class of automatic graphs is as hard as possible, namely $\Sigma_1^1$-complete. We (KNRS) proved this by a reduction from the isomorphism problem for computable trees, known to be $\Sigma_1^1$-complete.

{\bf Intrinsic Regularity:} With the aim of studying all automatic presentations of a given structure, we (KRS) call a relation $R \subset A^n$, not necessarily in the signature of $\A$, {\em intrinsically regular (IR) for $\A$} if it is regular in every automatic presentation of $\A$. Clearly relations definable in FO + $\existsinfty$ + $\existsmod$ are intrinsically regular. We completely described the IR relations for certain structures (for instance $\IR(\N,\leq) = \FO + \existsmod(\N,\leq))$, and provided pathological constructions such as an automatic presentation of $\NS$ in which the set $2\N$ is not intrinsically regular.
\end{document}
