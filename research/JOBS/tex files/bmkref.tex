\documentclass{letter}
\addtolength{\topmargin}{-0.5pc}
\addtolength{\oddsidemargin}{-3pc}
\addtolength{\textwidth}{6pc}
\setlength{\evensidemargin}{\oddsidemargin}
\font\helv=cmss12
\font\helvsmall=cmss9

\signature{\vspace{-3em} Professor Bakhadyr Khoussainov\\
Fellow of New Zealand Royal Society\\
Personal Chair\\
Department of Computer Science\\
University of Auckland}

\date{January, $2008$}

\textwidth = 6.25in
\textheight = 8.5in

\begin{document}
\begin{letter}{}
\opening{Dear Assessor}

I enthusiastically recommend Sasha Rubin for a permanent or temporary research position in the area of theoretical computer science and its applications. Sasha would be a great addition to any research group in the area; I have no doubt that he would contribute to research results and ideas, and is able to develop new techniques in areas related to logic, automata, verification, algorithms, and applications of these. His main research area is the intersection of Mathematical Logic and the Theory of Computation.

\smallskip


I have known Sasha for 10 years. I first met him when he came to discuss some coursework while he was an honours student in the Department of Mathematics at the University of Auckland.
I immediately noticed his sharp and keen mind. I subsequently supervised his Doctorate between 1999 and 2004. During this time he developed mathematical maturity and co-published papers in the top tier conferences LICS, STACS and CAV. He showed his enthusiasm, determination, and creativity to solve hard
combinatorial problems related to logic and automata. He is one of the founders of the theory of automatic structures; an area that has attracted the attention of many theoretical computer scientists and has the reputation of being quite difficult to work in. For example, Sasha has presented invited lectures on automatic structures at the following workshops: Dagstuhl Seminar 07441 `Algorithmic-Logical Theory of Infinite Structures' in October 2007; and Durham `Finite and Algorithmic Model Theory' in January 2006. I trust that Sasha's PhD thesis results were instrumental in attracting so many researchers into the area of automatic structures.

\smallskip

After his PhD Sasha was awarded the prestigious New Zealand Science and Technology Postdoctoral Fellowship. Although formally in the theory group in the Department of Computer Science in Auckland, I encouraged him to visit experts in theoretical computer science. This was a successful strategy that ensured he forged collaborations and raised his profile
in the community. During a four month visit to RWTH Aachen he collaborated with three of Erich Graedel's students, resulting in the solution of a 10 year old conjecture in finite model theory, and some difficult open questions in automatic structures. Both papers were accepted to STACS08.

Generally, Sasha has established a good collaboration environment with top experts in the area of automata and its applications. These include Moshe Vardi, Erich Gradel,
and Valentin Goranko and their students.

\smallskip

I have seen his mathematical abilities become more sophisticated over the duration of his Postdoc. I feel that there is still more room for him to grow and that a rich collaborative environment is best suited to his style. He works well when talking about problems. I have noticed that he enjoys engaging graduate students in discussions about their work, and so coming to grips with fields not directly in his area of expertise, such as enumerative combinatorics, computational biology, computable model theory, Kolmogorov randomness.
I think this is an excellent attribute that has and will continue to broaden his research.

%recent results
\smallskip

I will briefly describe some of his main contributions to automatic structures in fairly non-technical language. Roughly, a mathematical structure is automatic if its basic
operations can be described by automata; there are different types of automatic structures depending on the flavour of automata at hand. Sasha has worked on fundamental problems in this relatively new area (introduced by Nerode and myself in 1995). The most difficult of these are lower-bound type results. For instance, sometimes with various co-authors, he has devised proof techniques for showing that certain structures (such as the random graph and its variations, the multiplicative group of rationals) are not automatic. Sometimes this extended previous work [Blumensath/Graedel 2001, Delhomm{\'e} 2001, Stephan 2001] and sometimes it involved novel analysis.  Another lower bound result shows that the problem of deciding whether two automatic structures are isomorphic, given the automata, is hard for the class $\Sigma_1^1$ of the analytic hierarchy.

He has recently been working on extending the fundamental theorem of automatic structures (that FO model checking is effective on automatic structures) by allowing certain additional quantifiers. For instance, it was known that one could allow certain counting quantifiers (for instance 'there exists countably many elements...') on omega-string automatic structures [Kuske and Lohrey 2005]. In his recent STACS08 paper with B\'ar\'any and Kaiser he has shown that this is also possible on quotients of omega-string automatic structures by omega-string regular relations, eventhough these quotients may not be omega-string automatic themselves. He has also taken the first steps in automatic model theory: for instance, together with myself and Frank Stephan he exhibited automatic versions of K\"onig's lemma and Ramsey's theorem. These results are in stark contrast to the more general scenario, where for instance there is a computable tree with exactly one infinite path, and that path is not computable.

He has also worked in Formal Verification with Moshe Vardi, one of the initiators of the automata-theoretic approach to verification.

In closing, Sasha has always shown enthusiasm to develop ideas with others. Sasha is always keen to learn, he is an excellent collaborator, and he is also a good presenter of results. All in all, Sasha is an excellent candidate and I can already see him developing into a well-established researcher.

\closing{Sincerely} \end{letter}

\end{document} 