\documentclass{article}[12pt]

\usepackage{amsmath}
\usepackage{amssymb}
\usepackage{amsthm}
\usepackage{amscd}
\usepackage{amsfonts}
\usepackage{graphicx}%
\usepackage{fancyhdr}
\def\buchi{B{\"u}chi}
\def\str{\mbox{\tt str}}
\def\wstr{\mbox{\tt wstr}}
\def\tree{\mbox{\tt tree}}
\def\trees{\mbox{\tt trees}}
\def\FO{\mbox{\rm FO}}
\def\MSO{\mbox{\rm MSO}}
\def\WMSO{\mbox{\rm WMSO}}
%\def\FO{FO}
%\def\MSO{MSO}
%\def\WMSO{WMSO}

%%%%%%%%%USED IN BSL
%\def\waut{\mbox{\tt W-AutStr}}
%\def\taut{\mbox{\tt T-AutStr}}
%\def\baut{\mbox{\tt $\omega$W-AutStr}}
%\def\raut{\mbox{\tt $\omega$T-AutStr}}
%\def\aut{\mbox{\tt AutStr}}
%%%%%%%%%%%%%%

\def\waut{\mbox{\tt S-AutStr}}

\def\taut{\mbox{\tt T-AutStr}}
\def\injtaut{\mbox{\tt inj-T-AutStr}}

\def\baut{\mbox{\tt $\omega$S-AutStr}}
\def\injbaut{\mbox{\tt inj-$\omega$S-AutStr}}

\def\raut{\mbox{\tt $\omega$T-AutStr}}
\def\injraut{\mbox{\tt inj-$\omega$T-AutStr}}

\def\aut{\mbox{\tt AutStr}}

%\newcommand{\aut}{\diamondsuit}

\def\suc{\text{suc}}
\def\edom{\equiv_{\text{dom}}}
\def\el{\text{el}}

\def\qed{\nolinebreak \hfill $\lhd$}
\def\qex{\hfill $\lhd$}


\newcommand{\dom}{\mbox{dom}}
\newcommand{\st}{\ \vert \ }
\def\Fraisse{Fra\"{\i}ss\'{e} }

\newcommand{\U}{{\mathcal U}}
\newcommand{\Power}{{\mathbb P}}
%\renewcommand{\L}{{\mathcal L}}
\newcommand{\eL}{{\mathcal L}}

\newcommand{\A}{{\mathcal A}}
\newcommand{\B}{{\mathcal B}}
\newcommand{\C}{{\mathcal C}}
\newcommand{\D}{{\mathcal D}}
\newcommand{\E}{{\mathcal E}}
\newcommand{\F}{{\mathcal F}}
\newcommand{\G}{{\mathcal G}}
\newcommand{\J}{{\mathcal J}}
\newcommand{\M}{{\mathcal M}}
\newcommand{\Hch}{{\mathcal H}}
\newcommand{\T}{{\mathcal T}}
\newcommand{\Oh}{{\mathcal O}}
\newcommand{\I}{{\mathcal I}}
\newcommand{\R}{{\mathcal R}}
\newcommand{\V}{{\mathcal V}}
\newcommand{\W}{{\mathcal W}}
\newcommand{\eS}{{\mathcal S}}

\newcommand{\frakA}{\mathfrak{A}}
\newcommand{\frakD}{\mathfrak{D}}

\newcommand{\NS}{\mbox{$(\N,\textrm{S})$}}     %successor
\newcommand{\NL}{\mbox{$(\N,\leq)$}}  %ordering
\newcommand{\IR}{\mbox{\rm IR}}

\newcommand{\tup}[1]{\mbox{$\overline {#1}$}}

\newcommand{\K}{{\mathbf K}}
\newcommand{\mb}[1]{{\mathbf {#1}}}

\newcommand{\class}{\mathsf C}
\newcommand{\klass}{\mathsf K}
\newcommand{\klassaut}{{\mathsf K}^{\rm aut}}

\newcommand{\0}{{\bf 0}}
%\newcommand{\1}{{\bf 1}}


%\newcommand{\kb}{<_{kb}}
%\newcommand{\kbeq}{\leq_{kb}}
%\newcommand{\kbg}{>_{kb}}
%\newcommand{\val}{{\rm val}}
%\newcommand{\num}{{\rm num}}
%\newcommand{\base}{{\rm base}}
%\newcommand{\size}{{\rm size}}
%\newcommand{\last}{{\rm last}}
%\newcommand{\first}{{\rm first}}
%\newcommand{\WMSO}{\mbox{${\rm WMSO}$}}

%\newcommand{\modm}{\, (\rm{mod \,} m) \,}
%\newcommand{\modk}{\, (\rm{mod \,} k) \,}
%\newcommand{\modp}{\, (\rm{mod \,} p) \,}
%\newcommand{\mult}{\mbox{\rm mult}}
%\newcommand{\el}{\mbox{\rm el}}
\newcommand{\FOADD}{\mbox{\rm FO} + \exists^\infty + \exists^{\rm mod}}
%\def\true{\mbox{${\rm {\bf true}}$}}
%\def\false{\mbox{${\rm {\bf false}}$}}

\newcommand{\N}{{\mathbb N}}
\newcommand{\Nt}{\mathcal (\N,+,|_2)}
\def\Nn{\mathcal N}
\newcommand{\Np}{{\mathbb N}^{+}}
\newcommand{\Q}{{\mathbb Q}}
\renewcommand{\R}{{\mathbb R}}
\newcommand{\Quan}{{\mathbf Q}}
\newcommand{\Z}{{\mathbb Z}}

\newcommand{\pair}[2]{\langle{#1},{#2}\rangle}
%\renewcommand{\diamond}{{\perp}}
\newcommand{\blank}{{\perp}}
\newcommand{\zz} {{0 \choose 0}}
\newcommand{\oo} {{1 \choose 1}}
\newcommand{\oz} {{1 \choose 0}}
\newcommand{\zo} {{0 \choose 1}}
\newcommand{\zd} {{0 \choose \blank}}
\newcommand{\dz} {{\blank \choose 0}}
\newcommand{\oD} {\mbox{$\pmatrix{1\cr \blank \cr}$}}
\newcommand{\Do} {\mbox{$\pmatrix{\blank \cr 1\cr}$}}

\newcommand{\szz} {\mbox{$ {0 \choose 0} $}}
\newcommand{\soo} {\mbox{$ {1 \choose 1} $}}
\newcommand{\szo} {\mbox{$ {0 \choose 1} $} }
\newcommand{\soz} {\mbox{$ {1 \choose 0} $} }
\newcommand{\szd} {\mbox{$ {0 \choose \diamond} $}}
\newcommand{\sdz} {\mbox{$ {\diamond \choose 0} $}}
\newcommand{\sdo} {\mbox{$ {\diamond \choose 1} $}}
\newcommand{\sdd} {\mbox{$ {\diamond \choose \diamond} $}}
\newcommand{\sod} {\mbox{$ {1 \choose \diamond} $}}
\newcommand{\stack}[2] {\mbox{$ {#1 \choose #2} $} }

\newcommand{\PR}{{\mathbb P}}
\newcommand{\PA}{\mbox{$(\N,+)$}}
\newcommand{\pref}{\prec_{\tt prefix}}
\newcommand{\prefeq}{\preceq_{\tt prefix}}
\newcommand{\subtree}{\preceq_{\tt ext}}
\newcommand{\con}{\otimes}

%\newcommand{\lex}{\preceq_{\tt lex}} %lexeq used in bsl
\newcommand{\lex}{<_{\mathrm{lex}}} %THIS ONE USED FOR DAGSTUHL
\newcommand{\llex}{<_{\text{llex}}} %THIS ONE USED FOR DAGSTUHL

\newcommand{\ext}{\prec_{\tt ext}}
\newcommand{\exteq}{\preceq_{\tt ext}}


%\newcommand{\ll}[1]{ {\, <_{{#1}\text{-llex}}\,} }
%\newcommand{\lleq}[1]{ {\, \leq_{{#1}\text{ -llex}} \,} }
\newcommand{\kllex}{\, <_{k\text{-llex}}\,}
\newcommand{\kllexeq}{\, \leq_{k\text{-llex}}\,}


%\newcommand{\FO}{\mathrm FO}
\newcommand{\existsinfty}{\exists^{\infty}}
\newcommand{\existsmod}{\exists^{\text{mod}}}
\newcommand{\existscount}{\exists^{\leq \aleph_0}}
\newcommand{\existsuncount}{\exists^{> \aleph_0}}

\newcommand{\FOEXT}{\mbox{\rm FO} + \exists^\infty + \exists^{\rm mod} + \existscount + \existsuncount}
\newcommand{\FOinfty}{\FO[\existsinfty]}
\newcommand{\FOmod}{\FO[\existsmod]}
\newcommand{\qreg}{{\mathbf Q}^{\text{reg}}}

\newcommand{\op}{\mathrm{O}}
\newcommand{\one}{\mbox{$(\N,\mathrm{S})$}}
\newcommand{\two}{\mbox{$(\{0,1\}^{\star},\mathrm{s}_0, \mathrm{s}_1)$}}
\newcommand{\twolr}{\mbox{$(\{l,r\}^{\star},\suc_l, \suc_r)$}}
\newcommand{\oracleB}{\mbox{$(\{0,1\}^{\star},\mathrm{s}_0,
\mathrm{s}_1,\tup{B},Y)$}}
\newcommand{\oracle}{\mbox{$(\{0,1\}^{\star},\mathrm{s}_0, \mathrm{s}_1,\op)$}}



\theoremstyle{plain} \numberwithin{equation}{section}
\newtheorem{theorem}{Theorem}[section]
\newtheorem{corollary}[theorem]{Corollary}
\newtheorem{conjecture}{Conjecture}
\newtheorem{lemma}[theorem]{Lemma}
\newtheorem{proposition}[theorem]{Proposition}
\theoremstyle{definition}
\newtheorem{definition}[theorem]{Definition}
\newtheorem{finalremark}[theorem]{Final Remark}
\newtheorem{remark}[theorem]{Remark}
\newtheorem{example}[theorem]{Example}
\newtheorem{question}{Question} \topmargin-2cm

\textwidth6in

\setlength{\topmargin}{0in} \addtolength{\topmargin}{-\headheight}
\addtolength{\topmargin}{-\headsep}

\setlength{\oddsidemargin}{0in}

\oddsidemargin  0.0in \evensidemargin 0.0in \parindent0em
\pagestyle{plain}

\lhead{Brief Teaching Statement} \rhead{February 2011}
\chead{{\large{\bf Sasha Rubin}}} \lfoot{} \rfoot{} \cfoot{\thepage}

\setlength{\parskip}{1ex plus 0.5ex minus 0.2ex}

\begin{document}

%\raisebox{1cm}
\thispagestyle{fancy}

I have taught the following courses which are relevant to the visiting position.

\subsubsection{Undergraduate {\em logic and computation} at the University of Cape Town (2010).}

Student abilities in the class were very mixed which meant I had to carefully structure the course. I slowed the lectures down a bit and moved harder questions to the tutorials. One of the strongest students sent me an email at the end of the term:
\begin{quote}
I have enjoyed your maths course the most of all the maths courses I've taken so far at UCT. Even more than the content, your delivery was excellent.
I made the decision during one of your lectures to do honours in mathematics and computer science. 
\end{quote}

\subsubsection{{\em Calculus} at Cornell (2008 - 2009).}
This was essentially a second course in calculus for engineering students that follows a textbook very closely.
After each lecture I reflected on how I could improve presentation of the material. This usually involved improving the pace, refining what to write on the board, and finding better ways to break up the material into chunks that the students could follow. These students were generally capable of acquiring, on their own, the skills to work routine problems.
%; but they need to feel that the mathematics is relevant.
Consequently my main goal  was to get them to reason mathematically, both verbally and in writing. 
I noticed that students are very sensitive to the wording I use.
I keep a list of phrases they respond positively to, such as  `can anyone help A with her answer?', `can you explain B's idea to me?', `what do you mean by X?',
`are you sure?',  and `who will summarise today's class?'.
%, `if all I do is teach you skills I'm short-changing you'.

%I worked hard finding angles on the standard material that would engage the students. This invariably involved finding a question that
%sounded fun to try to solve. My favourite moments in class were those that involved discussing the big ideas in calculus. After writing a theorem on the board I asked: `why should we believe this?' (a soft version of `how do we prove it?') and the often overlooked `how is this theorem useful?'.

I received reviews to confirm my strategy had the desired effect:
\begin{quote}
Professor was very well prepared \dots Instead of memorizing formulas, he helped us understand what we were learning, why we
were learning it and why it was useful.
\end{quote}

\subsubsection{{\em Discrete structures in mathematics and computer science} and {\em Mathematical foundations of Software Engineering} at the University of Auckland (2007).}

\subsubsection{{\it $18$th European Summer School in Logic, Language and Information, University of Malaga, $2006$}
This was a five day course `Logic and computation in finitely presentable infinite structures' at the advanced level co-taught with V. Goranko.
 
 
 {\it University of Auckland, Department of Computer Science} \\
 $2003$: Introduction to Formal Verification, Stage $4$.\\
 $2002$: Automata Theory, Stage $3$.\\

{\bf Supervision}

In 2009 I supervised six exceptional undergraduate students for a two month research experience (REU). We worked on two projects and have interesting results, some of which have been submitted to {\em Theoretical Computer Science}. During this time I learned the value of giving students a few days to brew and filter their ideas before group discussions.
Overall it was a rich experience for both me and, I gather, for my students. One exceptional student expressed to me that the experience helped him decide to pursue a career in research.

In 2007 I collaborated with a number of graduate students at RWTH-Aachen, resulting in two co-authored publications at STACS 2008.

{\bf Graduate teaching}

%In 2006 I co-taught a five day course at ESSLLI with Valentin Goranko in the area of logic and computation. 

In 2009 I taught a course on logical aspects of random graphs. I improved my ability to give intuition
behind tricky definitions and proofs. The course was self-contained and the students remained engaged throughout. 
There were no exams -- each student presented a lecture on a topic that built on the material presented in class.
\end{document}

{\bf Future ideas}

 I am keen to teach a course that follows the historical development of the subject. I think it helps appreciation of a subject if one understands the
problems people were struggling with that led to the theory. %I imagine this would be a worthwhile experiment with a first course in calculus; students would appreciate how
%certain problems are hard or impossible to solve without calculus.

%I also enjoy participating in a course that follows a book, and requires each student to present some material from it. As a graduate student I learned modal logic and some model theory this way.

My particular teaching strengths, based on experience and familiarity with the material, are in mathematical logic and theory of computation (automata theory, algorithms, discrete mathematics, computational complexity), and calculus. However, I am always interested in teaching something new, especially if it forces me to learn unfamiliar mathematics. % I also hope to run a course based on Moore's method.

\end{document}

\subsubsection*{Ideas about teaching}
%Although I have not taught a class using this method, 
The experience has informed my teaching - the best way to absorb mathematics is to do it. 

\subsubsection*{Undergraduate teaching}

%I have taught undergraduate courses at all levels, however always with a predetermined syllabus.
Within the constraints of a predetermined syllabus I get my students to
reason mathematically, both verbally and in writing.

I do this by asking questions, expecting answers, and sometimes breaking for a
short written exercise. This interaction often helps students realise that they
don't yet understand an idea or technique well enough to work with it on their
own. 

In order to coax the most out of my freshmen I keep a list of phrases that I've noticed encourage
interaction, 
I find it very useful to have the students read the relevant chapter of the textbook
before coming to class. I noticed that the students who regularly read ahead would ask much more precise and interesting
questions during class. I have ensured this with online pre-class quizzes that tested comprehension.


\subsubsection*{Graduate teaching and supervision}

%I am comfortable teaching any undergraduate course in pure mathematics or theoretical computer science.
%With additional preparation, I would be able to teach undergraduate courses in applied mathematics or introductory programming.
\end{document} 
