\documentclass[10,a4paper,sans]{moderncv}       
% possible options include font size ('10pt', '11pt' and '12pt'), paper size ('a4paper', 'letterpaper', 'a5paper', 'legalpaper', 'executivepaper' and 'landscape') and font family ('sans' and 'roman')
% moderncv themes
\moderncvstyle{classic}                             % style options are 'casual' (default), 'classic', 'banking', 'oldstyle' and 'fancy'
\moderncvcolor{blue}                               % color options 'black', 'blue' (default), 'burgundy', 'green', 'grey', 'orange', 'purple' and 'red'
%\renewcommand{\familydefault}{\sfdefault}         % to set the default font; use '\sfdefault' for the default sans serif font, '\rmdefault' for the default roman one, or any tex font name
%\nopagenumbers{}                                  % uncomment to suppress automatic page numbering for CVs longer than one page
% character encoding
%\usepackage[utf8]{inputenc}                       % if you are not using xelatex ou lualatex, replace by the encoding you are using
%\usepackage{CJKutf8}                              % if you need to use CJK to typeset your resume in Chinese, Japanese or Korean
% adjust the page margins
\usepackage[scale=0.75]{geometry}
%\setlength{\hintscolumnwidth}{3cm}                % if you want to change the width of the column with the dates
%\setlength{\makecvtitlenamewidth}{10cm}           % for the 'classic' style, if you want to force the width allocated to your name and avoid line breaks. be careful though, the length is normally calculated to avoid any overlap with your personal info; use this at your own typographical risks...
% personal data
\name{Sasha}{Rubin}
\title{Teaching Statement} 
\address{November 2017}{}
\email{rubin@unina.it}                               % optional, remove / comment the line if not wanted

% optional, remove / comment the line if not wanted
% \address{University of Naples ``Federico II''}{}% optional, remove / comment the line if not wanted; the "postcode city" and "country" arguments can be omitted or provided empty
% \phone[mobile]{+1~(234)~567~890}                   % optional, remove / comment the line if not wanted; the optional "type" of the phone can be "mobile" (default), "fixed" or "fax"
% \phone[fixed]{+2~(345)~678~901}
% \phone[fax]{+3~(456)~789~012}
% \email{rubin@unina.it}                               % optional, remove / comment the line if not wanted
% \homepage{forsyte.at/alumni/rubin/}                         % optional, remove / comment the line if not wanted
% \social[linkedin]{john.doe}                        % optional, remove / comment the line if not wanted
% \social[twitter]{jdoe}                             % optional, remove / comment the line if not wanted
% \social[github]{jdoe}                              % optional, remove / comment the line if not wanted
% \extrainfo{additional information}                 % optional, remove / comment the line if not wanted
% \photo[70pt][0.4pt]{RUBIN_Sasha.jpg}                       % optional, remove / comment the line if not wanted; '64pt' is the height the picture must be resized to, 0.4pt is the thickness of the frame around it (put
% bibliography adjustements (only useful if you make citations in your resume, or print a list of publications using BibTeX)
%   to show numerical labels in the bibliography (default is to show no labels)
\makeatletter\renewcommand*{\bibliographyitemlabel}{\@biblabel{\arabic{enumiv}}}\makeatother
%   to redefine the bibliography heading string ("Publications")
\renewcommand{\refname}{References}
% 
% \usepackage{amsmath}
% \usepackage{amssymb}
% \usepackage{amsthm}
% \usepackage{amscd}
% \usepackage{amsfonts}
% \usepackage{graphicx}%
% \usepackage{fancyhdr}


% bibliography with mutiple entries
%\usepackage{multibib}
%\newcites{book,misc}{{Books},{Others}}
%----------------------------------------------------------------------------------
%            content
%----------------------------------------------------------------------------------
\begin{document}


\makecvtitle

 

The most influential course I've attended was a graduate level introduction to
mathematical logic at the University of Auckland.  It was taught by a topologist, David McIntyre, based on Moore's method --- 
we were given basic definitions, followed by statements of
fundamental theorems that we were to prove ourselves and present the following
lesson. The material was to form the background of my
graduate study. I probably did my best learning when my classmates presented a
proof that I was unable to produce beforehand. Identifying the points in the proof that I
did not think of trained me to get a focus on why and where my own reasoning
had fallen short. It is these sorts of insights that I try generate in my own classroom.

\

% It seems that the best way, and possibly the only way, to properly absorb mathematics is to do mathematics. 
% %This makes the usual lecture format for teaching mathematical ideas challenging.
%  I often discuss teaching strategies with colleagues.
% Our discussions usually revolve around techniques to engage students to enjoy and think deeply about the material.

\subsection{Undergraduate teaching}

I take every opportunity to improve my teaching.

\

I taught three semesters of Calculus for Engineers at Cornell University, essentially a second course in calculus for first year students that follows a textbook very closely.

\

After each lecture I reflected on how I could improve presentation of the material. This usually involved improving the pace, refining what to write on the board, and finding better ways to break up the material into chunks that students could follow.

\

My students were generally capable of acquiring, on their own, the skills to work routine problems. %; but they need to feel that the mathematics is relevant.
Consequently my main goal 
was to get them to reason mathematically, both verbally and in writing. I noticed that students are very sensitive to the wording I use.
I keep a list of phrases to which they seem to respond, such as  `can anyone help A with her answer?', `can you explain B's idea to me?', `what do you mean by X?',
`are you sure?', `who will summarise today's class?', `if all I do is teach you computational procedures I'm short-changing you'.

\

I worked hard finding angles on the material that would engage my students. This invariably involved finding a question that
sounded fun to try to solve. My favourite moments in class were those that involved discussing the big ideas in calculus. After writing a theorem on the board I asked: `why should we believe this?' (a soft version of `how do we prove it?') and the often overlooked `how is this theorem useful?'.

\

I also learned to ask questions that probe student knowledge and understanding. For instance `why would you say that?' and `tell me more' helps to diagnose their logic, while asking easy recall-level questions lets me see whether students have been listening.

\

Although my students were most comfortable with being given algorithms to solve problems I instead focused on problem solving techniques \textit{ala} P\'olya: 1) identify the unknown, 2) if you can't solve the problem find a problem that you can solve that has a similar unknown. 

\

I experimented with small group work in class to alter the pacing of lectures and encourage my students to think and reflect on what they had and had not understood.

\

%Finally, in 2009 I lectured to non-mathematics/computer science majors as part of a course called {\em totally amazing maths}. 
%I spoke about Hilbert's hotel and infinite cardinals as well as algorithms and termination.

Early in 2010 I gave the undergraduate course on logic and computation in the mathematics department at the University of Cape Town. Student abilities in the class were very mixed which meant I had to structure the course and pace very carefully. I slowed the lectures down a bit and moved harder questions to the tutorials. One of the strongest students sent me an email at the end of the term:

\begin{verbatim}
  I have enjoyed your maths course the most of all the maths courses 
  I've taken so far at UCT. Even more than the content, your delivery 
  was excellent.

  I made the decision during one of your lectures to do honours in 
  mathematics and computer science next year at UCT. 
\end{verbatim}
\

\subsection{Graduate Teaching}

I've designed and taught three graduate-level courses: a generalist course ``Milestones in Solving Games on Graphs'' at the Technical University of Vienna (2017), a PhD course 
``Games on Graphs'' at the University of Naples (2017), and a PhD course ``Logical Definability and Random Graphs'' at Cornell University (2009), 
an advanced course ``Logic and Computation in Finitely Presentable Infinite Structures'' at ESSLLI (2006, with Valentin Goranko).

I list a student comment from the most recent course:
\begin{verbatim}
 I especially liked the fact that you let us engage with the ideas. 
 I think I was able to deepen my understanding of graph games a 
 lot because I never took an actual course in it.
\end{verbatim}

\begin{verbatim}
 During the lecture Sasha Rubin promotes logically and mathematically 
 rigorous thinking and achieves a rare level of engagement among the students. 
 This combination results in a very enjoyable and stimulating course.
\end{verbatim}

\subsection{Undergraduate Supervision}

My usual approach to supervision is to discuss possible problems with students and let them pick one to work on. 
If the student lacks confidence or is unsure about how to proceed, I create a mini-proposal and timeline for them to follow. I meet with students once a week to discuss progress and troubleshoot. I consider undergraduate research successful if the student a) has fun, b) is challenged, and c) produces and publishes novel research.

\

In 2007 I started collaborations with a number of graduate students, two of which resulted in publications at STACS 2008.

\

In 2009 I supervised six undergraduate students for a two month research experience (REU). The student selection process was very competitive and so I received exceptionally talented undergraduates. The students formed two groups and worked on two projects. During this time I learned the value of giving students a few days to brew and filter their ideas before group discussions. Overall it was a rich experience for both me and, I gather, for my students. One exceptional student expressed to me that the experience helped him decide to pursue a career in research. The results were subsequently published in the journal {\em Theoretical Computer Science} and the conference {\em GandALF}.

\

In 2012 I co-supervised an undergraduate summer project which led to a publication in the conference {\em LATA}.  While the student was writing up I realised that a proof required some formalities that the student did not know. At that point the student was keen to learn {\em how} I came to realise there was a problem. This episode taught me the value of modelling good mathematical thinking for learners. % as well as `thinking ahead' of the student.

\

In 2017 I supervised an undergraduate thesis, that included theoretical and practical components, on ``graphical games'', a topic at the interface of game-theory and graph-theory. We are writing up this work.

\

I recently worked with four junior PhD students, resulting in papers published at VMCAI'14, IJCAI'16, AAMAS'16, and VMCAI'18. In all cases I learned the value of 
helping graduate students to structure their thinking so they could contribute more to the collaboration than they otherwise might.

% \subsection{Graduate teaching}
% 
% In 2006 I co-taught a five day course at ESSLLI with Valentin Goranko in the area of logic and computation. 
% 
% \
% 
% In 2009 I taught a course on logical aspects of random graphs. I improved my ability to pace lectures and give intuition
% behind tricky definitions and proofs. The course was self-contained and my students remained engaged throughout. 
% There were no exams --- each student presented a lecture on a topic that built on the material presented in class.

% \subsection{Future ideas}
% 
% I am keen to teach a course that follows the historical development of a subject. I believe that understanding the
% motivating problems helps one apply the material and ideas elsewhere. 

%  %I imagine this would be a worthwhile experiment with a first course in calculus; students would appreciate how
% %certain problems are hard or impossible to solve without calculus.
% 
% %I also enjoy participating in a course that follows a book, and requires each student to present some material from it. As a graduate student I learned modal logic and some model theory this way.
% 
% My particular teaching strengths, based on experience and familiarity with the material, are in mathematical logic and theory of computation (automata theory, algorithms, discrete mathematics, computational complexity), and calculus. I am comfortable teaching pure mathematics at the undergraduate level (e.g., real analysis, measure theory, probability theory, point-set topology, combinatorics, number theory, abstract algebra, linear algebra, graph theory) and am always interested in teaching something new, especially if it forces me to learn unfamiliar mathematics. % I also hope to run a course based on Moore's method.

\end{document}




























\subsubsection*{Ideas about teaching}
%Although I have not taught a class using this method, 
The experience has informed my teaching --- the best way to absorb mathematics is to do it. 

\subsubsection*{Undergraduate teaching}

%I have taught undergraduate courses at all levels, however always with a predetermined syllabus.
Within the constraints of a predetermined syllabus I get my students to
reason mathematically, both verbally and in writing.

I do this by asking questions, expecting answers, and sometimes breaking for a
short written exercise. This interaction often helps students realise that they
don't yet understand an idea or technique well enough to work with it on their
own. 

In order to coax the most out of my freshmen I keep a list of phrases that I've noticed encourage
interaction, 
I find it very useful to have students read the relevant chapter of the textbook
before coming to class. I noticed that students who regularly read ahead would ask much more precise and interesting
questions during class. I have ensured this with online pre-class quizzes that tested comprehension.


\subsubsection*{Graduate teaching and supervision}

%I am comfortable teaching any undergraduate course in pure mathematics or theoretical computer science.
%With additional preparation, I would be able to teach undergraduate courses in applied mathematics or introductory programming.
\end{document} 
