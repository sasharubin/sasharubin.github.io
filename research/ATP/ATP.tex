\documentclass{book}

\usepackage{morgan}

                                % Author-Year bibliography
\usepackage[square,comma,sort]{natbib}
                                % numbered bibliography
%usepackage[square,comma,sort,numbers]{natbib}
\usepackage[raggedsec]{morplay}
\usepackage[draft]{ttdefs}

\setcounter{secnumdepth}{2}
%% PACKAGES %%
\usepackage{latexsym}
\usepackage{amsmath}
\usepackage{amssymb}
%\usepackage{amsthm} %


\usepackage{color}
\usepackage{graphicx}
\usepackage{tikz,pgf}
  \usetikzlibrary{automata,positioning,matrix,calc,petri,arrows}

%% ENVIRONMENTS %%
%\theoremstyle{plain}
%\newtheorem{theorem}{Theorem}
%\newtheorem{lemma}{Lemma}
%\newtheorem{fact}{Fact}
%\newtheorem{example}{Example}
%\newtheorem{definition}{Definition}
%%\newtheorem{corollary}{Corollary}
%\newtheorem{proposition}{Proposition}
%%\newtheorem*{proof*}{Proof}

%% COMMENTS %%

\newcommand\note[1]{{\color{red}{#1}}}
\newcommand{\todo}[1]{{{\color{blue} #1}}}
%\renewcommand{\todo}[1]{}

%% LATEX SHORTCUTS %%
% cause problems with aamas style file
%\def\it{\begin{itemize} }
%\def\-{\item[-] }
%\def\ti{\end{itemize} }
%\def\en{\begin{enumerate} }
%\def\ne{\end{enumerate} }

%% COMPLEXITY CLASSES %%

\def\UPTime{\textsc{up}\xspace}
\def\CoUPTime{\textsc{coup}\xspace}


\def\exptime{\textsc{exptime}\xspace}
\def\exptimeC{\exptime-complete}

\def\pspace{\textsc{pspace}\xspace}
\def\pspaceC{\pspace-complete}

\def\logspace{\textsc{logspace}\xspace}
\def\nlogspace{\textsc{nlogspace}\xspace}

\def\ptime{\textsc{ptime}}
\def\np{\textsc{np}}



%% LOGIC %%
\def\fol{\mathsf{FOL}}
\def\SL{\textsf{SL}}

\def\msol{\mathsf{MSOL}}
\def\fotc{\mathsf{FOL+TC}}

\def\know{\mathbb{K}}
\def\dknow{\mathbb{D}}
\def\cknow{\mathbb{C}}
\def\eknow{\mathbb{E}}

\def\dualknow{\widetilde{\mathbb{K}}}
\def\dualdknow{\widetilde{\mathbb{D}}}
\def\dualcknow{\widetilde{\mathbb{C}}}
\def\dualeknow{\widetilde{\mathbb{E}}}

\renewcommand\implies{\rightarrow}

%% TEMPORAL LOGIC %%
\newcommand{\sqsq}[1]{\ensuremath{[\negthinspace[#1]\negthinspace]}}

\DeclareMathOperator{\ctlE}{{\mathsf{E}}}
\DeclareMathOperator{\ctlA}{\mathsf{A}}


\newcommand{\atlE}[1][A]{\ensuremath{\langle\!\langle{#1}\rangle\!\rangle}}
\newcommand{\atlA}[1][A]{\ensuremath{[[{#1}]]}}

\DeclareMathOperator{\nextX}{\mathsf{X}}
\DeclareMathOperator{\yesterday}{\mathsf{Y}}
\DeclareMathOperator{\until}{\mathbin{\mathsf{U}}}
\DeclareMathOperator{\weakuntil}{\mathbin{\mathsf{W}}}
\DeclareMathOperator{\since}{\mathbin{\mathsf{S}}}
\DeclareMathOperator{\releases}{\mathbin{\mathsf{R}}}
\DeclareMathOperator{\always}{\mathsf{G}}
\DeclareMathOperator{\hitherto}{\mathsf{H}}
\DeclareMathOperator{\eventually}{\ensuremath{\mathsf{F}}\xspace}
\DeclareMathOperator{\previously}{\mathsf{P}}
\newcommand{\true}{\mathsf{true}}
\newcommand{\false}{\mathsf{false}}


\newcommand{\LTL}{\ensuremath{\mathsf{LTL}}\xspace}
\newcommand{\PLTL}{\textsf{PROMPT-}\LTL}

\newcommand{\CTL}{\ensuremath{\mathsf{CTL}}\xspace}
\newcommand{\CTLS}{\ensuremath{\mathsf{CTL}^*}\xspace}
\newcommand{\PCTLS}{\textsf{PROMPT-}\CTLS}
\newcommand{\PCTL}{\textsf{PROMPT-}\CTL}
\newcommand{\CLTL}{\ensuremath{\textsf{C-}\LTL}\xspace}
\newcommand{\PCLTL}{\ensuremath{\textsf{PROMPT-C}\LTL}\xspace}

\newcommand{\ATL}{\ensuremath{\mathsf{ATL}}\xspace}
\newcommand{\ATLS}{\ensuremath{\mathsf{ATL}^*}\xspace}
\newcommand{\PATLS}{\textsf{PROMPT-}\ATLS}
\newcommand{\PATL}{\textsf{PROMPT-}\ATL}

\newcommand{\KATL}{\ensuremath{\mathsf{KATL}}\xspace}
\newcommand{\KATLS}{\ensuremath{\mathsf{KATL}^*}\xspace}
\newcommand{\PKATLS}{\textsf{PROMPT-}\KATLS}
\newcommand{\PKATL}{\textsf{PROMPT-}\KATL}


\def\red{{red}}
\def\col{{col}}
\def\alt{\, | \,}

%% PROMPT 
\def\kmodels{\models^k}
\def\twokmodels{\models^{2k}}
\DeclareMathOperator{\Fp}{\eventually_\mathsf{P}}
\DeclareMathOperator{\Gp}{\always_\mathsf{P}}
\DeclareMathOperator{\within}{\mathsf{within}}

\newcommand{\AP}{{AP}}
\def\Ag{{Ag}}
\def\Act{{Act}}

%% MATH OPERATIONS %%
\newcommand{\tpl}[1]{\langle {#1} \rangle }
\newcommand{\tup}[1]{\overline{#1}}
\def\proj{\mathsf{proj}}
\newcommand{\defeq}{\ensuremath{\triangleq}}

%% STRUCTURES and STRATEGIES %%
\newcommand{\cgs}{\ensuremath{\mathsf{S}}}
\newcommand{\LTS}{\mathsf{S}}
\newcommand{\Comp}{\mathsf{cmp}}
\newcommand{\Hist}{\mathsf{hist}}
\newcommand{\out}{{out}}

\newcommand{\Paths}{\mathsf{pth}}

\newcommand{\nat}{\mathbb{N}}
\def\int{\mathbb{Z}}
\newcommand{\natzero}{\mathbb{N}_0}

\newcommand{\trans}[3]{#1 \stackrel{\mathsf{#3}}{\rightarrow} #2}


%% HEADINGS ETC %%
\newcommand{\head}[1]{\noindent {\bf #1}.}

%% COUNTER MACHINES %%
\newcommand{\cm}{M}
\newcommand{\CMinc}{\mathsf{inc}}
\newcommand{\CMdec}{\mathsf{dec}}
\newcommand{\CMzero}{\mathsf{ifzero}}
\newcommand{\CMnonzero}{\mathsf{nzero}}
\newcommand{\CMcommit}{\mathsf{end}}


%% CLTL %%
\def\var{{\sf var}}
\def\ovar{{\sf ovar}}
\def\avar{{\sf avar}}
\def\svar{{\sf svar}}
\def\bvar{{\sf bvar}}

\def\MOD{\equiv}

%% PVP %%
\def\PVP{\mathsf{PVP}}




\begin{document}

 \frontmatter
\chapter*{Abstract}


\mainmatter      


\setcounter{tocdepth}{3}
\tableofcontents

%%%%%%%%%%%%%%%%%%%%%%%%%%%%%%%%%


\chapter{Introduction}
\section{Target Audience}
Graduate students who have taken basic courses in discrete mathematics, data structures, logic and computation;
who want to add to their toolbox a powerful, versatile, tool

\chapter{Formal Methods}

\section{Verification}

\section{Synthesis}

\section{Monitoring}

Testing?
\section{What is not in this book}

??
%%%%%%%%%%%%%%%%%%%%%%%%%%%%%%%%%%%%%%%%%%5

\chapter{Algorithmic Foundations: Decision problems, Automata theory}

\section{Decision problems}

\section{Finite Automata, Regular expressions}

\section{Automata on infinite words}

\section{Automata on infinite trees}


%%%%%%%%%%%%%%%%%%%%%%%%%%%%%%

\chapter{Foundations: Logics for Verification}

\section{What are logics? Syntax, Semantics.}

\section{What are logics good for?}

\section{model checking, satisfiability, realisability.}

\subsection{Linear-time}

\subsection{Branching-time}

\subsubsection{Computational Tree Logic}

\subsubsection{$\mu$-calculus}

\section{Alternating-time}

\section{Epistemic logics}

\section{first-order logics}

\section{second-order logics}

%%%%%%%%%%%%%%%%%%%%%%%%%%%%%%%%%%%



%%%%%%%%%%%%%%%%%%%%%%%%%%%%%%%%%%%
\chapter{Limitations of the method}

\section{Inherent limitations}

\section{Other approaches}

\section{Comparison with other approaches}

%%%%%%%%%%%%%%%%%%%%%%%%%%%%%%%%%%%%

\chapter{History of the method}


\chapter{Inverview? Biography of the inventors}

\chapter{tex}


 %% preliminaries: tree automata

\subsection{Automata Theory Background} \label{sec:prelims:automata}

We use alternating B\"uchi automata over infinite trees, written \ABT. All results can be found in \cite{DBLP:conf/dagstuhl/2001automata}, although we note that
here we use slight notational variants of the definitions.\footnote{Our trees $t$ have the unconvential but convenient property that their domains do not contain the empty-string. This leads to the need for an initial formula rather than the more common an initial state.}

Let $\B^+(X)$ denote the set of positive Boolean formulas over $X$, i.e., it consists of the constants $\true,\false$, and the formulas from the set containing the elements of $X$ closed under the operations $\vee$ and $\wedge$. Let $\Sigma$ be an alphabet, and let $D$ be a finite non-empty set of \emph{directions}. A $D$-ary $\Sigma$-tree is a function $t:D^+ \to \Sigma$. 


Tree automata recognised sets of $D$-ary $\Sigma$-trees. An \emph{alternating B\"uchi tree automaton (\ABT)} over $\Sigma$ and $D$ is a tuple 
$T = (Q,\iota,\delta,B)$ where $Q$ is a finite non-empty set of states, $\iota \in \B^+(D \times Q)$ is the initial formula, 
$\delta:Q \times \Sigma \to \B^+(D \times Q)$ is the transition function, and $B \subseteq Q$ is the set of {\em accepting} states. The set $\F = \{\iota\} \cup \{\delta(q,\sigma) : q \in Q, \sigma \in \Sigma\}$ is called the set of \emph{formulas} of $T$.

Since it is intuitive and concise, we give a game-theoretic semantics of what it means for an \ABT to accept or reject a tree.
An \ABT $T$ accepts tree $t$ if player Automaton has a winning strategy in the following ``membership game'' played against player Pathfinder. 
Positions of the game are elements of $D^+ \times Q$. From a position $(u,q)$ player Automaton picks (if he can) a set $S \subseteq D \times Q$ satisfying $\delta(q,t(u))$; if there is no such set Automaton loses the game, if the set is empty then Automaton wins, and otherwise Pathfinder picks a position 
$(d',q') \in S$ and the position is updated to $(ud',q')$. To start the game off Automaton picks $S$ satisfying $\iota$ and Pathfinder picks a position $(d,q) \in S$.
If the game lasts infinitely many rounds it generates an infinite sequence $(d_0,q_0) (d_0d_1,q_1) (d_0d_1d_2,q_2) \cdots$ (the sequence can be thought of as a path through $t$ labeled by states from $Q$). In this case Player Automaton wins iff there exists infinitely many $n \in \nat$ such that $q_n \in B$. 

% the players play on the parse-tree of the formula $\delta(q,t(u))$ with Automaton resolving disjuncts, Pathfinder resolving conjuncts, resulting in either $\true,\false$ or an expression of the form $(d',q')$; the players begin the game by playing on the parse-tree of the formula $\iota$. In case a parse-tree resolves to $\true$ then Automaton wins the game, and to $\false$ then Automaton loses the game. Otherwise, the game proceeds for another round from position $(ud',q')$. 


A \emph{nondeterministic B\"uchi tree automaton (\NBT)} is an \ABT whose formulas are either of the form $\true$, $\false$, or $\bigvee_i \bigwedge_{d \in D} (d,q_{i,d})$, i.e., they are in DNF in which every conjunct contains every direction exactly \todo{at most?} once.
A \emph{universal B\"uchi tree automaton (\UBT)} is an \ABT in which none of the formulas contain disjunction.
A \emph{deterministic B\"uchi tree automaton (\DBT)} is an \NBT that is also a \UBT, i.e., every formula that is not $\true$ or $\false$ is of the form $\bigwedge_{d \in D} (d,q_d)$.

We define \emph{alternating co-B\"uchi tree automata (\ACT)} like \ABT but dualise the acceptance condition, i.e., player Automaton wins iff there are only finitely many $n \in \nat$ such that $q_n \in B$.

A $D$-ary $\Sigma$-tree $t$ is \emph{regular} if it is determined by a transducer with input alphabet $D$ and output alphabet $\Sigma$.


\todo{define size of automaton.}



\begin{fact}[Dualising \ABT] \label{fact:dual}
Fix $\Sigma,D$.
 The \emph{dual} of an \ABT $T = (Q,\iota,\delta,B)$ is the \ACT $T^\partial = (Q,\iota^\delta,\delta^\partial,B)$ where $\delta^\partial(q,\sigma) := \delta(q,\sigma)^\partial$ and where $\theta^\partial$ is the transformation that swaps $\vee$ with $\wedge$ and swaps $\true$ with $\false$, i.e., $\true^\partial \doteq \false$, $\false^\partial \doteq \true$, $(\theta_1 \wedge \theta_2)^\partial \doteq \theta_1^\partial \vee \theta_2^\partial$ and $(\theta_1 \vee \theta_2)^\partial \doteq \theta_1^\partial \wedge \theta_2^\partial$. It follows from the definitions that, for all trees $t$, $T$ accepts $t$ if and only if $T^\partial$ does not accept $t$. 
 The size of $T^\partial$ is the same as the size of $T$. 
\end{fact}

\begin{fact}[Intersection of \ABT] \label{fact:intersection}
 Fix $\Sigma, D$ and \ABT $T_i = (Q_i,\iota_i,\delta_i,B_i)$ for $i = 1,2$. 
 Then the following \ABT accepts those trees $t$ that are accepted by both $T_1$ and $T_2$: it is the disjoint union of $T_1$ and $T_2$ with initial formula $\iota_1 \wedge \iota_2$.  
 \end{fact}


 


\begin{fact}[Emptiness of \ABT] \label{fact:ABT-emptiness} 
Emptiness of \ABT is decidable in \exptime, i.e., 
$2^{poly(n)}$ where $n$ is the number of states of the \ABT and $poly$ is a fixed polynomial.
If the \ABT $T$ is non-empty, the procedure can also return a regular tree in $T$. 
\end{fact}



\subsection{Automata with Finitary Acceptance}

An \ABT $(Q,\iota,\delta,B)$ has \emph{finitary acceptance condition} if $\delta(b,\sigma) = b$ for all $b \in B, \sigma \in \Sigma$. In particular, in the membership game, player Automaton wins an infinite play $\pi$ iff there exists $n \in \nat$ such that $\pi_n \in B$, i.e., a reachability condition. We write \AFT for these automata. 

\sr{WARNING: We need better notation. AFT usually means input trees are FINITE.}

\begin{fact}[Intersection of \NFT] \label{fact:nft-intersection}
 Fix $\Sigma, D$ and \NFT $T = (Q,\iota,\delta,B)$ and $T' = (Q',\iota',\delta',B')$. Without loss of generality:
 \begin{itemize}
 \item Assume $B = B' = \emptyset$ (this can be done redefining $\delta$ to map $q \in B$ and $\sigma \in \Sigma$ to $\true$).\footnote{We remark that this simplification is not valid for \DFTf 
 since the automaton would have to guess the end of the branch (although it would be valid if we assumed end-of-branch markers to all input trees).}
 \item  Assume that $Q,Q'$ are disjoint.
 \end{itemize}
 Let $S \doteq (Q \times Q') \cup Q \cup Q'$.
 For $\theta = \bigvee_{x \in X} \bigwedge_{d \in D} (d,q_{x,d}) \in \B^+(D \times Q)$ and 
 $\theta'  = \bigvee_{y \in Y} \bigwedge_{d \in D} (d,q_{y,d}) \in \B^+(D \times Q')$, define $\theta \otimes \theta' \in \B^+(D \times S)$ as follows:
 \begin{itemize}
 \item  $\theta \otimes \false \doteq \false$ and $\false \otimes \theta \doteq \false$, 
 \item  $\theta \otimes \true \doteq \theta$ and $\true \otimes \theta' \doteq \theta'$, 
 \item  $\theta \otimes \theta' \doteq
  \bigvee_{(x,y) \in X \times Y} \bigwedge_{d \in D} (d,\tpl{q_{x,d},q_{y,d}})$.
 \end{itemize}

 
 
 Then the following \NFT accepts those trees $t$ that are accepted by both $T_1$ and $T_2$: 
 its states are $S$, its initial state is $\iota \otimes \iota'$, and its transition function on input $\sigma$ sends state $\tpl{q,q'}$ to formula 
 $\delta(q,\sigma) \otimes \delta'(q',\sigma)$.
 \todo{This construction needs to be checked/debugged}
 \end{fact}
 
 
Just as the acceptance condition of a tree automaton can be viewed as a ``membership game'', so too the emptiness check can be viewed as an ``emptiness game''. For \NFT this means that classic fixpoint algorithms (used for solving games) can be applied to solve the emptiness problem of tree automata, see, e.g., \cite{DBLP:conf/dagstuhl/2001automata}. 

\begin{fact}[Emptiness of \NFT] \label{fact:nft-emptiness}
Emptiness of \NFT is decidable in linear time. Moreover, if an \NFT $T$ is non-empty, one can also return a regular tree in $T$. 
\sr{Can be cut... Emptiness of \NFT $N$ is equivalent to deciding if Player Automaton has a winning strategy in the following two-player zero-sum game of perfect information played on $N$. The adversary is called Pathfinder. Play goes as follows: from a state $q$ Player Automaton picks a symbol $\sigma$ and a conjunct $\bigwedge_{d \in D} (d,q_d)$ in $\delta(q,\sigma)$, and player Pathfinder picks $d \in D$, and play proceeds to state $q_d$; play starts with Automaton picking a disjunct in $\iota$ and Pathfinder picking the next direction and state; if $\delta(q,\sigma) = \false$ then Automaton loses, and if $\delta(q,\sigma) = \true$ then Automaton wins.}
\end{fact}

\head{Automata for finite sequences \NFWf} We only need the definition of nondeterministic word automaton. For conciseness, we reuse our existing definitions.

% We define word automata. For conciseness we reuse the definition of tree automata. 
% A \emph{$\Sigma$-word} is a tree with $|D| = 1$. Define word automata as special cases of tree automata in which $|D| = 1$. To ease notation, we drop mention of $D$, and thus write formulas in $\B^+(Q)$. We write \ABW, \NBW, \UBW, \DBW, etc.


Formally, a \emph{finite $\Sigma$-word} is a partial function $t:D^+ \to \Sigma$ with $|D| = 1$, whose domain 
is finite and closed under taking non-empty prefixes. Define \NFWf to be like an \NFT except that it operates on finite words, 
and the membership game has the following additional rule:  from a position $(u,q)$, if the word has ended, then Automaton wins iff $q \in B$. 
Thus, \NFWf is simply a notational variants of the usual nondetermistic word automata (sometimes denoted NFA in the literature).

 

%% Preliminaries: synthesis, realisability

\section{Preliminaries} \label{sec:prelims}

An \emph{alphabet} is a finite non-empty set $X$ of symbols, e.g., $\{0,1,2\}$. Let $X^+$ (resp. $X^*$) denote the set of non-empty (resp. possibly empty) finite sequences over alphabet $X$. The main technical objects of this paper are functions of the form $X^+ \to Y$ for alphabets $X,Y$. Such functions represent strategies in game-theory and trees in automata-theory. 

Sequences are indexed starting at $0$, thus we write $x = x_0 x_1 \cdots$. 



\subsection{Linear-time Temporal Logic (\LTL)} \label{sec:prelims:LTL}

%\LTL syntax is as follows.
%\head{Syntax}
	Fix a finite non-empty set $\AP$ of atomic propositions.
	The \emph{formulas of \LTL (over $\AP$)} are generated by the following grammar:
	$\varphi \!::=\! p \!\mid\! \varphi \vee \varphi \!\mid\! \neg \varphi \!\mid\!  \nextX \! \varphi \!\mid \! \varphi \until \varphi$
	where $p \in \AP$. 
	
	Formulas are interpreted over infinite traces $\alpha \in (2^{AP})^\omega$. Define the satisfaction relation
	$\models$ as follows:
	\begin{enumerate}
	\item  $(\alpha,n) \models p$ iff $p \in \alpha_n$;
% 	\vspace{-0,35em}
	\item  $(\alpha,n) \models \varphi_1 \vee \varphi_2$ iff $(\alpha,n) \models \varphi_i$ for some $i \in \{1,2\}$;
% 	\vspace{-0,35em}
	\item 	$(\alpha,n) \models \neg \varphi$ iff it is not the case that $(\alpha,n) \models \varphi$;
% 	\vspace{-0,35em}
	\item   $(\alpha,n) \models \nextX \varphi$ iff $(\alpha,n+1) \models \varphi$;
% 	\vspace{-0,35em}
	\item  $(\alpha,n) \models \varphi_1 \until \varphi_2$ iff there exists $i \geq n$ such that $(\alpha,i) \models \varphi_2$ and for all $i \leq j < n$, $(\alpha,j) \models \varphi_1$.
	\end{enumerate}
% 	\renewcommand{\baselinestretch}{1}
	Write $\alpha \models \varphi$ if $(\alpha,0) \models \varphi$ and say that $\alpha$ \emph{satisfies} $\varphi$ and that $\alpha$ is a \emph{model} of $\varphi$.
	We use the following abbreviations, $\varphi \limp \varphi' \doteq \neg \varphi \vee \varphi'$, $\true := p \vee \neg p$, $\false \doteq \neg \true$, $\eventually \varphi \doteq \true \until \varphi$, and
	$\always \varphi \doteq \varphi \until \false$.
% 	$\varphi \releases \varphi' \doteq \neg (\neg \varphi \until \neg \varphi')$ (read ``$\varphi$ releases $\varphi'$'', with the meaning that either $\varphi'$ always holds, or if it fails then at some strictly earlier point in time $\varphi$ must have held).
	%$\varphi \weakuntil \varphi' \doteq (\varphi \until \varphi') \vee \always \varphi$.
	
%	We write $mod(\phi)$ for the words that satisfy $\phi$.


\subsection{Reactive Realizability and Synthesis} \label{sec:prelims:synthesis}

Reactive Synthesis is the problem of producing a finite-state reactive module that satisfies a given property no matter how the environment behaves. 
For the most part we follow the notation in \cite{PnRo89}.\footnote{As in the modern literature, % e.g.,\cite{KlPn10}, 
we have conveniently simplified the formulation in \cite{PnRo89} by considering the predicates for each player to be Boolean variables, rather than predicates over terms over static and dynamic variables.} The set of propositions $\AP$ is partitioned into two sets: those controllable by the agent $\apa$, and those not controllable by the agent $\ape$.
Let $\Sigma_{{\ag}} = 2^{\apa}$ and $\Sigma_{{\env}} = 2^{\ape}$ be the corresponding sets of {\em assignments}, and let $\Sigma = \Sigma_{\ag} \cup \Sigma_{\env}$.
A {\em reactive module} or \emph{agent-strategy} is a function $\sigma:(\Sigma_{\env})^+ \to \Sigma_{\ag}$. Say that $\sigma$ is {\em finite-state} if it is determined by a transducer with input alphabet $\Sigma_{\env}$ and output alphabet $\Sigma_{\ag}$.\footnote{A \emph{transducer} is a deterministic finite-state machine that is fed symbols from an input alphabet $I$, and symbol by symbol, it produces a symbol from an output alphabet $O$, and changes its internal state. A transducer determines a function $I^+ \to O$. \todo{Formalise this more?}
}
A sequence $\pi = \pi_1 \pi_2 \dots$ is {\em consistent with agent-strategy $\sigma$} if for every $k \geq 1$, 
  $\pi_{k} \cap \Sigma_{\ag} = \sigma((\pi_1 \cap \Sigma_{\env}) \cdots (\pi_k \cap \Sigma_{\env}))$.
 
% \begin{definition}[Realizable]
% A strategy $\sigma$ \emph{realizes} an \LTL formula $\varphi$, written $\sigma \sat \varphi$, if every
% $\pi$ consistent with $\sigma$ satisfies $\varphi$.  
% \end{definition}





In \LTL Reactive Synthesis the full specification is given as a single \LTL formula $\varphi$.

\begin{definition}[\LTL Realisability]
Given an \LTL-formula $\varphi$ over alphabet $\Sigma = \Sigma_{\ag} \cup \Sigma_{\env}$, decide if there exists an agent-strategy 
$\sigma$ such that every play $\pi$ consistent with $\sigma$ satisfies $\varphi$. In this case, say that $\sigma$ \emph{realises} $\varphi$, and write 
$\sigma \sat \varphi$.
\end{definition}


\begin{definition}[\LTL Synthesis]
 Assume that $\varphi$ is realisable. The synthesis problem asks to return a finite-state agent-strategy realising $\varphi$.
\end{definition}


\begin{theorem}\label{thm:classic}
\LTL realizability is $2$\exptime-complete, and synthesis can be solved in $2$\exptime. \todo{Does it make sense to talk about synthesis being complete?}
\end{theorem}

{\bf Notation.} 
From now on we drop the adjective ``Reactive`` and just say, e.g., ``\LTL Synthesis'' instead of ``\LTL Reactive Synthesis''.

% A {\em history for ${\env}$} is a sequence from  $(\Sigma_{\ag})^*$, collectively denoted $\hist_{\env}$,  
% A {\em history for ${\ag}$} is a sequence from  $(\Sigma_{\env})^+$, collectively denoted $\hist$.
% An {\em $x$-strategy} is a function $\sigma:(\Sigma_{\env})^+ \to \Sigma_x$ (for $x \in \{{\ag},{\env}\}$).
% A play $\pi = \pi_1 \pi_2 \dots$ is {\em consistent with the strategy $\sigma$} if for every $k \geq 1$, 
%   $\pi_{k} \cap \Sigma_{\ag} = \sigma((\pi_1 \cap \Sigma_{\env}) \cdots (\pi_k \cap \Sigma_{\env}))$; 
%   and a play $\pi = \pi_1 \pi_2 \dots$ is {\em consistent with an {\env}-strategy $\sigma$} if for every $k \geq 0$, 
%   $\pi_{k+1} \cap \Sigma_{\env} = \sigma((\pi_1 \cap \Sigma_{\ag}) \cdots (\pi_k \cap \Sigma_{\ag}))$.
% For an $x$-strategy $\sigma$, write $\sigma \sat \varphi$ if every play consistent with $\sigma$ satisfies $\varphi$, and say that $\sigma$ realises $\phi$.
% The agent has an objective $\phi$, and because the game is zero-sum, the environment's objective is $\neg \phi$.


\subsection{Algorithm and Complexity} \label{sec:algorithm}

We now give the building-blocks that will be used in our Algorithm.%~\ref{alg:LTL-SuA}.

\begin{proposition}[From \LTL to \NBW] \label{prop:VW}
For every \LTL formula $\psi$ we can build an \NBW $N_\psi$ of size $2^{O(|\psi|)}$ that accepts the models of $\psi$~\cite{DBLP:journals/iandc/VardiW94}.  
\end{proposition}

\begin{lemma} \label{lem:existspath}
Let $\phi$ be an \LTL formula. 
There is an \NBT $T_\exists$ of size $2^{O(|\phi|)}$ that accepts all strategies $\sigma$ such that $\sigma \not \sat \neg \phi$.
\end{lemma}

\begin{proof}
We use the following fact: For every \NBW $N$ there is an \NBT $T$ of size linear in $N$ that accepts all strategies $\sigma$ such that some play consistent with $\sigma$ is labeled by an infinite word accepted by $N$. To see this, the \NBT $T$ guesses the path and runs the \NBW $N$ on that path, guessing the accepting run of $N$. That is, given $N = (Q,\iota,\delta,B)$ define $T = (Q,\iota',\delta',B)$ by replacing every formula $\bigvee_i q_i$ of $N$ by $\bigvee_i \bigvee_{d \in D} (d,q_i)$.

Apply Proposition~\ref{prop:VW} to $\phi$ to build an \NBW $N_\phi$ that accepts the models of $\phi$. By the fact, build an \NBT that accepts $\sigma$ iff some play consistent with $\sigma$ satisfies $\phi$, i.e., $\sigma \not \sat \neg \phi$.
\end{proof}


We now give the dual lemma.
\begin{lemma} \label{lem:allpaths}
Let $\phi$ be an \LTL formula. 
There is an \UCT $T_\forall$ of size $2^{O(|\phi|)}$ that accepts all strategies $\sigma$ such that $\sigma \sat \phi$.
\end{lemma}

\begin{proof}
Apply Lemma~\ref{lem:existspath} to $\neg \phi$ and build an \NBT that accepts all strategies $\sigma$ such that $\sigma \not \sat \phi$. Dualise the \NBT (by Fact~\ref{fact:dual}) 
to get a \UCT that accepts the complement, i.e., all strategies $\sigma$ such that $\sigma \sat \phi$.
\end{proof}



We are ready to give an algorithm for the upper bound of Theorem~\ref{thm:synth-under-assumption}.

\begin{enumerate}
 \item Build \NBT $T_1$ by Lemma~\ref{lem:existspath} applied to $\formula{\asmp}$. 
 \item Build \UCT $T_2$ by Lemma~\ref{lem:allpaths} applied to $\formula{\asmp} \limp \formula{\goal}$.
 \item Build \ABT $T$ by Fact~\ref{fact:intersection} as the intersection of $T_1$ and $T_2$.
 \item Test the emptiness, by Fact~\ref{fact:ABT-emptiness}, of the \ABT $T$, and if non-empty return a finite-state strategy.
\end{enumerate}
  

This algorithm constructs an \ABT $T$ that accepts exactly the strategies $\sigma$ such that i) $\sigma \not \sat \neg \formula{\asmp}$ and ii) $\sigma \sat \formula{\asmp} \limp \formula{\goal}$. Indeed, the \NBT $T_1$ accepts the $\sigma$ that satisfy i) and the \UCT $T_2$ accepts the $\sigma$ satisfying ii). 
The size of $T$ is $|T_1| + |T_2|  = 2^{O(|\formula{\asmp}| + |\formula{\goal}|)}$. By Fact~\ref{fact:ABT-emptiness}, testing for $T$'s emptiness results in a total cost of $2^{2^{O(|\formula{\asmp}| + |\formula{\goal}|)}}$.





\subsection{Algorithm and Complexity}

The following is proved in \cite{DBLP:conf/ijcai/GiacomoV15}, and is the finitary version of Proposition~\ref{prop:VW}.
\begin{proposition} \label{prop:VW-finite} 
For every \LTLf formula $\phi$ there is a \NFWf $N_\phi$ of size $2^{O(|\phi|)}$ that accepts the models of $\phi$.
\end{proposition}

% \begin{proof} First put $\phi$ into negation-normal form by introducing the dual operators: $\wedge$ (the dual of $\vee$), $\releases$ (the dual of $\until$), and $\Wnext$ (the dual of $\nextX$), and then
% pushing negations to the atoms (linear time). Then form an \AFWf $A_\phi$ (linear size) by
% translating the logic of the formula into the logic of the \AFWf and using
% one-step unfoldings for the temporal operators $\until$ and $\releases$.  Then
% build an equivalent \NFWf $N_\phi$ by removing alternation using a subset
% construction~\cite{ChandraKozenStockmeyer?}.  \sr{give algorithms, or cite GdGVardi} 
% \end{proof}


We now refine Lemmas~\ref{lem:existspath} and~\ref{lem:allpaths} for finite sequences. We remark that strategies are still infinite trees, 
and thus we will use \NFT and \DFT to define sets of strategies. The first lemma is proved just as Lemma~\ref{lem:existspath} but replacing Proposition~\ref{prop:VW} by Proposition~\ref{prop:VW-finite}.

\begin{lemma} \label{lem:existspath-finite}
Let $\phi$ be an \LTLf formula. 
There is an \NFT $T_\exists$ of size $2^{O(|\phi|)}$ that accepts all strategies $\sigma$ such that $\sigma \not \sat \neg \phi$. 
\end{lemma}


% \begin{proof}
%             To see the first item, apply Proposition~\ref{prop:VW-finite} to $\phi$ to build
%             an \NFWf $N_\phi$ that accepts the models of $\phi$. By Lemma~\ref{lem:guess path},
%             build an \NFT $T_\exists$ that accepts $\sigma$ iff $\sigma \not \sat \neg \phi$. This is done
%             by guessing a finite path and a finite run of the \NFWf $N_\phi$.
%             
%             To see the second item, build the \NFWf $N_\phi$ as before, and then determinise it, i.e., build an equivalent 
%             \DFWf $D_\phi$ by removing non-determinism using the classic subset construction~\cite{Rabin-Scott}. Note that the size
%             of $D_\phi$ is exponential in the size of $N_\phi$.
%             As in the proof of Lemma~\ref{lem:guess path}, build a \DFT $T_\forall$ that runs the \DFWf $D_\phi$ on all paths of the input tree $t$. 
% \end{proof}

The second lemma is similar to Lemma~\ref{lem:allpaths} except that we determinise the formula $\varphi$ (paying the extra exponent now, rather than later).
\begin{lemma} \label{lem:allpaths-finite}
Let $\phi$ be an \LTLf formula. There is an \DFT $T_\forall$ of size $2^{2^{O(|\phi|)}}$ that accepts all strategies $\sigma$ such that 
$\sigma \sat \phi$. Note that $T_\forall$ is deterministic.
\end{lemma}

\subsection{Algorithm and Complexity} \label{sec:alg:finite}

Here is the algorithm for the upper bound in Theorem~\ref{thm:synth-under-assumption:finite}.

\begin{enumerate}
\item Apply Lemma~\ref{lem:existspath-finite} to formula $\formula{\asmp}$ to get \NFT $T_1$.
\item Apply Lemma~\ref{lem:allpaths-finite} to formula $\formula{\asmp} \limp \formula{\goal}$ to get \DFT $T_2$.
\item Apply Fact~\ref{fact:nft-intersection} to get an \NFT $T$ whose language is the intersection of the languages of $T_1$ and $T_2$.
\item Use Fact~\ref{fact:nft-emptiness} to test the emptiness of the \NFT $T$, and if non-empty, return a regular tree which encodes the desired finite-state ${\ag}$-strategy.
\end{enumerate}


The size of $T$ is $|T_1| \times |T_2|$, and emptiness of $T$ is linear in $|T|$; thus the total cost of this algorithm is 
$2^{2^{O(|\formula{\asmp}| + |\formula{\goal}|)}}$.






%%% Local Variables:
%%% mode: latex
%%% TeX-master: "main"
%%% save-place: t
%%% End:



\bibliographystyle{plainnat}
\bibliography{lit}


 \backmatter                    % back matter


\end{document}
