\documentclass[a4paper,10pt]{article}
\usepackage[utf8]{inputenc}

%opening
\title{project/paper}
\author{}
\date{}
\usepackage{hyperref}

\begin{document}

\maketitle

\section*{Optimal FONDPs}

The aim of this project is to study the theoretical and practical problem 
of finding optimal solutions to fully-observable non-deterministic planning (FOND) problems.

Given a FOND planning domain, define
\begin{itemize}
 \item the *cost of a plan* to be the sum of the costs of the actions up until
 the plan reaches the goal (and $\infty$ if the goal is never reached).
 \item the *cost of a policy* to be the sup of the costs of all plans
  generated by the policy (i.e., no matter what the environment does). 
  \item *policy A is better than policy B* if the cost of policy A is
  smaller than the cost of policy B.
\end{itemize}

Such planning domains amount to "min-cost reachability games" from the
verification literature [1]. That paper shows that such games have
values that can be computed. In particular, the cost of a given
finite-state policy can be computed, and thus one can decide the
"better than" relation between finite-state policies.

[1] http://dblp.org/rec/conf/concur/BrihayeGHM1


What is there to do:
\begin{enumerate}
 \item do full literature search
 \item formalise the above, give examples, supply tight-complexity bonds
 \item reduce the search for optimal plans to the search for non-optimal plans, and run experiments using FOND planners.
\end{enumerate}

% \bibliographystyle{plain}
% \bibliography{refs}

\end{document}
