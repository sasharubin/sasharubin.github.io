\documentclass[10pt,a4paper,sans]{moderncv}       
% possible options include font size ('10pt', '11pt' and '12pt'), paper size ('a4paper', 'letterpaper', 'a5paper', 'legalpaper', 'executivepaper' and 'landscape') and font family ('sans' and 'roman')
% moderncv themes
\moderncvstyle{classic}                             % style options are 'casual' (default), 'classic', 'banking', 'oldstyle' and 'fancy'
\moderncvcolor{blue}                               % color options 'black', 'blue' (default), 'burgundy', 'green', 'grey', 'orange', 'purple' and 'red'
%\renewcommand{\familydefault}{\sfdefault}         % to set the default font; use '\sfdefault' for the default sans serif font, '\rmdefault' for the default roman one, or any tex font name
%\nopagenumbers{}                                  % uncomment to suppress automatic page numbering for CVs longer than one page
% character encoding
%\usepackage[utf8]{inputenc}                       % if you are not using xelatex ou lualatex, replace by the encoding you are using
%\usepackage{CJKutf8}                              % if you need to use CJK to typeset your resume in Chinese, Japanese or Korean
% adjust the page margins
\usepackage[scale=0.75]{geometry}
%\setlength{\hintscolumnwidth}{3cm}                % if you want to change the width of the column with the dates
%\setlength{\makecvtitlenamewidth}{10cm}           % for the 'classic' style, if you want to force the width allocated to your name and avoid line breaks. be careful though, the length is normally calculated to avoid any overlap with your personal info; use this at your own typographical risks...
% personal data
\name{Sasha}{Rubin}
\title{Curriculum Vitae, 12/2017}                               % optional, remove / comment the line if not wanted
\address{University of Naples ``Federico II''}{Naples, Italy}% optional, remove / comment the line if not wanted; the "postcode city" and "country" arguments can be omitted or provided empty
% \phone[mobile]{+1~(234)~567~890}                   % optional, remove / comment the line if not wanted; the optional "type" of the phone can be "mobile" (default), "fixed" or "fax"
% \phone[fixed]{+2~(345)~678~901}
% \phone[fax]{+3~(456)~789~012}
\email{rubin@unina.it}                               % optional, remove / comment the line if not wanted
\homepage{sasharubin.github.io}                         % optional, remove / comment the line if not wanted
% \social[linkedin]{john.doe}                        % optional, remove / comment the line if not wanted
% \social[twitter]{jdoe}                             % optional, remove / comment the line if not wanted
% \social[github]{jdoe}                              % optional, remove / comment the line if not wanted
% \extrainfo{additional information}                 % optional, remove / comment the line if not wanted
\photo[70pt][0.4pt]{../../data/RUBIN_Sasha.jpg}                       % optional, remove / comment the line if not wanted; '64pt' is the height the picture must be resized to, 0.4pt is the thickness of the frame around it (put it to 0pt for no frame) and 'picture' is the name of the picture file
% \quote{Some quote}                                 % optional, remove / comment the line if not wanted

\newif\ifmetrics
\metricstrue

\newif\ifref
% \reftrue

\usepackage[gen]{eurosym}

%%%%%%%%%%%%%%%%%%%%%%%%%%%%%%%%%%%%%
%%%%%%%%%% BIBLIO STUFF %%%%%%%%%%%%%
\usepackage[backend=bibtex, sorting=ydnt,maxbibnames = 10,citestyle = ieee,defernumbers]{biblatex}

\bibliography{../../data/rubin-core.bib}


\def\TOTALCITATIONS{796}
\def\citeKNRS04{100} % 100 in Dec 2017, 97 in Sept 2017, 96 in July 2017, 95 in April 2017, 94 in Feb 2017, 91 in July 2016, 
\def\citeRubin08{93} % 93 in dec 17, 91 in Sept 17, 90 in July 2017,  87 in April 2017, 85 in Feb 2017, 85 in Sept 2016, 84 in July 2016 
\def\citeThesis{73} %  73 in dec 17, 72 in July 2017, 67 in Feb 2017, 64 in August 2016
\def\citeAminofJKR14{36} % 36 in sept 17, 34 in Feb 2017, 
\def\citeBloem15{31} % 31, 30 in dec 17, 25 in sept 17
  

%%%%%%%%%%%%%%%%%%%%%%%%%%%%%%%%%%%%%%%%%%%%



\DeclareRangeChars{~,;-+/{}} % add '{}' as page range delimiter

% \addbibresource{\jobname.bib}

\DeclareFieldFormat{file}{\hfill \textbf{#1}}

\AtEveryBibitem{%
    \csappto{blx@bbx@\thefield{entrytype}}{% put at end of entry
        \iffieldundef{file}{}{%
       \space \printfield{file}
% 		{}
      }
    }
  }

% \usepackage{hyperref}

% bibliography adjustements (only useful if you make citations in your resume, or print a list of publications using BibTeX)
%   to show numerical labels in the bibliography (default is to show no labels)
% \makeatletter\renewcommand*{\bibliographyitemlabel}{\@biblabel{\arabic{enumiv}}}
\makeatletter\renewcommand*{\bibliographyitemlabel}{\@biblabel{\arabic{enumiv}}}\makeatother

%   to redefine the bibliography heading string ("Publications")
%\renewcommand{\refname}{Articles}

% bibliography with mutiple entries
%\usepackage{multibib}
%\newcites{book,misc}{{Books},{Others}}
%----------------------------------------------------------------------------------
%            content
%----------------------------------------------------------------------------------
\begin{document}
%\begin{CJK*}{UTF8}{gbsn}                          % to typeset your resume in Chinese using CJK
%-----       resume       ---------------------------------------------------------
\makecvtitle


\section{Personal Information}
% \cvitem{Date of Birth}{16.02.76}
% \cvitem{Place of Birth}{South Africa}
\cvitem{Citizenship}{New Zealand}
\cvitem{Languages}{English (first language), Italian (beginner)}
% , Italian (basic)}
% <Afrikaans (basic)}
% \cventry{year--year}{Degree}{Institution}{City}{\textit{Grade}}{Description}

\section{Current Appointment}
\cventry{2017-present}{PostDoc. (Computer Science)}{University of Naples ``Federico II''}{Department of Electrical Engineering and Information Technology}{Fellow of the ASTREA lab headed by Aniello Murano, for the study of formal methods, artificial intelligence, and multi-agent systems}{}
% 1/5/17 - 30/4/18
% “MODELLI E METODI PER LA METROLOGIA AVANZATA” IN RIFERIMENTO AL WORK PACKAGE “WP2 - TECNOLOGIE ABILITANTI PER L’INTERNET OF EVERYTHING” - ID PROGETTO UGOV 27635 - CODICE PROGETTO UGOV 040489_-_Ce.S.M.A._-_(Prog._ric._int.)_3 - DENOMINAZIONE PROGETTO UGOV 040489 - Modelli e Metodi per la Metrologia Avanzata.  

\section{Academic Qualifications}
\cventry{Begun 1999,\\Defence 2004,\\Awarded 2007.}{PhD}{Mathematics and Computer Science}{University of Auckland}{Best Doctoral Thesis in the Faculty of Science (one of seven awardees in 2004)}
{}
\cventry{1997-1998}{MSc}{Mathematics}{University of Auckland}{First Class}{}
\cventry{1994-1996}{BSc}{Department  of Mathematics and Department  of Computer Science}{University of Cape Town}{Dean's Merit List}{}
% awarded 7/5/20007, completed 2004

\section{Previous Appointments}
\cventry{2015-2017}{Marie Curie Fellow of INdAM ``F. Severi''}{University of Naples ``Federico II''}
{}{}
{Research in collaboration with Aniello Murano and his group}
%? 1/03/15 - 28/2/17

\cventry{2014-2015}{PostDoc. (Computer Science)}{TU Wien and TU Graz}
{}{}
{Research in collaboration with Helmut Veith and Roderick Bloem, and their groups} 
% project assistant, ``Rigourous systems engineering (RiSE)''
% 1/11/14 - 31/1/15 (Graz)


\cventry{2012-2014}{PostDoc. (Computer Science)}{TU Wien and IST Austria}
{}{}
{Research in collaboration with Helmut Veith and Krishnendu Chatterjee and their groups}% project assistant, ``Rigourous systems engineering (RiSE)''
% 13/3/12 - 12/3/13 (TU Wien/IST)
% 13/3/13 - 12/3/14 (TU Wien/IST)
% -- 30/9/14 (TU Wien)

\cventry{2010-2012}{Honorary Research Fellow (Computer Science)}{University of Auckland}
{}{}
{}

\cventry{}{Visiting Researcher (Computer Science)}{University of Tel Aviv}
{}{}
{Research in collaboration with Alexander Rabinovich, one semester in 2011}
% $5.2011$ -- $8.2011$\\


\cventry{}{Visiting Lecturer (Mathematics)}{University of Cape Town}
{}{}
{1 semester in 2010}
% \cventry{2010}{Visiting Researcher (Computer Science)}{University of Auckland}{}{}{}{}

\cventry{2008-2009}{Visiting Assistant Professor (Mathematics)}{Cornell University}
{}{}
{Lecturer for 3 semesters}{} % aug 09 - dec 10 , and Research in collaboration with Anil Nerode and his group,

\cventry{2007-2008}{Visiting Researcher (Computer Science)}{University of Auckland}{}{}{}{}

\cventry{2004-2007}{Foundation for Research, Science and Technology (FRST) New Zealand Science and Technology (NZST) Postdoctoral Fellowship}{University of Auckland}
{}{}
{NZ Government funded 3 year fellowship} % 1/12/2004 until 30/11/2007.





\newpage

\section{Research Portfolio}
% I am a computer scientist with an interest in Artificial Intelligence, especially knowledge representation, logic, multi-agent systems, and automated planning. 
My main research interest is in artificial intelligence, including logic, knowledge representation, multi-agent systems and automated planning. %particularly strategic and epistemic reasoning for multi-agent systems.
% I work in formal methods, a branch of theoretical computer science, and study the power of automata theory (broadly construed) and mathematical logic for describing, reasoning and controlling systems. 
I have contributed to the following areas:
\begin{itemize}
 \item Formal methods (modeling, verification, synthesis) of Multi-agent Systems (including parameterised systems, distributed systems, probabilistic systems, timed systems).
 \item Logics (for games and strategic reasoning)
 \item Foundations of planning and generalised planning.
 \item Automated reasoning (including verification and synthesis).
 \item Automata theory.
 \item Finite and algorithmic model theory.
\end{itemize}


% (including Parameterised Systems, Probabilistic Systems, Distributed Systems, Timed Systems; 


% \subsection{ACM Computing Classification System}
% \cvitem{}{Theory of Computation [Models of Computation, Logic, Formal Languages and Automata Theory]}
% 
% \cvitem{}{Computing Methodologies [Artificial Intelligence: Planning and Scheduling, Knowledge Representation and Reasoning, Distributed Artificial Intelligence]}
% 
% % \subsection{Mathematics Subject Classification}
% % {03Bxx (General Logic), 03Cxx (Model Theory); 68Qxx (Theory of Computing), 68Txx (Artificial Intelligence)}



\subsection{Research Accomplishments} 

Citation source: google scholar 10/12/17
% : \newline \url{scholar.google.com/citations?hl=en&user=auUS1rMAAAAJ}.

%list  their  five  most  important publications since their last review, along with brief explanations  of  why  each  paper  is  significant.

% \cvitem{Output}{
% I published 16 articles in CORE A* conferences, 7 in CORE A conferences, 4 in CORE B conferences; and 4 articles in Q1 journals, 2 articles in Q2 journals, 1 book, and 1 book chapter.}


\cvitem{2017}{I published 5 CORE A* conference papers in 2017~\cite{DBLPconflicsBerthonMMRV17,BDGR17,GMPRW17,BLMRIJCAI17,BLMR17}.}

\cvitem{2016}{I published 5 CORE A* conference papers in 2016~\cite{GMRSIJCAI16,DBLPconfatalAminofMMR16,DBLPconfkrAminofMRZ16,DBLPconfatalAminofMRZ16,DBLPconfcadeAminofR16}.}


\cvitem{2016-2017}{I teamed up with world-experts in automated planning and multi-agent systems and made theoretical and practical contributions to verification and synthesis under imperfect information. In particular, I extended logics for strategic reasoning in two ways: allowing imperfect information (with Moshe Vardi et.al.~\cite{DBLPconflicsBerthonMMRV17} and Alessio Lomuscio et.al.~\cite{BLMRIJCAI17}), and by graded modalities (with Aniello Murano et.al.~\cite{AMMRSRjournal16}). 
I extended the belief-space construction to infinite games (with Giuseppe De Giacomo et. al.~\cite{GMRSIJCAI16}), and
I studied an application of these ideas to Generalised Planning (with Giuseppe de Giacomo, Blai Bonet, and Hector Geffner)~\cite{BDGR17}.}

\cvitem{2014-2015}{I opened the direction of formal methods for parameterised light-weight mobile agents with \cite{DBLPconfatalRubin15}.  Subsequently (with my co-authors) I continued this direction with \cite{DBLPconfprimaRubinZMA15} (which won a best-paper award) and \cite{DBLPconfatalAminofMRZ16}.}

\cvitem{2015}{I co-authored a book surveying decidability results in parameterised verification \cite{DBLPseriessynthesis2015Bloem}, 
published by Morgan \& Claypool, (\citeBloem15 citations).}

\cvitem{2012-2014}{I (with my co-authors) generalised a cornerstone paper on verification of parameterised systems ("Reasoning about Rings", E.A. Emerson, K.S. Namjoshi, {POPL}, 1995) from ring topologies to arbitrary topologies (\citeAminofJKR14\ citations) \cite{DBLPconfvmcaiAminofJKR14}.}

\cvitem{2008-2011}{I published a survey and extension of the main results in my thesis in the Bulletin of Symbolic Logic \cite{DBLPjournalsbslRubin08}. 
With a PhD student of Erich Gr\"adel's (Tobias Ganzow) I solved a 12 year-old conjecture of Courcelle's \cite{DBLPconfstacsGanzowR08}.}

\cvitem{1999-2007}{During and after my PhD I (and my co-authors) pioneered the development of the theory of automatic structures. My most cited publications in this area are:
\cite{DBLPconflicsKhoussainovNRS04} (\citeKNRS04\ citations) and 
\cite{DBLPjournalsbslRubin08} (\citeRubin08\ citations). My PhD thesis has \citeThesis\ citations.}
% My PhD thesis (\citeThesis\ citations) was awarded  the Vice-chancellor's prize for the best doctoral thesis in the Faculty of Science (one of 7 in 2004), and Montgomery memorial prize in logic from the Department of Philosophy. }



\subsection{Bibliometrics}

\cvitem{conf. papers}{16 x CORE A* (including 5 LICS, 4 IJCAI, 4 AAMAS, 1 KR, 1 IJCAR, 1 CAV), 7 x CORE A and 5 x CORE B}
\cvitem{journal articles}{3 x SJR Q1, and 4 x SJR Q2}
% \cvitem{citations}{\TOTALCITATIONS citations on Google Scholar}
% \cvitem{H-index}{According to Google Scholar, my H-index is 15 (since 2002), and 12 (since 2012)}



\section{Awards}
\cventry{PhD Prize}{Best doctoral thesis in the Faculty of Science, University of Auckland, 2004}{At most five awarded per year}{}{}{}
\cventry{PhD Prize}{Montgomery memorial prize in logic from the Department of Philosophy, 2004}{At most one awarded per year}{}{}{}
\cventry{Paper Prize}{Best-paper award at PRIMA 2015, \cite{DBLPconfprimaRubinZMA15}}{One awarded per year}{}{}{}{}
\cventry{Competition}{As part of a team of three, won the national heats and represented New Zealand in the world finals of the 1998 ACM Programming Contest}{Atlanta, Georgia USA}{}{}{}{}

\section{Funding and Grant writing}
\subsection{Individual Funding}
\cventry{2004-2007}
{Foundation for Research, Science and Technology (FRST) New Zealand Science and Technology (NZST) Post Doctoral Fellowship (n. UOAX0413, ``Automatic Structures'')}
{3 years}
{NZ\$224532}
{}{}


\cventry{2015-2016}
{Marie Curie Fellowship of INdAM ``F. Severi''}
{2 years}
{\euro{107000}}
{one of four awarded in 2015}{}

% {\url{https://cofund.altamatematica.it/2012/main/website?page=call-1}} %44310} + RCC \euro{5800} + Travel \euro{2000}}{}{} Ranked 4th out of 27 applicants 

\cventry{2011}
{Exchange Grant within the framework of the European Science Foundation (ESF) activity on ``Games for Design and Verification``}
{12 weeks}
{\euro{5300}}
{}{}

\cventry{2015-2016}
{GNSAGA grants for research travel}
{}
{\euro{4909}}
{}{}

\cventry{2011}
{Short Visit Grant (n. 4391) of the European Science Foundation (ESF) activity on ``Games for Design and Verification``}
{15 days}
{\euro{1475}}{}{}

\cventry{2011}
{Short Visit Grant (n. 4500) of the European Science Foundation (ESF) activity on ``Games for Design and Verification``}
{10 days}
{\euro{1350}}{}{}

% \cventry{2017}
% {\'Ecole normale sup\'erieure (ENS) de Rennes grant for la visiting Sophie Pinchinat and her group}{Finanziamenti ottenuto e visita da realizzare nell mese di 12/2017}
% {}{}{} %}{3 giorni}{\euro{600}}{Pianificato 12/2017}{}


\subsection{Grant writing}
\cvitem{2017}{Assisted Aniello Murano with writing a Programma Operativo Nazionale (PON) ''Ricerca e Innovazione`` 2014-2020 grant application.} %ERC
\cvitem{2017}{Assisted Giuseppe De Giacomo with writing an ERC grant application.} %ERC
\cvitem{2017}{Assisted Aniello Murano's postdoc with writing an INdAM postdoctoral application.}
\cvitem{2016-2017}{Assisted Florian Zuleger with writing an Austrian Science Fund grant application.}
\cvitem{2014}{Assisted Helmut Veith with writing and editing Austrian Science Fund grant applications and reports for the National Research Network (NFN). \newline \url{http://arise.or.at/}}



\newpage


\section{Supervision and Mentoring}

I've supervised/mentored a number of students, resulting in publications.

\subsection{PhD Mentoring}
\cvitem{2015-2017}{Closely worked with PhD students of Aniello Murano (Vadim Malvone, Antonio di Stasio, Loredanna Sorrentino) and produced \cite{DBLPconfatalAminofMMR16,AMMRSR16,GMRSIJCAI16}}
\cvitem{2007}{Closely worked with a PhD student of Erich Gradel's (Tobias Ganzow) and solved a 12-year open problem \cite{DBLPconfstacsGanzowR08}}

\subsection{Masters Supervision}
% While at Cornell I mentored six students for three months. This resulted in two publications \cite{DBLP:journals/tcs/GrinshpunPRT14,DBLP:journals/corr/abs-1210-2462}.
% %and gave students a taste of research to help them decide if they should pursue a PhD. 
% While at IST Austria I co-mentored one intern which resulted in \cite{DBLP:conf/lata/ChatterjeeCR13}. 
% % I recently co-mentored an undergraduate thesis at the University of Naples.

\cventry{2017}{Masters Internship: Aurele Barriere}{University of Naples}{Topic: Epistemic Logic}{}{co-supervision, ongoing} %(4 months)

\subsection{Undergraduate Supervision}
\cventry{2017}{Undergraduate thesis: Paolo Lambiase}{University of Naples}{Topic: Graphical Games}{}{co-supervision, ongoing} %(4 months)
\cventry{2012}{Summer undergraduate project: Siddhesh Chaubal}{IST Austria}{Topic: Edit-distance and Formal Languages}{}{co-supervision, produced \cite{DBLPconflataChatterjeeCR13}} %(3 months)
\cventry{2009}{Summer research experience for undergraduates: Andrey Grinshpun, Pakawat Phalitnonkiat, Andrei Tarfulea}{Cornell University}{Topic: Parity Games}{}{supervision, produced \cite{DBLPjournalstcsGrinshpunPRT14}} % (3 months)
\cventry{2009}{Summer research experience for undergraduates: Alex Kruckman, John Sheridan, Ben Zax}{Cornell University}{Topic: Automatic Structures with Advice}{}{supervision, produced \cite{DBLPjournalscorrabs-1210-2462}} %(3 months)

\section{Teaching}

\subsection{Graduate courses}
\cventry{2017}{PhD course}{Milestones in Solving Games on Graphs}{Technical University of Vienna}{Duties: design, present and examine}{} %20 hours, 7 students 
\cventry{2017}{PhD course}{Games on Graphs}{University of Naples}{Duties: designed and presented}{} %9 attendants 10 hour 
\cventry{2009}{PhD course}{Logical Definability and Random Graphs}{Cornell University}{Duties: designed and presented}{} %5 attendants 1 semester

\subsection{School course}
\cventry{2006}{Advanced course}{Logic and Computation in Finitely Presentable Infinite Structures}{European Summer School in Logic, Language and Information (ESSLLI 2006)}{Duties: designed and presented}{co-taught} %5 day , approx. 20 attendants


% I have a proactive approach to learning best-teaching practices. 

\subsection{Undergraduate courses}
% While  at Cornell, I sought out a number of teaching mentors including Maria Terrell (Department of Mathematics) and David Way (associate director of the Cornell University Centre for Teaching Excellence) to discuss successful teaching strategies, both philosophical and concrete. 
% According to my student evaluations, I was clear, organised, proactively willing to help, and motivating.

% \cventry{2016/2017}{System specification}{University of Naples ''Federico II``}{Duties: Teaching assistant}{}{}{}{}
\cventry{2015/2016}{System specification}{University of Naples ''Federico II``}{Duties: Teaching assistant}{}{}{}{}

\cventry{2010}{Logic and Computation}{Department of Mathematics, University of Cape Town}{Duties: Lecturer, course design}{}{}{}{} %, 40 students, 30 lectures, 12 tutorials, 1 class test, 1 final exam}

\cventry{2008-2009}{Calculus for Engineers}{Department of Mathematics, Cornell University}{Duties: Lecturer, weekly online quizzes, marking}{}{taught the course 5 times} %25-30 students,




%Computational Biology (undergraduate, TA)\\
%{Department of Computer Science, University of Auckland} $(2008)$\\


\cventry{2007}{Discrete Structures in Mathematics and Computer Science}{Department of Computer Science, University of Auckland}{Duties:  Lecturerer (including tutorials), course design}{}{co-taught}{}
\cventry{2007}{Mathematical Foundations of Software Engineering }{Department of Computer Science, University of Auckland}{Duties:  Lectures (including tutorials), course design}{}{co-taught}{}
\cventry{2003}{Introduction to Formal Verification}{Department of Computer Science, University of Auckland}{Duties:  Lecturer (including tutorials), course design}{}{co-taught}{} 
\cventry{2002}{Automata theory}{Department of Computer Science, University of Auckland}{Duties:  Lecturer (including tutorials), course design}{}{co-taught}{}

\cventry{2000-2001}{Pre-calculus}{Department of Mathematics, University of Wisconsin, Madison}{Duties: Lecturer, tutorials, marking}{}{taught the course 2 times} %+-30 students, 



%  {\bf Tutoring}\\
%  {University of Auckland, Department of Mathematics} \\
%  $1998$ and $1999$: Undergraduate Mathematics\\


\iffalse $1998$ and $1999$: Stages $1$, $2$ and $3$ in the Mathematics \\
   	 %Assistance Room \\
	 $1999$: Assistant tutor for `Discrete Mathematics', Stage $2$
	Mathematics\\
	 $1999$: Demonstrator for `Combinatorial Computing', Stage $3$
	Mathematics\\
\fi

%\pagebreak 

% 
%  
\section{Dissemination and Outreach}


\subsection{Recent Talks of Accepted Papers}
\cventry{2017}{IJCAI, Melbourne}{Generalised Planning: Non-Deterministic Abstractions and Trajectory Constraints}{}{}{}
\cventry{2017}{FMAI, Naples}{Verification of Multi-agent Systems with Imperfect Information and Public Actions}{}{}{}
\cventry{2016}{SR, New York}{LTL Reactive Synthesis under Assumptions}{}{}{}
\cventry{2016}{KR, Cape Town}{Model Checking Prompt Alternating-Time Epistemic Logics}{}{}{}
\cventry{2016}{IJCAI, New York}{Imperfect-Information Games and Generalized Planning}{}{}{}
\cventry{2016}{AAMAS, Singapore}{Automatic Verification of Multi-Agent Systems in Parameterized Grid-Environments}{}{}{}
\cventry{2015}{PRIMA, Bertinoro}{Verification of Asynchronous Mobile-Robots in Partially-Known Environments}{}{}{}
\cventry{2015}{HIGHLIGHTS, Prague}{The Composition Method and Parameterised Verification}{}{}{}
\cventry{2015}{AAMAS, Istanbul}{Parameterised Verification of Autonomous Mobile-Agents}{}{}{}
\cventry{2014}{VMCAI, San Diego}{Cutoffs for Parameterised Token-Passing Systems}{}{}{}
\cventry{2014}{SR, Grenoble}{First Cycle Games}{}{}{}
\cventry{2014}{HIGHLIGHTS, Paris}{First Cycle Games}{}{}{}
\cventry{2014}{FRIDA, Vienna}{Using automata and logic to reason about
parameterised robot protocols}{}{}{}
\cventry{2013}{LATA, Bilbao}{How to Travel between Languages}{}{}{}

\subsection{Outreach}
\cvitem{2010}{I briefly volunteered at a secondary school in Accra, Ghana, teaching, observing and commenting on grade $5$ mathematics classes.}
\cvitem{2010}{I briefly volunteered in Khayelitsha, South Africa, helping high-school students prepare for their mathematics exams.}
\cvitem{2009}{I taught two interactive lectures to non-mathematics majors at Cornell University on i) Hilbert's Hotel and Infinite Cardinals and ii) Algorithms and Termination.}


% \newpage
 \section{Esteem}
 
%%%%%%%%%% COMMITTEES 
\subsection{Chair}
\cvitem{2017}{Co-chair of the Italian Conference on Theoretical Computer Science (ICTCS)\newline \url{http://ictcs2017.unina.it/}}
\cvitem{2017}{Co-chair of the International Workshop on Strategic reasoning (SR) \newline \url{http://sr2017.csc.liv.ac.uk/}}
\cvitem{2017}{Co-chair of the Workshop on Formal Methods in AI (FMAI)\newline  \url{https://sites.google.com/site/fmai2017homepage/}}

\subsection{Organiser}
\cvitem{2017}{Co-organiser of the Italian Conference on Theoretical Computer Science (ICTCS)\newline \url{http://ictcs2017.unina.it/}}
\cvitem{2017}{Co-organiser of the Italian Conference on Computational Logic (CILC)\newline  \url{http://cilc2017.unina.it/}}
\cvitem{2017}{Co-organiser of the First Workshop on Formal Methods in AI (FMAI)\newline  \url{https://sites.google.com/site/fmai2017homepage/}}
\cvitem{2013}{Co-organiser of the IST Austria Young Scientist Symposium on the topic `Understanding Shape: {in silico} and {in vivo}'\newline 
\url{ist.ac.at/young-scientist-symposium-2013/}}
\cvitem{2012}{Founded and organised the computer science seminar at IST Austria whose goal was to foster collaborations within the institute between computer scientists and, at the time, biologists.\newline  \url{ist.ac.at/computer-science-seminar/}}

\subsection{PC Membership}
\cvitem{2018}{PC member for IRISA Master Research Internship}
\cvitem{2018}{PC member of the International Workshop on Strategic reasoning (SR) \newline \url{http://projects.lsv.fr/sr18/}}
\cvitem{2018}{PC member of the International Conference on Autonomous Agents and Multi-agent Systems (AAMAS)}
\cvitem{2018}{PC member of the AAAI Conference on Artificial Intelligence (AAAI)\newline\url{https://aaai.org/Conferences/AAAI-18/}}
\cvitem{2017}{External reviewer for Icelandic Research Fund}
\cvitem{2017}{PC member of the International Joint Conference on Artificial Intelligence (IJCAI) }
\cvitem{2017}{PC member of the AAAI Conference on Artificial Intelligence (AAAI)}
\cvitem{2017}{PC member for IRISA Master Research Internship}
\cvitem{2016}{PC member of the International Workshop of Strategic Reasoning (SR)}
\cvitem{2016}{PC member of the International Symposium on Games, Automata, Logics and Formal Verification (GandALF)}
\cvitem{2016}{PC member of the European Conference on Artificial Intelligence (ECAI)}

\subsection{Editorship}

\cvitem{2017}{Guest-editor, Special issue of SR 2017, Information and Computation, In process.}
\cvitem{2017}{Guest-editor, Special issue of ICTCS 2017 and CILC 2017, Theoretical Computer Science, In process.}
\cvitem{2017}{Editor, Joint proceedings of ICTCS 2017 and CILC 2017, CEUR Workshop proceedings, ISSN 1613-0073, \url{ceur-ws.org/Vol-1949/}}

\subsection{Project co-ordinator}
\cvitem{2013-2016}{Handbook of Model Checking, to be published by Springer, and edited by Edmund Clarke, Thomas Henzinger, Helmut Veith and Roderick Bloem. Duties included: assisted editors in managerial, organisational, and technical matters, including: organising reviews, reviewers, and copy-editors; liasing between editors and Springer editor. ISBN 978-3-319-10575-8, \url{http://www.springer.com/us/book/9783319105741}}

\subsection{Reviewing}
\cvitem{Funding}{Icelandic Research Fund}
\cvitem{Book}{Handbook of Model Checking}
\cvitem{Journals}
{Artificial Intelligence (AIJ), Journal of Symbolic Logic (JSL), Logical Methods in Computer 
Science (LMCS), Transactions on Computational Logic (ToCL), 
Theory of Computing Systems (ToCS), Central European Journal of Mathematics, 
Information and Computation (IC), Journal of Logic and Computation (JLC), Annals of 
Mathematics and Artificial Intelligence (AMAI), Theory and Practice of Logic 
Programming (TLP), Science of Computer Programming (SCP)}
\cvitem{Conferences}{IJCAI, KR, AAMAS, AAAI, EUMAS, ECAI, LICS, STACS, ICALP, 
MFCS, CONCUR, CSL, FoSSaCS, FSTTCS, SR, KRR@SAC, CiE, GandALF, RV, LPAR, LATA}



% \section{Service}

% \subsection{Grant writing}

% As project coordinators, they assisted the editors in all managerial,
% organizational, and technical matters necessary for bringing such a
% large project to fruition:
% they managed the collaboration software and the interaction with the
% authors, reviewers, copy editors, and the publisher throughout much
% of the project.

% \item In 2014, I volunteered for the Vienna Summer of Logic, the largest event in the history of logic.\\
% \textsf{http://vsl2014.at/}


\subsection{Recent Research Visits}

\cventry{2015,2016,2017}{Host: Giuseppe De Giacomo, Sapienza University of Rome}% \hfill $12.2015$\\
{Topic 1: Synthesis under Assumptions;
Topic 2: Generalised Planning with Partial Observability}{}{}{}

\cventry{2016,2017}{Host: Mike Wooldridge, Oxford University}{Topic: Rational Synthesis}{}{}{}% \hfill $03.2016, 01.2017$\\

\cventry{2016,2017}{Host: Alessio Lomuscio, Imperial College London}% \hfill  $03.2016, 01.2017$\\
{Topic: Strategic-Epistemic logics for Multi-Agents Systems}{}{}{}

\cventry{2016}{Host: Diego Calvanese and Marco Montali, University of Bolzanno}% \hfill $07.2016$\\
{Topic 1: Data-aware strategic logics;
Topic 2: Knowledge Representation for Business Process Management}{}{}{}


\cventry{2016}{Hosts: Frank Stephan and Sanjay Jain, National University of Singapore}% \hfill $05.2016$\\
{Topic: Learning Theory and Verification}{}{}{}


\cventry{2015}{Host: Helmut Veith, TU Wien}% \hfill $08.2015$\\
{Topic 1: Logic and Impossibility Results in Distributed Computing;
Topic 2: Abstractions for Fault-tolerant Distributed Algorithms}{}{}{}

% %\item Host: \L ukasz Kaiser, Universit\'e Paris Diderot, France \hfill  $10.2011$\\
% %Topic: Application of Logic to AI
% 
% \item Host: Aniello Murano, Universit\`a degli Studi di Napoli ``Federico II'' \hfill $08.2011$\\
% Topic: Games of Imperfect Information and Pushdown Automata 
% 
% \item Host: Alexander Rabinovich, Tel Aviv University, Israel \hfill $5.2011$ -- $8.2011$\\
% Topic: Logical-Interpretability and Trees 
% 
% %Visited Vince 20 August? till end of September, 2010 in Warsaw.
% 
% \item Host: Erich Gr\"adel, RWTH Aachen \hfill $08.2006-01.2007$
% 
% 
% \item Host: Moshe Vardi, Rice University \hfill $01.2001-05.2001$
% 
%\item Steffen Lempp at UW Madison \hfill $08.2000-12.2000$


 \subsection{Invited Workshop Talks}
 
 \cventry{2017}{Games of Imperfect-information with Public Actions}{RoboLog, Rennes}{}{}{} %$02.2017$
\cventry{2017}{Verification of Multi-Agent Systems with Imperfect Information 
and Public Actions}{FMAI17, Napoli}{}{}{} %$02.2017$
\cventry{2012}{Finite and Algorithmic Model Theory}{Les Houches, France}{}{}{} %$05.2012$
\cventry{2011}{Automata theory and Applications}{IMS programme,  Singapore}{}{}{}% \hfill $09.2011$
\cventry{2008}{Computational Model Theory}{CNRS SIG, Bordeaux, France}{}{}{}% \hfill $06.2008$
\cventry{2007}{Algorithmic-Logical Theory of Infinite Structures}{Dagstuhl, Germany}{}{}{}% \hfill $10.2007$
\cventry{2006}{Finite and Algorithmic Model Theory}{Newton Institute, England}{}{}{}% \hfill $01.2006$
\cventry{2004}{Workshop on Automata, Structures and Logic}{Auckland, New Zealand}{}{}{}% \hfill $12.2004$


%
%\subsection*{Recent Seminar Talks}  IST Austria and TU Vienna ($2011,2012$), CNRS Liafa Paris 7 ($2011$), Tel Aviv University ($2011$), University of Cape Town ($2010$) \\
%Cornell ($2007,2008,2009$),
%\ifcut LSV Cachan ($2008$), CNRS LIAFA Paris 7 ($2007$), Heidelberg ($2007$)\\
%\fi
%\subsection*{Invited Talks}

\subsection{Invited Seminar Talks}

\cventry{2018}{To be determined}{Yale-NUS, Singapore}{}{}{}
\cventry{2018}{To be determined}{University of Auckland, New Zealand}{}{}{}
\cventry{2018}{To be determined}{IRIF, Universit\'e Paris-Diderot}{}{}{}
\cventry{2017}{Complexity of strategic reasoning under partial observability}{IMT Lucca, Italy}{}{}{}
\cventry{2017}{Complexity of strategic reasoning under partial observability}{GSSI, Italy}{}{}{}
\cventry{2017}{Temporal-Strategic Reasoning for Partial-Observation Games}{University of New South Wales, Australia}{}{}{}
\cventry{2016}{Imperfect-Information Games and Generalized Planning}{Free university of Bolzano, Italy}{}{}{}
\cventry{2014}{Verification of Mobile Agents in Partially Known Environments}{University of Naples, Italy}{}{}{}
\cventry{2014}{Memoryless Determinacy of Cycle Games}{University of California, San Diego, USA}{}{}{}
\cventry{2012}{Automata theoretic approach to mixed integer and rational arithmetic}{IST Austria, Austria}{}{}{}
\cventry{2011}{Representing infinite structures by automata}{TU Wien, Austria}{}{}{}
\cventry{2011}{An introduction to automatic structures}{Tel Aviv University, Israel}{}{}{}
\cventry{2011}{Representing infinite structures by automata}{EPFL, Switzerland}{}{}{}
\cventry{2008}{Generalised Quantifiers on Automatic Structures}{LSV Cachan, France}{}{}{}
\cventry{2007}{Generalised Quantifiers on Automatic Structures}{LIAFA Paris, France}{}{}{}
\cventry{2007}{Decidable extensions of the Monadic Second-order theory of one successor by unary predicates}{Cornell University, USA}{}{}{}
\cventry{2007}{Automatic Structures}{Heidelberg University, Germany}{}{}{}


%\subsection*{Conference Talks} CiE ($2008$),

%$2008 - 2009$: Logic Seminar, Cornell Talks in the Logic Seminar at Cornell. \\
%Ongoing work with Anil Nerode and Dexter Kozen.\\

%$2.2007$: Attended the 'Model Theory and Computable Model Theory' workshop, part
%of the University of Florida's Special Year in Logic.\\



\newpage
\section{Refereed Publications}
% IN ITALY, all papers must be ``equal'' participants
% I was lead or co-lead author for all publications \emph{except} the following where I played significant (e.g., supervisory) but secondary roles: \cite{DBLP:conf/lics/BerthonMMRV17, DBLP:journals/corr/abs-1210-2462,DBLP:journals/tcs/GrinshpunPRT14}.


\ifmetrics
The cited bibliometrics are as follows: conferences are given their CORE (\url{http://portal.core.edu.au/conf-ranks/}) letter ranking, followed by the acceptance rate, followed by the number of submissions (where available); journal are given their SJR letter ranking (\url{http://www.scimagojr.com/journalrank.php}) at time of publication (if a ranking is not available for the current year, then an average of the last 5 years is taken).
% These bibliometrics are a measure of conference/journal influence and do not, apriori, accurately reflect the quality of an individual paper. 
% I have 16 articles in CORE A* conferences, 
% 7 in CORE A conferences, 4 in CORE B conferences; and 4 articles in Q1 journals, 2 articles in Q2 journals, 1 book, and 1 book chapter.



% \renewcommand{\listitemsymbol}{-~} % Changes the symbol used for lists
% \subsection{Summary (the cited bibliometrics are a measure of journal influence and do not, apriori, accurately reflect the quality of an individual paper)}
% \cvitem{Ranking}{Number of publications}
% \cvitem{CORE A*}{16}
% \cvitem{CORE A}{7}
% \cvitem{CORE B}{4}
% % \cvitem{SJR Q1}{3}
% % \cvitem{SJR Q2}{2}

% {5 LICS papers (A*)}
% {3 IJCAI2017 (A*, approx 25\% acceptance, 2540 submissions)}
% {1 IJCAI2016 (A*, approx 25\% acceptance, 2294 submissions)}
% {2 AAMAS2017 (A*, approx 26\% acceptance, 595 submissions)}
% {1 AAMAS 2016 (A*, approx, 25\% acceptance, 550 submissions}
% {1 AAMAS2015 (A*, approx 25\% acceptance, 670 submissions)}
% {1 KR paper (A*)} 182, 26.9%
% {1 ICALP paper (A)}{}
% {1 CAV paper (A*)}{}
% {1 CONCUR paper (A)}{}
% {2 LPAR papers (A)}
% 1 IJCAR A*

%1 VMCAI B
% CSL B
% {3 STACS papers (B)}{}
% {1 KR paper (A*)} 182, 26.9%
% {1 ICALP paper (A)}{}
% {1 CAV paper (A*)}{}
% {1 CONCUR paper (A)}{}
% {2 LPAR papers (A)}
% 2 PRIMA (B)
% 
% 
%   {1 ACM TOCL (Q1)}
%   {1 TCS (Q1)}
%   {1 I\&C (Q2)}
%   {1 LMCS (unranked at time, currently Q1)}
%   {1 BSL (Q1)}


% 
% I have 40 refereed publications, most in top conferences and journals, 
% including 1 (co-authored) book, 1 (sole authored) book-chapter, 6 journal 
% articles (5 of them invited), 5 {\textsc LICS} papers, 4 {\textsc AAMAS} papers, 4 {\textsc IJCAI} papers, 
% 3 {\textsc STACS} papers,  1 {\textsc KR} paper, and a best-paper at {\textsc 
% PRIMA}. Not listed, are 6 invited journal articles (in preparation or under 
% evaluation) and an invited chapter in a handbook on automata theory and 
% applications (in preparation, J.E. Pin (ed.), to be published by EMS).


\nocite{*}

\printbibliography[heading=subbibliography,title={Book}, prefixnumbers={B},type=book]

\printbibliography[heading=subbibliography,title={Book Chapter}, prefixnumbers={BC},type=incollection]

\printbibliography[heading=subbibliography,title={Conference Articles}, prefixnumbers={C},type=inproceedings,notkeyword={workshop}]

\printbibliography[heading=subbibliography,title={Journal Articles}, prefixnumbers={J},type=article]

\printbibliography[heading=subbibliography,title={Workshop Articles}, prefixnumbers={W}, keyword={workshop}]


% Publications from a BibTeX file without multibib
%  for numerical labels: \renewcommand{\bibliographyitemlabel}{\@biblabel{\arabic{enumiv}}}% CONSIDER MERGING WITH PREAMBLE PART
%  to redefine the heading string ("Publications"): \renewcommand{\refname}{Articles}
% \nocite{*}
% 
% \bibliographystyle{plain}
% \bibliography{/home/sr/svn/forsyte-publications/trunk/rubin.bib}                        % 'publications' is the name of a BibTeX file

% Publications from a BibTeX file using the multibib package
%\section{Publications}
%\nocitebook{book1,book2}
%\bibliographystylebook{plain}
%\bibliographybook{publications}                   % 'publications' is the name of a BibTeX file
%\nocitemisc{misc1,misc2,misc3}
%\bibliographystylemisc{plain}
%\bibliographymisc{publications}                   % 'publications' is the name of a BibTeX file


% \section{Online Profiles}
% % \cvitem{CORE}{A*x16, Ax7, Bx4}
% % \cvitem{SJR}{Q1x2, Q1/Q2x1, Q2x1}
% \cvitem{Current list of publications}{\url{http://forsyte.at/alumni/rubin/publications/?nocache}}
% \cvitem{Google scholar}{\url{https://scholar.google.it/citations?user=auUS1rMAAAAJ&hl=en&oi=ao}}
% \cvitem{SCOPUS}{\url{https://www.scopus.com/authid/detail.uri?authorId=7201922792}}
% \cvitem{Semantic Scholar}{\url{https://www.semanticscholar.org/author/Sasha-Rubin/2807596}}


\ifref
\newpage
\section{References}

\subsection{Teaching}

\cventry{Mentor}
{Maria Terrell}
{Director of Teaching Assistant Programs}
{Cornell University}
{}
{maria@math.cornell.edu}


\cventry{Mentor}{David Way}
{Associate Director of Instructional Support}
{Center for Teaching Excellence}
{Cornell University}
{dgw2@cornell.edu}

% \subsection{Supervision}
% 
% \cventry{Mentor}{Bob Strichartz}
% {Department of Mathematics}
% {Cornell University}
% {}
% {str@math.cornell.edu}
% 

\subsection{Academic}

% \cventry{Previous Employer}{Roderick Bloem}
% {Institute for Applied Information Processing and Communication}
% {Technische Universit\"at Graz}
% {}
% {roderick.bloem@iaik.tugraz.at}

\cventry{PhD Supervisor}{Bakhadyr Khoussainov} 
{Department of Computer Science}
{University of Auckland}
{}
{bmk@cs.auckland.ac.nz}

\cventry{Current Collaborator}{Alessio Lomuscio}
{Faculty of Engineering, 
Department of Computing}
{Imperial College London}
{}
{a.lomuscio@imperial.ac.uk}

\cventry{Current Collaborator}{Giuseppe De Giacomo}
{Dipartimento di Ingegneria Informatica, Automatica e Gestionale} 
{Sapienza Universit\`a di Roma}
{}
{degiacomo@dis.uniroma1.it}
% 
% \cventry{Past Collaborator}{Erich Gr\"adel}
% {Mathematische Grundlagen der Informatik}
% {RWTH Aachen}
% {}
% {graedel@logic.rwth-aachen.de}

\cventry{Current Collaborator}{Michael Wooldridge}
{Department of Computer Science}
{University of Oxford}
{}
{mjw@cs.ox.ac.uk}

\cventry{Current Employer}{Aniello Murano}
{Dipartimento di Ingegneria Elettrica e Tecnologie dell'Informazione} 
{Universit\`a degli Studi di Napoli ``Federico II''}
{}
{murano@na.infn.it}


% \cventry{Past Collaborator}{Frank Stephan}
% {School of Computing}
% {National University of Singpore}
% {}
% {fstephan@comp.nus.edu.sg}

\fi

\end{document}




\section{\mysidestyle{Awards and Distinctions}}
\begin{itemize}
\item 2 individual fellowships (Marie Curie fellow of INdAM, New Zealand Science and Technology Postdoctoral Fellowship).
\item 2 PhD prizes (best doctoral thesis in the Faculty of Science, Montgomery memorial prize in logic from the Department of Philosophy).

\end{itemize}


\begin{itemize}


\end{itemize}

%
%
%$1998$: Competed as part of a team of three, in the world finals of the $1998$
%ACM Programming Contest in Atlanta, Georgia USA, representing the University of Auckland
%and New Zealand. 
%%The same team won the Regional Programming Contest in $1997$.\\ 
%\fi

\section{\mysidestyle{Recent service}}

%%%HBMC
%I am a chair of the ...

\begin{itemize}

%One reviewer wrote "We are extremely grateful to this reviewer for his/her careful reading of the paper, and for his/her constructive suggestions."

%equaleducation.org.za

%\pagebreak

\end{itemize}

% \newpage


\section{\mysidestyle{References}}

%\begin{multicols}{2}
%\subsection*{Primary}

\subsubsection*{\sc{Academic}}




\section{\mysidestyle{Refereed Publications}}



\nocite{*}

\printbibliography


%prima x2, SR special issue, concur, 

%The following conference papers were invited to journals \cite{
%DBLP:journals/jalc/KhoussainovR01,
%DBLP:journals/jalc/KhoussainovR03,
%DBLP:conf/lics/KhoussainovNRS04,
%DBLP:journals/corr/AminofR14,
%DBLP:journals/lmcs/KhoussainovNRS07,
%DBLP:conf/concur/AminofKRSV14}. 
%The journal versions of the following are under evaluation, and not listed below: \cite{DBLP:conf/concur/AminofKRSV14}.





%\begin{description}

%\item[Book]\
%
%{Decidability of Parameterized Verification} with R. Bloem, S. Jacobs, A. Khalimov, I.
%    Konnov, H. Veith and J. Widder, in {Synthesis Lectures in Distributed Computing Theory}, N. Lynch Ed., September 2015, 170 pages
%    
%    
%\item[Book chapters]\
%
% {Automatic Structures} in {Automata: From mathematics to applications}, J.E. Pin, Ed., to be published by EMS.
%
% {Automata based presentations of infinite structures} with V. B{\'a}r{\'a}ny and E. Gr{\"a}del,
%in {Finite and Algorithmic Model Theory}, J. Esparza, C. Michaux, and C. Steinhorn, Eds.,
%Series: London Mathematical Society Lecture Note Series (379), $1-76$, $2011$. \cc{15}
%
%\item[Journals]\
%
%{First-Cycle Games} with B. Aminof, Information and Computation, $2016$.
%
%{Alternating Traps in Parity Games} with P. Phalitnonkiat, A. Grinshpun, A.Tarfulea, Theoretical Computer Science,  $73-91, 2014$.
%
%{Automata presenting structures: A survey of the finite-string case}, The Bulletin of Symbolic Logic, 
%$14 (2), 169-209, 2008$. \cc{66}
%
%{Automatic Structures: Richness and Limitations}, with B. Khoussainov, A. Nies and F. Stephan, 
%Logical Methods in Computer Science, Vol $3$, $2007$.  \cc{78}
%
%{Automatic linear orders and trees}, with B. Khoussainov and F. Stephan, 
%ACM Transactions on Computational Logic,
%$6 (4), 675-700, 2005$.  \cc{49}
%
%
%%{Automatic Linear Orders and Trees: Revised}, CDMTCS Technical Report $208$,
%%Department of Computer Science, University of Auckland, $2003$.   
% 
% %{Definability and Regularity in Automatic Presentations of Subsystems of
%%Arithmetic}, CDMTCS Technical Report $209$,
%%Department of Computer Science, University of Auckland, $2003$.   
% 
%{Automatic Structures - Overview and Future Directions}, with 
%B. Khoussainov,
%Journal of Automata, Languages and Combinatorics, $8(2), 287-301, 2003$.   \cc{28}
%
%{Graphs with Automatic Presentations over a Unary Alphabet}
%Journal of Automata, Languages and Combinatorics, $6(4), 467-480, 2001$. \cc{15}  
%
%{Finite Automata and Well Ordered Sets},
%New Zealand Journal of Computing, $7(2), 39-46, 1999$. 
%
%
%\item[IJCAI Proceedings]\
%
%{Imperfect-Information Games and Generalized Planning}, with
%G. De Giacomo, A. Di Stasio, A. Murano, $2016$.
%
%\item[KR Proceedings]\
%
%{Prompt Alternating-Time Epistemic Logics}, with B. Aminof, A. Murano, F. Zuleger, $2016$.
%
%\item[AAMAS Proceedings]\
%
%{Automatic verification of multi-agent systems in parameterised grid-environments}, with
%B. Aminof, A. Murano, F. Zuleger, $2016$.
%
%{Graded Strategy Logic: Reasoning about Uniqueness of Nash Equilibria}, with
%B. Aminof, V. Malvone, A. Murano, $2016$.
%
%{Parameterised Verification of Autonomous Mobile-Agents in Static but Unknown Environments}, $2015$.
%
%\item[PRIMA Proceedings]\
%
%{Multi-Agent Path Planning in Known Dynamic Environments}, with A. Murano, G. Perelli, $2015$. 
%
%\item[LICS Proceedings]\
%
%{Interpretations in trees with countably many branches}, with A. Rabinovich, $551-560$, $2012$. \cc{3}
%
%
%{Automatic Structures: Richness and Limitations}, with B. Khoussainov, A. Nies and F. Stephan, 
%$44-53$, $2004$. \cc{78} 
%
%{Automatic Partial Orders}, with B. Khoussainov and F. Stephan, $168-177$, $2003$. \cc{33}
%
%{Some Results on Automatic Structures}, with B. Khoussainov
%and H. Ishihara,  $235-244$, $2002$. \cc{13}
%
%\item[STACS Proceedings]\
%
%{Cardinality and counting quantifiers on omega-automatic structures}, with V.  B{\'a}r{\'a}ny and \L. Kaiser, $385-396$, $2008$.  \cc{22}
%
%{Order invariant MSO is stronger than counting MSO}, with T. Ganzow, $313-324$,  
% $2008$.  \cc{9}
% 
%{Definability and Regularity in Automatic Structures}, with B. Khoussainov
%and F. Stephan,  $440-451, 2004$.  \cc{23}
%
%%International Workshop on Logic and Computational Complexity $2007$.\\
%
%\item[CONCUR Proceedings]\
%
%{ Parameterized model checking of Rendezvous Systems}, with B. Aminof, T. Kotek, F. Spegni and H. Veith, $109-124$, $2014$
%
%\item[CAV Proceedings]\
%
%{Verifying $\omega$-regular Properties of Markov Chains}, with D. Bustan and
%M. Vardi, $189-201, 2004$. \cc{18}
%
%\item[ICALP Proceedings]\
%
%{Liveness of Parameterized Timed Networks}, with B. Aminof, F. Spegni and F. Zuleger, $375-387, 2015$.
%
%
%\item[VMCAI Proceedings]\
%
%{Parameterized Model Checking of Token-Passing Systems}, with B. Aminof, S. Jacobs and A. Khalimov, $262-281, 2014$. \cc{6}
%
%
%\item[Other Refereed Proceedings]\
%
%{Model Checking Parameterised Multi-Token Systems via the Composition Method}, with
%B. Aminof, IJCAR $2016$.
%
%
%{On CTL* with Graded Path Modalities}, with
%B. Aminof, A. Murano, {LPAR}, $2015$.
%
%{On the expressive power of communication primitives in parameterised systems},
%B. Aminof and F. Zuleger, {LPAR} $2015$.
%
%{Cycle Games} with B. Aminof, {Strategic Reasoning}, ETAPS workshop, $2014$.
%
%{How to Travel Between Languages} with  K. Chatterjee and S. Chaubal, {LATA}, $2013$.
%
%{A Myhill-Nerode Theorem for Automata with Advice} with A. Kruckman, J. Sheridan and B. Zax, {GandALF}, $238-246$, $2012$. \cc{2}
%
%%{Quantifiers on Automatic Structures} with V. Goranko, {CiE}, $2008$.
%

%\item[Thesis]\
%
%{Automatic Structures}, University of Auckland, $2004$ \cc{55}
%
%%{Satisfiability of $CTL^*$ with graded path modalities}, with B. Aminof, A. Murano.
%    
%\end{description}







\end{document}



%----------------------------------------------------
%----------------------------------------------------
%----------------------------------------------------
%----------------------------------------------------
%----------------------------------------------------
%----------------------------------------------------


\iffalse
{\bf Dr. Valentin Goranko}\\
School of Mathematics\\
University of Witwatersrand\\ 
Private Bag 3, WITS 2050\\
Johannesburg, South Africa\\
goranko@maths.wits.ac.za\\
Phone : +$27$ $11$ $717$ $6243$ \\
%Fax   : +$27$ $11$ $717$ $6259$ \\
\fi
%- Tutorial co-ordinator for undergraduate paper `Discrete Mathematics', %1999 \\
%- Project and summer scholarship supervisor for `Application of Elementary \\
%\phantom{- }Submodels to Topology', 1998\\ 
%- Lecturer for `Logic and Set Theory', 1998\\ 
% \enlargethispage*{1cm}

%\section*{Conferences Attended}
%\begin{tabular}{@{}ll}
% {\bf 2000} & Computational Group Theory, Sydney\\
% {\bf 1999} & Third New Zealand Computer Science Research Students'
%Conference \\
%	    & April, Waikato\\
% {\bf 1999} & ACSC - DMTCS/CATS \\
%	    & January, Auckland \\
% {\bf 1999} & NZMRI summer workshop, Harmonic Analysis\\
%	    & January, Raglan, New Zealand\\
% {\bf 1998} & Second Japan-New Zealand Workshop on 
%		{\em Logic in Computer Science}\\
%	    & October, Auckland\\
% {\bf 1998} & First International Conference on 
%		{\em Unconventional Models of Computation}\\
%	    & January, Auckland \\
% {\bf 1997} & First Japan-New Zealand Workshop on {\em Logic in Computer Science}\\
%	    & August, Auckland\\
% {\bf 1997} & Fifth Australasian Mathematics Convention\\
%	    & July, Auckland\\
%\end{tabular}


\iffalse
\subsection*{Additional}
{\bf Dr. David McIntyre}\\
Department of Mathematics\\
University of Auckland, New Zealand\\
mcintyre@math.auckland.ac.nz\\
Phone : (+64 9) 373 7599 Ext 8763\\
%- Tutorial co-ordinator for undergraduate paper `Discrete Mathematics',
%1999 \\
%- Project and summer scholarship supervisor for `Application of Elementary \\
%\phantom{- }Submodels to Topology', 1998\\
%- Lecturer for `Logic and Set Theory', 1998\\
\enlargethispage*{1cm}


%\pagebreak
\section*{Non Academic}
\subsection*{University of Auckland}
{Membership}\\
  Auckland University Dramatic Society (1998/1999/2000)\\
  Auckland University Comedy and Improvisation Club (1997) Secretary (1998)\\


{Performance}\\
  `Morte Accidentale Di Un Anarchico', (Dario Fo) (1999)\\
  Celebration of Performing Arts, Auckland Town Hall (Stoppard, Simon) (1999)\\
  Three comedy shows, Auckland University (1997 and 1998)\\
  A duet in an evening of short plays, Auckland University (Stoppard) (1998)\\
  Cultural Mosaic Festival, Auckland University (Stoppard) (1997)\\
\begin{center}
--------------------
\end{center}
\fi


\section*{Seminars/Talks} 

 
' Temporal-Strategic Reasoning for Games of Imperfect Information', UNSW computer science seminar, 2017\\

`Techniques to prove non-automaticity', University of Heidelberg Logic Seminar, 2002 \\


`Some Results on Automatic Structures', with Bakhadyr Khoussainov
and Hajime Ishihara, 17th Annual IEEE Symposium on Logic in Computer Science, 2002. \\


`Automata-theoretic approach to verification of probabilistic systems',
Rice University Computer Science Theory Seminar, 2001. \\


`Automatic Structures', University of Notre Dame Logic Seminar, 2001, and \\
University of Madison, Wisconsin, Logic Seminar, 2001.\\


`Finite Automata and Relational Structures', with Bakhadyr Khoussainov, 
DCAGRS, July 2000, London, Ontario \\


 `Finite Automata and Well Ordered Sets', 
3rd New Zealand Computer Science Research Conference, 1999, Waikato, New
Zealand \\


{\em Auckland Department of Computer Science:}\\
 `Finite Model Theory - Ehrenfeucht-Fraisse Theorem', 2000\\
 `Extracting Algebraic Information from Finite State Machines', 1999\\
 `Finite Automata and Regular Languages', 1999\\


{\em Auckland Department of Mathematics:}\\
 `Algebraic Structures and Finite Automata', 1999\\
 `Applications of Elementary Submodels to Topology', 1999\\ 
\fi

\iffalse
		
  {\bf Marking -} {University of Auckland} \\
	 $1998$: Stage $3$ Assignments for the Department of Mathematics\\
  	 1997: Stage 1 Assignments for the Departments of 
	 Computer Science and Mathematics\\
\fi


