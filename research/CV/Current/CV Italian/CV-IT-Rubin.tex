\documentclass[10pt,a4paper,sans]{moderncv}       
% possible options include font size ('10pt', '11pt' and '12pt'), paper size ('a4paper', 'letterpaper', 'a5paper', 'legalpaper', 'executivepaper' and 'landscape') and font family ('sans' and 'roman')
% moderncv themes
\moderncvstyle{classic}                             % style options are 'casual' (default), 'classic', 'banking', 'oldstyle' and 'fancy'
\moderncvcolor{blue}                               % color options 'black', 'blue' (default), 'burgundy', 'green', 'grey', 'orange', 'purple' and 'red'
%\renewcommand{\familydefault}{\sfdefault}         % to set the default font; use '\sfdefault' for the default sans serif font, '\rmdefault' for the default roman one, or any tex font name
%\nopagenumbers{}                                  % uncomment to suppress automatic page numbering for CVs longer than one page
% character encoding
%\usepackage[utf8]{inputenc}                       % if you are not using xelatex ou lualatex, replace by the encoding you are using
%\usepackage{CJKutf8}                              % if you need to use CJK to typeset your resume in Chinese, Japanese or Korean
% adjust the page margins
\usepackage[scale=0.75]{geometry}
%\setlength{\hintscolumnwidth}{3cm}                % if you want to change the width of the column with the dates
%\setlength{\makecvtitlenamewidth}{10cm}           % for the 'classic' style, if you want to force the width allocated to your name and avoid line breaks. be careful though, the length is normally calculated to avoid any overlap with your personal info; use this at your own typographical risks...
% personal data
\name{Sasha}{Rubin}
\title{Curriculum Vitae, 11/2017}                               % optional, 
%remove / comment the line if not wanted
\address{Universit\`a di Napoli ``Federico II''}{Napoli, Italia}% optional, 
%remove / comment the line if not wanted; the "postcode city" and "country" 
%arguments can be omitted or provided empty
% \phone[mobile]{+1~(234)~567~890}                   % optional, remove / comment the line if not wanted; the optional "type" of the phone can be "mobile" (default), "fixed" or "fax"
% \phone[fixed]{+2~(345)~678~901}
% \phone[fax]{+3~(456)~789~012}
\email{rubin@unina.it}                               % optional, remove / comment the line if not wanted
\homepage{sasharubin.github.io}                         % optional, remove / comment the line if not wanted
% \social[linkedin]{john.doe}                        % optional, remove / comment the line if not wanted
% \social[twitter]{jdoe}                             % optional, remove / comment the line if not wanted
% \social[github]{jdoe}                              % optional, remove / comment the line if not wanted
% \extrainfo{additional information}                 % optional, remove / comment the line if not wanted
\photo[70pt][0.4pt]{RUBIN_Sasha.jpg}                       % optional, remove / comment the line if not wanted; '64pt' is the height the picture must be resized to, 0.4pt is the thickness of the frame around it (put it to 0pt for no frame) and 'picture' is the name of the picture file
% \quote{Some quote}                                 % optional, remove / comment the line if not wanted

\newif\ifmetrics
\metricstrue

\newif\ifref
\reftrue

\usepackage[gen]{eurosym}

%%%%%%%%%%%%%%%%%%%%%%%%%%%%%%%%%%%%%
%%%%%%%%%% BIBLIO STUFF %%%%%%%%%%%%%
\usepackage[backend=bibtex, sorting=ydnt,maxbibnames = 10,citestyle = ieee,defernumbers]{biblatex}

\bibliography{../data/rubin-core.bib}


\def\TOTALCITATIONS{796}
\def\citeKNRS04{100} % 100 in Dec 2017, 97 in Sept 2017, 96 in July 2017, 95 in April 2017, 94 in Feb 2017, 91 in July 2016, 
\def\citeRubin08{93} % 93 in dec 17, 91 in Sept 17, 90 in July 2017,  87 in April 2017, 85 in Feb 2017, 85 in Sept 2016, 84 in July 2016 
\def\citeThesis{73} %  73 in dec 17, 72 in July 2017, 67 in Feb 2017, 64 in August 2016
\def\citeAminofJKR14{36} % 36 in sept 17, 34 in Feb 2017, 
\def\citeBloem15{31} % 31, 30 in dec 17, 25 in sept 17
  

%%%%%%%%%%%%%%%%%%%%%%%%%%%%%%%%%%%%%%%%%%%%



\DeclareRangeChars{~,;-+/{}} % add '{}' as page range delimiter

\addbibresource{\jobname.bib}

\DeclareFieldFormat{file}{\hfill \textbf{#1}}

\AtEveryBibitem{%
    \csappto{blx@bbx@\thefield{entrytype}}{% put at end of entry
        \iffieldundef{file}{}{%
       \space \printfield{file}
% 		{}
      }
    }
  }

% \usepackage{hyperref}

% bibliography adjustements (only useful if you make citations in your resume, or print a list of publications using BibTeX)
%   to show numerical labels in the bibliography (default is to show no labels)
% \makeatletter\renewcommand*{\bibliographyitemlabel}{\@biblabel{\arabic{enumiv}}}
\makeatletter\renewcommand*{\bibliographyitemlabel}{\@biblabel{\arabic{enumiv}}}\makeatother

%   to redefine the bibliography heading string ("Publications")
%\renewcommand{\refname}{Articles}

% bibliography with mutiple entries
%\usepackage{multibib}
%\newcites{book,misc}{{Books},{Others}}
%----------------------------------------------------------------------------------
%            content
%----------------------------------------------------------------------------------
\begin{document}
%\begin{CJK*}{UTF8}{gbsn}                          % to typeset your resume in Chinese using CJK
%-----       resume       ---------------------------------------------------------
\makecvtitle


\section{Informazioni Personali}
% \cvitem{Date of Birth}{16.02.76}
% \cvitem{Place of Birth}{South Africa}
\cvitem{Cittadinanza}{Neozelandese}
\cvitem{Lingua}{Inglese (madrelingua), Italiano (scolastico)}
% , Italian (basic)}
% <Afrikaans (basic)}
% \cventry{year--year}{Degree}{Institution}{City}{\textit{Grade}}{Description}

\section{Posizione corrente}
\cventry{2017-ad oggi}{Assegnista di ricerca (Computer Science)}{Universit\`a degli Studi di Napoli ``Federico II''}{Borsa di studio}{Presso il Dipartimento di Ingegneria e Tecnologie dell'Informazione,}{Laboratorio ASTREA responsabile scientifico Prof. Aniello Murano, per ricerche in metodi formali, intelligenza artificiale e sistemi multi-agenti.}{}

\section{Istruzione e Formazione}
\cventry{1999-2007}{PhD}{Mathematics and Computer 
Science}{University of Auckland}{(Completato nel 2004, Premiato nel 2007)}{ 
Miglior tesi di dottorato nella Facolt\`a di Scienze}
\cventry{1997-1998}{MSc}{Mathematics}{University of Auckland}{}{First Class}
\cventry{1994-1996}{BSc}{Department  of Mathematics and Department  of Computer Science}{University of Cape Town}{}{Dean's Merit List}

\section{Precedenti incarichi}
\cventry{2015-2017}{PostDoc Biennale (Computer Science)}{Universit\`a di Napoli 
``Federico II''}{Borsa di studio Marie Curie dell' INdAM ``F. Severi''.}{Ricerche condotte in collaborazione con il Prof. Aniello Murano e il suo gruppo di ricerca.}{}{}

\cventry{2014-2015}{PostDoc Annuale (Computer Science)}{TU Wien e TU Graz}{In collaborazione con Helmut Veith e Roderick Bloem e i loro gruppi di ricerca}{}{}{}

\cventry{2012-2014}{PostDoc Biennale (Computer Science)}{TU Wien e IST Austria}{In collaborazione con Helmut Veith e Krishnendu Chatterjee e i loro gruppi di ricerca}{}{}{}
\cventry{2010-2012}{Honorary Research Fellow Biennale (Computer Science)}{University of Auckland}{In collaborazione con Bakhadyr Khoussainov e il suo gruppo di ricerca}{}{}{}

\cventry{2011}{Visiting Researcher Semestrale (Computer Science), University of Tel Aviv}{In collaborazione con Alexander Rabinovich}{}{}{}

\cventry{2010}{Visiting Lecturer Semestrale (Mathematics)}{University of Cape Town}{}{}{}{}

% \cventry{2009-2010}{Visiting Researcher Biennale (Computer Science)}{University of Auckland}{In collaborazione con Bakhadyr Khoussainov e il suo gruppo di ricerca}{}{}{}

\cventry{2008-2009}{Visiting Assistant Professor Biennale (Mathematics)}{Cornell University}{In collaborazione con Anil Nerode e il suo gruppo di ricerca}{}{}{}

\cventry{2004-2007}{PostDoc Triennale (Computer Science)}{Borsa di studio Foundation for Research, Science and Technology (FRST) New Zealand Science and 
Technology (NZST) Post Doctoral Fellowship}{}{}{}{}


\newpage

\section{Interessi di ricerca}
I miei principali interessi di ricerca riguardano lo studio dei metodi formali 
per l'intelligenza artificiale, compreso il ragionamento automatizzato e 
rappresentazione della conoscenza. 
Pi\`u nello specifico, i miei lavori hanno portato dei forti contributi nelle 
seguenti aree:
% My main interest is in formal methods for artificial intelligence, particularly strategic and epistemic reasoning for multi-agent systems.
% I work in formal methods, a branch of theoretical computer science, and study the power of automata theory (broadly construed) and mathematical logic for describing, reasoning and controlling systems. 
% I have contributed to the following areas:
\begin{itemize}
 \item Metodi Formali (Modellazione, Verifica e Sintesi) di Sistemi 
			 Multi-agente (inclusi i Sistemi Parametrizzati, Sistemi 
       Distribuiti, Sistemi Probabilistici e Sistemi Temporizzati); 
%  \item Formal methods (Modeling, Verification, Synthesis) of Multi-agent Systems (including Parameterised Systems, Distributed Systems, Probabilistic Systems, Timed Systems)
 \item Logiche per giochi e ragionamento strategico;
 \item Generalised Planning;
 \item Teoria degli automi;
 \item Teoria dei modelli finiti ed algoritmici.
%  Finite and Algorithmic Model Theory
\end{itemize}


% \subsection{ACM Computing Classification System}
% \cvitem{}{Theory of Computation [Models of Computation, Logic, Formal Languages and Automata Theory]}
% 
% \cvitem{}{Computing Methodologies [Artificial Intelligence: Planning and Scheduling, Knowledge Representation and Reasoning, Distributed Artificial Intelligence]}
% 
% % \subsection{Mathematics Subject Classification}
% % {03Bxx (General Logic), 03Cxx (Model Theory); 68Qxx (Theory of Computing), 68Txx (Artificial Intelligence)}



\subsection{Risultati significativi di Ricerca} 

%list  their  five  most  important publications since their last review, along 
% with brief explanations  of  why  each  paper  is  significant.

Tutte le citazioni sono prese da google scholar il 11/10/17: \url{scholar.google.com/citations?hl=en&user=auUS1rMAAAAJ}.\newline
% All bibliometrics (e.g., citations) are from google scholar: 

%list  their  five  most  important publications since their last review, along with brief explanations  of  why  each  paper  is  significant.


\cvitem{2017}{Ho contribuito in modo significativo agli aspetti tecnici della presentazione del lavoro ``Deciding parity games in quasipolynomial time'' scritto da Cristian S. Calude, Sanjay Jain, Bakhadyr Khoussainov, Wei Li, Frank Stephan. Questo lavoro 
e fondamentale nell'informatica teorica perch\'e introduce per la prima volta un algoritmo quasi-polinomiale per i Parity Games. L'articolo 
ha ottenuto il Best Paper Award alla conferenza STOC 2017. Per il mio contributo sonno stato ringraziato alla fine del lavoro.}

\cvitem{2017}{Ho pubblicato 5 articoli a conferenze classificate CORE A* nel 
2017~\cite{DBLP:conf/lics/BerthonMMRV17,BDGR17,GMPRW17,BLMR17IJCAI,BLMR17}.}

% \cvitem{2017}{I initiated the study of the complexity of deciding if there exists equilibria in multi-player graphs games whose objectives are a mixture of qualitative and quantitative objectives.


% \cvitem{2017}{Ho contribuito in modo significativo agli lavoro della 

\cvitem{2016}{Ho pubblicato 5 articoli a conferenze classificate CORE A* nel 
2016~\cite{GMRS16IJCAI,DBLP:conf/atal/AminofMMR16,DBLP:conf/kr/AminofMRZ16,
DBLP:conf/atal/AminofMRZ16,DBLP:conf/cade/AminofR16}.}



\cvitem{2016-2017}{Ho collaborato con esperti mondiali  allo studio di sistemi 
									multi-agente e  pianificazione automatizzata, 
									fornendo importanti contributi teorici alla 
									verifica e sintesi con informazioni imperfetta. In particolare, ho esteso Strategy Logic (la logica recentemente introdotta per il ragionamento strategico da Moshe Vardi et.al.) in due modi: consentendo informazioni imperfette (con Moshe Vardi et.al.~\cite{DBLP:conf/lics/BerthonMMRV17} e Alessio Lomuscio et.al.~\cite{BLMR17IJCAI}) and con graded modalities (con Aniello Murano et.al.~\cite{AMMR16-SR-journal}). Ho esteso di belief-space construction a giochi infinite (con Giuseppe De Giacomo et. al.~\cite{GMRS16IJCAI}), e ho studiato la sua applicazione a Generalised Planning (con Giuseppe de Giacomo, Blai Bonet, e Hector Geffner~\cite{BDGR17}).}

\cvitem{2014-2015}{Ho aperto una nuova linea di ricerca di metodi formali 
per parameterised light-weight mobile agents producendo il lavoro~\cite{DBLP:conf/atal/Rubin15}. 
Successivamente (con i miei co-autori) ho portato avanti il lavoro in questa 
direzione, ottenendo i risultati riportati in~\cite{DBLP:conf/prima/RubinZMA15} 
(che ha vinto il premio come miglior paper) e~\cite{DBLP:conf/atal/AminofMRZ16}.}

\cvitem{2015}{Sono stato co-autore di un libro che riporta tutti i 
risultati di decidibilit\`a nella verifica parametrizzata 
\cite{DBLP:series/synthesis/2015Bloem}, 
pubblicato da Morgan \& Claypool (\citeBloem15 citazioni).}


\cvitem{2012-2014}{Insieme ai miei co-autori abbiamo generalizzato un paper 
pilastro riguardante la verifica dei sistemi parametrizzati ("Reasoning about 
Rings", E.A. Emerson, K.S. Namjoshi, {POPL}, 1995) passando da topologie ad 
anello a topologie arbitrarie 
(\citeAminofJKR14\ citazioni) \cite{DBLP:conf/vmcai/AminofJKR14}.}

% Inoltre abbiamo anche completato un libro, pubblicato da Morgan \& Claypool, 
% che riporta i risultati di decidibilit\`a nella verifica 
% parametrizzata (\citeBloem15\ citazioni) \cite{DBLP:series/synthesis/2015Bloem}.}

\cvitem{2008-2011}{Ho pubblicato una rassegna ed un'estensione dei risultati 
principali ottenuti nella mia tesi in Bulletin of Symbolic Logic 
\cite{DBLP:journals/bsl/Rubin08}. 
Insieme ad uno studente di dottorato di Erich Gr\"adel 
(Tobias Ganzow) ho risolto la congettura di 
Courcelle~\cite{DBLP:conf/stacs/GanzowR08} (Problema aperto da 12 anni).}

\cvitem{1999-2007}{Durante e dopo il mio dottorato, insieme ad i miei 
co-autori abbiamo aperto la strada alla sviluppo della teoria di strutture 
automatiche. Le pubblicazioni pi\`u citate in quest'area sono:
\cite{DBLP:conf/lics/KhoussainovNRS04} (\citeKNRS04\ citazioni) e 
\cite{DBLP:journals/bsl/Rubin08} (\citeRubin08\ citazioni). La mia tesi ha \citeThesis\ citazioni.}



\subsection{Bibliometrics}

\cvitem{Articoli}{16 x CORE A* (di cui 5 LICS, 4 IJCAI, 4 
AAMAS, 1 KR, 1 IJCAR, 1 CAV), e 7 x CORE A}
\cvitem{Journals}{3 x SJR Q1, e 4 x SJR Q2}



\cvitem{H-index}{In accordo a Google Scholar, il mio  H-index \`e 15 (dal 
2002), e 12 (dal 2012)}




\section{Premi}
\cventry{Premio PhD}{Miglior Tesi di Dottorato alla Facolt\`a di Scienze, 
				University of Auckland, 2004}{}{}{}{}
\cventry{Premio PhD}{Premio Montgomery in Logica al Dipartimento di 
Filosofia, 2004}{}{}{}{}
\cventry{Premio Paper}{Premio miglior paper a PRIMA 2015, 
\cite{DBLP:conf/prima/RubinZMA15}}{}{}{}{}{}
% \cventry{Competition}{Won the regional heats and represented New Zealand in the world finals of the 1998 ACM Programming Contest}{Atlanta, Georgia USA}{}{}{}{}
\cventry{Competizioni}{Ho vinto le  competizioni nazionali di programmazione, e quindi ho rappresentato la 
Nuova Zelanda nelle finali mondiali del 1998 di ACM Programming 
Contest}{Atlanta, Georgia USA}{}{}{}{}

\section{Finanziamenti e Grant}
\subsection{Finanziamenti individuali}
\cventry{2004-2007}{Borsa di studio Triennale, Foundation for Research, Science and Technology (FRST) New Zealand Science and 
Technology (NZST) Post Doctoral Fellowship (n. UOAX0413, ``Automatic Structures'')}{3 anni}{NZ\$224532}{}{}
% Salario, Viaggi, Spese varie}{}{}

\cventry{2015-2016}{Borsa di studio Marie Curie Individual Fellowship dell'INdAM ``F. Severi''}{2 anni}{\euro{107000}}{}{}

% {\url{https://cofund.altamatematica.it/2012/main/website?page=call-1}}{}{} 
%43200 living + mobility allowance} + RCC to uni \euro{5800} + Travel \euro{2500} + RCC2 \euro(2000)}{}{}

\cventry{2011}{Exchange Grant (n. 3471) all' interno del framework dell' European Science Foundation (ESF) activity su ``Games for Design and Verification``}{12 settimane }{\euro{5300}}{}{}

\cventry{2015-2016}{GNSAGA grants per periodi ricerca allestero}{\euro{4909}}{}{}{}

\cventry{2011}{Short Visit Grant (n. 4391) dell' European Science Foundation (ESF) activity su ``Games for Design and Verification``}{15 giorni}{\euro{1475}}{}{}

\cventry{2011}{Short Visit Grant (n. 4500) dell' European Science Foundation (ESF) activity su ``Games for Design and Verification``}{10 giorni}{\euro{1350}}{}{}
% 
% \cventry{2017}
% {\'Ecole normale sup\'erieure (ENS) de Rennes grant per la visita del gruppo di Sophie Pinchinat}{Finanziamenti ottenuto e visita da realizzare nell mese di 12/2017}
% {}{}{} %}{3 giorni}{\euro{600}}{Pianificato 12/2017}{}






% INDAM grant per periodi ricerca allestero
% Grant to go to LICS 2002 from IEEE
% Grant to travel from math + cs depts

% visited Heidelberg as student who paid?

\subsection{Partecipazione internazionale di ricerca}

\cvitem{2017-oggi}{Centro di Servizi Metrologici Avanzati (CESMA) ''Metedologia avanzate di intelligenza artificiale per il controllo e l'analisi di sistemi reattivi complessi`` presso il Universit\`a di Napoli 
``Federico II''}
\cvitem{2016,2017}{Michael Wooldridge (Oxford), European Research Commission (ERC) Advanced Grant ''RACE``.}
\cvitem{2016,2017}{Alessio Lomsucio (Imperial) e Aniello Murano (Universit\`a di Napoli ``Federico II''), Royal Society International Exchanges Scheme.}
\cvitem{2016}{Frank Stephan e Sanjay Jain, Academic Research Fund Tier 1 Grant (R146-000-181-112) "Automata Theoretic Aspects Of Predicting And Learning".} %75eu x5 days + acc
\cvitem{2014-2015}{Austrian Research Fund (National Research Network) "Rigorous systems engineering (RiSE)", presso il TU Wien e TU Graz.}
\cvitem{2014-2015}{Austrian Research Fund (National Research Network) "Rigorous systems engineering (RiSE)", presso il TU Wien e IST Austria.}
\cvitem{2000/2001}{Moshe Vardi (Rice University), Marshall Grant}

\subsection{Scrittura di Proposta Progettuale}
\cvitem{2017}{Supportato Aniello Murano alla scrittura di un Programma 
Operativo Nazionale (PON) ''Ricerca e Innovazione`` 2014-2020 grant 
application.} %ERC
\cvitem{2017}{Supporto alla scrittura di un progetto ERC advanced presentato da Giuseppe De Giacomo.} %ERC
\cvitem{2017}{Supporto alla scrittura di un una proposta progettuale INdAM per postdoc.}
\cvitem{2016-2017}{Supporto alla scrittura di un una proposta progettuale per Florian Zuleger, sovvenzione per il fondo scientifico austriaco.}
\cvitem{2014}{Supporto alla scrittura di un una proposta progettuale com Helmut Veith per la scrittura ed editing di una domanda di sovvenzione per il fondo scientifico austriaco e nei rapporti relativi alla Rete nazionale di ricerca (NFN). \url{http://arise.or.at/}}


\newpage

\section{Supervisione e Mentoring}

Sono stato il supervisore di diversi studenti, la cui collaborazione ha 
prodotto pubblicazioni.

\subsection{Mentoring/Tutor di studenti di dottorato}
\cvitem{2015}{Ho lavorato in stretta collaborazione con studenti di dottorato 
di Aniello Murano (Vadim Malvone, Antonio di Stasio, Loredanna Sorrentino) con i quali abbiamo prodotto
\cite{DBLP:conf/atal/AminofMMR16,AMMR16-SR,GMRS16IJCAI}.}
\cvitem{2007}{Ho lavorato in stretta collaborazione con studenti di dottorato 
di Erich Gradel (Tobias Ganzow), risolvendo insieme un problema aperto da 12 anni~
\cite{DBLP:conf/stacs/GanzowR08}.}

\subsection{Supervisione di tesi di laurea}
% While at Cornell I mentored six students for three months. This resulted in two publications \cite{DBLP:journals/tcs/GrinshpunPRT14,DBLP:journals/corr/abs-1210-2462}.
% %and gave students a taste of research to help them decide if they should pursue a PhD. 
% While at IST Austria I co-mentored one intern which resulted in \cite{DBLP:conf/lata/ChatterjeeCR13}. 
% % I recently co-mentored an undergraduate thesis at the University of Naples.

\cventry{2017}{Tirocinio di laurea. Tirocinante: Aurele Barriere}{Universit\`a 
di Napoli}{Argomento: Epistemic Logic}{}
{co-supervisore} %(4 months)

\cventry{2017}{Tesi di laurea (4 mesi): Paolo Lambiase}{Universit\`a di Napoli}
{Argomento: Graphical Games}{}{co-supervisore, recentemente terminata, articolo in scrittura} %(4 months)



 
 \subsection{Undergraduate Supervision}
\cventry{2012}{Undergraduate research project: Siddhesh Chaubal}{IST Austria}
{Argomento: Edit-distance e Formal Languages}{}{co-supervisore, pubblicato
\cite{DBLP:conf/lata/ChatterjeeCR13}} %(3 months)
\cventry{2009}{''Research Experience for Undergraduates`` (REU): Andrey 
Grinshpun, Pakawat Phalitnonkiat, Andrei Tarfulea}
{Cornell University}{Argomento: Parity Games}{}{pubblicato 
\cite{DBLP:journals/tcs/GrinshpunPRT14}} % (3 months)
\cventry{2009}{''Research Experience for Undergraduates`` (REU): Alex Kruckman, 
John Sheridan, Ben Zax}
{Cornell University}{Argomento: Automatic Structures with Advice}{}{pubblicato  
\cite{DBLP:journals/corr/abs-1210-2462}} %(3 months)



% \newpage

\section{Didattica}

% Mentre ero alla Cornell, ho collaborato con un certo numero di insegnanti tra 
% cui Maria Terrell (Dipartimento di Matematica) E David Way (direttore associato 
% della Cornell University Center for Teaching Excellence) per discutere 
% strategie di insegnamento di successo, sia filosofiche che concrete.
% Secondo le valuitazioni dei miei studenti,  ero chiaro, organizzato, 
% proattivamente disposto ad aiutare ed a motivare.

\subsection{Corsi di dottorato}

\cventry{2017}{Corso di dottorato di 18 ore}{Milestones in solving games on graphs}{Technical University of Vienna}{Pianificato 11/2017}{}

\cventry{2017}{Corso di dottorato di 10 ore}{Games 
on Graphs}{Universit\`a di Napoli}{Compiti: Organizzato e presentato}{} 
%9 partecipanti

\cventry{2009}
{Corso di dottorato semestrale}
{Topics in Applied Logic: Logical Definability and Random Graphs}
{Cornell University}{Compiti: Organizzato e presentato}{} % 5 partecipanti

\subsection{Corsi in Scuole estive}
\cventry{2006}{Corso avanzato di 5 giorni}
{Logic and Computation in Finitely Presentable Infinite Structures}
{European Summer School in Logic, Language and Information (ESSLLI 2006)}
{Compiti: Organizzato e presentato con Valentin Goranko}
{approssimativamente 20 partecipanti}



\subsection{Corsi universitari}


% \cventry{2016/2017}{Tecniche di Specifica}{Universit\`a di Napoli ''Federico 
% II``}{Corso di Laurea Magistrael in Informatica}{Compiti: Assistente di insegnamento}{}{}{}{}
\cventry{2015/2016}{Tecniche di Specifica}{Universit\`a di Napoli ''Federico 
II``}{Corso di Laurea Magistrale in Informatica}{Compiti: Assistenza alla didattica}{Il titolare del corso e' il Prof. Murano Aniello.}{}{}{}


\cventry{2010}
{Logic and Computation}
{Department of Mathematics, University of Cape Town, 40 studenti}
{Compiti: Strutturazione del corso, 30 lezioni teoriche, 12 lezioni di 
tutoraggio, 1 test, 1 esame finale}{}{}{}{}

\cventry{2008-2009}{Calculus for Engineers}{Department of Mathematics, Cornell 
University, 25-30 studenti. Corso semestrale tenuto 5 volte}{Compiti: Lezioni teoriche, quiz settimanali online, correzioni}{}{}

\cventry{2007}{Discrete Structures in Mathematics and Computer 
Science}{Department of Computer Science, University of Auckland}
{Compiti:  Lezioni teoriche, tutoraggio}{}{}{} %Corso di laurea, insegnamento condiviso

\cventry{2007}{Mathematical Foundations of Software Engineering}{Department of 
Computer Science, University of Auckland}
{Compiti:  Lezioni teoriche, tutoraggio}{}{}{}

\cventry{2003}{Introduction to Formal Verification}{Department of Computer 
Science, University of Auckland}
{Compiti:  Lezioni teoriche, tutoraggio}{}{}{} 
\cventry{2002}{Automata theory}{Department of Computer Science, University of 
Auckland}
{Compiti:  Lezioni teoriche, tutoraggio}{}{}{}

\cventry{2000-2001}{Pre-calculus}{Department of Mathematics, University of Wisconsin, 
Madison, +-30 studenti. Corso semestrale, tenuto 2 volte}{Compiti: Lezioni teoriche, 
tutoraggio, correzione}{}{}



%Computational Biology (undergraduate, TA)\\
%{Department of Computer Science, University of Auckland} $(2008)$\\





%  {\bf Tutoring}\\
%  {University of Auckland, Department of Mathematics} \\
%  $1998$ and $1999$: Undergraduate Mathematics\\


\iffalse $1998$ and $1999$: Stages $1$, $2$ and $3$ in the Mathematics \\
   	 %Assistance Room \\
	 $1999$: Assistant tutor for `Discrete Mathematics', Stage $2$
	Mathematics\\
	 $1999$: Demonstrator for `Combinatorial Computing', Stage $3$
	Mathematics\\
\fi

%\pagebreak 


 
\section{Disseminazione e Didattica extra-universitaria}


\subsection{Presentazione di lavori recentemente accettati a conferenza}
\cventry{2017}{IJCAI, Melbourne}{Generalised Planning: Non-Deterministic Abstractions and Trajectory Constraints}{}{}{}
\cventry{2017}{FMAI, Naples}{Verification of Multi-agent Systems with Imperfect Information and Public Actions}{}{}{}
\cventry{2016}{SR, New York}{LTL Reactive Synthesis under Assumptions}{}{}{}
\cventry{2016}{KR, Cape Town}{Model Checking Prompt Alternating-Time Epistemic Logics}{}{}{}
\cventry{2016}{IJCAI, New York}{Imperfect-Information Games and Generalized Planning}{}{}{}
\cventry{2016}{AAMAS, Singapore}{Automatic Verification of Multi-Agent Systems in Parameterized Grid-Environments}{}{}{}
\cventry{2015}{PRIMA, Bertinoro}{Verification of Asynchronous Mobile-Robots in Partially-Known Environments}{}{}{}
\cventry{2015}{HIGHLIGHTS, Prague}{The Composition Method and Parameterised Verification}{}{}{}
\cventry{2015}{AAMAS, Istanbul}{Parameterised Verification of Autonomous Mobile-Agents}{}{}{}
\cventry{2014}{VMCAI, San Diego}{Cutoffs for Parameterised Token-Passing Systems}{}{}{}
\cventry{2014}{SR, Grenoble}{First Cycle Games}{}{}{}
\cventry{2014}{HIGHLIGHTS, Paris}{First Cycle Games}{}{}{}
\cventry{2014}{FRIDA, Vienna}{Using automata and logic to reason about
parameterised robot protocols}{}{}{}
\cventry{2013}{LATA, Bilbao}{How to Travel between Languages}{}{}{}

\subsection{Outreach}
\cvitem{2010}{ Ho insegnato matematica come volontario alla 
scuola secondaria in Accra, Ghana, nelle classi di quinto grado.}
\cvitem{2010}{Sono stato volontario in Khayelitsha, 
South Africa, aiutando gli studenti di scuola superiore a preparare i 
loro esami di matematica.}
\cvitem{2009}{ Ho insegnato a due conferenze interattive alla Cornell 
University riguardo i) Hilbert's Hotel and Infinite Cardinals e ii) Algoritmi e 
Terminazioni.}

\newpage
%  \section{Considerazioni}
 
 \section{Importanti posizioni ricoperte nella ricerca scientifica negli ultimi 5 anni}

%\subsection*{Conference Talks} CiE ($2008$),

%$2008 - 2009$: Logic Seminar, Cornell Talks in the Logic Seminar at Cornell. \\
%Ongoing work with Anil Nerode and Dexter Kozen.\\

%$2.2007$: Attended the 'Model Theory and Computable Model Theory' workshop, part
%of the University of Florida's Special Year in Logic.\\

% \item Competed as part of a team of three, in the world finals of the 1998 ACM Programming Contest in Atlanta, Georgia USA, representing the University of Auckland and New Zealand.



\subsection{Chair, Organisation e partecipazione a comitati di programma}
\cvitem{2018}{Membro del Program Committee di International Conference on Autonomous Agents and Multi-agent Systems (AAMAS)}
\cvitem{2018}{Membro del Program Committee di AAAI Conference on Artificial 
Intelligence (AAAI)\newline\url{https://aaai.org/Conferences/AAAI-18/}}
\cvitem{2017}{Co-chair e co-organizzatore di Italian Conference su Theoretical Computer Science 
(ICTCS)\newline \url{http://ictcs2017.unina.it/}}
\cvitem{2017}{Co-chair di International Workshop su Strategic reasoning (SR) 
\newline \url{http://sr2017.csc.liv.ac.uk/}}
\cvitem{2017}{Co-organizzatore di Convegno Italiano di Logica 
Computazionale (CILC)\newline  
\url{http://cilc2017.unina.it/}}
\cvitem{2017}{Co-organizzatore e co-chair di First Workshop on Formal 
Methods in AI (FMAI)\newline  
\url{https://sites.google.com/site/fmai2017homepage/}}
\cvitem{2017}{Membro del Program Committees di International Joint 
Conference on Artificial Intelligence (IJCAI) }
\cvitem{2017}{Membro del Program Committees di AAAI Conference on Artificial 
Intelligence (AAAI) }
\cvitem{2017}{Membro del Program Committees per IRISA Master Research 
Internship}
\cvitem{2016}{Membro del Program Committees di the International Workshop of 
Strategic Reasoning (SR) }
\cvitem{2016}{Membro del Program Committees di International Symposium on Games, 
Automata, Logics and Formal Verification (GandALF) }
\cvitem{2016}{Membro del Program Committees di European Conference on Artificial 
Intelligence (ECAI) }
\cvitem{2013}{Co-organizzatore di IST Austria Young Scientist Symposium on 
the topic `Understanding Shape: {in silico} e {in vivo}'\newline 
\url{ist.ac.at/young-scientist-symposium-2013/}}
\cvitem{2012}{Fondatore del seminario di computer science a IST Austria il 
cui obiettivo era quello di promuovere collaborazioni all'interno 
dell'istituto tra informatici e biologi.\newline 
\url{ist.ac.at/computer-science-seminar/}}

% \section{Service}

\subsection{Editoriale}

\cvitem{2017}{Guest editor, Special issue of SR 2017, Information and Computation, In process.}
\cvitem{2017}{Guest editor, Special issue of ICTCS 2017 and CILC 2017, Theoretical Computer Science, In process.}
\cvitem{2017}{Editor, Joint proceedings of ICTCS 2017 and CILC 2017, CEUR 
Workshop proceedings, ISSN 1613-0073, \url{ceur-ws.org/Vol-1949/}}


\subsection{Coordinatore di progetti}
\cvitem{2013-2016}{Partecipazione alla realizzazione del Handbook of Model 
Checking,
pubblicato da Springer ed editato da Edmund Clarke, Thomas Henzinger, Helmut 
Veith e Roderick Bloem. Compiti inclusi: Assistente editore in questioni 
manageriali, organizzative e tecniche, tra cui: organizzazione di revisioni, 
revisori e copy-editors; assicurare il collegamento tra gli editori e l'editor 
di Springer. \url{http://www.springer. com/us/book/9783319105741}
}
% \subsection{Grant writing}

% As project coordinators, they assisted the editors in all managerial,
% organizational, and technical matters necessary for bringing such a
% large project to fruition:
% they managed the collaboration software and the interaction with the
% authors, reviewers, copy editors, and the publisher throughout much
% of the project.

% \item In 2014, I volunteered for the Vienna Summer of Logic, the largest event in the history of logic.\\
% \textsf{http://vsl2014.at/}

\subsection{Attivit\`a di revisione}
\cvitem{Book}{Handbook of Model Checking}
\cvitem{Funding}{Icelandic Research Fund}

\cvitem{Journals}
{Artificial Intelligence (AIJ), Journal of Symbolic Logic (JSL), Logical Methods in Computer 
Science (LMCS), Transactions on Computational Logic (ToCL), 
Theory of Computing Systems (ToCS), Central European Journal of Mathematics, 
Information and Computation (IC), Journal of Logic and Computation (JLC), Annals of 
Mathematics and Artificial Intelligence (AMAI), Theory and Practice of Logic 
Programming (TLP), Science of Computer Programming (SCP)}
\cvitem{Conferenze}{IJCAI, KR, AAMAS, AAAI, EUMAS, ECAI, LICS, STACS, ICALP, 
MFCS, CONCUR, CSL, FoSSaCS, FSTTCS, SR, KRR@SAC, CiE, GandALF, RV, LPAR, LATA}


\subsection{Visite di ricerca recenti}


\cventry{2015,2016,2017}{Prof: Giuseppe De Giacomo, Sapienza, Roma}
% \hfill $12.2015$\\
{Argomento 1: Synthesis under Assumptions;
Argomento 2: Generalised Planning with Partial Observability}{}{}{}

\cventry{2016,2017}{Prof: Mike Wooldridge, Oxford University}{Argomento: 
Rational Synthesis}{}{}{}% \hfill $03.2016, 01.2017$\\

\cventry{2016,2017}{Prof: Alessio Lomuscio, Imperial College London}% \hfill  
%$03.2016, 01.2017$\\
{Argomento: Strategic-Epistemic logics for Multi-Agents Systems}{}{}{}

\cventry{2016}{Prof: Diego Calvanese e Marco Montali, Universit\`a di Bolzano}% 
%\hfill $07.2016$\\
{Argomento 1: Data-aware strategic logics;
Argomento 2: Knowledge Representation for Business Process Management}{Talk: 
Removing partial observability from generalised planning}{}{}


\cventry{2016}{Prof: Frank Stephan and Sanjay Jain, National University of 
Singapore}% \hfill $05.2016$\\
{Argomento: Learning Theory and Verification}{}{}{}


\cventry{2015}{Prof: Helmut Veith, TU Wien}% \hfill $08.2015$\\
{Argomento 1: Logic and Impossibility Results in Distributed Computing;
Argomento 2: Abstractions for Fault-tolerant Distributed Algorithms}{}{}{}

% %\item Host: \L ukasz Kaiser, Universit\'e Paris Diderot, France \hfill  $10.2011$\\
% %Topic: Application of Logic to AI
% 
% \item Host: Aniello Murano, Universit\`a degli Studi di Napoli ``Federico II'' \hfill $08.2011$\\
% Topic: Games of Imperfect Information and Pushdown Automata 
% 
% \item Host: Alexander Rabinovich, Tel Aviv University, Israel \hfill $5.2011$ -- $8.2011$\\
% Topic: Logical-Interpretability and Trees 
% 
% %Visited Vince 20 August? till end of September, 2010 in Warsaw.
% 
% \item Host: Erich Gr\"adel, RWTH Aachen \hfill $08.2006-01.2007$
% 
% 
% \item Host: Moshe Vardi, Rice University \hfill $01.2001-05.2001$
% 
%\item Steffen Lempp at UW Madison \hfill $08.2000-12.2000$

 \subsection{Invited Workshop Talks}
%
%\subsection*{Recent Seminar Talks}  IST Austria and TU Vienna ($2011,2012$), CNRS Liafa Paris 7 ($2011$), Tel Aviv University ($2011$), University of Cape Town ($2010$) \\
%Cornell ($2007,2008,2009$),
%\ifcut LSV Cachan ($2008$), CNRS LIAFA Paris 7 ($2007$), Heidelberg ($2007$)\\
%\fi
%\subsection*{Invited Talks}

\cventry{2017}{Games of Imperfect-information with Public Actions}{RoboLog, Rennes}{}{}{} %$02.2017$
\cventry{2017}{Verification of Multi-Agent Systems with Imperfect Information 
and Public Actions}{FMAI17, Napoli}{}{}{} %$02.2017$
\cventry{2012}{Finite and Algorithmic Model Theory}{Les Houches, France}{}{}{} %$05.2012$
\cventry{2011}{Automata theory and Applications}{IMS programme,  Singapore}{}{}{}% \hfill $09.2011$
\cventry{2008}{Computational Model Theory}{CNRS SIG, Bordeaux, France}{}{}{}% \hfill $06.2008$
\cventry{2007}{Algorithmic-Logical Theory of Infinite Structures}{Dagstuhl, Germany}{}{}{}% \hfill $10.2007$
\cventry{2006}{Finite and Algorithmic Model Theory}{Newton Institute, England}{}{}{}% \hfill $01.2006$
\cventry{2004}{Workshop on Automata, Structures and Logic}{Auckland, New Zealand}{}{}{}% \hfill $12.2004$


\subsection{Invited Seminar Talks}
\cventry{2018}{To be determined}{Yale-NUS, Singapore}{}{}{}
\cventry{2018}{To be determined}{University of Auckland, New Zealand}{}{}{}
\cventry{2018}{To be determined}{IRIF, Universit\'e Paris-Diderot}{}{}{}
\cventry{2017}{Complexity of strategic reasoning under partial observability}{IMT Lucca, Italy}{}{}{}
\cventry{2017}{Complexity of strategic reasoning under partial observability}{GSSI, Italy}{}{}{}
\cventry{2017}{Temporal-Strategic Reasoning for Partial-Observation Games}{University of New South Wales, Australia}{}{}{}
\cventry{2016}{Imperfect-Information Games and Generalized Planning}{Free university of Bolzano, Italy}{}{}{}
\cventry{2014}{Verification of Mobile Agents in Partially Known Environments}{University of Naples, Italy}{}{}{}
\cventry{2014}{Memoryless Determinacy of Cycle Games}{University of California, San Diego, USA}{}{}{}
\cventry{2012}{Automata theoretic approach to mixed integer and rational arithmetic}{IST Austria, Austria}{}{}{}
\cventry{2011}{Representing infinite structures by automata}{TU Wien, Austria}{}{}{}
\cventry{2011}{An introduction to automatic structures}{Tel Aviv University, Israel}{}{}{}
\cventry{2011}{Representing infinite structures by automata}{EPFL, Switzerland}{}{}{}
\cventry{2008}{Generalised Quantifiers on Automatic Structures}{LSV Cachan, France}{}{}{}
\cventry{2007}{Generalised Quantifiers on Automatic Structures}{LIAFA Paris, France}{}{}{}
\cventry{2007}{Decidable extensions of the Monadic Second-order theory of one successor by unary predicates}{Cornell University, USA}{}{}{}
\cventry{2007}{Automatic Structures}{Heidelberg University, Germany}{}{}{}



\newpage
\section{Articoli pubblicati con referaggio}

% I was lead or co-lead author for all publications \emph{except} the following 
% where I played significant (e.g., supervisory) but secondary roles: 
% Sono stato ``lead or co-lead'' autore in tutte le pubblicazioni, eccetto le 
% seguenti in cui ho giocato un significativo ma secondario ruolo: 
% \cite{DBLP:conf/lics/BerthonMMRV17, 
% DBLP:journals/corr/abs-1210-2462,DBLP:journals/tcs/GrinshpunPRT14}.


Le bibliometrie citate sono le seguenti: per quanto riguarda le conferenze 
vengono dati i rispettivi CORE 
(\url{http://portal.core.edu.au/conf-ranks/}) letter ranking, seguito dal 
tasso di accettazione, seguito dal numero di sottomissioni; per quanto 
riguarda i journal sono dati i rispettivi SJR letter ranking 
(\url{http://www.scimagojr.com/journalrank.php}) al momento della 
pubblicazione (nell caso di pubblicazioni su rivista nell'anno in corso per le quali
non e encora noto il Ranking dell'anno, si riporta il Ranking medio degli ultimi 5 anni).
% Queste bibliometrie sono una misura di influenza di
% conferenza/journal e non riflettono, a priori, la qualit\`a del un singolo 
% paper.

% In sintesi: Ho 16 articoli in CORE A* conferences, 
% 7 in CORE A conferences, 4 in CORE B conferences e  4 articoli in Q1 journals, 
% 2 articoli in Q2 journals, 1 book, and 1 book chapter.

% \renewcommand{\listitemsymbol}{-~} % Changes the symbol used for lists
% \subsection{Summary (the cited bibliometrics are a measure of journal influence and do not, apriori, accurately reflect the quality of an individual paper)}
% \cvitem{Ranking}{Number of publications}
% \cvitem{CORE A*}{16}
% \cvitem{CORE A}{7}
% \cvitem{CORE B}{4}
% % \cvitem{SJR Q1}{3}
% % \cvitem{SJR Q2}{2}

% {5 LICS papers (A*)}
% {3 IJCAI2017 (A*, approx 25\% acceptance, 2540 submissions)}
% {1 IJCAI2016 (A*, approx 25\% acceptance, 2294 submissions)}
% {2 AAMAS2017 (A*, approx 26\% acceptance, 595 submissions)}
% {1 AAMAS 2016 (A*, approx, 25\% acceptance, 550 submissions}
% {1 AAMAS2015 (A*, approx 25\% acceptance, 670 submissions)}
% {1 KR paper (A*)} 182, 26.9%
% {1 ICALP paper (A)}{}
% {1 CAV paper (A*)}{}
% {1 CONCUR paper (A)}{}
% {2 LPAR papers (A)}
% 1 IJCAR A*

%1 VMCAI B
% CSL B
% {3 STACS papers (B)}{}
% {1 KR paper (A*)} 182, 26.9%
% {1 ICALP paper (A)}{}
% {1 CAV paper (A*)}{}
% {1 CONCUR paper (A)}{}
% {2 LPAR papers (A)}
% 2 PRIMA (B)
% 
% 
%   {1 ACM TOCL (Q1)}
%   {1 TCS (Q1)}
%   {1 I\&C (Q2)}
%   {1 LMCS (unranked at time, currently Q1)}
%   {1 BSL (Q1)}


% 
% I have 40 refereed publications, most in top conferences and journals, 
% including 1 (co-authored) book, 1 (sole authored) book-chapter, 6 journal 
% articles (5 of them invited), 5 {\textsc LICS} papers, 4 {\textsc AAMAS} papers, 4 {\textsc IJCAI} papers, 
% 3 {\textsc STACS} papers,  1 {\textsc KR} paper, and a best-paper at {\textsc 
% PRIMA}. Not listed, are 6 invited journal articles (in preparation or under 
% evaluation) and an invited chapter in a handbook on automata theory and 
% applications (in preparation, J.E. Pin (ed.), to be published by EMS).


\nocite{*}

\printbibliography[heading=subbibliography,title={Libro}, prefixnumbers={B},type=book]

\printbibliography[heading=subbibliography,title={Capitoli di Libri}, prefixnumbers={BC},type=incollection]

\printbibliography[heading=subbibliography,title={Articoli a conferenza}, prefixnumbers={C},type=inproceedings,notkeyword={workshop}]

\printbibliography[heading=subbibliography,title={Articoli a rivista}, prefixnumbers={J},type=article]

\printbibliography[heading=subbibliography,title={Articoli in Workshops}, prefixnumbers={W}, keyword={workshop}]


% Publications from a BibTeX file without multibib
%  for numerical labels: \renewcommand{\bibliographyitemlabel}{\@biblabel{\arabic{enumiv}}}% CONSIDER MERGING WITH PREAMBLE PART
%  to redefine the heading string ("Publications"): \renewcommand{\refname}{Articles}
% \nocite{*}
% 
% \bibliographystyle{plain}
% \bibliography{/home/sr/svn/forsyte-publications/trunk/rubin.bib}                        % 'publications' is the name of a BibTeX file

% Publications from a BibTeX file using the multibib package
%\section{Publications}
%\nocitebook{book1,book2}
%\bibliographystylebook{plain}
%\bibliographybook{publications}                   % 'publications' is the name of a BibTeX file
%\nocitemisc{misc1,misc2,misc3}
%\bibliographystylemisc{plain}
%\bibliographymisc{publications}                   % 'publications' is the name of a BibTeX file


% \section{Online Profiles}
% % \cvitem{CORE}{A*x16, Ax7, Bx4}
% % \cvitem{SJR}{Q1x2, Q1/Q2x1, Q2x1}
% \cvitem{Current list of publications}{\url{http://forsyte.at/alumni/rubin/publications/?nocache}}
% \cvitem{Google scholar}{\url{https://scholar.google.it/citations?user=auUS1rMAAAAJ&hl=en&oi=ao}}
% \cvitem{SCOPUS}{\url{https://www.scopus.com/authid/detail.uri?authorId=7201922792}}
% \cvitem{Semantic Scholar}{\url{https://www.semanticscholar.org/author/Sasha-Rubin/2807596}}


\ifref
\newpage
\section{Referenze}

\subsection{Didattica}

\cventry{Mentore}
{Maria Terrell}
{Director of Teaching Assistant Programs}
{Cornell University}
{}
{maria@math.cornell.edu}


\cventry{Mentore}{David Way}
{Associate Director of Instructional Support}
{Center for Teaching Excellence}
{Cornell University}
{dgw2@cornell.edu}

% \subsection{Supervisione}
% 
% \cventry{Mentore}{Bob Strichartz}
% {Department of Mathematics}
% {Cornell University}
% {}
% {str@math.cornell.edu}


\subsection{Accademico}

\cventry{Datore di lavoro precedente}{Roderick Bloem}
{Institute for Applied Information Processing and Communication}
{Technische Universit\"at Graz}
{}
{roderick.bloem@iaik.tugraz.at}

\cventry{Collaboratore corrente}{Giuseppe De Giacomo}
{Dipartimento di Ingegneria Informatica, Automatica e Gestionale} 
{Sapienza, Universit\`a di Roma}
{}
{degiacomo@dis.uniroma1.it}

\cventry{Collaboratore passato}{Erich Gr\"adel}
{Mathematische Grundlagen der Informatik}
{RWTH Aachen}
{}
{graedel@logic.rwth-aachen.de}

\cventry{Supervisore del dottorato}{Bakhadyr Khoussainov} 
{Department of Computer Science}
{University of Auckland}
{}
{bmk@cs.auckland.ac.nz}

\cventry{Collaboratore corrente}{Alessio Lomuscio}
{Faculty of Engineering, 
Department of Computing}
{Imperial College London}
{}
{a.lomuscio@imperial.ac.uk}

\cventry{Datore di lavore corrente}{Aniello Murano}
{Dipartimento di Ingegneria Elettrica e Tecnologie dell'Informazione} 
{Universit\`a degli Studi di Napoli ``Federico II''}
{}
{murano@na.infn.it}


\cventry{Collaboratore passato}{Frank Stephan}
{School of Computing}
{National University of Singapore}
{}
{fstephan@comp.nus.edu.sg}

\cventry{Collaboratore corrente}{Michael Wooldridge}
{Department of Computer Science}
{University of Oxford}
{}
{mjw@cs.ox.ac.uk}
\fi

\end{document}




\section{\mysidestyle{Premi e Riconoscimenti}}
\begin{itemize}
\item 2 borse di studio individuali (1 borsa di studio Marie Curie indetta 
dall' INdAM, 1 assegno di ricerca presso New Zealand Science and Technology).
\item 2 premi dottorato (miglior tesi di dottorato nella Facolt\`a di 
Scienze ed il  ``Montgomery memorial prize'' in logica dal Dipartimento di 
Filosofia).

\end{itemize}


\begin{itemize}


\end{itemize}

%
%
%$1998$: Competed as part of a team of three, in the world finals of the $1998$
%ACM Programming Contest in Atlanta, Georgia USA, representing the University of Auckland
%and New Zealand. 
%%The same team won the Regional Programming Contest in $1997$.\\ 
%\fi

\section{\mysidestyle{Recent service}}

%%%HBMC
%I am a chair of the ...

\begin{itemize}

%One reviewer wrote "We are extremely grateful to this reviewer for his/her careful reading of the paper, and for his/her constructive suggestions."

%equaleducation.org.za

%\pagebreak

\end{itemize}

% \newpage


\section{\mysidestyle{References}}

%\begin{multicols}{2}
%\subsection*{Primary}

\subsubsection*{\sc{Academic}}




\section{\mysidestyle{Refereed Publications}}



\nocite{*}

\printbibliography


%prima x2, SR special issue, concur, 

%The following conference papers were invited to journals \cite{
%DBLP:journals/jalc/KhoussainovR01,
%DBLP:journals/jalc/KhoussainovR03,
%DBLP:conf/lics/KhoussainovNRS04,
%DBLP:journals/corr/AminofR14,
%DBLP:journals/lmcs/KhoussainovNRS07,
%DBLP:conf/concur/AminofKRSV14}. 
%The journal versions of the following are under evaluation, and not listed below: \cite{DBLP:conf/concur/AminofKRSV14}.





%\begin{description}

%\item[Book]\
%
%{Decidability of Parameterized Verification} with R. Bloem, S. Jacobs, A. Khalimov, I.
%    Konnov, H. Veith and J. Widder, in {Synthesis Lectures in Distributed Computing Theory}, N. Lynch Ed., September 2015, 170 pages
%    
%    
%\item[Book chapters]\
%
% {Automatic Structures} in {Automata: From mathematics to applications}, J.E. Pin, Ed., to be published by EMS.
%
% {Automata based presentations of infinite structures} with V. B{\'a}r{\'a}ny and E. Gr{\"a}del,
%in {Finite and Algorithmic Model Theory}, J. Esparza, C. Michaux, and C. Steinhorn, Eds.,
%Series: London Mathematical Society Lecture Note Series (379), $1-76$, $2011$. \cc{15}
%
%\item[Journals]\
%
%{First-Cycle Games} with B. Aminof, Information and Computation, $2016$.
%
%{Alternating Traps in Parity Games} with P. Phalitnonkiat, A. Grinshpun, A.Tarfulea, Theoretical Computer Science,  $73-91, 2014$.
%
%{Automata presenting structures: A survey of the finite-string case}, The Bulletin of Symbolic Logic, 
%$14 (2), 169-209, 2008$. \cc{66}
%
%{Automatic Structures: Richness and Limitations}, with B. Khoussainov, A. Nies and F. Stephan, 
%Logical Methods in Computer Science, Vol $3$, $2007$.  \cc{78}
%
%{Automatic linear orders and trees}, with B. Khoussainov and F. Stephan, 
%ACM Transactions on Computational Logic,
%$6 (4), 675-700, 2005$.  \cc{49}
%
%
%%{Automatic Linear Orders and Trees: Revised}, CDMTCS Technical Report $208$,
%%Department of Computer Science, University of Auckland, $2003$.   
% 
% %{Definability and Regularity in Automatic Presentations of Subsystems of
%%Arithmetic}, CDMTCS Technical Report $209$,
%%Department of Computer Science, University of Auckland, $2003$.   
% 
%{Automatic Structures - Overview and Future Directions}, with 
%B. Khoussainov,
%Journal of Automata, Languages and Combinatorics, $8(2), 287-301, 2003$.   \cc{28}
%
%{Graphs with Automatic Presentations over a Unary Alphabet}
%Journal of Automata, Languages and Combinatorics, $6(4), 467-480, 2001$. \cc{15}  
%
%{Finite Automata and Well Ordered Sets},
%New Zealand Journal of Computing, $7(2), 39-46, 1999$. 
%
%
%\item[IJCAI Proceedings]\
%
%{Imperfect-Information Games and Generalized Planning}, with
%G. De Giacomo, A. Di Stasio, A. Murano, $2016$.
%
%\item[KR Proceedings]\
%
%{Prompt Alternating-Time Epistemic Logics}, with B. Aminof, A. Murano, F. Zuleger, $2016$.
%
%\item[AAMAS Proceedings]\
%
%{Automatic verification of multi-agent systems in parameterised grid-environments}, with
%B. Aminof, A. Murano, F. Zuleger, $2016$.
%
%{Graded Strategy Logic: Reasoning about Uniqueness of Nash Equilibria}, with
%B. Aminof, V. Malvone, A. Murano, $2016$.
%
%{Parameterised Verification of Autonomous Mobile-Agents in Static but Unknown Environments}, $2015$.
%
%\item[PRIMA Proceedings]\
%
%{Multi-Agent Path Planning in Known Dynamic Environments}, with A. Murano, G. Perelli, $2015$. 
%
%\item[LICS Proceedings]\
%
%{Interpretations in trees with countably many branches}, with A. Rabinovich, $551-560$, $2012$. \cc{3}
%
%
%{Automatic Structures: Richness and Limitations}, with B. Khoussainov, A. Nies and F. Stephan, 
%$44-53$, $2004$. \cc{78} 
%
%{Automatic Partial Orders}, with B. Khoussainov and F. Stephan, $168-177$, $2003$. \cc{33}
%
%{Some Results on Automatic Structures}, with B. Khoussainov
%and H. Ishihara,  $235-244$, $2002$. \cc{13}
%
%\item[STACS Proceedings]\
%
%{Cardinality and counting quantifiers on omega-automatic structures}, with V.  B{\'a}r{\'a}ny and \L. Kaiser, $385-396$, $2008$.  \cc{22}
%
%{Order invariant MSO is stronger than counting MSO}, with T. Ganzow, $313-324$,  
% $2008$.  \cc{9}
% 
%{Definability and Regularity in Automatic Structures}, with B. Khoussainov
%and F. Stephan,  $440-451, 2004$.  \cc{23}
%
%%International Workshop on Logic and Computational Complexity $2007$.\\
%
%\item[CONCUR Proceedings]\
%
%{ Parameterized model checking of Rendezvous Systems}, with B. Aminof, T. Kotek, F. Spegni and H. Veith, $109-124$, $2014$
%
%\item[CAV Proceedings]\
%
%{Verifying $\omega$-regular Properties of Markov Chains}, with D. Bustan and
%M. Vardi, $189-201, 2004$. \cc{18}
%
%\item[ICALP Proceedings]\
%
%{Liveness of Parameterized Timed Networks}, with B. Aminof, F. Spegni and F. Zuleger, $375-387, 2015$.
%
%
%\item[VMCAI Proceedings]\
%
%{Parameterized Model Checking of Token-Passing Systems}, with B. Aminof, S. Jacobs and A. Khalimov, $262-281, 2014$. \cc{6}
%
%
%\item[Other Refereed Proceedings]\
%
%{Model Checking Parameterised Multi-Token Systems via the Composition Method}, with
%B. Aminof, IJCAR $2016$.
%
%
%{On CTL* with Graded Path Modalities}, with
%B. Aminof, A. Murano, {LPAR}, $2015$.
%
%{On the expressive power of communication primitives in parameterised systems},
%B. Aminof and F. Zuleger, {LPAR} $2015$.
%
%{Cycle Games} with B. Aminof, {Strategic Reasoning}, ETAPS workshop, $2014$.
%
%{How to Travel Between Languages} with  K. Chatterjee and S. Chaubal, {LATA}, $2013$.
%
%{A Myhill-Nerode Theorem for Automata with Advice} with A. Kruckman, J. Sheridan and B. Zax, {GandALF}, $238-246$, $2012$. \cc{2}
%
%%{Quantifiers on Automatic Structures} with V. Goranko, {CiE}, $2008$.
%

%\item[Thesis]\
%
%{Automatic Structures}, University of Auckland, $2004$ \cc{55}
%
%%{Satisfiability of $CTL^*$ with graded path modalities}, with B. Aminof, A. Murano.
%    
%\end{description}







\end{document}



%----------------------------------------------------
%----------------------------------------------------
%----------------------------------------------------
%----------------------------------------------------
%----------------------------------------------------
%----------------------------------------------------


\iffalse
{\bf Dr. Valentin Goranko}\\
School of Mathematics\\
University of Witwatersrand\\ 
Private Bag 3, WITS 2050\\
Johannesburg, South Africa\\
goranko@maths.wits.ac.za\\
Phone : +$27$ $11$ $717$ $6243$ \\
%Fax   : +$27$ $11$ $717$ $6259$ \\
\fi
%- Tutorial co-ordinator for undergraduate paper `Discrete Mathematics', %1999 \\
%- Project and summer scholarship supervisor for `Application of Elementary \\
%\phantom{- }Submodels to Topology', 1998\\ 
%- Lecturer for `Logic and Set Theory', 1998\\ 
% \enlargethispage*{1cm}

%\section*{Conferences Attended}
%\begin{tabular}{@{}ll}
% {\bf 2000} & Computational Group Theory, Sydney\\
% {\bf 1999} & Third New Zealand Computer Science Research Students'
%Conference \\
%	    & April, Waikato\\
% {\bf 1999} & ACSC - DMTCS/CATS \\
%	    & January, Auckland \\
% {\bf 1999} & NZMRI summer workshop, Harmonic Analysis\\
%	    & January, Raglan, New Zealand\\
% {\bf 1998} & Second Japan-New Zealand Workshop on 
%		{\em Logic in Computer Science}\\
%	    & October, Auckland\\
% {\bf 1998} & First International Conference on 
%		{\em Unconventional Models of Computation}\\
%	    & January, Auckland \\
% {\bf 1997} & First Japan-New Zealand Workshop on {\em Logic in Computer Science}\\
%	    & August, Auckland\\
% {\bf 1997} & Fifth Australasian Mathematics Convention\\
%	    & July, Auckland\\
%\end{tabular}


\iffalse
\subsection*{Additional}
{\bf Dr. David McIntyre}\\
Department of Mathematics\\
University of Auckland, New Zealand\\
mcintyre@math.auckland.ac.nz\\
Phone : (+64 9) 373 7599 Ext 8763\\
%- Tutorial co-ordinator for undergraduate paper `Discrete Mathematics',
%1999 \\
%- Project and summer scholarship supervisor for `Application of Elementary \\
%\phantom{- }Submodels to Topology', 1998\\
%- Lecturer for `Logic and Set Theory', 1998\\
\enlargethispage*{1cm}


%\pagebreak
\section*{Non Academic}
\subsection*{University of Auckland}
{Membership}\\
  Auckland University Dramatic Society (1998/1999/2000)\\
  Auckland University Comedy and Improvisation Club (1997) Secretary (1998)\\


{Performance}\\
  `Morte Accidentale Di Un Anarchico', (Dario Fo) (1999)\\
  Celebration of Performing Arts, Auckland Town Hall (Stoppard, Simon) (1999)\\
  Three comedy shows, Auckland University (1997 and 1998)\\
  A duet in an evening of short plays, Auckland University (Stoppard) (1998)\\
  Cultural Mosaic Festival, Auckland University (Stoppard) (1997)\\
\begin{center}
--------------------
\end{center}
\fi


\section*{Seminari e Talk} 

 
' Temporal-Strategic Reasoning for Games of Imperfect Information', UNSW computer science seminar, 2017\\

`Techniques to prove non-automaticity', University of Heidelberg Logic Seminar, 2002 \\


`Some Results on Automatic Structures', con Bakhadyr Khoussainov
e Hajime Ishihara, 17th Annual IEEE Symposium on Logic in Computer Science, 
2002. \\


`Automata-theoretic approach to verification of probabilistic systems',
Rice University Computer Science Theory Seminar, 2001. \\


`Automatic Structures', University of Notre Dame Logic Seminar, 2001, e \\
University of Madison, Wisconsin, Logic Seminar, 2001.\\


`Finite Automata and Relational Structures', con Bakhadyr Khoussainov, 
DCAGRS, July 2000, London, Ontario \\


 `Finite Automata and Well Ordered Sets', 
3rd New Zealand Computer Science Research Conference, 1999, Waikato, New
Zealand \\


{\em Auckland Department of Computer Science:}\\
 `Finite Model Theory - Ehrenfeucht-Fraisse Theorem', 2000\\
 `Extracting Algebraic Information from Finite State Machines', 1999\\
 `Finite Automata and Regular Languages', 1999\\


{\em Auckland Department of Mathematics:}\\
 `Algebraic Structures and Finite Automata', 1999\\
 `Applications of Elementary Submodels to Topology', 1999\\ 
\fi

\iffalse
		
  {\bf Marking -} {University of Auckland} \\
	 $1998$: Stage $3$ Assignments for the Department of Mathematics\\
  	 1997: Stage 1 Assignments for the Departments of 
	 Computer Science and Mathematics\\
\fi

