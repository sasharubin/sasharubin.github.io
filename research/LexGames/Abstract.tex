% Begin of file Abstract.tex

\begin{abstract}

%	We study the problem of finding Nash Equilibria of games in which the agents goals are a combination of qualitative and quantitative objectives.
%	The qualitative side is represented by \LTL formulas, while the quantitative part is designed by applying a rational cost for each action taken by the agents and aggregating with a mean-payoff flavour.
%	Every agents wants to first satisfy its \LTL objective and then, as a secondary goal, to do it by optimizing (in terms of mean-payoff costs) the actions needed to achieve the qualitative objective.
%	We show that, for turn-based games, there always exists a Nash Equilibrium and the problem of synthesizing it is decidable.
%	For what regards concurrent games, we provide a decidable procedure to decide whether there exists a Nash Equilibrium.
%	
	We introduce lexicographic games, which are multi-player games where players have goals given by a two-level lexicographic order: in the first place, a player is interested in satisfying a temporal logic goal, represented with an \LTL formula, and in the second place they want to do so at the lowest possible cost.
	These games are played in (weighted) graphs and can be used to model in a natural way the behaviour of concurrent and multi-agent systems.
	In particular, in this paper, we study the problem of computing Nash equilibrium strategy profiles, whenever they exist.
	Our main result is that%, unlike for similar games and decision questions in the literature,
	the problem for lexicographic games, where players have \LTL goals and costs are given by a mean-payoff function, is decidable. % and can be solved using automata-theoretic techniques.
	We also study the turn-based case and show that in this case the question as to whether a lexicographic game has any Nash equilibria always has a positive answer.  
	
\end{abstract}




% End of file Abstract.tex
