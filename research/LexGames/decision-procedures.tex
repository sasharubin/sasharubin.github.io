\section{Decision Procedures} \label{sec:decision procedures}



\begin{theorem}
 NE-emptiness of $\LEX(\LTL,\mp)$-games is decidable. \todo{Todos: 1) debug, 2) polish, 3) calculate complexity of the algorithm}
\end{theorem}

\begin{proof}[Sketch]
We prove this in three steps. In the first step we push the weights of $G$ into the states and translate the result into a $\LEX(\parity,\mp)$ game $G'$ such that $\NE(G) = \NE(G')$. Here $\parity$ is an aggregation function that maps a sequence of priorities (integer weights) to $0$ if the smallest priority occuring infinitely often is even, and to $1$ otherwise (the priorities are given as a second weight function, in addition to $\kappa$). In the second step we show how to reduce $NE$-emptiness of $G'$ 
to the problem of solving two-player zero-sum games $H$ in which player $0$ has a $\LEX(\parity,\mp)$ objective. We do this by adapting the proof that shows how to decide $NE$-emptiness of mean-payoff games~\cite{DBLP:conf/concur/UmmelsW11}.  In the third step we reduce $H$ to solving mean-payoff parity games. These are two-player zero-sum games in which the objective of player $0$ is to ensure both the parity condition holds and that the mean-payoff is maximised~\cite{DBLP:conf/lics/ChatterjeeHJ05}. Formally: the payoff is $-\infty$ if the smallest priority occuring infinitely often is even, otherwise it is the mean-payoff of the weights.

% $\parity(\alpha) \in \{0,1\}$ is $0$ iff the $min(inf(\alpha))$ is even, where $inf(\alpha)$ consists of the weights occuring infinitely often in $\alpha$.

\head{Replace \LTL by Parity}
For the first step, push the weights to the states. This is done by replacing $St$ by $St \times Act^{Ag}$ and defining the weight of $(s,d)$ to be $\kappa(d)$. Convert each \LTL formula $\varphi_a$ into a deterministic parity automaton $D_a$ of size double-exponential in $\varphi_a$~\cite{??}. To build the game $G'$ form the product weighted arena $\St' = \St \times \prod_a D_a$. Each agent has a pair of weights associated with each state, i.e., a priority (coming from $D_a$) and the integer weight (coming from $\kappa$).  Given a play, the payoff for an agent is $(x,y)$ where $x = 0$ iff the smallest priority occuring infinitely often on the play is even, and $y$ is the meanpayoff of the integer weights. By construction we have that $\NE(G) = \NE(G')$.

\head{Reduce \NE\ to path-finding}
For the second step, we adapt the proof in Section~6 of \cite{DBLP:conf/concur/UmmelsW11} that shows how to decide $NE$-emptiness for mean-payoff games. For $a \in Ag$ and $s \in \St$ define the \emph{punishing value} $p_a(s)$ to be the $\lex$-largest $(x,y)$ that player $a$ can achieve from state $s$ by ``going it alone'', i.e., by playing against the coalition $Ag \setminus \{a\}$.  We will show how to compute these values in the third step. 

Definition: For an agent $a$ and $\bar{z} \in \mathbb{R}^{|Ag|}$, a pair $(s,d) \in \St \times Act^{Ag}$ is \emph{$\bar{z}$-secure for $a$} if $p_a(tr(s,d')) \leq z_a$ for every $d' \in Act^{Ag}$ that agrees with $d$ except possibly at $a$. 

Claim: $\NE(G')$ is non-empty iff there exists $\bar{z}$ where $z_a \in \{p_a(s) : s \in \St\}$ and there exists an execution $\pi = s_0 d_0 s_1 d_1 \cdots$ in $G'$ such that for every agent $a$,  i) $z_a \leq pay_a(\pi)$ and ii) for all $i \in \mathbb{N}$, the pair $(s_i,d_i)$ is $\bar{z}$-secure for $a$. 


Sketch proof of Claim: Suppose $\NE(G')$ is non-empty. Let $\pi$ be the execution resulting from some Nash-profile. 
Let $z_a = \max\{p_a(\delta(s_n,d'_n)) : n \in \mathbb{N}, \wedge_{b \neq a} d'_n(b) = d_n(b)\}$, i.e., $z_a$ is the largest value player $a$ can get by deviating from $\pi$. Clearly $(s_n,d_n)$ is $\tup{z}$-secure for $a$. Moreover, $z_a \leq pay_a(\pi)$: indeed, if $z_a = p_a(\delta(s_n,d'_n)) > pay_a(\pi)$ then player $a$ would deviate at step $n$ by playing $d'_n$ and achieve a higher payoff, contradicting the choice of $\pi$ as the execution of a Nash-profile.

For the converse direction, let $\bar{z}$ and $\pi$ be given with the stated properties. The following is a Nash-profile: every agent plays ``follow $\pi$'' until, and if, a single player, say $a$, deviates, say at state $s$; in that case the other agents play a punishing strategy, i.e., a strategy that ensures player $a$ achieves no more than he possibly can given the current history. It can be shown that this value is $p_a(s)$ for $LEX(\parity,\mp)$ payoff functions (since such payoff functions are prefix-independent). This completes the proof of the claim.





Note that $p_a(s)$ is the value of the two-player zero-sum game $H_a(s)$ in which the first player is trying to maximise $a$'s lexicographic payoff $pay_a(\cdot)$ and the second player is adversarial (i.e., trying to minimise $pay_a(\cdot)$). 
Thus, we have 
reduced $NE$-emptiness of $\LEX(\parity,\mp)$ games to the problem of computing the values of these games $H_a(s)$. 

\head{Reduce to solving mean-payoff parity games} 
In the third and final step we show how to reduce the games $H = H_a(s)$ to solving mean-payoff parity games. We consider two games $J$ and $K$, both on the same weighted arena as $H$, but with different objectives: the first player's objective in $J$ is the mean-payoff parity objective, and the first player's objective in $K$ is the mean-payoff objective. Let $j$ be the value of $J$ and $k$ the value of $K$. It is easy to see that the value of $H$ is $(1,j)$ if $j \neq -\infty$ and $(0,k)$ otherwise.
\end{proof}

\begin{theorem}
 E-NASH of $\LEX(\LTL,\mp)$-games is decidable. 
\end{theorem}
\begin{proof}
Check that previous works with: $\NE(G') \subseteq \Phi$ iff there exists $\bar{z}$ where $z_a \in \{p_a(s) : s \in \St\}$ and there exists an execution 
$s_0 d_0 s_1 d_1 \cdots$ in $G'$ \textbf{satisfying $\Phi$} such that each transition $tr(s_i,d_i) = s_{i+1}$ (for $i < \omega$) is $\bar{z}$-secure. 
\end{proof}

