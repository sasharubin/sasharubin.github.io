\documentclass{article}
\usepackage{ijcai17}
\usepackage{times}

%\usepackage{pdfinfo}

% \usepackage{latexsym} 
\pdfinfo{
/Title (Nash Equilibria in Games with Lexicographic Goals)
/Author (TBA)}

\title{Nash Equilibria in Games with Lexicographic Goals}

%\title{Rational Cost Games}
%\title{Nash Equilibria in Games with Combined Qualitative and Quantitative Goals}


\author{XXX}



% % \usepackage{times,mathtools}
% % \usepackage{latexsym,upgreek,wasysym,graphicx,textcomp}
 \usepackage{xspace,color,upgreek}
% % \usepackage[inline]{enumitem}
%% PACKAGES %%
\usepackage{latexsym}
\usepackage{amsmath}
\usepackage{amssymb}
%\usepackage{amsthm} %


\usepackage{color}
\usepackage{graphicx}
\usepackage{tikz,pgf}
  \usetikzlibrary{automata,positioning,matrix,calc,petri,arrows}

%% ENVIRONMENTS %%
%\theoremstyle{plain}
%\newtheorem{theorem}{Theorem}
%\newtheorem{lemma}{Lemma}
%\newtheorem{fact}{Fact}
%\newtheorem{example}{Example}
%\newtheorem{definition}{Definition}
%%\newtheorem{corollary}{Corollary}
%\newtheorem{proposition}{Proposition}
%%\newtheorem*{proof*}{Proof}

%% COMMENTS %%

\newcommand\note[1]{{\color{red}{#1}}}
\newcommand{\todo}[1]{{{\color{blue} #1}}}
%\renewcommand{\todo}[1]{}

%% LATEX SHORTCUTS %%
% cause problems with aamas style file
%\def\it{\begin{itemize} }
%\def\-{\item[-] }
%\def\ti{\end{itemize} }
%\def\en{\begin{enumerate} }
%\def\ne{\end{enumerate} }

%% COMPLEXITY CLASSES %%

\def\UPTime{\textsc{up}\xspace}
\def\CoUPTime{\textsc{coup}\xspace}


\def\exptime{\textsc{exptime}\xspace}
\def\exptimeC{\exptime-complete}

\def\pspace{\textsc{pspace}\xspace}
\def\pspaceC{\pspace-complete}

\def\logspace{\textsc{logspace}\xspace}
\def\nlogspace{\textsc{nlogspace}\xspace}

\def\ptime{\textsc{ptime}}
\def\np{\textsc{np}}



%% LOGIC %%
\def\fol{\mathsf{FOL}}
\def\SL{\textsf{SL}}

\def\msol{\mathsf{MSOL}}
\def\fotc{\mathsf{FOL+TC}}

\def\know{\mathbb{K}}
\def\dknow{\mathbb{D}}
\def\cknow{\mathbb{C}}
\def\eknow{\mathbb{E}}

\def\dualknow{\widetilde{\mathbb{K}}}
\def\dualdknow{\widetilde{\mathbb{D}}}
\def\dualcknow{\widetilde{\mathbb{C}}}
\def\dualeknow{\widetilde{\mathbb{E}}}

\renewcommand\implies{\rightarrow}

%% TEMPORAL LOGIC %%
\newcommand{\sqsq}[1]{\ensuremath{[\negthinspace[#1]\negthinspace]}}

\DeclareMathOperator{\ctlE}{{\mathsf{E}}}
\DeclareMathOperator{\ctlA}{\mathsf{A}}


\newcommand{\atlE}[1][A]{\ensuremath{\langle\!\langle{#1}\rangle\!\rangle}}
\newcommand{\atlA}[1][A]{\ensuremath{[[{#1}]]}}

\DeclareMathOperator{\nextX}{\mathsf{X}}
\DeclareMathOperator{\yesterday}{\mathsf{Y}}
\DeclareMathOperator{\until}{\mathbin{\mathsf{U}}}
\DeclareMathOperator{\weakuntil}{\mathbin{\mathsf{W}}}
\DeclareMathOperator{\since}{\mathbin{\mathsf{S}}}
\DeclareMathOperator{\releases}{\mathbin{\mathsf{R}}}
\DeclareMathOperator{\always}{\mathsf{G}}
\DeclareMathOperator{\hitherto}{\mathsf{H}}
\DeclareMathOperator{\eventually}{\ensuremath{\mathsf{F}}\xspace}
\DeclareMathOperator{\previously}{\mathsf{P}}
\newcommand{\true}{\mathsf{true}}
\newcommand{\false}{\mathsf{false}}


\newcommand{\LTL}{\ensuremath{\mathsf{LTL}}\xspace}
\newcommand{\PLTL}{\textsf{PROMPT-}\LTL}

\newcommand{\CTL}{\ensuremath{\mathsf{CTL}}\xspace}
\newcommand{\CTLS}{\ensuremath{\mathsf{CTL}^*}\xspace}
\newcommand{\PCTLS}{\textsf{PROMPT-}\CTLS}
\newcommand{\PCTL}{\textsf{PROMPT-}\CTL}
\newcommand{\CLTL}{\ensuremath{\textsf{C-}\LTL}\xspace}
\newcommand{\PCLTL}{\ensuremath{\textsf{PROMPT-C}\LTL}\xspace}

\newcommand{\ATL}{\ensuremath{\mathsf{ATL}}\xspace}
\newcommand{\ATLS}{\ensuremath{\mathsf{ATL}^*}\xspace}
\newcommand{\PATLS}{\textsf{PROMPT-}\ATLS}
\newcommand{\PATL}{\textsf{PROMPT-}\ATL}

\newcommand{\KATL}{\ensuremath{\mathsf{KATL}}\xspace}
\newcommand{\KATLS}{\ensuremath{\mathsf{KATL}^*}\xspace}
\newcommand{\PKATLS}{\textsf{PROMPT-}\KATLS}
\newcommand{\PKATL}{\textsf{PROMPT-}\KATL}


\def\red{{red}}
\def\col{{col}}
\def\alt{\, | \,}

%% PROMPT 
\def\kmodels{\models^k}
\def\twokmodels{\models^{2k}}
\DeclareMathOperator{\Fp}{\eventually_\mathsf{P}}
\DeclareMathOperator{\Gp}{\always_\mathsf{P}}
\DeclareMathOperator{\within}{\mathsf{within}}

\newcommand{\AP}{{AP}}
\def\Ag{{Ag}}
\def\Act{{Act}}

%% MATH OPERATIONS %%
\newcommand{\tpl}[1]{\langle {#1} \rangle }
\newcommand{\tup}[1]{\overline{#1}}
\def\proj{\mathsf{proj}}
\newcommand{\defeq}{\ensuremath{\triangleq}}

%% STRUCTURES and STRATEGIES %%
\newcommand{\cgs}{\ensuremath{\mathsf{S}}}
\newcommand{\LTS}{\mathsf{S}}
\newcommand{\Comp}{\mathsf{cmp}}
\newcommand{\Hist}{\mathsf{hist}}
\newcommand{\out}{{out}}

\newcommand{\Paths}{\mathsf{pth}}

\newcommand{\nat}{\mathbb{N}}
\def\int{\mathbb{Z}}
\newcommand{\natzero}{\mathbb{N}_0}

\newcommand{\trans}[3]{#1 \stackrel{\mathsf{#3}}{\rightarrow} #2}


%% HEADINGS ETC %%
\newcommand{\head}[1]{\noindent {\bf #1}.}

%% COUNTER MACHINES %%
\newcommand{\cm}{M}
\newcommand{\CMinc}{\mathsf{inc}}
\newcommand{\CMdec}{\mathsf{dec}}
\newcommand{\CMzero}{\mathsf{ifzero}}
\newcommand{\CMnonzero}{\mathsf{nzero}}
\newcommand{\CMcommit}{\mathsf{end}}


%% CLTL %%
\def\var{{\sf var}}
\def\ovar{{\sf ovar}}
\def\avar{{\sf avar}}
\def\svar{{\sf svar}}
\def\bvar{{\sf bvar}}

\def\MOD{\equiv}

%% PVP %%
\def\PVP{\mathsf{PVP}}





\def\hist{{\sf Hst}} %history
\def\plays{{\sf Ply}} %plays



\def\DCLPC{{\sf{DCL-PC}}}
\def\ATL{{\sf{ATL}}}
\def\ATLS{{\sf{ATLS}\xspace}}
\def\CTLS{{\sf{CTLS}\xspace}}
\def\CLPC{{\sf{CL-PC}}}
\def\LCGS{{\sf LCGS}\xspace}
\def\CGS{{\sf CGS}\xspace}
\def\cgs{\CGS}
\def\CCGS{{\sf CCGS}\xspace}
\def\CCGA{{\sf CCGA}\xspace}

\def\Pth{{\sf Pth}} %set of plays
\def\playf{{\sf play}}
\def\Reach{{\sf Reach}}
\newcommand{\Hst}{{\sf Hst}}
\newcommand{\St}{St}
\newcommand{\Str}{{\sf Str}}
\newcommand{\exec}{{\sf Exec}}

\def\mp{{\sf mp}}
\def\max{{\sf max}}
\def\parity{{\sf parity}}

\newcommand\card[1]{|{#1}|}

\newcommand{\C}{{\mathfrak{C}}}

%LOGICS 
\newcommand{\LATL}{{\sffamily LATL}\xspace}
\newcommand{\LATLS}{{\sffamily LATLS}\xspace}


\def\lex{\prec_{lex}}
\def\lexeq{\preceq_{lex}}
\def\LEX{{\sf Lex}}

\def\NE{{\sf NE}}


\newcommand{\longtrans}[3]{#1 \stackrel{\mathsf{#3}}{\leadsto} #2}
\newcommand{\longtranszero}[3]{#1 \stackrel{\mathsf{#3}}{\leadsto_0} #2}
%\newcommand{\todo}[1]{{{\color{blue} (#1)}}}

\newcommand{\coop}[2][]{\langle\!\langle{#2}\rangle\!\rangle_{_{\!\mathit{#1}}}}

\newcommand{\noavoid}[1]{[\![{#1}]\!]}




\begin{document}




\section*{Decisions}

We agreed i) to have weights on actions and a note/remark about the extension to weights on states or edges; ii) to focus on NE for arbitrary strategies and, if we can prove something useful/interesting, $\epsilon$-NE for finite-memory strategies; iii) we probably have enough material with mean-payoff and can consider max for a journal version; iv) it is probably enough to focus on the following decision problems: NE-emptiness, E-Nash, NE-membership.


\section*{Todos}

\begin{enumerate}
 \item Settle on structure of paper

 \item Fill in the details of the proof sketches already in the draft. Note that bisimilar CGSs may not preserve existence of NE if histories are sequences of states. However, if histories alternate states and actions, then indeed they do. So, we must carefully check the papers we cite/use.

 \item Decide if we want LTL or Lex formulas $\Phi$ for E-Nash. Question: how does one motivate E-Nash with LEX objectives for the government? Potential answer: social welfare. Technically, add an extra component to each tuple of actions in the CGS that belongs to the government.
  
 \item Instead of reducing E-Nash to NE-emptiness it seems that one can simply adapt the proof that NE-emptiness is decidable for Lex Games to proving that E-Nash is decidable for Lex Games. Indeed, instead of looking for a secure path in G', look for a secure path in G' that satisfies the given formula $\Phi$. Also, Giuseppe 
checked that E-Nash reduces to NE-emptiness using the usual matching-pennies gadget. We should decide whether to present NE-emptiness or E-nash ``first''.


 \item Discuss ``motivating examples''/``applications''.  Example scenario: i) one wholesaler and a bunch of retailers, each trying to maximise profit (this needs to be formalised); ii) taxation schemes. Perhaps look at motivating examples from papers on mean-payoff games for ideas.
 
 \item If we need extra: If we need more dec probs, we might look at strong NE; other restrictions on strategy memory, myopic.
 
 \item Absorb definition and techniques for $\epsilon$-Nash equilibria.
\end{enumerate}

\clearpage

\maketitle

\section{Introduction}

\section{Preliminaries}



In this section we define our game model: our games are played on finite graphs (rather than games in extensive-form or normal-form), games are multi-player, agents move concurrently (which includes the special case that they move sequentially), and agents play deterministic (rather than randomised) and memoryfull (rather than memoryless or finite-memory) strategies, and agents are trying to maximise their payoffs which are given as a lexicographic combination of a qualitative \LTL formula and a quantitative long-term average of the rewards of its actions.

We fix some notation. If $X$ is a set, then $X^\omega$ is the set of all infinite sequences over $X$. 
If $X$ and $Y$ are sets then $X^Y$ is the set of all functions $f:Y \to X$.

\subsection{Lexicographic Games}
	
		A \emph{weighted arena} is a
		tuple $A=\tpl{Ag,  Act, Ap, W, St, \iota,tr, \lambda,\kappa}$ where
		\begin{enumerate} \item $Ag$ is a finite non-empty set of \emph{agents} (write 
			$N			= \card{Ag}$);
			\item $Act$ is a finite non-empty set of the \emph{actions};
						\item $Ap$ is a finite non-empty set of \emph{atoms};

			\item $W \subset \mathbb{Z}$ is a finite non empty-set of integer \emph{weights}; 
			\item $St$ is a finite non-empty set of \emph{states} and $\iota$
						is the \emph{initial state};
			
			\item $tr: St \times Act^{Ag} \rightarrow St$ is
						a \emph{transition function} mapping each pair, consisting of a
						state and an action for each agent, to a successor state;
			\item $\lambda:St \to 2^{Ap}$ is the \emph{labeling function}; 
			\item $\kappa: Act \to W$ is a \emph{weight function} mapping each action to an integer.
		\end{enumerate}

		

\head{Executions}
An \emph{execution} $\pi = s_0 d_0 s_1 d_1 \cdots$ is an infinite sequence over $St \times Act^{Ag}$ such that $s_0 = \iota$ and 
$tr(s_i,d_i) = s_{i+1}$ for all $i$.
In particular, $d_i(a)$ is the action of agent $a$ in step $i$. Let $\exec$ denote the set of all
executions. Extend $\lambda$ and $\kappa$ to executions: i.e., each execution $\pi = s_0 d_0 s_1 d_1 \cdots $ induces
\begin{enumerate}

\item a sequence $\lambda(\pi) = \lambda(s_0) \lambda(s_1) \cdots$  of labels; and

\item for each agent $a$, a sequence $\kappa_a(\pi) = \kappa(d_0(a)) \kappa(d_1(a)) \cdots$.
\end{enumerate}



\head{Lexicographic games}
% A function of the form $F:W^\omega \to \mathbb{R} \cup \{-\infty, +\infty\}$ is called an \emph{aggregation function}.
For a sequence $\alpha \in \mathbb{R}^\omega$, let $\mp(\alpha)$ be the \emph{mean-payoff} of $\alpha$, i.e., 
\[
 \mp(\alpha) = \lim \inf_{n \to \infty} \frac{1}{n} \Sigma_{j=0}^{n-1} \alpha_j.
\]



% A \emph{qualitative \LTL objective} is a tuple $\Psi = (\varphi_a)_{a \in Ag}$ of \LTL formulas.
A \emph{$\LEX(\LTL,\mp)$ game} is a tuple $G = \tpl{A,\Psi}$ where $A$ is a weighted arena and $\Psi = (\varphi_a)_{a \in Ag}$ is a tuple of \LTL formulas.
The \emph{payoff for agent $a$ of an execution $\pi$} is the pair $pay_a(\pi) = (x,y) \in \{0,1\} \times \mathbb{R}$ where
$x = 1$ iff $\lambda(\pi) \models \varphi$, and $y = \mp(\kappa_a(\pi))$. We define a total ordering on payoffs:
$(x,y) \lex (x',y')$ iff, either $x < x'$, or $x = x'$ and $y < y'$. Each agent is trying to maximise its payoff. In other words, agent $a$'s primary goal is to make $\varphi_a$ true, and its secondary goal is to maximise it's reward $\mp(\kappa_a(\cdot))$. 

\begin{remark}
We consider weights to be rewards to be maximised. One may, instead, consider them costs to be minimised. All our results hold for such cost-games. Indeed, given a weighted arena $A$ with weights $\kappa$, form a weighted arena $A'$ in which all weights are replaced by their negations. 
Then, an agent, say $a$, that maximises its payoff in $A'$ has the primary goal of making $\varphi_a$ true and the secondary goal of maximising $\mp(-\kappa_a(\cdot))$. 
But maximising $\mp(-\kappa_a(\cdot)) = - \mp(\kappa_a(\cdot))$ is the same as minimising $\mp(\kappa_a(\cdot))$.
\end{remark}
 
\head{Strategies and Nash Equilibria}
A \emph{history} is a finite sequence $s_0 d_0 \cdots s_{n-1} d_{n-1} s_n$ such that $s_0 = \iota$ and $tr(s_i,d_i) = s_{i+1}$ for $i < n$.
A \emph{strategy profile} is a function $\sigma: Ag \to (\hist \to Act)$. A strategy profile $\sigma$ induces a unique execution $\pi_\sigma$, i.e., $\pi_\sigma = s_0 s_1 \cdots$ where $s_0 = \iota$ and $s_{i+1} = tr(s_i,d)$ where $d(a) = \sigma(a)(s_0 s_1 \cdots s_i)$ for $i > 0$.
A strategy profile $\sigma$ is a \emph{Nash-Equilibrium} if for every strategy profile $\sigma'$ that disagrees with $\sigma$ on exactly one component, say $\sigma'(a) \neq \sigma(a)$ and $\sigma'(b) = \sigma(b)$ for all $b \neq a$, we have that $pay_a(\pi_{\sigma'})  \lexeq  pay_a(\pi_\sigma)$.

% % \head{Concrete aggregation functions}
% % We will consider various aggregation functions.
% % For $\alpha \in W^\omega$, 
% % % let $occ(\alpha)$ be the values occuring in $\alpha$, and let $inf(\alpha)$ be the values occuring infinitely often in $\alpha$.
% % 
% % \begin{enumerate}
% %  \item $\mp(\alpha) \in \mathbb{R}$ is the mean-payoff of $\alpha$, i.e., the lim-inf of the average of the first $n$ terms of $\alpha$; 
% %  \item $\max(\alpha) \in \mathbb{Z}$ is the maximum value of the weights in $\alpha$.   
% % \end{enumerate}

\head{Turn-based Arenas} Arenas in which every state is ``controlled'' by a single agent are called turn-based (sometimes called sequential). We define such arenas as a special case of concurrent arenas. Formally, 
a weighted weighted arena $A$ is called \emph{turn-based} if for every $s \in St$ there exists an agent $a \in Ag$ such that
$tr(s,d) = tr(s,d')$ for all $d,d' \in Act^Ag$ such that $d(b) = d'(b)$ for all $b \neq a$. A game is called \emph{turn-based} if its arena is turn-based.


\subsection{Decision Problems}

The \emph{NE-language $\NE(G)$} of a game $G$ is the set of strategy-profiles that are Nash-Equilibria. We consider the following decision problems:

\begin{enumerate}
 \item \emph{NE-membership} is the problem of deciding, given $G$ and a finite-state profile $\sigma$, if $\sigma \in \NE(G)$.
 \item \emph{NE-emptiness} is the problem of deciding, given $G$, if $\NE(G)$ is empty.
 \item \emph{E-Nash} is the problem of deciding, given $G$ and an \LTL formula $\varphi$, if there exists $\sigma \in \NE(G)$ such that $\lambda(\sigma) \models \varphi$.
\end{enumerate}

\subsection{Justification of our Model}
In this section we justify lexicographic games and our focus on pure memoryfull strategies.

First, $\LEX(\LTL,\mp)$-games generalise a number of important models, i.e., \LTL-games (let all weights be $0$) and $\mp$-games 
(let all formulas be $\true$). Note that \LTL-games generalise safety and reachability games. \todo{mention more papers, e.g., taxation model}

Second, we focus on pure strategies instead of mixed strategies. This is because $\LEX(\LTL,\mp)$-games need not have mixed NE (this is true even for two-player zero-sum games with reachability objectives~\cite{DBLP:journals/corr/abs-1109-6220}), and it is undecidable whether or not a given game has a mixed NE (this is true even for mean-payoff games~\cite{DBLP:journals/corr/abs-1109-6220}).

Third, we focus on decision problems for memoryfull strategies. This is because pure NE need not exist, e.g., matching pennies is a reachability game with no pure NE. Moreover, even if a pure NE exists, it need not have finite-memory NE. This can be seen by considering the following one-player game with \LTL objective $\always \eventually p$: $Act = \{l,r\}$, $Ap = \{p\}$, $W = \{0,10\}$, $St = \{\iota,v_1,v_2\}$, $tr(\iota,l) = v_1$, $tr(\iota,r) = v_2$, $tr(v_1,x) = tr(v_2,x) = \iota$ for $x \in Act$, $\lambda(v_1) = p$, $\kappa(\iota) = \kappa(v_1) = 10$ and $\kappa(v_2) = 0$. It is shown in \cite[Example $1$]{DBLP:journals/corr/abs-1210-3539} that there is a pure strategy that achieves payoff $(1,10)$, but every finite-state strategy achieves less. 

Finally, we focus on decision problems for concurrent games rather than turn-based games because, as we now show, turn-based games always have NE.

\begin{theorem} \label{thm:TB}
 Turn-based $\LEX(\LTL,\mp)$-games always have NE.
\end{theorem}
\begin{proof}[Sketch]
The proof is in three steps: first, push the weights to the states and replace the \LTL formulas by parity conditions on the states, to get a 
$\LEX(\parity,\mp)$ game with the same set of NE. Second, 
prove that the following games are determined, i.e., they have a value: two-player zero-sum games with a $\LEX(\parity,\mp)$ objective 
(i.e., one player is trying to maximise $pay(\pi)$ while the other is trying to minimise this same payoff). Third, use the existence of optimal strategies (from the second item) to prove the existence of a NE in multiplayer $\LEX(\parity,\mp)$ games. This last step uses the fact that $\LEX(\parity,\mp)$ is prefix-independent, i.e., if player $a$ prefers execution $\pi'$ to $\pi$ then, for all histories $h$, player $a$ also prefers $h \cdot \pi'$ to $h \cdot \pi$ (assuming these are also executions). 

Here are the details. First, push the weights to the states. This is done by replacing $St$ by $St \times Act^{Ag}$ and defining the weight of $(s,d)$ to be $\kappa(d)$. Now translate each \LTL formula $\varphi_i$ into a deterministic parity word automaton  $D_i$. Form the product weighted arena $A' = A \times \prod_i D_i$ where each state is labeled by both the weight from $A$ and a tuple of priorities $\tup{p}$ from the $D_i$s. The payoff for player $i$ is the pair $(x,y) \in \{0,1\} \times \mathbb{R}$ where $x = 1$ iff the smallest priority in co-ordinate $i$ that occurs infinitely often is even, and $y$ is the mean-payoff of the weights. The resulting $\LEX(\parity,\mp)$ game $A$ has the same NE as the original $\LEX(\LTL,\mp)$ game.


Second, let $A_i$ be the two-player game, based on the weighted arena $A$, in which the first player is trying to maximise player $i$'s $\LEX(\parity,\mp)$ objective, and the adversary is trying to minimise this same objective (thus each state is labeled by a single weight and a single priority). We are required to show that such games are determined. To this end, consider the zero-sum game played on weighted arena $A$ in which player $i$ is trying to achieve his parity objective \emph{and} maximise his mean-payoff (and the opponent is adversarial). These are called mean-payoff parity games, and are known to be determined~\cite{DBLP:conf/lics/ChatterjeeHJ05}. Let $v_i \in \mathbb{R} \cup \{-\infty\}$ be the value of this game. Let $w_i$ be the value of the zero-sum game played on weighted arena $A$ in which player $i$ is trying to maximise his mean-payoff (i.e., he ignores his parity condition). The value of $A_i$ is equal to $v_i$ if $v_i \neq -\infty$, and equal to $w_i$ if $v_i = -\infty$.

Third, the promised reduction works as follows: player $i$ plays optimally in $A_i$ (thus ensuring at least $v_i$), but the moment a player, say $j$, deviates from this strategy all the other players punish $j$ by ensuring that player $j$ receives no more than $v_i$. To define this profile formally, let 
$\sigma_i$ be an optimal strategy for player $i$ in $A_i$ (note that it is also a strategy in $A$), 
let $\sigma^{-i}$ be an optimal strategy for the opponent in $A_i$, and let 
$val_i(v)$ be the value of the game $A_i$ starting in vertex $v$. These exist by determinacy. 
Let $\rho_j^{-i}$ be the strategy for player $j$ in $A$ derived from $\sigma^{-i}$. Let $\pi$ be the path induced by the strategy profile $(\sigma_1, \cdots, \sigma_N)$. Since $\sigma_j$ is optimal, we get that $v_j \leq pay_j(\pi)$.  

Define a strategy profile $\bar{\xi}$ where $\xi_i$ follows $\sigma_i$ as long as the play stays on $\pi$; the moment the play deviates, say in state $v$ on player $j$'s move, the strategy $\xi_i$ switches to $\rho_j^{-i}$ forever. To see that $\bar{\xi}$ is a NE reason as follows: let $\tau_j$ be any strategy for agent $j$ different from $\xi_j$ and consider the play $\pi'$ induced by $(\xi_1, \cdots, \xi_{j-1}, \tau_j, \xi_{j+1}, \cdots, \xi_N)$ starting from initial state $\iota$. Let $v$ be the vertex of player $j$ where $\tau_j$ first deviates from $\xi_j$. By construction of $\bar{\xi}$, we have that $pay_j(\pi'') \leq val_j(v)$ where $\pi''$ is the suffix of $\pi'$ starting with the deviation. Since the objectives are prefix-independent, we get that
$pay_j(\pi') = pay_j(\pi'')$. Since player $j$ played optimally until $v$, we get that $val_j(v) = val_j(\iota)$. Thus, $pay_j(\pi') \leq pay_j(\pi)$, as required.


The only thing left to check is that $\LEX(\parity,\mp)$ is prefix-independent. This follows from the definition of the lexicographic order and the fact that both the parity condition and the mean-payoff condition are prefix-independent, i.e., if the smallest priority occuring infinitely often in $\pi$ (resp. $\pi'$) is even (resp. odd), then for every $h$, the smallest priority occuring infinitely often in $h \cdot \pi$ (resp. $h \cdot \pi'$) is even (resp. odd); and if $\mp(\kappa(\pi)) < \mp(\kappa(\pi'))$ then, for every $h$, $\mp(\kappa(h \cdot \pi)) < \mp(\kappa(h \cdot \pi'))$. 
\end{proof}



\section{Decision Procedures} \label{sec:decision procedures}



\begin{theorem}
 NE-emptiness of $\LEX(\LTL,\mp)$-games is decidable. \todo{Todos: 1) debug, 2) polish, 3) calculate complexity of the algorithm}
\end{theorem}

\begin{proof}[Sketch]
We prove this in three steps. In the first step we push the weights of $G$ into the states and translate the result into a $\LEX(\parity,\mp)$ game $G'$ such that $\NE(G) = \NE(G')$. Here $\parity$ is an aggregation function that maps a sequence of priorities (integer weights) to $0$ if the smallest priority occuring infinitely often is even, and to $1$ otherwise (the priorities are given as a second weight function, in addition to $\kappa$). In the second step we show how to reduce $NE$-emptiness of $G'$ 
to the problem of solving two-player zero-sum games $H$ in which player $0$ has a $\LEX(\parity,\mp)$ objective. We do this by adapting the proof that shows how to decide $NE$-emptiness of mean-payoff games~\cite{DBLP:conf/concur/UmmelsW11}.  In the third step we reduce $H$ to solving mean-payoff parity games. These are two-player zero-sum games in which the objective of player $0$ is to ensure both the parity condition holds and that the mean-payoff is maximised~\cite{DBLP:conf/lics/ChatterjeeHJ05}. Formally: the payoff is $-\infty$ if the smallest priority occuring infinitely often is even, otherwise it is the mean-payoff of the weights.

% $\parity(\alpha) \in \{0,1\}$ is $0$ iff the $min(inf(\alpha))$ is even, where $inf(\alpha)$ consists of the weights occuring infinitely often in $\alpha$.

\head{Replace \LTL by Parity}
For the first step, push the weights to the states. This is done by replacing $St$ by $St \times Act^{Ag}$ and defining the weight of $(s,d)$ to be $\kappa(d)$. Convert each \LTL formula $\varphi_a$ into a deterministic parity automaton $D_a$ of size double-exponential in $\varphi_a$~\cite{??}. To build the game $G'$ form the product weighted arena $\St' = \St \times \prod_a D_a$. Each agent has a pair of weights associated with each state, i.e., a priority (coming from $D_a$) and the integer weight (coming from $\kappa$).  Given a play, the payoff for an agent is $(x,y)$ where $x = 0$ iff the smallest priority occuring infinitely often on the play is even, and $y$ is the meanpayoff of the integer weights. By construction we have that $\NE(G) = \NE(G')$.

\head{Reduce \NE\ to path-finding}
For the second step, we adapt the proof in Section~6 of \cite{DBLP:conf/concur/UmmelsW11} that shows how to decide $NE$-emptiness for mean-payoff games. For $a \in Ag$ and $s \in \St$ define the \emph{punishing value} $p_a(s)$ to be the $\lex$-largest $(x,y)$ that player $a$ can achieve from state $s$ by ``going it alone'', i.e., by playing against the coalition $Ag \setminus \{a\}$.  We will show how to compute these values in the third step. 

Definition: For an agent $a$ and $\bar{z} \in \mathbb{R}^{|Ag|}$, a pair $(s,d) \in \St \times Act^{Ag}$ is \emph{$\bar{z}$-secure for $a$} if $p_a(tr(s,d')) \leq z_a$ for every $d' \in Act^{Ag}$ that agrees with $d$ except possibly at $a$. 

Claim: $\NE(G')$ is non-empty iff there exists $\bar{z}$ where $z_a \in \{p_a(s) : s \in \St\}$ and there exists an execution $\pi = s_0 d_0 s_1 d_1 \cdots$ in $G'$ such that for every agent $a$,  i) $z_a \leq pay_a(\pi)$ and ii) for all $i \in \mathbb{N}$, the pair $(s_i,d_i)$ is $\bar{z}$-secure for $a$. 


Sketch proof of Claim: Suppose $\NE(G')$ is non-empty. Let $\pi$ be the execution resulting from some Nash-profile. 
Let $z_a = \max\{p_a(\delta(s_n,d'_n)) : n \in \mathbb{N}, \wedge_{b \neq a} d'_n(b) = d_n(b)\}$, i.e., $z_a$ is the largest value player $a$ can get by deviating from $\pi$. Clearly $(s_n,d_n)$ is $\tup{z}$-secure for $a$. Moreover, $z_a \leq pay_a(\pi)$: indeed, if $z_a = p_a(\delta(s_n,d'_n)) > pay_a(\pi)$ then player $a$ would deviate at step $n$ by playing $d'_n$ and achieve a higher payoff, contradicting the choice of $\pi$ as the execution of a Nash-profile.

For the converse direction, let $\bar{z}$ and $\pi$ be given with the stated properties. The following is a Nash-profile: every agent plays ``follow $\pi$'' until, and if, a single player, say $a$, deviates, say at state $s$; in that case the other agents play a punishing strategy, i.e., a strategy that ensures player $a$ achieves no more than he possibly can given the current history. It can be shown that this value is $p_a(s)$ for $LEX(\parity,\mp)$ payoff functions (since such payoff functions are prefix-independent). This completes the proof of the claim.





Note that $p_a(s)$ is the value of the two-player zero-sum game $H_a(s)$ in which the first player is trying to maximise $a$'s lexicographic payoff $pay_a(\cdot)$ and the second player is adversarial (i.e., trying to minimise $pay_a(\cdot)$). 
Thus, we have 
reduced $NE$-emptiness of $\LEX(\parity,\mp)$ games to the problem of computing the values of these games $H_a(s)$. 

\head{Reduce to solving mean-payoff parity games} 
In the third and final step we show how to reduce the games $H = H_a(s)$ to solving mean-payoff parity games. We consider two games $J$ and $K$, both on the same weighted arena as $H$, but with different objectives: the first player's objective in $J$ is the mean-payoff parity objective, and the first player's objective in $K$ is the mean-payoff objective. Let $j$ be the value of $J$ and $k$ the value of $K$. It is easy to see that the value of $H$ is $(1,j)$ if $j \neq -\infty$ and $(0,k)$ otherwise.
\end{proof}

\begin{theorem}
 E-NASH of $\LEX(\LTL,\mp)$-games is decidable. 
\end{theorem}
\begin{proof}
Check that previous works with: $\NE(G') \subseteq \Phi$ iff there exists $\bar{z}$ where $z_a \in \{p_a(s) : s \in \St\}$ and there exists an execution 
$s_0 d_0 s_1 d_1 \cdots$ in $G'$ \textbf{satisfying $\Phi$} such that each transition $tr(s_i,d_i) = s_{i+1}$ (for $i < \omega$) is $\bar{z}$-secure. 
\end{proof}



\section{Related Work}

NE-emptiness is the main decision problem studied in this paper, i.e., decide if a given multiplayer concurrent game has a Nash Equilibrium or not. 
We studied this problem for $\LEX(\LTL,\mp)$-games, i.e., graph-games in which agents are trying to maximise their lexicographic payoff whose first co-ordinate is an \LTL formula and whose second co-ordinate is the mean-payoff of its weights. We proved the problem is $2$\exptime-complete.

\head{Other objectives}
NE-emptiness has been studied for graph-games with other objectives, a sample of which are given in Table~\ref{tab:NE-emptiness}.
\begin{table}\label{tab:NE-emptiness}
\begin{center} 
\begin{tabular}{llll}
Objectives			& Complexity 			& Reference\\
\hline
$\LEX(\LTL,\mp)$		& $2$\exptime-complete		& This paper\\
\LTL 				& $2$\exptime-complete		& \cite{DBLP:journals/tocl/MogaveroMPV14}\\
mean-payoff (\mp)			& \np-complete			& \cite{DBLP:journals/corr/abs-1109-6220}\\
reachability			& \np-complete			& \cite{DBLP:journals/corr/BouyerBMU15}\\
lexicographic reachability	& \pspace-complete		& \cite{DBLP:journals/corr/BouyerBMU15}\\
B\"uchi				& \ptime-complete		& \cite{DBLP:journals/corr/BouyerBMU15}\\
lexicographic B\"uchi		& \ptime-hard, in \np		& \cite{DBLP:journals/corr/BouyerBMU15}\\
% counting ???		& ???				& \cite{DBLP:journals/corr/BouyerBMU15}
% parity			& ${\sf P}^\np_{||}$-complete	& \cite{DBLP:journals/corr/BouyerBMU15}\\
\end{tabular}
\caption{Complexity of NE-emptiness for multiplayer concurrent-games with objectives from various classes.}
\end{center}
\end{table}
$\LEX(\LTL,\mp)$ generalises both mean-payoff games (simply set every formula to be $\true$) and \LTL games (simply set all the weights of the arena to be $0$). That NE-emptiness is decidable in $2$\exptime for \LTL games follows from a more general result, i.e., that model checking \emph{strategy logic} on concurrent game structures (i.e., arenas without the weights) is decidable \cite{DBLP:journals/tocl/MogaveroMPV14}. To get the $2$\exptime upper bound one writes the existence of a NE in a fragment of strategy logic (i.e., nested-goal with alternation number $1$) whose model checking problem is decidable in $2$\exptime. We cannot apply strategy logic to $\LEX(\LTL,\mp)$ games since the latter have a quantitative component while strategy logic, being built on top of \LTL, can only express qualitative objectives. 
Instead, as discussed in Section~\ref{sec:decision procedures}, we generalised the proof in~\cite{DBLP:journals/corr/abs-1109-6220} of the fact that NE-emptiness is in \np\ for mean-payoff games. 

We remark that the hardness results for lexicographic reachability and lexicographic B\"uchi are for lexicographic orders on an arbitrary (but finite) number of co-ordinates.

\head{Turn-based}
Very simple concurrent games, such as matching pennies, do not have NE. However, there are many classes of turn-based graph-games that always have NE. For instance, turn-based mean-payoff games~\cite{Alpe91}, turn-based games with $\omega$-regular objectives (which include \LTL objectives)~\cite{CJM04}, turn-based quantitative-reachability games~\cite{Brihaye2013}. Actually, this phenonemon has a general explanation: under certain conditions the existence of (finite-memory) NE for a class of objectives follows from the (finite-memory) determinacy of two-player zero-sum games with those objectives~\cite{DBLP:journals/corr/0001P16,Brihaye2013}. Although the latter citations do not cover the case of turn-based $\LEX(\LTL,\mp)$-games, we use a similar strategy in Theorem~\ref{thm:TB} to prove that these too always have NE. Determinacy (and the complexity of determining the winner) of two-player zero-sum games has been studied for a variety of objectives, 
i.e., mean-payoff~\cite{EM79,ZwPa95}, $\omega$-regular (which include $\LTL$-objectives)~\cite{DBLP:conf/dagstuhl/2001automata}.
%In addition, there are various uniform explanations for when such zero-sum games have positional or finite-memory winning strategies~\cite{Demri?,AmRu14,halfpositional}.

\head{Randomised Strategies}
We have considered deterministic strategies. Randomised strategies are functions that map a history to a probability distribution over actions. 
NE-emptiness is undecidable for randomised strategies on games with reachability objectives for each player~\cite{DBLP:conf/fsttcs/BouyerMS14}. Since reachability objectives are a special case of $\LEX(\LTL,\mp)$ objectives, we have undecidability of NE-emptiness for randomised strategies on $\LEX(\LTL,\mp)$ games.

\todo{IMPORTANT RELATED WORK THAT NEEDS TO BE SUMMARISED: cite{DBLP:journals/corr/abs-1303-0789}; as well as work on secure-Equilibria}

% \head{Other representations of games}
% Nash's famous result states that every finite game in strategic form has a randomised NE~\cite{??}. Computing such a NE is complete for the complexity class PPAD \cite{DBLP:journals/siamcomp/DaskalakisGP09}.\todo{check}
% 
% \todo{Stochastic games...}



 
% \begin{theorem} \cite{DBLP:journals/corr/abs-1109-6220}
%  The following problem is NP-complete: given $\bar{x} \leq \bar{z} \in \mathbb{Q}^{|Ag|}$ and a mean-payoff game, decide if there is a
%  NE, say with payoff $\bar{y}$, that satisfies $\bar{x} \leq \bar{y} \leq \bar{z}$.
%  \end{theorem}



% 
% \head{``Pure Nash Equilibria in Concurrent Deterministic Games'', Bouyer et al. \cite{DBLP:journals/corr/BouyerBMU15}}
% 
% The paper then gives the exact complexity of NE-emptiness for many $\omega$-regular 
% objectives, and semi-quantitative versions of these (like counting or 
% lexicographic). E.g., 
% \begin{theorem}[\cite{DBLP:journals/corr/BouyerBMU15}]
% The complexity of NE-emptiness  for:
% \begin{enumerate}
%  \item reachability- and safety-games are $\np$-complete, 
%  \item Buchi-games are $\ptime$-complete,
%  \item parity-games is ${\sf P}^\np_{||}$-complete,
%  \item Muller-games is $\pspace$-complete,
%  \item lexicographic Buchi-games are in $\np$.
% \end{enumerate}
% \end{theorem}

% \head{  Synthesis from LTL Specifications with Mean-Payoff Objectives`` \cite{DBLP:journals/corr/abs-1210-3539}}
% Two player, synthesis of specifications written in LTL with mean-payoff weights. 
% Objective is ''formula must be true and weights must be maximised``. Reduced to parity games.
% 
% \head{ A four-person chess-like game without Nash equilibria in pure stationary strategies}
% https://arxiv.org/abs/1411.0349
% 
% This paper shows that NE need not be memoryless for games in which the payoff depends on the
% first cycle formed. The example in their paper can be expressed in terms of meanpayoff.
% 
% 
% 
% \head{``Extending finite memory determinacy: General techniques and an application
%                to energy parity games'' \cite{DBLP:journals/corr/0001P16}}
%              
%              
%            Shows how to extend determinacy of win/lose games to finite-state NE of multiplayer games.
%            Turn based.
%            
%            
%       \section{OLD --- The model}


	
		\begin{definition}[CCGS]
		A \emph{$d$-dimensional Cost Concurrent Game Arena} (\CCGA) is a
		tuple $A=\tpl{\Ag, \Stt, \Act,\iota,\kappa,d}$
		\begin{enumerate} \item $\Ag$ is a finite non-empty set of \emph{agents} (we write 
			$\NElm			= \card{\Ag}$);
			\item $\Act$ is a set of the actions;
			\item $\Stt$ is a finite non-empty set of \emph{states} and $\iota$
						is the \emph{initial state};
			\item $\trn: \Stt \times \Act^{\Ag} \rightarrow \Stt$ is
						a \emph{transition function} mapping each pair consisting of a
						state and an action for each agent to a state,
			\item $d \in \nat$ is the {\em dimension} and $\kappa: \Ag \to (\Stt \times \Act^{\Ag} \rightarrow \mathbb{Z}^d)$ is a \emph{cost function}.
		\end{enumerate}
		\end{definition}


An \emph{execution} $\pi$ is an infinite sequence over $\Stt \times \Act^{\Ag}$ that
respects the transition function.  Let $\exec$ denote the set of all
executions. 

A {\em payoff function} is a function $\mu: \Ag \to (\exec \to \mathbb{R})$, i.e., it
assigns a payoff/utility to each agent.  The payoff may depend on the costs and
the states along the execution. Each agent is trying to maximise its payoff.
A \emph{$d$-dimensional Cost Concurrent Game Structure} (\CCGS) is a tuple $G = (A,\mu)$ where $A$ is a \CCGA and $\mu$ is a payoff function.
In case the payoffs are always in the set $\{0,1\}$ such games are called ``win-lose''.

A \emph{strategy profile} is a strategy for each agent. We write these as $\sigma: \Ag \to (\hist \to \Act)$ where $\hist$ is the set of finite sequences of states.
\todo{Note: we do not include actions in histories... this seems a reasonable choice for discussing NE in distributed systems}
A strategy profile $\sigma$ induces a unique execution $\pi_\sigma$.

\todo{todo: define turn-based game as a special case}

\subsection{Payoff functions}

In order to define concrete payoff functions we typically aggreggate the costs incurred for each player. 
Each execution $\pi$ induces, for each agent $i$, a sequence of costs
$c_i(\pi) \in (\mathbb{Z}^d)^\omega$. Write $c_i^j(\pi) \in \mathbb{Z}^\omega$ for the sequence of costs
for player $i$ in dimension $j \leq d$.




In what follows, the name of the classes of games (mean-payoff, parity, LTL) should really be prefixed by 
``multi-player'' so as not to cause confusion with the two-player zero-sum case.
For $\alpha \in \mathbb{Z}^\omega$, define $mp(\alpha)$ to be the long-term average of the sequence $\alpha$, i.e.,
the lim-inf of the the average of the first $n$ terms of $\alpha$.


\begin{example}[Mean-payoff games]
Let $d = 1$, and let $\mu(i)(\pi) = mp(c_i(\pi))$ be the long term average of
$c_i(\pi)$. Call these {\em mean-payoff games}.
\end{example}

  \begin{example}[$F$-mean payoff games]
  Fix a function $F:\mathbb{R}^d \to \mathbb{R}$ and define $\mu(i)(\pi) = F(mp(c_i^1(\pi)), \cdots, mp(c_i^d(\pi)))$,
  i.e., the payoff for player $i$ is the value of $F$ on the long-term averages of its costs. 
  Call such games {\em $F$-mean payoff games}. Here are some illustrative examples of functions $F$:
\begin{enumerate}
\item $F$ is a threshold function. For instance, if $d = 1$ and $F(x) = 1$ if $x > 0$ and $0$ otherwise.
Let us call these \emph{Threshold Mean-Payoff games}.
\item $F$ is the average of its arguments, 
\item $F$ is the max of its arguments,
\item $F$ counts the number of arguments that are positive,
\item $F$ is the lexicographic. For instance, if $d = 2$, define $F(x,y) = 0$ if $x \leq 0$, $1$ if $x \geq 0$, and $2$ if $x \geq 0$ and $y \geq 0$.
Call these \emph{lexicographic threshold mean-payoff games}. 
\end{enumerate}

\end{example}

We give some more examples with $d = 1$.


\begin{example}[Parity-games]
Let $d = 1$, and let $\mu(i)(\pi)$ equal $1$ if the largest cost of player $i$ seen infinitely often on $\pi$ is even, and equal to $0$ otherwise. 
Call such games {\em parity-games}. 
\end{example}

\begin{example}[Generalised Muller games]
 Let $d = 1$, and let $\mu(i)(\pi)$ depend only on the set of states that occur infinitely often in $\pi$. Call such games {\em generalised Muller-games} after that fact
 that a Muller-game is such a game in which the payoffs are from $\{0,1\}$.
\end{example}

\begin{example}[LTL-games and $\omega$-regular games]
A \emph{LTL-game} is one in which each player has an LTL objective that it is trying to satisfy. 
Formally, replace the cost function in CGCS's by state-labelings $\lambda:\Stt \to 2^{\Ap}$ over some fixed set of atoms $\Ap$.
Given $N$ LTL formulas $\phi_i, i \leq N$, the payoff for player $i$ is $1$ if $\pi \models \phi_i$ and $0$ otherwise.

More generally, if $\phi_i$ is an $\omega$-regular language (i.e., $\phi_i \subseteq (2^{\Ap})^*$) then we have \emph{$\omega$-regular games}.
\end{example}

\begin{example}[Oxford Games]
Let $d = 1$ and suppose every weight is positive. Let $\phi_i$ be \LTL formulas, one for each agent. 
An \emph{Oxford} game (because we discussed some variation of this in Oxford)
has payoff that maps a play $\pi$ to $0$ if $\pi \models \neg \phi_i$ and to $mp(c_i(\pi))$ otherwise. The goal is to maximise the payoff.
\end{example}

\note{Can we capture zero-sum mean-payoff games in which one player maximises and the other minimises?}

\section{The decision problems}

Let $\C$ be a set of games. We define two decision problems.
\begin{definition}
 \emph{E-NASH($\C$)}: Decide if a given game from $\C$ has a Nash-Equilibrium.
\end{definition}

\begin{definition}
 Memoryless-E-NASH($\C$): Decide if a given game from $\C$ has a Nash-Equilibrium made of memoryless strategies.
\end{definition}


We now discuss the relationship between some of the games.

\begin{definition}[Equivalent games]
Call two games $G,G'$ \emph{(memoryless-)equivalent} 
if for every strategy profile $\sigma$ (of memoryless strategies) in $G$ there exists a strategy profile 
$\sigma'$ (of memoryless strategies) in $G'$, and vice versa, such that $\mu(i)(\pi_\sigma) = \mu'(i)(\pi_{\sigma'})$ for each agent $i$. Here $\pi_\sigma$ and $\pi_{\sigma'}$
are the plays induced by the strategy profiles $\sigma$ and $\sigma'$ respectively.
\end{definition}

It is useful to have effective transformations of games from one class into games of another class. Indeed, if $\C,\C'$ are two classes of games for which 
there is an effective transformation of games in $\C$ into equivalent games in $\C'$ then E-NASH($\C$) can be reduced to E-NASH($\C'$). A similar statement holds for the memoryless case.

For instance, it is immediate from the definitions that every parity-game is (trivially) equivalent to a Muller game (i.e., itself), 
and every Muller game is (trivially) equivalent to an $\omega$-regular game (i.e., itself). 

Here are some more interesting equivalences. 

\begin{theorem}\label{thm:LTL to parity}
Every LTL-game can be effectively translated into an equivalent parity-game with $2$-exponential blowup. \todo{This should also work for objective-LTL games.}
\end{theorem}

\begin{proof}[Sketch]
Let $G$ be an LTL-game.
The idea (which is standard for two-player zero-sum games) 
is to convert the LTL formulas into deterministic parity-word automata (DPW), and then form the product of the original game with these automata. 
Being deterministic, the automata
simply annotate the game $G$. 

Here are some details. Translate each LTL formula $\phi_i$ into a deterministic parity-word automaton $A_i$ in which priorities are on edges 
(instead of, as usual, states). Say $A_i$ has state set $Q_i$. 
Then form the parity-game $G'$ as the product of $G$ and $\prod Q_i$. The cost for player $i$ of transition of $G'$ is defined to be the priority of the corresponding 
transition in $A_i$. Then $G$ and $G'$ are equivalent.
\end{proof}


\begin{theorem} \label{thm:PG-MP}
Every parity-game can be effectively translated into a memoryless-equivalent threshold mean-payoff game with no blowup.
\end{theorem}

\begin{proof}[Sketch]
Let $G$ be a parity-game. The idea (cf. e.g., \cite{Jurd98}) is to replace each priority $p$ by the cost $(-m)^p$ where $m$ is 
the number of states of $G$. Then, if $\pi$ is the play resulting from a memoryless-strategy profile in $G$ then the highest priority for player $i$ on $\pi$ is even 
if and only if $mp(c_i(\pi)) > 0$.
\end{proof}

%\todo{Similarly, the lexicographic payoff is a special case of $F$-mean payoff games.}

\begin{corollary}
 For every LTL-game there is a memoryless-equivalent threshold mean-payoff game with $2$-exponential blowup.
\end{corollary}

\todo{Question. Is this same true if we drop ``memoryless''?}

\begin{conjecture} \label{conj:oxford-mp}
Oxford games are memoryless-equivalent to lexicographic threshold mean-payoff games with $2$-exponential blowup.
\end{conjecture}
\begin{proof}[Sketch]
 Let $G$ be an Oxford game. Replace the LTL formulas by DPW, as in Theorem~\ref{thm:LTL to parity}. Then replace DPWs by mean-payoff condition, as in Theorem~\ref{thm:PG-MP}.
\end{proof}


%The following property is useful: a payoff function is \emph{prefix-independent} if $\mu(i)(\pi) = \mu(i)(\pi')$ for every suffix $\pi'$ of $\pi$.
%All the payoff functions above are prefix-independent, except some LTL and $\omega$-regular games.


\section{Related Work and Conjectures}

We now state some known results and sketch their proofs. For the sake of the draft, we also
conjecture some results and sketch proof-ideas.

\subsection{Concurrent Setting}
The authors in \cite{DBLP:journals/corr/BouyerBMU15} give a general reduction from multiplayer concurrent games with preference relations to two-player zero-sum games such that the former has a NE
iff the later has a winning strategy. In their words:

\begin{verbatim}
We propose a novel transformation of the multi-player
concurrent game (with a preference relation for each player) 
into a two-player zero-sum turn-based game, which we call 
the suspect game. Intuitively, in the suspect game, one of 
the players suggests a global move (one action per player 
of the original game), with the aim to progressively build 
a Nash equilibrium; while the second player aims at proving 
that what the first player proposes is not a Nash equilibrium.
This transformation can be applied to arbitrary concurrent games 
(even those with infinitely many states) and preference relations
for the players, and it has the property that there is a 
correspondence between Nash equilibria in the original game and 
winning strategies in the transformed two-player turn-based 
game. The winning condition in the suspect game of course depends 
on the preference relations of the various players in the 
original game. 
\end{verbatim}

The paper then gives the exact complexity of ENASH for many $\omega$-regular 
objectives, and semi-quantitative versions of these (like counting or 
lexicographic). E.g., 
\begin{theorem}[\cite{DBLP:journals/corr/BouyerBMU15}]
The complexity of ENASH for:
\begin{enumerate}
 \item reachability- and safety-games are $\np$-complete, 
 \item Buchi-games are $\ptime$-complete,
 \item parity-games is ${\sf P}^\np_{||}$-complete,
 \item Muller-games is $\pspace$-complete,
 \item lexicographic Buchi-games are in $\np$.
\end{enumerate}
\end{theorem}



\begin{conjecture} \label{conj:ENASH-mpgs}
E-NASH for threshold mean-payoff games is decidable in \expspace.
Memoryless E-NASH for threshold mean-payoff games is decidable in \exptime. The same statements hold for lexicographic mean-payoff games.
\end{conjecture}
\begin{proof}[Idea]
% Perhaps the suspect game can be applied, or GP's idea of reducing E-NASH to synthesis and satisfiability can be applied.
 We first sketch the proof for threshold mean-payoff games (i.e., not lexicographic).
 Given a game $G$, let $\phi_a$ denote the condition that the mean-payoff of the costs of a play for agent $a$ is greater than $0$. 
 Guess the set of losers $L \subseteq \Ag$ and consider the following turn-based two-player zero-sum game $G_L$ played on the states of $G$
 between $E$ and $A$, who alternate turns, with $E$ going first. $E$ plays tuples of actions $m$, one for each player,  
 and $A$ plays states.  There are $|\Ag|+1$ modes the game can be in: neutral or $a$-deviating for $a \in \Ag$. Before any of his moves, $A$ can decide to switch modes from neutral to $a$-deviating for some $a \in \Ag$, and once in $a$-deviating mode, the play stays in $a$-deviating mode.  Play starts from the initial state, in neutral mode. The modes are used as follows. Suppose the current state is $q$ and $E$ has chosen the tuple $m$. 
In the neutral mode $A$ plays $q' = \delta(q,m)$. In $a$-deviating mode $A$ may play $q' = \delta(q,m')$ where $m'$ agrees with $m$ except possibly on $a$'s action. A play $\pi$ is won by $E$ iff 
 $\pi \models \bigwedge_{a \in L} \neg \phi_a \wedge \bigwedge_{a \in \Ag \setminus L} \phi_a$. Then: $E$ has a (finite-memory) winning strategy in $G_L$ iff $G$ has a (finite-memory)  Nash Equilibrium whose losers are $L$.
 
 Thus, we have reduced the (memoryless) ENASH problem to solving (in memoryless strategies) exponentially-many polynomially-sized two-player zero-sum games $H$ with objective that is a boolean combination of mean-payoff thresholds. Each game $H$ is solvable for arbitrary strategies \expspace (using Petri-Nets). So the whole procedure can be done in \expspace.
 Each game $H$ is solvable for memoryless strategies in \exptime (just run through all the memoryless strategies, and look for a lasso satisfying the winning condition). So the whole procedure can be done in \exptime.
 
 To get the result for the lexicographic case, adapt the formula that $E$ needs to satisfy to reflect the lexicographic order on the threshold objectives.
\end{proof}
The complexity can probably be improved. Use nondeterminism to guess $L$, and then solve a single game $H$. It is known that if the boolean combination of the objectives is a conjunction of ``greater-than'' conditions, then the complexity is co-\np-complete for finite-strategies, and \np-complete for memoryless strategies~\cite{DBLP:conf/fsttcs/ChatterjeeDHR10}. 

 \begin{conjecture}
  Memoryless-ENASH for Oxford-games is decidable in 3\exptime.
 \end{conjecture}
\begin{proof}[Sketch]
 By Conjecture~\ref{conj:oxford-mp}, we can reduce the problem to Memoryless-ENASH of lexicographic threshold mean-payoff games. By Conjecture~\ref{conj:ENASH-mpgs}
 these can be solved in \exptime.
\end{proof}

\todo{Q: What about a lower bound? See \cite{DBLP:journals/corr/BouyerBMU15}}


% The following is folklore: one can decide if a given player in a two-player 
% concurrent game has a winning strategy by reducing to a turn-based game in which 
% that player always moves first.

\subsection{Turn-based Setting}

The first paper studies mean-payoff games.

\begin{theorem}[\cite{Alpe91}] \label{thm:turnbased:MP}
In turn-based mean-payoff games:
\begin{enumerate}
 \item Memoryless NE may not exist if $N > 2$, \todo{Q: what about $N \leq 2$?}
 \item Finite-state NE exist, and the number of states may be taken to be $|\Stt|!$. \todo{Q: can this be improved to memoryless?}
\end{enumerate}
\end{theorem}
\begin{proof}[Sketch]
The main part of the proof of 2. is to show that such games are equivalent to certain 
finite-duration games: play as usual, but stop once a state is repeated and a 
simple cycle is formed, and define the utility of the play for agent $i$ to be 
mean-payoff costs of the cycle.  Given this, use backward induction (i.e., 
Zermelo-Kuhn theorem) to deduce that the finite duration game (and hence the 
original CCGS) admits a NE. Each strategy needs to remember at most the ordering 
of the states as they are encountered, i.e., $|\St|!$ memory is enough. This 
bound on the strategies of the finite-duration game transfers to the original 
game. 
\end{proof}

The next paper studies multi-player stochastic games with $\omega$-regular objectives.
Deterministic games are a special case.

\begin{theorem}[special case of \cite{CJM04}]
Turn-based games with prefix-independent $\omega$-regular objectives have
finite-state NE. \todo{Check if for parity objectives memoryless strategies suffice}
\todo{Check if paper mentions prefix-independent}
\end{theorem}
\begin{proof}[Sketch] 
The idea is that a NE is one where the players play to try and achieve their 
own 
objectives, but the moment a player deviates all the others punish that player 
by ensuring she still can't achieve her objective if she couldn't in the first 
place.

In more detail, let $W_i$ be the 
winning region of the game in which player $i$ tries to achieve its objective 
and all the others form an adversarial-coalition. This is a two-player zero-sum 
game (i.e., player $i$ vs a single player representing all the other players, 
called the adversarial coalition) with $\omega$-regular objective. These games 
are determined, and thus the states $L_i$ not in $W_i$ are won by the 
adversarial coalition. Let $\sigma_i$ be a strategy for player $i$ that wins 
from $W_i$, and let $\sigma_i^j$ be a strategy for players $j \neq i$ such that 
they, collectively, ensure that player $i$ loses on $L_i$. Define a strategy 
$\rho_i$ for player $i$ as follows: as long as all other players $j \neq i$ 
behave according to $\sigma_j$, then $\rho_i$ behaves like $\sigma_i$. However, 
the moment some player $j \neq i$ deviates from $\sigma_j$ then $\rho_i$ 
punishes player $j$ by playing $\sigma_j^i$. The strategy profile $\tpl{\rho_1, 
\cdots, \rho_n}$ is a NE.
Thus each $\rho_i$
just needs to remember whether or not it is in normal behaviour or punishing 
behaviour. Thus if the $\sigma_i$ and $\sigma_i^j$ were finite-state 
then the $\rho_i$ are finite-state.
But two-player zero-sum games with $\omega$-regular objectives admit finite-state strategies.
\end{proof}

The next paper studies generalised Muller games. Here the payoff is a function that depends only on the set of states seen infinitely often on a play. 
\begin{theorem}[\cite{DBLP:conf/fsttcs/PaulS09}]
Every turn-based generalised Muller game admits finite-state NE.
\end{theorem}
The proof does not go via threat-strategies, but via backward induction. The 
authors also prove that one can decide if such a game has a subgame-perfect 
equilibrium.

\begin{conjecture}
In turn-based $F$-mean-payoff games, finite-state NE exist, and the number of states may be taken to be $|\Stt|!$. 
\end{conjecture}
\begin{proof}[Idea]
Proof should follow that of Theorem~\ref{thm:turnbased:MP}. One could also adapt the proof in \cite{AmRu14}.
\end{proof}








  



%The fact that one-player mean-payoff games can be solved in \ptime\ (Karp 1978) 
%means that one can compute a NE in \expspace.

	

%\begin{theorem}[Ehrenfeuct + Mycielski, 1979]
%Two player zero-sum mean-payoff games are memoryless determined, i.e., exactly 
%one of the players has a winning strategy, and that strategy may be taken to be 
%memoryless.
%\end{theorem}











\subsection{Other related work that should be checked}

\begin{enumerate}
 \item \url{http://arxiv.org/abs/1210.3539} covers synthesis of specifications written in LTL with mean-payoff weights. Reduced to parity games.
\item Rational synthesis, e.g., \cite{CFRR16}

\end{enumerate}








\bibliographystyle{plain}

\bibliography{refs}


\end{document}

\section{Conclusion}

\head{Future Work}
\begin{enumerate}
 \item Decide if there is a unique NE (cite GSL paper).
 \item The constrained NE-emptiness problem, i.e., does there exist a NE whose value (for each player) is in some given interval? This is studied for mean-payoff games in \cite{DBLP:journals/corr/abs-1109-6220} where it is shown that constrained NE-emptiness is \np-complete for pure NE.
\end{enumerate}



\bibliographystyle{abbrv}

\bibliography{refs}

\end{document}

