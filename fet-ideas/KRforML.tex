\documentclass[a4paper,10pt]{article}
\usepackage[utf8]{inputenc}

%opening
\title{Idea for a project: KR for ML}
\author{}

%% PACKAGES %%
%\usepackage{latexsym}
\usepackage{amsmath}
\usepackage{amssymb}
%\usepackage{amsthm}


\usepackage{xspace,color,hyperref,upgreek} 
\usepackage{graphicx}
\usepackage{tikz,pgf}
\usetikzlibrary{arrows,shapes,automata,positioning,matrix,calc,petri}

%% ENVIRONMENTS %%
%\theoremstyle{plain}
%\newtheorem{theorem}{Theorem}
%\newtheorem{lemma}{Lemma}
%\newtheorem{fact}{Fact}
%\newtheorem{example}{Example}
%\newtheorem{definition}{Definition}
%\newtheorem{corollary}{Corollary}
%\newtheorem{proposition}{Proposition}
%\newtheorem*{proof*}{Proof}

%% COMMENTS %%

%\newcommand\note[1]{{\color{red}{#1}}}

\newcommand\sr[1]{{\color{red}{(sr:#1)}}}
% \renewcommand{\sr}[1]{}
\newcommand\ba[1]{{\color{purple}{(ba:#1)}}}
\newcommand\fz[1]{{\color{blue}{(fz:#1)}}}

\newcommand{\todo}[1]{{{\color{blue} #1}}}
%\renewcommand{\todo}[1]{}

%% LATEX SHORTCUTS %%

\def\it{\begin{itemize} }
\def\-{\item }
\def\ti{\end{itemize} }
\def\en{\begin{enumerate} }
\def\ne{\end{enumerate} }

%% COMPLEXITY CLASSES %%


\def\exptime{\textsc{exptime}\xspace}
\def\exptimeC{\exptime-complete}

\def\expspace{\textsc{expspace}\xspace}

\def\pspace{\textsc{pspace}\xspace}
\def\pspaceC{\pspace-complete}

\def\logspace{\textsc{logspace}\xspace}
\def\nlogspace{\textsc{nlogspace}\xspace}

\def\ptime{\textsc{ptime}}
\def\np{\textsc{np}}



%% LOGIC %%
\def\fol{\mathsf{FOL}}
\def\msol{\ensuremath{\mathsf{MSOL}}\xspace}
\def\fotc{\mathsf{FOL+TC}}

%% TEMPORAL LOGIC %%
\newcommand{\sqsq}[1]{\ensuremath{[\negthinspace[#1]\negthinspace]}}

\DeclareMathOperator{\ctlE}{{\mathsf{E}}}
\DeclareMathOperator{\ctlA}{\mathsf{A}}

\newcommand{\atlE}[1]{\ensuremath{\langle\!\langle{\text{#1}}\rangle\!\rangle}}
\newcommand{\atlA}[1]{\ensuremath{[\:\!\![\text{#1}]\:\!\!]}}

\DeclareMathOperator{\nextX}{\mathsf{X}}
\DeclareMathOperator{\yesterday}{\mathsf{Y}}
\DeclareMathOperator{\until}{\mathbin{\mathsf{U}}}
\DeclareMathOperator{\weakuntil}{\mathbin{\mathsf{W}}}
\DeclareMathOperator{\since}{\mathbin{\mathsf{S}}}
\DeclareMathOperator{\releases}{\mathbin{\mathsf{R}}}
\DeclareMathOperator{\always}{\mathsf{G}}
\DeclareMathOperator{\hitherto}{\mathsf{H}}
\DeclareMathOperator{\eventually}{\ensuremath{\mathsf{F}}\xspace}
\DeclareMathOperator{\previously}{\mathsf{P}}
\newcommand{\true}{\mathsf{true}}
\newcommand{\false}{\mathsf{false}}


\newcommand{\LTL}{\ensuremath{\mathsf{LTL}}\xspace}
\newcommand{\PLTL}{\textsf{PROMPT-}\LTL}

\newcommand{\CTL}{\ensuremath{\mathsf{CTL}}\xspace}
\newcommand{\CTLS}{\ensuremath{\mathsf{CTL}^*}\xspace}
\newcommand{\PCTLS}{\textsf{PROMPT-}\CTLS}
\newcommand{\PCTL}{\textsf{PROMPT-}\CTL}
\newcommand{\CLTL}{\ensuremath{\textsf{C-}\LTL}\xspace}
\newcommand{\PCLTL}{\ensuremath{\textsf{PROMPT-C}\LTL}\xspace}

\newcommand{\ATL}{\ensuremath{\mathsf{ATL}}\xspace}
\newcommand{\ATLS}{\ensuremath{\mathsf{ATL}^*}\xspace}
\newcommand{\PATLS}{\textsf{PROMPT-}\ATLS}
\newcommand{\PATL}{\textsf{PROMPT-}\ATL}


\def\alt{\, | \,}


\DeclareMathOperator{\Fp}{\eventually_\mathsf{P}}

\newcommand{\AP}{\mathrm{A\!P}}

%% MATH OPERATIONS %%
\newcommand{\tpl}[1]{\langle {#1} \rangle }
\newcommand{\tup}[1]{\overline{#1}}


%% STRUCTURES %%
\newcommand{\LTS}{\mathsf{S}}
\newcommand{\Paths}{\mathsf{pth}}
\newcommand{\nat}{\mathbb{N}}
\def\int{\mathbb{Z}}
\newcommand{\natzero}{\mathbb{N}_0}

\newcommand{\trans}[3]{#1 \stackrel{{#3}}{\rightarrow} #2}


%% HEADINGS ETC %%
\newcommand{\head}[1]{\noindent{\bf #1}}

%% COUNTER MACHINES %%
\newcommand{\cm}{\ensuremath{M}}
\newcommand{\CMinc}{\mathsf{inc}}
\newcommand{\CMdec}{\mathsf{dec}}
\newcommand{\CMzero}{\mathsf{ifzero}}
\newcommand{\CMnonzero}{\mathsf{nzero}}
\newcommand{\CMcommit}{\mathsf{end}}


%% CLTL %%
\def\var{{\sf var}}
\def\ovar{{\sf ovar}}
\def\avar{{\sf avar}}
\def\svar{{\sf svar}}
\def\bvar{{\sf bvar}}

\def\MOD{\equiv}

%% PVP %%
\def\PVP{\mathsf{PVP}}



\def\bool{\mathbb{B}}
\def\real{\mathbb{R}}

\usepackage{amsmath}

\begin{document}

\maketitle

\begin{abstract}
% The purpose of these notes is to try and formulate precise questions that are of 
% interest to both ML and CAV/KR people. The starting point is the feeling that ML 
% consists of a bunch of black boxes for doing isolated tasks. Given this is true, 
% the challenge is integrate a set of black boxes in such a way that a) the 
% integration is sound, and b) we can extract knowledge from the integrated 
% system.
% 
Hypothesis: ML has focuses on solving concrete learning/prediction tasks. In order to harness this technology, it needs to be imbued with a KR philosophy.
\end{abstract}

\section{Next steps}
\begin{enumerate}
 \item What is the killer-ap? can combination of ontology and ML help guide ML
 techniques? what does one need ontologies to do?
 \url{http://dl-learner.org/about/ontology-learning/}, 
 \item How to find patterns in data... 
 \url{https://www.youtube.com/watch?v=X6Qt_BxWKMo},

 \item \url{http://www.ttic.edu/SNL2017/}
 \item Find and engage the best ML people interested in CAV/KR.
 \item Do a literature search to find out what has been done at the interface of ML and KR (see Section~\ref{sec:lit}).
 \begin{enumerate}
  \item \url{https://sites.google.com/site/krr2015/home/schedule} was a workshop on symbolic and neural approaches to KRR.
  \item \url{http://www1.icsi.berkeley.edu/~shastri/shruti/} has work on modeling logic programs as NN and running the NN to evaluate queries.
  \item \url{http://www.ntu.edu.sg/home/asahtan/} has worked on integrating domain knowledge into NNs.
  \item Lise Getoor: "we need machine learning for graphs"
 \end{enumerate}
  \item Can one solve synthesis using ML? Yes, genetic programming.
 \item The five tribes of ML: \url{https://www.youtube.com/watch?v=B8J4uefCQMc}
 \item \url{https://en.wikipedia.org/wiki/Markov_logic_network}

 \item 
Pour resoudre des jeux, j'avais parle de Daniel Neider :
 \url{https://www.lii.rwth-aachen.de/en/18-mitarbeiter/wissenschaftliche-mitarbeiter/team-prof-grohe/121-publications-of-daniel-neider.html}
 Celui-la par ex : \url{https://arxiv.org/abs/1507.05612}

 Sur l'application de l'algo de Angluin dans la verif compositionnelle :
 Cobleigh, J.M., Giannakopoulou, D., and Pasareanu, C.S. "Learning Assumptions
 for Compositional Verification", TACAS 2003.
 Voir ici : \url{https://ti.arc.nasa.gov/profile/dimitra/}

 P Madhusudan fait des choses aussi avec D Neider entre autres :
 \url{http://madhu.cs.illinois.edu/pub.html}
\item \url{http://ti.tuwien.ac.at/cps/teaching/practicals/?id=178} Neurons controlling locomotion.
\end{enumerate}

\section*{Glossary}

\begin{itemize}
 \item NN = artificial neural network

\item KR = knowledge representation

\item ML = machine learning

\end{itemize}

%Question: how much progress has there been on extracting/injecting
%rules/predicates/automata from/into neural networks?

%Given that the answer is "not much", we propose a way to use NN as building
%blocks of a KR system, ala automatic structures using NN instead of automata.

\section{NNs 101}
An \emph{activation function} is a function $F:\real \to \real$. Two common activation functions are
\[
 F_1(z) = \begin{cases}
         1 & \mbox{ if } z > 0\\
         0 & \mbox{ otherwise.}
        \end{cases}
\]
and
\[
 F_2(z) = \frac{1}{1+e^{-z}}.
\]

A \emph{neuron} is a function $G:\bool^n \to \bool$ where $G(\tup{x}) = F(\tup{x} \cdot \tup{w})$ for some
\emph{weights} $w_i \in \real$. Neurons with activation functions $F_1$ are called \emph{perceptrons}, and with activation functions $F_2$ are
called \emph{sigmoid neurons}.

% \begin{remark}
% Perceptrons are also known as \emph{weighted voting games} in co-operative game theory. What tools can be transfered between these domains? Does it make sense 
% to talk about a network of games?
% \end{remark}
% 
A \emph{neural network (NN)} is a DAG whose nodes are neurons. A NN with $N$ inputs and $M$ outputs 
computes a function $F:\bool^N \to \real^M$. One may round the output of $F$ to the nearest integer $0$ or $1$ to get a function $F:\bool^N \to \bool^M$.

\begin{fact}
 NN built from perceptrons can compute arbitrary Boolean functions. E.g., $N = 3, M = 1, w_1 = w_2 = 1, w_3 = -1$ determines $x_1 \wedge x_2$ if we fix $x_3 = 1$; and $N = 2, M = 1, w_1 = -2, w_2 = 1$ determines $\neg x_1$ if we fix $x_2 = 1$.
\end{fact}

One might be tempted to consider the synthesis problem is: given a function $G$, find a NN that computes it. The problem is that $G$ is typically not known so one can't know if a NN actually computes $G$. Instead, some tuples $(x,G(x))$ are known. This data can be used as training and validation data, and there are measures
of how well a NN represents $G$ (e.g., cross-validation). Thus from a ML point of view, this is a \emph{optimisation} problem which one can think of as minimising some error.

The typical approach in NN seems to be: a) get a bunch of data, b) take an 
educated guess at the DAG, c) apply an optimisation technique (e.g., gradient 
descent) to \emph{learn} the weights that minimise the error.

\begin{remark}
 I read that there are no good learning algorithms for NNs using $F_1$ unless the DAG has very special form (e.g., consists of a single neuron).
\end{remark}

\section{KR for NN}

Suppose you've trained a bunch of NN to do certain tasks in the same domain.
Are these NNs consistent with each other? Can one compose these NNs to do more complex tasks?

\begin{example}
Suppose $N_1$ recognises pictures of birds and $N_2$ recognises pictures of animals. 

\begin{enumerate}
 \item \emph{consistency}: check if every input recognised by $N_1$ is also recognised by $N_2$? 
 \item \emph{composition}: build a NN that recognises all animals that are not birds.
\end{enumerate}
\end{example}

\begin{definition}
 Given a finite directed graph $(V,E)$ and a NN $N_v$ for every $v \in V$. This setting is \emph{consistent} if $(v,w) \in E$ implies $N_v(\tup{x}) \to N_w(\tup{x})$ for all $\tup{x}$ (intuitively, $E$ represents the ``is a'' relation).
\end{definition}

\begin{theorem}
Assume we are using the activation function $F_1$. Consistency is decidable. 
\end{theorem}

\begin{proof}
Recall that the FO-theory of $\mathcal{R} := (\mathbb{R},+,<)$ is decidable, a fact that can be proved using automata on infinite words. 
But every NN $N(\tup{x})$ can be written as a FO formula $A(\tup{x})$ over $\mathcal{R}$. Thus we can decide consistency, i.e., decide if $\forall \tup{x}. A(\tup{x}) \to B(\tup{x})$. 
\end{proof}

Actually, $A(\tup{x})$ is an existential formula, i.e., of the form $\exists \tup{y}. \phi(\tup{x},\tup{y})$ where $\phi$ is a quantifier free formula in the signature of $\mathcal{R}$. So, consistency can be checked in EXPTIME. Can it be done in PSPACE?

What about composition? We can build a formula for the composition, e.g., $C(\tup{x}) = A(\tup{x}) \wedge \neg B(\tup{x})$. But, how does one turn $C$ into a NN? Does one need to turn it into a NN?
 
Can we say anything if the activation function is $F_2$?

\subsection{DL for ML}

NN can be used to represent binary relations between objects, e.g., ``friend'' relation, ``like'' relation, etc. 

So, the description logic $FL^{-1}$ seems like a start for doing basic reasoning about the objects and their relationships. We describe this logic. Given a set of unary predicates $C_i$ and binary predicates $R_j$, a \emph{concept} is generated by the unary predicates, and closed under the following operations: $C_i \cap C_j$, 
$\{x : \forall y. R_j(x,y) \to C_i(y)\}$, and $\{x : \exists y. R_j(x,y)\}$.

In DL, one basic question is ``subsumption'', i.e., $\forall x. C(x) \to D(x)$. This is what we called consistency before.

Issue: One problem with subsumption/consistency is that $\forall x$ is probably too strong. Perhaps one needs something like ``almost all x''. Or, assuming that the data is labeled consistently, what is the error/probability that a data point $x$ is classified as being in $C$ but not in $D$.
% \todo{idea. have some sort of semantic graph representing objects and their
% relationships. assume that some of the objects and relationships are represented
% by NNs. use tarski or presburger to solve these questions if using activation
% function $F_1$. What to do if using function $F_2$? instead of FOL, perhaps it
% is neater to use a description logic like $FL^{-1}$? to do this we need to
% represent roles not just concepts. e.g., "father(x,y)"}
% 
% \subsection{Wild Ideas}
% 
% \begin{enumerate}
% \item Can one adapt Angluin's algorithm for learning DFAs to learn a neural net?
% \end{enumerate}
% 



\section{Wild ideas}

\begin{enumerate}
\item Ontologies guide feature extraction?
\item Can FOL deduction be implemented in a NN? See Smolensky 1988 "On the proper treatment of Connectionism".
\end{enumerate}

\section{Literature} \label{sec:lit}

\subsubsection{Some basic machine terminology}

\subsubsection{Towell and Shavlik 1994, KR for ANN}

\subsubsection{Markov Logic Networks}

\subsubsection{Misc}
Modern AI systems can generally be thought of as observing some input and recovering some (hidden)
structure of interest:
– We observe an image and recover some description of the scene.
– We observe a sentence of English and recover a syntax tree, a meaning representation, a translation into
Chinese, etc.
– We are given a goal or reward function and recover a plan to earn rewards.
– We observe some facts expressed in a knowledge representation language and recover some other facts
that can be logically deduced or statistically guessed from them.
– We observe a dataset and recover the parameters of the probability distribution that generated it.
\end{document}
