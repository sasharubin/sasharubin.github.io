\section{Background}  \label{sec:prelim}
We will use basic notions from mathematical logic including monadic second-order logic~\cite{EbFl95}, automata theory~\cite{HMU03}, 
and linear-temporal logic~\cite{CGP1999}. We will introduce them below, as needed.
%, and the theorem of McNaughton and Papert connecting star-free languages and first-order logic, see~\cite{Thomas96})
%
Write $B^\omega$ for the set of infinite sequences over alphabet $B$, and write $B^*$ for the set of finite sequences. The empty sequence is denoted $\epsilon$.
Sequences are indexed starting at $0$, thus we write $\alpha = \alpha_0 \alpha_1  \cdots$. 
Write $[n]$ for the set $\{1,2,\cdots,n\}$.

\todo{In the rest of this paper we use the term ``robot'' instead of ``mobile agent''.}

% The reader may refer to this section as needed.
\subsection{Graphs, \LTL and \msol} \label{subsec:graphs-MSOL}

\head{Graphs.} Graphs will serve as the environment in which robots move.
Let $\Sigma$ be a finite set of {\em edge labels}.
A {\em $\Sigma$-graph}, or {\em graph}, $G$, is a tuple $(V,\lambda)$ where $V$ is a finite set of {\em vertices} and $\lambda:V^2 \to \Sigma$ is a partial function called the {\em edge labeling function}. The domain of $\lambda$ is denoted $E \subseteq V \times V$, and is called the
{\em edge relation}. We write $\trans{v}{w}{\sigma}$ to mean $\lambda(v,w) = \sigma$.
A graph is \emph{deterministic} if for every $v \in V, \sigma \in \Sigma$ there exists at most one $v'$ such that $\lambda(v,v') = \sigma$. Here are some examples of classes of deterministic graphs.

% \sr{Q: do we want to consider nondeterministic graphs? what about vertices from which there may not be an edge labeled $\sigma$?}

%The {\em out-degree} of a vertex $v$, written $\deg(v)$, is the cardinality of the set $\{(v,w) \in E : w \in V\}$ of {\em outgoing edges} of $v$.
%
%Sometimes the edge labels give a sense of direction:
%
%\begin{example} An {\em $L$-labeled grid} is a graph with $V = [n] \times [m]$, $\Sigma =  L \times \{N,S,E,W\}$, and labels $\lambda((x,y),(x+1,y)) \in L \times \{E\}$, etc., where $L$ is a finite set and $n,m \in \nat$. An {\em $L$-labeled line} is an $L$-labeled grid with $V = [n] \times [1]$. If $|L| = 1$ then we call the grid (or line) {\em unlabeled}.
%\end{example}
%


% \label{ex:tree}
\head{Trees.} A {\em $\Delta$-ary tree} (for $\Delta \in \nat$) is a $\Sigma$-graph $(V,\lambda)$ where $(V,E)$ is a tree, $\Sigma = [\Delta] \cup \{up\}$, and $\lambda$ labels the edge leading to the node in direction $i$ (if it exists) by $i$, and the edge leading to the parent of a node (other than the root) is labelled by $up$. We may rename the labels for convenience, e.g., for binary trees ($\Delta = 2$) we let $\Sigma = \{lc,rc,up\}$ where $lc$ replaces $1$ and $rc$ replaces $2$. For instance, see Figure~\ref{fig:tree}.

\head{Lines.} The \emph{$n$th line} is the $\Sigma$-graph $L_n = (V_n,\lambda_n)$, where $\Sigma = \{l,r\}$, $V_n = [n]$, and $\lambda_n(i,i+1) = r$ and $\lambda_n(i+1,i) = l$. Thus, the domain of $\lambda_n$ is  $E_n = \cup_{i,j \in [n]} \{(i,j) : |i-j| = 1\}$. \label{def:line}

\head{Grids.} The \emph{$n$th grid} is the $\Sigma$-graph $G_n = (V_n,\lambda_n)$, where $\Sigma = \{u,d,l,r\}$, $V_n = [n] \times [n]$,  and the labels are as expected with $u$ standing for ``up'', $d$ for ``down'', $l$ for ``left'' and $r$ for ``right''. E.g., $\trans{(i,j)}{(i+1,j)}{r}$ and $\trans{(i,j)}{(i,j+1)}{u}$.
% Thus the edge relation $E_n$ consists of all pairs $((i,j),(i',j'))$ such that $|i-i'| = 1$ xor $|j = j'| = 1$.

%A $\Delta$-ary tree with $\Delta = 1$ is called a {\em line}. A $\Delta$-ary tree in which every vertex has degree at most $2$ is called a {\em labelled line}.


% The degree of a graph $G$ is the maximum degree of all its vertices.


% To simplify the presentation we do not distinguish between the syntactic symbols used in formulas (such as the binary relation symbol $E$) and the semantic symbols used in structures; thus, e.g., we write the formula $E(x,y)$ where $E$ is a binary relation symbol, as well as the expression $(u,v) \in E$ where $E$ is the edge relation of a graph.


\head{Linear Temporal Logic.} \label{dfn:LTL}
This is a canonical modal logic for the specification of reactive systems. Here we will use it as a foundation for specifying robot tasks.

For a set of \emph{atoms} $\AP$, write $\LTL(\AP)$ (or simply \LTL) for the logic with the following syntax and semantics:
$\varphi ::= p \mid \varphi \wedge \varphi \mid \neg \varphi \mid \nextX \varphi \mid \varphi \until \varphi$, where $p$ varies over $\AP$.
We use standard abbreviations, e.g., $\false := p \wedge \neg p, \true := \neg \false, \eventually \varphi := \true \until \varphi$,
	$\always \varphi := \neg \eventually \neg \varphi$, etc.
% 	 $\varphi \weakuntil \varphi' := (\varphi \until \varphi') \vee \always \varphi$,
	The \emph{safety} formulas are those of the form $\always b$ where $b$ is a Boolean combination of atoms.
	
	
	Formulas are interpreted over infinite strings $\alpha = \alpha_0 \alpha_1 \alpha_2 \cdots \in (2^\AP)^\omega$. Define the satisfaction relation
	$\models_\LTL$ as follows:
	\it
	\- $(\alpha,n) \models_\LTL p$ iff $p \in \alpha_n$;
	\- $(\alpha,n) \models_\LTL \varphi_1 \wedge \varphi_2$ iff $(\alpha,n) \models_\LTL \varphi_i$ for $i = 1,2$;
	\-	$(\alpha,n) \models_\LTL \neg \varphi$ iff it is not the case that $(\alpha,n) \models_\LTL \varphi$;
	\-  $(\alpha,n) \models_\LTL \nextX \varphi$ iff $(\alpha,n+1) \models_\LTL \varphi$;
	\- $(\alpha,n) \models_\LTL \varphi_1 \until \varphi_2$ iff there exists $i \geq n$ such that $(\alpha,i) \models_\LTL \varphi_2$ and for all $i \leq j < n$, $(\alpha,j) \models_\LTL \varphi_1$.
	\ti
	Write $\alpha \models_\LTL \varphi$ if $(\alpha,0) \models_\LTL \varphi$ and say that $\alpha$ \emph{satisfies} $\varphi$.



  \head{Monadic Second-order Logic.} \label{dfn:msol} This is an extension of first-order logic by set variables. It will be used to define powerful testing abilities of robots, as well as a technical tool in the decidability proofs.

Formulas are interpreted in $\Sigma$-graphs $G$. Define the set of monadic second-order formulas $\msol(\Sigma)$ as follows. Formulas of $\msol(\Sigma)$ are built using {\em first-order variables} $x,y,\cdots$ that vary over vertices, and {\em set variables} $X,Y, \cdots$ that vary over sets of vertices. The {\em atomic formulas} (when interpreted over $\Sigma$-graphs) are:
\it
\- $x = y$ (denoting that vertex $x$ is the same as vertex $y$),
\- $x \in X$ (denoting that vertex $x$ is in the set of vertices $X$),
% \- $init(x)$ (denoting that $x$ is the initial vertex $v_0$), and
\- $\lambda(x,y) = \sigma$ (denoting that there is an edge from $x$ to $y$ labeled $\sigma \in \Sigma$).
% \- $\true$ (the formula that is always true).
\ti
The formulas of $\msol(\Sigma)$ are built from the atomic  formulas using the Boolean connectives (i.e., $\neg,\vee, \wedge, \limp$) and variable quantification (i.e., $\forall,\exists$ over both types of variables). The fragment of $\msol(\Sigma)$ which does not mention set variables is called {\em first-order logic}, denoted $\fol(\Sigma)$.
Write $\msol_k(\Sigma)$ for formulas with at most $k$ many free first-order variables and no free set-variables. We abbreviate $z_1, \cdots, z_k$ by $\tup{z}$. We write $\phi(x_1, \cdots, x_k)$ to mean that the free variables from the formula $\phi$ are amongst the set $\{x_1, \cdots, x_k\}$. Note that the formula $\phi(x_1,\cdots,x_k)$ does not need to use all the variables from the set $\{x_1,\cdots,x_k\}$. When $\Sigma$ is clear from the context, it may be dropped.

The semantics of \msol is defined in a similar way as for first-order logic. Instead of giving the formal definitions (see~\cite{EbFl95}) it is enough to introduce some useful notation. Given a graph $G$, a \emph{valuation $\nu$} is a function that maps each first-order variable to a vertex in $G$ and each monadic second-order variable to a set of vertices of $G$. Thus, we write $(G,\nu) \models \phi(x_1,\cdots,x_k,X_1, \cdots, X_l)$ to mean that $\phi$ holds in $G$ with variable $x_i$ (resp. $X_i$) simultaneously substituted by vertex $\nu(x_i)$ (resp. $\nu(X_i)$).
A {\em sentence} is a formula with no free variables. In this case we write $G \models \phi$ to mean that $\phi$ holds in $G$. A sentence $\phi$ is \emph{satisfiable} if there exists a graph $G$ such that $G \models \phi$.



%  A sentence $\phi$ is \emph{valid} if for all graphs
%  $G$, $G \models \phi$.
% \sr{remove dfns of sat and valid if they are not needed}


%\footnote{In the section on \RLTL\ we will introduce the satisfaction relation $\models_{\tup{R},\Omega}$.}

Here are some examples of formulas and their meanings:
\example{\label{ex:formulas}
\begin{itemize}

% \item  We abbreviate $\bigwedge_i x_i = y_i$ by $\tup{x} = \tup{y}$, and $\bigwedge_i x_i \in X_i$ by $\tup{x} \in \tup{X}$.

\item The formula $\forall x (x \in X \limp x \in Y)$ means that $X \subseteq Y$. Similarly, there are formulas for the set operations $\cup, \cap,=$, and relative complement $X \setminus Y$.

 \item The formula $E(x,y) := \bigvee_{\sigma \in \Sigma} \lambda(x,y) = \sigma$ says that there is an edge from $x$ to $y$.

 \item On lines (so $\Sigma = \{l,r\}$) the formula $\neg \exists y. \lambda(x,y) = r$ states that $x$ is on the right end-point of the line.

 \item On binary-trees (so $\Sigma = \{lc,rc,up\}$) the formula $\neg \exists y. \lambda(y,x) = up$ states that $x$ is a leaf, $\neg \exists y. \lambda(x,y) = up$ states that $x$ is the root, and $\exists y. \lambda(y,x) = lc$ states that $x$ is a left-child.

%\item The formula $\exists x \exists y (x\neq y \wedge  edg(z,x) \wedge edg(z,y))$ means that $\deg(z) \geq 2$.
 \item The formula $E^*(x,y) := \forall Z [(closed_{E}(Z) \wedge x \in Z) \limp y \in Z]$ defines the transitive closure of $E$, where $closed_{E}(Z)$ is $\forall a \forall b[(a \in Z \wedge E(a,b)) \limp b \in Z]$.


%  $\phi^*(x,y)$ holds in a graph $G$ if and only if there exists a finite sequence of vertices $v_1 v_2 \cdots v_m \in V^*$  (here $m \geq 1$) such that $x = v_1, y = v_m$ and $\phi(v_i,v_{i+1})$ holds in $G$ for all $i < m$ (for example, if $\phi(x,y) = edg(x,y)$ then $G \models \phi^*(x,y)$  if and only if $G$ has a path from $x$ to $y$). This shows that that if a binary relation is $\msol$-definable, then so is its transitive-closure.
% %%Write $[[\phi]]_G := \{\tup{v} \in V^k : G \models \tup{v}\}$.
%%We say that a graph $G$ {\em satisfies} a sentence $\phi$, and write $G \models \phi$, if $\phi$ is true in the graph $G$.
\end{itemize}
}
% of $\msol$-definable formulas \label{ex:TC} \begin{itemize}
%\item
%
%
%
%%Similarly, $\phi^+(x,y) := \exists z (\phi(x,z) \wedge \phi^{*}(z,y))$ is as before, but replace $m \geq 1$ by $m > 1$.
%\item Similarly, define $\phi^\omega(x,y) := \phi^*(x,y) \wedge \exists z (\phi(y,z) \wedge \phi^{*}(z,y))$. Note $\phi^\omega(x,y)$ holds in a graph $G$ if and only if there is an infinite sequence of vertices $v_1 v_2 \cdots \in V^\omega$ with $x= v_1$, $v_i = y$ for infinitely many $i \in \nat$, and $\phi(v_i,v_{i+1})$ holds in $G$ for all $i \in \nat$. This uses the fact that $V$ is finite.
%
%\item Note that if $\phi(x,y,\tup{z})$ is an $\msol(\Sigma)$ formula we can define an $\msol(\Sigma)$ formula $\phi^*(x,y,\tup{z})$ that expresses  that there exists a finite sequence of vertices $v_1 v_2 \cdots v_m \in V^*$  (here $m \geq 1$) such that $x = v_1, y = v_m$ and $\phi(v_i,v_{i+1},\tup{z})$ holds in $G$ for all $i < m$. In other words, we can treat $\tup{z}$ as parameters. We can similarly add parameters to $\phi^\omega$.
%\end{itemize}

%no space
%\item[MF5.] \label{ex:ktc}


%The set of formulas in $\fotc(\Sigma)$ is defined to be the set of formulas $\fol(\Sigma)$  closed under the transitive-closure operator (for $k=1$). We have seen that every formula in $\fotc(\Sigma)$ is expressible as a formula in $\msol(\Sigma)$, which justifies writing $\fotc(\Sigma) \subset \msol(\Sigma)$, a slight abuse of notation. \sr{is FOTC used?}






%\subsection{Automata and Regular Expressions.}
%\sr{define automata? universality problem?}
%
% Ordinary {\em regular-expressions} over a finite alphabet $B$ are built from the
% sets $\emptyset$, $\{\epsilon\}$, and $\{b\}$ ($b \in B$), and the operations
% union $+$, concatenation $\cdot$, and Kleene-star $\phantom{}^*$.
% Kleene's Theorem states that the languages definable by regular expressions over
% alphabet $B$ are exactly those recognised by finite automata over alphabet $B$.


%
% An {\em $\omega$-regular expression} over alphabet $B$ is inductively defined to
% be of the form: $exp^\omega$, $exp \cdot r$, or $r + r'$, where $exp$ is an
% ordinary regular-expression over $B$, and $r,r'$ are $\omega$-regular
% expressions over $B$. An {$\omega$-regular language} is one defined by an
% $\omega$-regular expression. A variation of Kleene's Theorem says that the
% languages definable by $\omega$-regular expressions over alphabet $B$ are
% exactly the languages recognised by B\"uchi automata over alphabet $B$ (which
% are like finite automata except they take infinite words as input, and a run is successful if
% some accepting state occurs infinitely often).


%A regular language over alphabet $B$ is {\em star-free} if it is the language of a regular-expression over $B$ that does not use the Kleene-star, but may use the complementation operator (written $\neg{r}$).
%A {\em star-free $\omega$-regular language} is the language of an $\omega$-regular expression that may mention the complement operator, but may not mention the $\omega$-power $exp^\omega$ nor the Kleene-star $exp^*$ operators.

\iffalse
We will use the McNaughton-Papert Theorem (and its generalisation to infinite-words) (see \cite{Thomas96,DiGa08}): it states that a regular language of ($\omega$)-words is star-free if and only if it is definable in first-order logic. For instance, the language $0^*1^*$ over alphabet $\{0,1\}$ is definable by the formula $\exists z. \forall x. [x \leq z \limp P_0(x)] \wedge [x \not \leq z \limp P_1(x)]$, and by the star-free regular expression $\neg( \neg{\emptyset} \cdot 1 \cdot 0 \cdot  \neg{\emptyset})$. Formally, a word $w \in B^*$ is coded as a structure with domain $[|w|]$, unary predicates $P_b := \{i \leq |w| : w_i = b\}$ (for $b \in B$), and the usual linear order $\leq$ on $[|w|]$; and thus the first-order definition may make use of these unary predicates and the linear order.
\fi

%\footnote{Although this is the standard encoding, if one prefers uniformity with the earlier definition graph, one may instead code the word $w$ as an edge-labeled graph with $|w|+1$ vertices, and an additional edge label for $\leq$.}

%\sr{use 'FO language' and 'FO models'?}
%Graphs of degree at most $\Delta \in \nat$ will be called $\Delta$-graphs.


\section{The Model of Multi-Robot Systems and Robot Task Logic} \label{sec:model}


In this section we provide a framework for modeling multi-robot systems parameterised by their environment. Here is a brief outline.
%We generalise the definition of robot systems from \cite{Rubin15AAMAS} to allow asynchronous behaviour.
An environment is modeled as a $\Sigma$-graphs $G$, and robots are modeled as finite-state automata over an alphabet of guarded commands.
A guarded command tells the robot to move along an edge with a given label or stay where it is, subject to a test being true.
Robots can test their own current positions, as well as the published positions and local states of other robots. That is, robots can publish their current
position and local state (by entering special `publishing' states) --- this can be thought of as shared memory, or a bulletin-board, where each robot overwrites
its current state and position. Robots move asynchronously, i.e., at least one robot (chosen nondeterministically) moves at a time.
We also provide a definition of a logic, based on Linear Temporal Logic (\LTL), in which one can express various tasks such as gathering, exploring, and reconfiguring.  The logic is called Robot Linear Temporal Logic (\RLTL): it is like \LTL except that its atoms are tests. We assign one \RLTL formula to each robot that specifies its goal, and we require that each of the robots achieve its goal. Thus, co-operative global specifications are given by specifying the individual requirement for each robot.
%e.g., to specify that all robots gather we assign to each robot the individual task of reaching a position containing all the other robots.
% All tests (whether for robots or specfication formula) are based on \msol.
%As discussed in Section~\ref{sec:DC}, our model is general enough to be able to express a popular model of robot systems found in the distributed computing literature~\cite{FIPPP04,Diks200438,Cohen05graphexploration,KKR06,GR08,Das13}.

\subsection{The Model of Multi-Robot Systems}

In the rest of this section, $\Sigma$ denotes a finite set of edge-labels, and $k \in \nat$ denotes the number of robots.

\head{Commands and Tests.}
%
A {\em command} is a symbol from $\upmu(\Sigma) := \{move(\sigma) : \sigma \in \Sigma\} \cup
\{stay\}$.  The command $move(\sigma)$ tells the robot to move
from its current vertex along an edge labeled $\sigma$, and the command
$stay$ tells the robot to stay at its current vertex.

% \sr{should we instead use $pub\_{pos_i}$, $cur\_pos$, $pub\_{st_i}$, $cur\_st$? no.}

% \sr{i changed $cur$ to $pos_{cur}$.}

 A {\em position-test} is a formula $\tau$ from $\msol_{k+1}(\Sigma)$ whose free variables come from the set 
 $\{\VARbcpos_1, \cdots, \VARbcpos_k,\VARpos_{cur}\}$. Informally, a position test
  \[
\tau(\VARbcpos_1,\cdots,\VARbcpos_k,\VARpos_{cur})  
 \]
allows the robot to test that $\tau$ holds in $G$
 where $\VARbcpos_{cur}$ is the current position of the robot doing the testing, and each $\VARbcpos_i$ is the last published position of the $i$th robot.
 Thus, the test $\VARbcpos_i = \VARbcpos_{cur}$ states that the current position (of the robot doing the testing) is equal to the last published position of robot $i$,
 the test $\bigwedge_{i,j} \VARbcpos_i = \VARbcpos_j$ states that the last published positions of all the robots are equal,
 and $E(\VARbcpos_i,\VARbcpos_j) \vee E(\VARbcpos_j,\VARbcpos_i)$ tests if the last published positions of robots $i$ and $j$ are adjacent.

 A {\em state-test} is an expression of the form $st_i = q$, or an expression of the form $st_{cur} = q$, where $i \in [k]$ and $q$ is a symbol representing a state (for concreteness, we may take $q \in \nat$).
 The meaning of $st_i = q$ is that
 the last published state of robot $i$ is $q$, and the meaning of $st_{cur} = q$ is that the current state of the robot doing the testing is $q$.
% For convenience, we assume that all robots have states from $\nat$, i.e., that $q \in \nat$.

% \begin{definition}[Tests]
 The set of \emph{tests for $k$ robots on $\Sigma$-labeled graphs} $\uptau_k(\Sigma)$ is the smallest set containing state-tests, position-tests, and closed under Boolean combinations ($\wedge,\neg$).
% \end{definition}

 For convenience we use the following natural shorthands. If $Q$ is a finite set of states,
 \begin{enumerate}
  \item $\bigvee_{q\in Q} (st_i = q \wedge st_j = q)$ is a test, which is abbreviated $st_i = st_j$. Similarly, we may write $st_{cur} = st_j$.
  \item $\bigvee_{q \in X} st_i = q$ (for $X \subseteq Q$) is abbreviated $st_i \in X$. Similarly, we may write $st_{cur} \in X$.
 \end{enumerate}



 % no space...  Here is a more complex test: is there a path from robot $i$ to
 % robot $j$ that only uses edges labeled $\sigma$? It can be expressed using
 % the transitive closure formula (see Example~MF4 in
 % Section~\ref{sec:prelim}). We emphasise that the test ``$\true$'' is always
 % satisfied.

%
%
% \begin{definition}[Guarded Commands]
The set of {\em guarded commands $\ins$} (which depends on $k,\Sigma$) consists of all expressions of the
form $\gc{\tau}{\kappa}$ where $\tau \in \uptau_k(\Sigma)$ is a test and $\kappa \in \mu(\Sigma)$ is a command.
% \end{definition}

\begin{example}
The guarded command $\gc{(\VARbcpos_1 = \VARbcpos_2 \wedge st_1 = st_2)}{move(u)}$
tells the robot to move in direction $u$ if the last published states and positions of robots $1$ and $2$ coincide.
\end{example}





%For a sequence of instructions $ins \in ((\ins_{\Sigma,k})^k)^*$ define  $[[ins]]_G \subseteq V^{2k}$ such that $(\tup{u},\tup{v}) \in [[ins]]_G$ if and only if, in $G$, one can reach $\tup{v}$ from $\tup{u}$ by successfully executing the instructions in $ins$. To improve readability, the notation $[[\phantom{n}]]_G$ does not mention $k$.
%%
% Formally,\sr{this can be simplified to what is means for one robot to make one move}
%  \begin{enumerate}
%  \item If $ins = \epsilon$, then $(\tup{u},\tup{v}) \in [[ins]]_G$ if and only if $\tup{u} = \tup{v}$;
%  \item If $ins = (d_1,\cdots,d_k) \in (\ins_{\Sigma,k})^k$
%  then $(\tup{u},\tup{v}) \in [[ins]]_G$ if, for each $i \in [k]$, writing $d_i = \tau_i \to \kappa_i$, %\footnote{This is where we introduce the assumption that robots act {\em synchronously}.}
% it is the case that $G \models \tau_i(\tup{u})$ and
% \begin{enumerate}
% \item  if $\kappa_i = \uparrow_\sigma$ then $\lambda(u_i,v_i) = \sigma$,
% \item if $\kappa_i = stay$ then $u_i = v_i$.
%  \end{enumerate}
%  \item If $ins = d \cdot e$ then $(\tup{u},\tup{v}) \in [[ins]]_G$ if and only if there exists $\tup{z} \in V$ such that $(\tup{u},\tup{z}) \in [[d]]_G$ and $(\tup{z},\tup{v}) \in [[e]]_G$.
%
%%  $[[ins]]_G := [[d]]_G \circ [[e]]_G$ (where $\circ$ denotes composition of relations).
%\end{enumerate}


\head{Robots.}
%
A {\em $k$-robot ensemble} is a sequence $\tup{R} = \tpl{R_1, \cdots, R_k}$ where each {\em robot} $R_i$ is a tuple $\tpl{Q_i,B_i,I_i,\delta_i}$ where
\begin{itemize}
 \item $Q_i$ is a finite set of {\em states},
 \item $B_i \subseteq Q_i$ is a set of {\em publishing} states,
 \item $I_i \subseteq B_i$ is a set of {\em initial} states --- so every initial state is also a publishing state, and
 \item $\delta_i \subseteq Q_i \times \ins \times Q_i$ is a finite {\em transition} relation.
\end{itemize}

For convenience, we assume that no publishing state has a self-loop, i.e., there is no $b \in B_i$ such that $(b,\gc{\tau}{\kappa},b) \in \delta_i$.


\begin{remark}
If a robot wanted to test its current state (rather than its last published state),
it can do this implicitly, e.g., the guarded command ``if my current state is $q$ then $move(\sigma)$ and change to $q'$'' can be expressed by adding the
transition $(q,\gc{\true}{move(\sigma)},q')$. The reason we add an explicit test $st_{cur} = q$ is so that the robot's specification formula can talk about its state, see \ref{sec:TL}.
\end{remark}


\head{Configurations.}
%
Fix a $\Sigma$-graph $G$ and $k$-robot ensemble $\tup{R}$.
A {\em configuration} $c$ is a function $c$ with domain $[k]$ such that for all $i \in [k]$,
\begin{enumerate}
 \item $c(i) \in V \times V \times Q_i \times Q_i$, we write $c(i) =  \tpl{pos_i(c), \bcpos_i(c), st_i(c), \bcst_i(c)}$,
 \item $st_i(c) \in B_i$ implies $\bcpos_i(c) = pos_i(c)$ and $\bcst_i(c) = st_i(c)$,
i.e., if robot $i$ is in a publishing state then it publishes its position and state.
\end{enumerate}


Informally, $pos_i(c)$ and $st_i(c)$ consist of the current position and state of robot $i$, while
$\bcpos_i(c)$ and $\bcst_i(c)$ consist of the last published position and state of robot $i$.

A configuration is {\em initial} if the following holds for all $i$: $\bcst_i(c) = st_i(c) \in I_i$ and $pos_i(c) = \bcpos_i(c)$, i.e., for each robot, the published data is equal to the current data, and the state is initial.\footnote{We remark that our $\Sigma$-graphs do not contain initial states. The vertices in which robots start can be specified in the specification logic, see Example~\ref{ex:RLTL-deploy}.}

\head{Definition of $Sat(-,-,-)$.}
Define the following property, written $Sat(c,\tau,i)$, that \emph{robot $i$ satisfies test $\tau$ in configuration $c$}  by the following inductive definition: \label{def:sat}
\begin{itemize}
\item if $\tau$ is a position-test $\tau(\VARbcpos_1, \cdots, \VARbcpos_k,\VARpos_{cur})$ then we require that
\[
G \models \tau(\bcpos_1(c), \cdots, \bcpos_k(c),pos_i(c)).
\]
\item if $\tau$ is a state-test ``$st_j = q$''then we require that $\bcst_j(c) = q$, and if it is a state test ``$st_{cur} = q$'' then we require that
$st_i(c) = q$.
\item if $\tau = \varphi_1 \wedge \varphi_2$ then we require that $Sat(c,\varphi_j,i)$ for $j = 1,2$.
\item if $\tau = \neg \varphi$ then we require that $Sat(c,\varphi,i)$ does not hold.
\end{itemize}

Informally, $Sat(c,\tau,i)$ says that the test $\tau$ is true when $pos_j$ (resp. $st_j$) is interpreted as the published position (resp. state) of robot $j$, while $\VARpos_{cur}$ (resp. $st_{cur}$) is interpreted as the current position (resp. state) of robot $i$.
% For two configurations $c = \tpl{\tup{u},\tup{p}}$, $d = \tpl{\tup{v},\tup{q}}$ and Let $K \subseteq [k]$ a non-empty set of indices (of robots),
% write $c \vdash d$ if:
% for each $i \in [k] \setminus K$:
% \it
% \- $u_i = v_i$ and $p_i = q_i$,
% \ti
% and for each $i \in K$:
% \it
% \- if $\kappa_i = \uparrow_\sigma$ then $\lambda(w_i,v_i) = \sigma$, and
% \- if $\kappa_i = stay$ then $w_i = v_i$,
% \ti


\head{Definition of $Mv(-,-,-,-)$.} Given configurations $c,d$, command $\kappa$ and robot $i$, define the property $Mv(c,d,\kappa,i)$, that \emph{robot $i$ moves from $c$ to $d$ using command $\kappa$}, by: \label{def:mv}
%\footnote{Since $c,d$ are assumed to be consistent, there is no need to update the published states.}
\begin{itemize}
\- if $\kappa = stay$ then $pos_i(c) = pos_i(d)$ (i.e., the $stay$ command does not change the current position),
\- if $\kappa = move(\sigma)$ then $\trans{pos_i(c)}{pos_i(d)}{\sigma}$ (i.e., the $move(\sigma)$ command moves the robot along an edge labeled $\sigma$), and
\- if $st_i(d) \not \in B_i$ then $\bcpos_i(d) = \bcpos_i(c)$ and $\bcst_i(d) = \bcst_i(c)$ (i.e., if the target state is not a publishing state then the published data does not change),
\- if $st_i(d) \in B_i$ then $\bcpos_i(d) = pos_i(d)$ and $\bcst_i(d) = st_i(d)$ (i.e., if the target state is a publishing state then the published data is updated).
\end{itemize}

\head{Runs.}
Fix a $\Sigma$-graph $G$ and $k$-robot ensemble $\overline{R}$.
For a pair of configurations $c,d$ and a non-empty set $K \subseteq [k]$ of \emph{active robots}, write \emph{$\onestep{c}{d}{K}$}
if the non-active robots do not change state or position, and each active robot nondeterministically takes a transition. Formally: if $i \not\in K$ then
$c(i) = d(i)$, and if $i \in K$ then there exists a transition $(st_i(c),\gc{\tau_i}{\kappa_i},st_i(d)) \in \delta_i$ such that
$Sat(c,\tau_i,i)$ (i.e., robot $i$ satisfies test $\tau_i$ at configuration $c$), and
$Mv(c,d,\kappa_i,i)$ (i.e., robot $i$ moves from $c$ to $d$ using command $\kappa_i$).


% Let $\alpha = c_0 c_1 c_2 \cdots c_n$ be a finite sequence of configurations.
% Let $St(\alpha)(i)$, resp. $Pos(\alpha)(i)$, be the last publishing state, resp. position, of robot $i$, assuming that
% the first publishing states and positions are given by $c_0$ and that the robots proceed as in $\alpha$.
% Formally, for each $i \in [k]$ let $f(i)$ be the largest index $m \in [0,n]$, if it exists, such that $st(c_m)(i) \in B_i$.
% Define
% \[
%  St(\alpha)(i) =
%       \begin{cases}
% 	    st(c_{f(i)})(i) & \mbox{ if $f(i)$ exists,}\\
%             st(c_0)(i) & \mbox{ otherwise.}
%       \end{cases}
% \]
% and similarly
% \[
%  Pos(\alpha)(i) =
%       \begin{cases}
% 	    pos(c_{f(i)})(i) & \mbox{ if $f(i)$ exists,}\\
%             pos(c_0)(i) & \mbox{ otherwise.}
%       \end{cases}
% \]


% If $\tau$ is a test, $i \leq [k]$ a robot, and $c_0 c_1 \cdots c_n$ a finite-sequence of configurations over graph $G$,
% say that \emph{$\tau$ is true for robot $i$ on $c_0 \cdots c_n$} if the following inductive definition holds:
% \begin{enumerate}
%   \- if $\tau = \psi_1 \wedge \psi_2$ then each $\psi_i$ is true for robot $i$ on $c_0 \cdots c_n$,
%   \- if $\tau = \neg \psi_1$ then it is not the case that $\psi_1$ is true for robot $i$ on $c_0 \cdots c_n$.
%  \- if $\tau(x_1,\cdots,x_k,z)$ is a position-test then $G \models \tau$ where the position variable $z$ is interpreted by $pos(c_n)(i)$ and
%   the position variables $x_j$ (for $j \leq k$) are interpreted by $Pos_{c_0,c_0c_1 \cdots c_n}(j)$ (i.e., robot $i$ tests its current position $z$ and the last published positions of all the robots $x_j$),
%   \- if $\tau$ is a state-test $st_j = q$ then $St_{c_0,c_0c_1 \cdots c_n}(j) = q$ (i.e., robot $i$ tests the last publishing state of robot $j$),
% \end{enumerate}




A \emph{partial run of $\tup{R}$ on $G$} is a finite sequence of the form $\rho = c_0 K_0 c_1 \dots K_{N-1} c_N$
where $\onestep{c_n}{c_{n+1}}{K_n}$ for all $n \in [0,N-1]$ (note that, in a partial run, $c_0$ need not be an initial configuration).
A \emph{run of $\tup{R}$ on $G$} is an infinite sequence $\pi = c_0 K_0 c_1 K_1 c_2 K_2 \cdots$ where $c_0$ is an initial configuration, and for each $n \geq 0$, $\onestep{c_n}{c_{n+1}}{K_n}$.  The set of all runs of $\tup{R}$ on $G$ is denoted $\runs(G,\tup{R})$.

\head{Definition of $\activeproj_i(\cdot)$.} We now define
$\activeproj_i(\rho)$, the projection of $\rho$ onto robot $i$, to be
the sequence of configurations of $\rho$ at the indices where robot $i$ is
active (if $\rho$ is a run and the robot $i$ is only active finitely often,
then repeat the last configuration forever).

Formally, for a partial run $\rho = c_0 K_0 c_1 K_1 \cdots c_{N-1} K_N c_N$ write $\activeproj_i(\rho)$ for the
sequence $c_{j_0} c_{j_1} c_{j_2} \cdots$ such that $j_0 = 0$
and for every $n\geq 1$, $j_n$ is the least index (if it exists) such that i) $j_n \in K_{n-1}$ and ii) $j_{n-1} < j_n$.
For a run $\pi$, define $\activeproj_i(\pi)$ as for partial runs, except that if
$\activeproj_i(\pi)$ is finite $c_{j_0} c_{j_1} c_{j_2} \cdots c_{j_m}$ (this happens if $i$ is only active finitely often), then redefine
$\activeproj_i(\pi)$ to be the infinite sequence $c_{j_0} c_{j_1} c_{j_2} \cdots c_{j_m} (c_{j_m})^\omega$.


\begin{remark}
Observe that by restricting the active robots, one can capture robots moving a) synchronously (i.e., at every step, every robot that has a true guard is active at that step), b) asynchronously (i.e., at most one robot is active at every step), and anything inbetween.
%Thus, even if the robots are deterministic and the graphs are deterministic, there may be more than one run (unless the robots are designed so that at every point in time at most one robot can take a transition, e.g., see the proofs in Section~\ref{sec:PVPundec}). \todo{only true for a single robot moving}
\end{remark}

\head{Deterministic Robots.}
Fix a $k$-robot ensemble $\tup{R}$. The robot $R_i = \tpl{Q_i,B_i,I_i,\delta_i}$ is \emph{deterministic} if for every $\Sigma$-graph $G$ and configuration $c$,
there is at most one transition of the form $(st_i(c),\gc{\tau}{\kappa},q) \in \delta_i$ such that $Sat(c,\tau,i)$ holds. Suppose also that $G$ is deterministic, i.e., for every $v \in V, \sigma \in \Sigma$ there exists at most one $v'$ such that $\lambda(v,v') = \sigma$. Then, the partial runs starting in a given configuration $c$ and in which only robot $i$ is active are linearly ordered under the prefix relation:

\begin{lemma} \label{lem:deterministic}
Fix a $k$-robot ensemble $\tup{R}$ such that robot $R_i$ is deterministic, and fix a deterministic $\Sigma$-graph $G$.
If
\[
\onestep{c_0}{c_1}{\{i\}}\onestep{}{c_2}{\{i\}} \cdots \onestep{c_{N-1}}{c_N}{\{i\}} \mbox{ and }
\onestep{d_0}{d_1}{\{i\}}\onestep{}{d_2}{\{i\}} \cdots \onestep{d_{M-1}}{d_M}{\{i\}}
\]
are two partial runs with $c_0 = d_0$, then
$c_j = d_j$ for all $j \leq \min(N,M)$.
\end{lemma}

 This property will be used in Section~\ref{sec:extensions} where we show how to formalise and model check
exploration tasks.

% \- {\em accepting} states $A_i \subseteq Q_i$, \todo{are these used?}
% \- {\em halting} states $H_i \subseteq Q_i$ each with the property that the only transition from $p \in H_i$ is $(p,\true,p)$ (thus, halting is modeled as staying in the same state forever).




\begin{example}
% \todo{Check example is correct.}
The robot in Figure~\ref{fig:dfs}, when started at the root of a binary tree (not necessarily a full binary tree), does a depth-first traversal of the tree (going left before going right) and stops at the root. The transitions use the following tests: $\haslc$ states that the current node has a left-child, $\hasrc$ states that the current node has a right-child, $\islc$ states that the current node is a left-child, $\isrc$ states that the current node is a right-child, $\isleaf$ states that the current node is a leaf. E.g., $\hasrc$ is the \msol-formula  $\exists y. \lambda(\VARpos_{cur},y) = rc$. Also, we write $move(\sigma)$ for the command to move along an edge labeled $\sigma \in \{up,lc,rc\}$. The transitions are as follows:
\begin{align*}
a &= \gc{\haslc}{move(lc)}\\
b &= \gc{\hasrc \wedge \neg \haslc}{move(rc)}\\
c &= \gc{\isleaf \wedge \islc}{move(up)}\\
d &= \gc{\isleaf \wedge \isrc}{move(up)}\\
e &= \gc{\hasrc}{move(rc)}\\
f &= \gc{\islc \wedge \neg \hasrc}{move(up)}\\
g &= \gc{\isrc \wedge \neg \hasrc}{move(up)}\\
h &= \gc{\isrc}{move(up)}\\
i &= \gc{\islc}{move(up)}
\end{align*}
The idea is that in state $A$ the robot has not explored its left or its right subtrees, in state $B$ the robot has explored its left subtree but not its right subtree, and in state $C$ the robot has explored both its left and its right subtrees.
% Tests are written $leaf?, lc?,rc?,root?$ (whose meanings are ``is the current node a leaf?'', ``left-child?'', ``right-child?'', ``the root?'', see Example~\ref{ex:formulas} for the formulas), and commands are written ${up}$, ${lc}$, ${rc}$ (whose meanings are ``move up to the parent'', ``move to the left-child'', ``move to the right-child''). To save space, we allow transitions to be labeled by sequences of tests and commands. Thus, e.g., an edge from $q$ to $q'$ labeled $up;lc?:up;rc$ represents three transitions: $(q,\gc{\true}{up},A)$, $(A,\gc{lc?}{up},B)$, and $(B,\gc{\true}{rc},q')$ where $A,B$ are fresh states.
\end{example}


\begin{figure}[htbp]
\centering
% \begin{tikzpicture}[->,>=stealth',shorten >=1pt,auto,node distance=3cm,thick, line width = 1pt]
%   \tikzstyle{every state}=[fill=black,draw=none,text=white,minimum size = 0.4cm]
% 
% {\node[state] (A){};}
%   
%  
%    {\node[state]         (B) [right of=A] {};}
%  
%   {\node[state]         (C) [below of=B] {};}
%   
%   \node[state]         (D) [right of=B] {};
% 
% {  \node[state,initial]         (R) [below of=A] {};}
% 
% 
%   \path (A) edge  [loop above]            node {$\neg leaf?$; ${lc}$} (A)
%             edge    node {$leaf?; {up}; {rc}$} (B)
%         (B) edge [bend left] node {$\neg leaf?$; ${lc}$} (A)
%             edge  node {$leaf?;{up}$} (C)
%         (C) edge [bend right, right] node {$rc?$} (D)
%             edge  [bend left] node {$lc?;{up}; {rc}$} (B)
%             edge [bend right,above] node {$root?$} (R)
%         (D) edge [loop above] node {${up}; rc?$} (D)
%             edge              node {${up};lc?;{up};{rc}$} (B)
%           (R) edge [loop right] node {$leaf?$} (R)
%           (R) edge node {$\neg leaf?; {lc}$} (A);
%             
% \end{tikzpicture}


\begin{tikzpicture}[->,>=stealth',shorten >=1pt,auto,node distance=3cm,thick, line width = 1pt]
  \tikzstyle{every state}=[fill=black,draw=none,text=white,minimum size = 0.4cm]

{\node[state,initial]	(A)			{A};}
  
{\node[state]		(B) [below left of=A] 	{B};}
  
{\node[state]		(C) [below right of=A] 	{C};}


\path	(A) edge  [loop above] 	node {a,b} (A)
            edge  [bend right]	node {c} (B)
	    edge  [bend left]	node {d} (C)

        (B) edge [loop below] 	node {f} (B)
	    edge 		node {g} (C)
	    edge [bend right]	node {e} (A)
	
	(C) edge [loop below] 	node {h} (C)
	    edge [bend left] 	node {i} (B)
% 	    edge [bend left]	node {f} (A)
	 ;            
\end{tikzpicture}

%
%\caption{asdfadfs}


\caption{Robot that does a depth-first traversal of binary trees.}
\label{fig:dfs}
%\hfill
%\begin{minipage}[b]{0.45\linewidth}
%    \centering
%    \input{butterfly}
%    \caption{}
%    \label{fig: butterfly}
%\end{minipage}
%
%\begin{minipage}[b]{0.45\linewidth}
%    \centering
%    \input{eye}
%    \caption{}
%    \label{fig: eye}
%\end{minipage}
\end{figure}

% \head{Scheduling}
% Informally, a schedule $\calS$ states which agents are active at every time step.
% We need not think of the schedule as something that a centralised agent produces, but rather as a mathematical way to talk about different timing assumptions, such as synchronous, asychronous, round-robin, etc. In this work we consider asynchronous scheduling, i.e., at every 	
%
% We define typical schedules $\calS$.
%
% Let $c_0, c_1, c_2 \dots$ be a run and let $K_0, K_1, K_2 \dots$ be the active robots at each time step.
%
% The run is \emph{synchronous} if $K_n = [k]$ for every $n$.
%
% The run is \emph{asynchronous} if $|K_n| = 1$ for every $n$.
%
% The run is \emph{fair} if for every $i \leq k$ there are infinitely many $n$ such that $i \in K_n$.
%
% The run is \emph{round-robin} if $K_n = \{n \mod k\}$ for every $n$. Such runs are also fair and asynchronous.
%
% The set of runs following a schedule $\calS$ is denoted $\runs_\calS(G,\tup{R})$.
%
% \todo{formalise the fact that these schedules are definable in MSO}

% \begin{example}
% \todo{give deployment robots example. publish when you move left.}
% \end{example}

\subsection{Robot Linear Temporal Logic (\RLTL) --- a logic of tasks} \label{sec:TL}
%\head{Robot Tasks.} \label{ex:tasks}
%
%: a {\em $k$-robot task}, or simply a {\em task}, $\T$, is a function that maps a graph $G$ to a set of sequences of positions of $G$, i.e.,  $\T(G) \subseteq (V^k)^\omega$. A robot-ensemble $\tup{R}$ {\em achieves} $\T$ on $G$ if for every run $\alpha$ of $\tup{R}$ on $G$ it holds that $\alpha' \in \T(G)$, where $\alpha'$ is the sequence of positions of the run $\alpha$.
%%\sr{should define tasks to be properties of runs? i.e., include states}
%
%\item A robot {\em explores and halts} if, no matter where it starts, it a) eventually halts, and b) visits every vertex of the graph at least once.
%
%\item A robot {\em explores and returns} if, no matter where it starts, it a) eventually halts where it started, and b) visits every vertex of the graph at least once.
%Formally, $v_1 v_2 \cdots \in T(G)$ if and only if there exists $i$ such that $\{v_1,\cdots,v_i\} = V$ and for all $j \geq i$, $v_i = v_j$.
%\sr{but the robot does not know it halts. tasks should also talk about the sequence of states, not just sequence of positions?}
%\item[RT3.] A robot {\em perpetually explores} if, no matter where it starts, it visits every vertex of $G$ infinitely often.
Robots should achieve some task in their environment.
We give some examples of foundational robot tasks \cite{KKR07handbook}:

\it
\- A robot ensemble {\em deploys} or {\em reconfigures} if they move, in a collision-free way, to a certain target configuration.

\- A robot ensemble {\em gathers} if, no matter where each robot starts, there is a vertex $z$, such that eventually every robot is in $z$.

\- A robot ensemble {\em collaboratively explores} a graph if, no matter where they start, every node is eventually visited by at least one robot.

\- All of these tasks have {\em safe} variations: the robots complete their task without entering certain pre-designated ``bad'' nodes of the graph.

\ti


%
%\item[RT4.] An ensemble of robots {\em catches} a single robot if no matter where they start, at some point in time one robot in the ensemble is in the same vertex as the single robot. Note that unlike the previous examples, this task is adversarial.
% A robot {\em explores} if, no matter where it starts, it eventually visits every vertex of the graph at least once.


%no space
%Note that there are some natural relations between these tasks. If robots can explore and return, then in particular they can explore and halt. If they can explore and halt, then they can perpetually explore. If they can perpetually explore then, using the fact that they have IDs, they can gather --- every robot searches for the robot with the smallest ID, which stays still (with two agents this is called ``wait for mommy'').


%
We now define \RLTL, a logic for formally expressing tasks from the robots' points of view. The idea is that \RLTL is \LTL in which the atoms are tests.

% The syntax is \LTL over the alphabet of tests.
% The models of \RLTL are sequences of configurations. The semantics $\models_i$ (for $i \leq k$) of \RLTL are given as in \LTL, where the atomic case is
% ``robot $i$ satisfies test $\tau$ at the current configuration''. Finally, we interpret \RLTL over models of the form $\activeproj_i(\pi)$ where $i \leq k$ and $\pi \in \runs(G,\tup{R})$.


{\bf Syntax.} Let $k \in \nat$ and $\Sigma$ be a set of edge-labels.
Formulas of $\RLTL_k(\Sigma)$ are those of $\LTL(\uptau_k(\Sigma))$ where $\uptau_k(\Sigma)$ is the set of tests for $k$ robots on $\Sigma$-labeled graphs.
If $k$ and $\Sigma$ are understood from the context, we write $\RLTL$ instead of $\RLTL_k(\Sigma)$.

{\bf Semantics.} Fix $i \leq k$. For a sequence of configurations $\alpha = c_0 c_1 \cdots$ and an $\RLTL_k(\Sigma)$ formula $\varphi$, write
$\alpha \models_i \varphi$ if $\alpha$ satisfies $\varphi$ as an \LTL formula whose atoms are tests interpreted wrt robot $i$.
Formally, for formula $\varphi$ and $n \geq 0$, define $(\alpha,n) \models_i \varphi$ inductively (all cases are like \LTL, the only difference is
the first item, i.e., the atomic case):
	\it
	\- $(\alpha,n) \models_i \tau$ iff $Sat(\alpha_n,\tau,i)$ (i.e., robot $i$ satisfies test $\tau$ in $\alpha_n$);
	\- $(\alpha,n) \models_i \varphi_1 \wedge \varphi_2$ iff $(\alpha,n) \models \varphi_j$ for $j = 1,2$;
	\-	$(\alpha,n) \models_i \neg \varphi$ iff it is not the case that $(\alpha,n) \models \varphi$;
	\-  $(\alpha,n) \models_i \nextX \varphi$ iff $(\alpha,n+1) \models \varphi$;
	\- $(\alpha,n) \models_i \varphi_1 \until \varphi_2$ iff there exists $j \geq n$ such that $(\alpha,j) \models_i \varphi_2$ and for all $l \leq j < n$, $(\alpha,l) \models_i \varphi_1$.
	\ti
	Write $\alpha \models_i \varphi$ if $(\alpha,0) \models_i \varphi$.

For a $\Sigma$-graph $G$ and a $k$-robot ensemble $\tup{R}$, write $(G,\tup{R}) \models_i \varphi$ if for all $\pi \in \runs(G,\tup{R})$,
it holds that $\activeproj_i(\pi) \models_i \varphi$. We say that \emph{Robot $i$ satisfies $\varphi$}.
Finally, given \RLTL formulas $\varphi_i$ for each robot $i \in [k]$, write $(G,\tup{R}) \models \tpl{\varphi_1,\cdots,\varphi_k}$ if $(G,\tup{R}) \models_i \varphi_i$ for all $i \in [k]$.



%To express that some run satisfies $\varphi$, we could write $(G,\tup{R}) \not \models \neg \varphi$.

%Thus, given a graph $G$ and $k$-ensemble of robots $\tup{R}$ where the $i$th robot $R_i$ has state set $Q_i$, initial-state set $I_i$,
%% repeating-state set $A$,
%and halting-state set $H_i$, accepting-state set $A_i$,

\begin{example}[avoid halting state]
The \RLTL safety formula $\always  (st_{cur} \not \in H)$ says that the last published state of the robot is never in the set $H$.
\end{example}

\begin{example}[gather]
The \RLTL formula 
\[ \eventually\left[ (\bigwedge_j \VARpos_{cur} = \VARbcpos_j) \wedge \always (\VARpos_{cur} = \nextX \VARpos_{cur})\right] 
\]
 states that eventually
the current position of the robot is equal to the last published position of all the robots (including itself), and that from that point on the robot never moves.
In words, every robot satisfies this formula if and only if the robots gather (and do not disperse) at a vertex of the graph at which they last published. %sent their last broadcast.
\end{example}
%The \RLTL safety formula $\always \textsc{diff}$ where $\textsc{diff} := \bigwedge_{i \neq j} x_i \neq x_j$ says that it is always the case that the published positions of different robots do not agree. Thus, if a robot broadcasts its position whenever it moves, then this formula says that the robots never collide.

\begin{example}[deploy to leaves] \label{ex:RLTL-deploy}
 Suppose $G$ is a tree and let $\isleaf(z)$ be the test that says that position $z$ is a leaf in the tree.
%  , e.g., for binary trees,
%  $\isleaf(z) = \neg \exists x.\lambda(z,x) = lc \vee \lambda(z,x) = rc$.
 Let $\mathit{diff}$ be the formula $\bigwedge_{i \neq j} pos_i \neq pos_j$.
The \RLTL formula $\mathit{diff} \limp\left[\eventually \isleaf(\VARpos_{cur}) \wedge \always(\bigwedge_{j\neq i} \VARpos_{cur} \neq pos_j)\right]$
states that if the robots begin in different vertices of the tree $G$, then
eventually the robot reaches a leaf, and it is never in the same position as a published position of another robot.
In words, every robot satisfies this formula if and only if, assuming robots begin at different vertices of the tree $G$, the robots deploy to the leaves while not colliding with the last published positions of any other robot.
%
% In words, every robot satisfies this formula if and only if, assuming robots begin at different vertices of the tree $G$, the robots deploy to the leaves while not colliding with any other robot (according to published positions).
%
\end{example}

In order to define collaborative exploration, we will need an extension of our logic to allow tests over the accumulated positions, see Section~\ref{sec:extensions}.


%\item The $\msol$-formula $\forall \tup{x} \exists \tup{y} \exists \tup{X}. \bigcup X_i = V \wedge \psi_\alpha(\tup{X},\tup{x},\tup{y})$ says that, no matter where they start, there is a run according to a schedule that follows $\alpha$ in which the robots collaboratively explore the graph.

%\item The $\msol$-formula (to be interpreted on trees)
%where $\textsc{nonleaf}(\tup{x})$ is an $\msol$-formula expressing that every $x_i$ is not a leaf, $\textsc{leaf}(\tup{y})$ expresses that every $y_i$ is a leaf, and $\textsc{diff}(\tup{z})$ expresses that $z_i \neq z_j$ for $i \neq j$. On trees this formula expresses that for every ordering $\alpha$ of processes that switch $N$-times, there is a run according to a schedule that follows $\alpha$ such that the robots reconfigure from different internal nodes to different internal leaves.\footnote{If one wants to add that the paths of the robots are collision-free one can replace $Reach$, for the case of two robots in which robot $1$ moves and then robot $2$ moves (for instance), by the formula
%$\exists \tup{Z} \exists \tup{Y} \exists \tup{z}\left[ \psi_1(\tup{Z},\tup{x},\tup{z}) \wedge \psi_2(\tup{Y},\tup{z},\tup{y})
%  \wedge Z_1 \cap Y_2 = \emptyset \right]
%$.}

% expresses, on trees, that no matter which internal nodes of the tree two robots start on, they have a collision-free path ($Z_1 \cap Y_2 = \emptyset$) in which they reach different leaves, according to a schedule in which robot $1$ is scheduled and then robot $2$ is scheduled. Similar more complex formulas can express the reconfiguration task of Example \ref{ex:recon} for (collaborative or adversarial) $b$-switching schedules.
%\item The atomic formula $Infty(X,x,y)$ expresses that the robot, starting in position $x$, visits position $y$ infinitely often, and the set of vertices the robot visits along this run is exactly $X$. Thus $\forall x Infty(V,x,x)$ is an \RLTL\ formula expressing that the robot ``perpetually explores'' the graph.

%\item The \RLTL\ formula $\exists \tup{x} \exists \tup{X} [Halt(\tup{X},\tup{x},\tup{x}) \wedge \cup_i X_i = V ]$, which talks about $k$ robots, expresses that each robot eventually returns and halts in its starting position, and every vertex of the graph is visited by at least one of the robots.


%\item The atomic formula $Infty(X,x,y)$ expresses that the robot, starting in position $x$, visits position $y$ infinitely often, and the set of vertices the robot visits along this run is exactly $X$. Thus $\forall x Infty(V,x,x)$ is an \RLTL\ formula expressing that the robot ``perpetually explores'' the graph.

%\item The \RLTL\ formula $\exists \tup{x} \exists \tup{X} [Halt(\tup{X},\tup{x},\tup{x}) \wedge \cup_i X_i = V ]$, which talks about $k$ robots, expresses that each robot eventually returns and halts in its starting position, and every vertex of the graph is visited by at least one of the robots.

%if $k=1$ and Lemma~\ref{lem:kcompile} if $k > 1$;
%$Q := \prod_{i \in [k]} Q_i$ is the set of tuples of states, $I := \prod_{i \in [k]} I_i$ is the set of tuples of initial states, $A := \prod_{i \in [k]} A_i$ are the repeating tuples, and $H := \prod_{i\leq k} H_i$ are the halting tuples.  We will often supress mention of $k$ and write, for instance, $Reach(\tup{X},\tup{x},\tup{y})$.

%an execution $\alpha \in (V^k \times \prod_{i \in [k]} Q_i)^*$ with $\alpha_1$ an initial configuration suppose $I_i \subseteq Q_i$ are the initial states, $A_i \subseteq Q_i$ are the accepting states, and $H_i \subseteq Q_i$ are the halting states:
%$(G,\tup{R}) \models Reach(\tup{x},\tup{y})$ iff there exists a  &\textrm{ iff } \alpha_1 = (\tup{q},\tup{x}) \textrm{ is initial, and } \exists i \in \nat\, \alpha_i = \tup{y}$


% such and every $k$-robot ensemble $\tup{R} = \tpl{R_1,\cdots,R_k}$, and all $k$-tuples of states $\tup{q},\tup{s} \in \prod_{i \in [k]} Q_i$, there is an atomic formula $\tup{R}_{\tup{p},\tup{q}}$  (of arity $2k$) where $\tup{R}_{\tup{p},\tup{q}}(\tup{x},\tup{y})$ expresses that there is an execution of the ensemble $\tup{R}$ starting with configuration $\tpl{\tup{x},\tup{p}}$ and containing\sr{ending?} the configuration $\tpl{\tup{y},\tup{q}}$.


%Formulas written in MSOL can quantify over sets (of vertices and edges). Thus they can express graph properties such as whether the graph is connected, whether it has  a $k$-colouring, whether it is planar, but not whether it is rigid.\sr{are these relevant properties?}

%A cornerstone of automata theory states that MSOL over finite/infinite words/trees coincides with automata operating on finite/infinite words/trees \cite{}.
%MSOL


\section{Parameterised Verification of Robot Systems}

In this short section, we formalise the parameterised verification problem (PVP) for robot protocols and provide the basic tools which allows one to prove undecidability of certain classes of PVP problems. We illustrate with a simple undecidability result that says that PVP is undecidable, even for a single robot, on grids. In the following two sections we provide deeper undecidability and decidability results.
%, show that it is undecidable already in some very restricted cases, and then describe a simple restriction on the orderings, namely bounded switching, that guarantees decidability, which we show by reducing to the validity problem of certain logics.
%
%\head{The Parameterised Verification Problem.}

% Fix a set of $\Sigma$-graphs $\gclass$.
% \begin{definition}
% The \textbf{parameterised verification problem} $\PVP(\gclass)$ is: given a number $k \in \nat$ of robots, a $k$-robot ensemble $\tup{R}$, and an \RLTL formula $\T$, decide whether for every graph $G \in \gclass$, $(G,\tup{R}) \models \T$.
% \end{definition}
\begin{definition}
Fix a set of $\Sigma$-graphs $\gclass$, positive integer $k$, a set  $\rclass$ of $k$-robot ensembles, and a set $\tclass$ of $\RLTL_k(\Sigma)$ formulas.
The \textbf{parameterised verification problem} $\PVP(\gclass,k,\rclass,\tclass)$ is the following decision problem: given a $k$-robot ensemble $\tup{R} \in \rclass$, and a $k$-tuple of \RLTL formulas $\tpl{\varphi_1,\cdots,\varphi_k}$ from $\tclass$, decide whether for every graph $G \in \gclass$ it holds that $(G,\tup{R}) \models \tpl{\varphi_1,\cdots,\varphi_k}$, i.e., whether for every $G \in \gclass$ every robot $i$ satisfies $\varphi_i$.
\end{definition}

% \emph{Notation.}
% \it
% \- In case $\tclass$ consists of a single formula $\T$ we write $\PVP(\gclass,k,\rclass,\T)$ instead.
%
% % \- We may write $\PVP$ instead of $\PVP(\gclass,k,\rclass,\tclass)$ to ease the reading.
% \ti


% \begin{example} \sr{ordering}
% Let $\gclass$ be the set of all binary trees, $\rclass$ be the set of all $k$-robot ensembles,  let $\Omega_b := \{\alpha \in [k]^* : ||\alpha|| = b\}$ be the set of {\em $b$-switch orderings}, and let $T$ be the task expressing that if the robots start on different internal nodes of a tree then they eventually reconfigure themselves to be on different leaves of the tree, no matter which ordering from $\Omega_b$ is chosen (cf. Example~\ref{ex:formulas}). We will see later that one can decide $\PVP_{\T,\Omega_b}(\gclass,\rclass)$ given $b \in \nat$. So, one can decide, given $b$, whether the protocol from the reconfiguration example (in the Introduction) succeeds for every ordering with $b$ switches.
% \end{example}

We now introduce two-counter machines, a Turing-complete model of computation that will be used in the undecidability proofs.




\subsection{Two-counter machines}
An \emph{input-free two-counter machine} (2CM)~\cite{Minsky67} is a deterministic program manipulating two non-negative integer counters, called Counter $1$ and Counter $2$, both initialised to zero, using commands that can increment a given counter by $1$, decrement a given counter by $1$ (if the counter is not zero), and branch depending on whether or not a given counter is equal to zero. We refer to the ``line numbers'' of the program code as the ``states'' of the machine.

\begin{definition}[2CM]
A 2CM is a tuple $M = (Q,q_0,h,C)$ where $Q = [m]$ is a finite set of numbered \emph{states} ($m \in \nat$), $q_0 \in Q$ is the \emph{initial state}, 
$h \in Q$ is the \emph{halting state}, and
$C$ is a function associating each state $q \in Q$ to a \emph{command} which is an expression of the form ``increment counter $i$ and goto state $q+1$'', ``decrement counter $i$ and goto state $q+1$'', and ``if counter $i$ is zero goto $q'$ else goto $q''$'' where $i = 1,2$ and $q',q'' \in L$.
\end{definition}

We assume that every command ``decrement counter $i$'' is guarded by a command that tests that counter $i$ is not zero.
A \emph{configuration of $M$} is a triple $(q,c_1,c_2) \in Q \times \nat \times \nat$. The \emph{initial configuration} of $M$ is $(q_0,0,0)$ and the \emph{halting configurations} of $M$ are $\{h\} \times \nat \times \nat$.  Given a configuration $(q,c_1,c_2)$ there is at most one \emph{successive configuration}, i.e., if the command at $q$ is ``increment counter $1$'' then the successive configuration is $(q+1,c_1+1,c_2)$, if it is ``decrement counter $1$'' then it is $(q+1,c_1-1,c_2)$, if it is ``if counter $1$ is zero goto $q'$ else goto $q''$`` then it is $(q',c_1,c_2)$ if $c_1 = 0$ and $(q'',c_1,c_2)$ otherwise, and similar definitions hold for counter $2$. Thus, there is a unique maximal (possibly finite) sequence that starts in the initial configuration and stops once, if at all, a halting configuration is reached. This is called the \emph{computation} of the 2CM. We say that the 2CM \emph{halts} if the computation is finite and ends in a halting configuration.
% Observe that if the computation is finite then the values of both counters are bounded by some integer $n$. 
The \emph{non-halting problem for 2CMs} is to decide, given a 2CM $\cm$, whether it does not halt. This problem is undecidable (cf. \cite{Minsky67}), and is usually a convenient choice for proving undecidability of problems concerning parameterised systems due to the simplicity of the operations of counter machines~\cite{Emerso03,AJKR14,DBLP:conf/cade/AminofR16}.

\subsection{Basic undecidability result}
To show undecidability we reduce the non-halting problem of two-counter machines to certain parameterised verification problems.

That is, given a 2CM machine $M$, we build robot(s) $\tup{R}$ and a tuple of formulas $\tpl{\varphi_1,\cdots,\varphi_k}$
such that the run of $M$ never reaches a halting state if and only if
$(G,\tup{R}) \models \tpl{\varphi_1,\cdots,\varphi_k}$ for all $G \in \gclass$. We illustrate with a basic undecidability result on grids that states that
PVP is undecidable already for a single robot on grids with simple testing abilities and a safety task.
The proof ideas were already observed in \cite{BlHe67} for a different setting.
% for ``2-dimensional automata''.


\begin{theorem}[Grid, boundary testing] \label{thm:undec-1robotgrid}
$\PVP(\gclass,k,\rclass,\tclass)$ is undecidable for the following data:
\it
\- $\gclass = \{G_n : n \in \nat\}$ is the set of grids (so $\Sigma = \{u,d,l,r\}$),
\- $k = 1$,
\- $\rclass$ is the set of $1$-robot ensembles, with no publishing states, that can test whether or not they are on a given boundary of the grid,
\- $\tclass$ consists of the single safety formula $\always (st_{cur}  \neq h)$.
\ti
\end{theorem}



% \todo{- add to statement of thm that robots don't need publishing states.}

% \todo{- fix initial states. either put them in the formula, or in the graph.}
\begin{proof}
 The idea is that counter values $(n,m) \in \nat^2$ are encoded by the robot being at position $(n,m)$ of the grid, and the only tests that are needed are to check whether or not the robot is on a given boundary. Indeed, given a 2CM $\cm$, the robot $R_\cm$ stores the current state of $\cm$. If the current state is $q$ and the current command is ``increment (resp. decrement) counter $1$'' then the robot tests if it is on the right (resp. left) boundary, and if not it moves one step to the right (resp. left) and updates its state to $q+1$, and if yes it enters a failure sink state $\bot$. ``Increment (resp. decrement) counter $2$'' is similar, and involves moving up (resp. down) and entering the failure sink $\bot$ if the tests fail. Finally ``test counter $1$ (resp. $2$) for zero'' is done by testing if the robot is on the left (resp. bottom) boundary. Before the robot starts the simulation, it moves to co-ordinate $(0,0)$ (which it can do since it can test for boundaries). Note that for every $n$ the set $\runs(G_n,R_\cm)$ contains a single run. Moreover, the run of $\cm$ never halts iff for every $n$ the run in $\runs(G_n,R_\cm)$ never reaches the halting state. \qed
\end{proof}

This theorem strongly suggests that one cannot get decidable PVP unless one limits at least one of the following: the testing abilities of the robots, the moving abilities of the robots, or the class of graphs. The above undecidability result for grids made use of only a single robot with very basic testing abilities (i.e., boundary detection). Thus, there is not much hope for an interesting model if we further restrict the testing abilities. Limiting the movement of robots on grids was studied in \cite{AAMAS16Grids}. There, one disallows robots from specifying their exact positions in the grid, e.g., a robot may decide to move to the left but not specify the exact number of steps. In the present work, instead, we consider robots with powerful remote-testing and the ability to specify their movements exactly. We restrict the set of graphs that are definable in monadic second-order logic and of bounded clique-width. We call these \courcellian sets of graphs. These do not include grids, but do include, e.g., lines, trees, and cliques, see section~\ref{sec:dec}.
% As we will see in Section~\ref{sec:DEC k=1} this restriction on the sets of graphs is enough to get decidable PVP for systems consisting of only a single robot.
However, this restriction on its own is not enough.
In the next section, we give undecidability results for multiple robots and possibly the simplest sets of \courcellian graphs, i.e.,
lines (thus indicating that further restricting the graphs will not yield interesting models).
The undecidability results imply that one has to impose limitations on their
``communications'' in order to regain decidability for multiple robots.
Specifically, the limitation on the robots state that there
is a bound on the number of times that a robot can test another robot's current
state and position, and a similar limitation must
be imposed on the specification logic.
%Interestingly enough, these restrictions are neccessary even if one limits robots to very basic tests (i.e., collision detection).

% \ba{Update rest of this section.}
%
%
% In the next section it is shown that the PVP is undecidable for two synchronous robots on a line, and very simple tasks. In light of this negative result, we explore in what ways we can restrict the robots to gain decidability. One possible direction is to limit the sensing/communication between the robots. Indeed, the above mentioned undecidability result assumes that each robot can sense/query whether the other robot is at the left-most position of the line. Another possibility is to consider the case of asynchronous robots. In Section \ref{sec:PVPundec} we show that the assumption of asynchronous robots alone does not guarantee decidability of the parameterised verification problem already in the very restricted case of reachability tasks on lines. Interestingly enough, our proof also applies \todo{check} to the synchronous case,  and assumes only very limited sensing capabilities, namely, that a robot can sense which of the other robots shares the same position with it (i.e., ``collision detection''). This fact strongly suggests that limiting the robots' sensing capabilities may not be a very fruitful direction. In Section \ref{sec:PVPdec} we show that for asynchronous robots with full testing abilities we can guarantee decidability by i) restricting to context-free environments, and ii) restricting to $B$-broadcast schedules. As we have just seen, dropping either of these two restrictions results in undecidability.
