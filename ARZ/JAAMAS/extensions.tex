\section{Extension allowing Exploration} \label{sec:extensions}

% \fz{-the definitions do not contain a precise definitions of the sets $POS_i$, which makes them a little unclear
%
% - In Lemma 6, I am missing cases for the updates of $POS_i$ (as in Lemma 2)
%
% - Lemma 8: in the definition of $\zeta^\subseteq$ it should be $closed_\zeta$ (instead of $closed_\phi$), right?
%
% - In don't understand the argument "For the other direction, suppose the MSOL formula (1) holds. ....". The argument seems to complicated.
% Isn't it simply that there is only one run in case of determinism?
%
% - the whole Section 8 does not contain an example, maybe we want to add one? }

% \fz{the whole Section 8 does not contain an example, maybe we want to add one?}

We now show how to extend our framework and decidability to handle specifications such as ``exploration with stop''. This is a useful extension for model checking, e.g., decide if a given robot does indeed know that it has (or has not) explored.  We do this by allowing robots to accumulate their visited states, to publish these sets, and to test these published sets.  Our proof makes use of the following additional assumptions:
\begin{itemize}
 \item All robots are deterministic,
 \item All graphs are deterministic,
 \item Neither robots nor specifications formulas can test a robot's current set of visited states, i.e., a robot must publish this set in order for it to be testable.
\end{itemize}

In what follows, we only give the ``delta'', i.e., the changes to be made to the previous sections in order to complete the definitions and the proofs for this new setting.

\head{Notation.}
In what follows we will use the following additional notation:
\begin{itemize}
 \item $X_i$ will denote the set of positions that robot $i$ has accumulated (i.e., visited since the beginning of the run),
 \item $Z_i$ will denote the set of published positions of robot $i$.
\end{itemize}

\head{Logic.} Let $\msol_{i,j}(\Sigma)$ denote $\msol$ formulas with $i$ free first-order variables and $j$ free set-variables.

\head{Tests.}
Redefine a {\em position-test} to be a formula $\tau$ from $\msol_{k+1,k}(\Sigma)$ whose free variables we denote
$\VARbcpos_1, \cdots \VARbcpos_k,\VARpos_{cur}$ and $\VARBCPOS_1, \cdots, \VARBCPOS_k$.
Let $\uptau^+_k(\Sigma)$ denote the set of tests using these new positions tests.
The intended meaning is that $\VARBCPOS_i$ is the last published set of positions that robot $i$ visited since the start of the run.

\head{Configurations.}
Redefine a {\em configuration} $c$ to be a function $c$ with domain $[k]$ such that for all $i \in [k]$,
\begin{enumerate}
 \item $c(i) \in V \times V \times Q_i \times Q_i \times 2^V \times 2^V$, we write
 \[
c(i) =  \tpl{pos_i(c), \bcpos_i(c), st_i(c), \bcst_i(c), POS_i(c),\bcPOS_i(c)},
 \]
 \item If $st_i(c) \in B_i$ then $st_i(c) = \bcst_i(c)$, $pos_i(c) = \bcpos_i(c)$, and $POS_i(c) = \bcPOS_i(c)$.
% i.e., if robot $i$ is in a publishing state then it publishes its position and state.
\end{enumerate}

Observe that the sets $POS_i(c)$ are not visible to the transition relation of the robots, nor to the specification formulas.
On the other hand, the sets $\bcPOS_i(c)$ can occur in tests, and thus used by the robots and mentioned in the specification formulas.

% The value of configuration $c$ for robot $i$ is:
% \[
% c(i) = \tpl{pos_i(c), \bcpos_i(c), st_i(c), \bcst_i(c), \bcPOS_i(c)}
% \]
% such that, for all $i$,

\head{Definition of $Sat(-,-,-)$.} In $Sat(c,\tau,i)$, change the clause:
\begin{itemize}
\item if $\tau$ is a position-test $\tau(pos_1, \cdots, pos_k,pos_{cur},POS_1, \cdots, POS_k)$ then we require that $G$ models the formula
\[\tau(\bcpos_i(c), \cdots, \bcpos_i(c),pos_i(c),\bcPOS_1(c), \cdots, \bcPOS_k(c)).
\]
% \item if $\tau$ is a state-test of the form ``$ST_j = Q$''then we require that $\bcST_j(c) = Q$.
\end{itemize}

\head{Definition of $Mv(-,-,-,-)$.} Add to $Mv(c,d,\kappa,i)$ the clauses:
\begin{itemize}
 \- $POS_i(d) = POS_i(c) \cup \{pos_i(d)\}$ (i.e., the current set of visited vertices $POS_i$ is updated to include the current position moved to),
 \- if $st_i(d) \not \in B_i$ then $\bcPOS_i(c) = \bcPOS_i(d)$  (i.e., if the target state is not a publishing state then the published data does not change),
 \- if $st_i(d) \in B_i$ then $\bcPOS_i(d) = POS_i(d)$ (i.e., if the target state is a publishing state then the published set of accumulated positions is updated).
\end{itemize}

\head{Runs.}
The definition of $\onestep{c}{d}{K}$, of partial run, and of run does not change (except for being based on the extended definitions of configurations and moves given above), and we will keep denoting the set of all runs of $\tup{R}$ on $G$ as $\runs(G,\tup{R})$.



\head{\RLTLV.} Define a logic \RLTLV to be like \RLTL except that it uses the new tests $\uptau^+_k(\Sigma)$.

\begin{example}[collaborative exploration]
The \RLTLV formula
\[
 \eventually \left[ \forall x. \bigvee_{i \in [k]} x \in \VARBCPOS_i \right]
\]
expresses that it is eventually the case that every position in the graph is in some robots' published accumulated set. 
\end{example}

\head{Lemmas} The statements of Lemmas~\ref{lem:istep}, \ref{lem:zeta}, \ref{lem:steps}, and \ref{lem:boundedly-many-publishing-points} are extended in the natural way to have additional free set variables $\tup{Z},\tup{Z}'$
corresponding to published sets of positions and set variables $\tup{X},\tup{X}'$ corresponding to the sets of accumulated states.

% \begin{lemma}[\ref{lem:hat-transforms}']
% \begin{enumerate}
%   \item For every test $\tau$ there exists an \msol formula $\hat{\tau}(\tup{z},x_i,\tup{Z})$ such that
%    $Sat(c,\tau,i)$ iff
%   \[
%    G \models \hat{\tau}(\tup{\bcpos_i(c)},pos_i(c),\tup{\bcPOS_i(c)}).
%   \]
%  \item For every command $\kappa$ there exists an \msol formula $\hat{\kappa}(x_i,x'_i,z_i,z'_i,Z_i,Z'_i)$ such that
%  $Mv(c,d,\kappa,i)$ iff
% \[
%  G \models \hat{\kappa}(pos_i(c),pos_i(d),\bcpos_i(c),\bcpos_i(d),\bcPOS_i(c),\bcPOS_i(d))
% \]
% \end{enumerate}
% \end{lemma}

The following is the extended version of Lemma~\ref{lem:istep}.
\begin{lemma}[extension: one step] \label{lem:ext:istep}
Fix a $k$-robot ensemble $\tup{R}$, tuples of states $\tup{p},\tup{p}' \in \prod_j Q_j$, tuples of publishing states
$\tup{b},\tup{b}' \in \prod_j B_j$, and a non-empty set of robot indices $K \subseteq [k]$.

One can build an \msol formula $\extfire(\tup{x},\tup{x}',\tup{z},\tup{z}',\tup{X},\tup{X}',\tup{Z},\tup{Z}')$ (that may depend on $\tup{R}, \tup{p}, \tup{p}', \tup{b}, \tup{b}', K$)
such that for all $\Sigma$-graphs $G$ and all valuations $\nu$ of the free variables:
we have that:
\[
 (G,\nu) \models \extfire \mbox{ iff there exists a transition } \onestep{c}{c'}{K}
\]
such that $c,c'$ satisfy, for all $j \in [k]$,
\begin{itemize}
 \item $c(j) = 	(\nu(x_j),	\nu(z_j),		p_j,		b_j,\nu(X_j),\nu(Z_j))$ and
 \item $c'(j) =	(\nu(x'_j),	\nu(z'_j),	p'_j,	b'_j,\nu(X'_j), \nu(Z'_j))$.
\end{itemize}
\end{lemma}

\begin{proof}
 The proof is a trivial extension of the proof of Lemma~\ref{lem:istep}, obtained by adding the requirement that the set variables $\tup{X},\tup{Z}$ are updated correctly to become $\tup{X}',\tup{Z}'$.
 \qed
\end{proof}
%  \todo{give short proof/explanation?}

% \begin{lemma}[\ref{lem:onestep}']
%  Fix a $k$-robot ensemble $\tup{R}$ and tuples of states $\tup{p},\tup{p}' \in \prod_i Q_i$ and tuples of publishing states
% $\tup{b},\tup{b'} \in \prod_i B_i$ such that $p_i \in B_i$ implies $b_i = p_i$, and $p'_i \in B_i$ implies $b'_i = p_i$.
% One can build an $\msol$ formula $\step$ (that depends on $\tup{R},\tup{p},\tup{p}',\tup{b},\tup{b}'$) with free variables
% $\tup{x}, \tup{x}', \tup{z},\tup{z}',\tup{Z},\tup{Z}'$, such that for every $\Sigma$-graph $G$,
% $G \models \step(\tup{x}, \tup{x}', \tup{z},\tup{z}',\tup{Z},\tup{Z}')$ iff there exists a non-empty set $K \subseteq [k]$ such that
% $\onestep{c}{c'}{K}$ is a partial run and, for all $i \in [k]$,
% $c_i = \tpl{x_i,z_i,p_i,b_i,Z_i}$ and $c'_i = \tpl{x'_i,z'_i,p'_i,b'_i,Z'_i}$.
%
% The formula $\step$ may be written $\step_{\tup{p},\tup{p}',\tup{b},\tup{b}'}$ or $\step^{\tup{R}}_{\tup{p},\tup{p}',\tup{b},\tup{b}'}$ to stress the parameters it depends on.
% \end{lemma}


The proof of the extended version of Lemma~\ref{lem:zeta} (Lemma~\ref{lem:ext:zeta} below)
is the one that contains the new ideas and and makes use of the deterministic restrictions stated at the beginning of this section.

We first need to define the Determinism Condition. Let $\Theta(y,y',Y,\tup{P})$ be an \msol formula where $\tup{P}$ is a tuple of first-order and second-order variables (the variables $\tup{P}$ will not play a major role in the next Proposition). A \emph{$\Theta$-path (of length $l \geq 0$) from $y_0$ to $y_l$ in a graph $G$} is a
sequence 
\[
y_0, Y_0 ,y_1, Y_1, \cdots, Y_{l-1}, y_l
\]
such that $G \models \Theta(y_i,y_{i+1},Y_i,\tup{P})$ for each $0 \leq i < l$. If $l= 1$ then we have a \emph{$\Theta$-edge}.
The set $\cup_{j < l} Y_j$ is called the set of nodes that the $\Theta$-path \emph{visits}.



\begin{definition}[Determinism Condition] \label{def:DC}
A formula $\Theta(y,y',Y',\tup{P})$ satisfies the Determinism Condition if for deterministic graphs $G$, 
all subsets $A,B$ of $G$, if there is a $\Theta$-path in $G$ from $y$ to $y'$ that visits $A$ and there exists is a $\Theta$-path in $G$ from $y$ to $y'$ that visits $B$, then either $A \subseteq B$ or $B \subseteq A$.
\end{definition}

The next proposition shows that if $\Theta$ satisfies the ``Determinism Condition''
one can define an \msol-formula $\Theta^=$ that captures, in deterministic graphs, that there exists a $\Theta$-path from $y$ to $y'$ that visits the set $Y'$.
% This will allow us in the proof of Lemma~\ref{lem:ext:zeta} to reason about paths that a robot takes that are concatenations of sub-paths each satisfying $\Theta$.


\begin{proposition} \label{prop:ext:TC}
Let $\Theta(y,y',Y',\tup{P})$ be an \msol formula satisfying the Determinism Condition.
Then there exists an \msol formula $\Theta^{=}(y,y',Y',\tup{P})$ that expresses, in all deterministic graphs, 
that there exists a $\Theta$-path from $y$ to $y'$ that visits the set $Y'$.
\end{proposition}

\begin{proof}
Let $\Theta(y,y',Y,\tup{P})$ be an \msol formula.

First, define the formula $\Theta^\subseteq(y,y',Y',\tup{P})$ that expresses that there exists a $\Theta$-path from $y$ to $y'$ that visits a (not necessarily proper) subset of $Y'$, i.e., $\cup_i Y_i \subseteq Y'$.  This is done by relativising the standard transitive-closure formula (from Proposition~\ref{prop:TC})
to only consider edges that stay inside $Y'$, i.e.,
\[
\Theta^\subseteq(y,y',Y',\tup{P}) :=
y \in Y' \wedge \forall W \subseteq Y' [(closed_{\Theta}(W,Y',\tup{P}) \wedge y \in W) \limp y' \in W]
\]
where $closed_{\Theta}(W,Y',\tup{P}) := \forall a \forall b [(a \in W \wedge \exists X \subseteq Y'. \Theta(a,b,X,\tup{P})) \limp b \in W]$.


% \sr{Can we show that $\Theta^{=}$ is not \msol-definable if all we know is that $\Theta$ is \msol?}

Second, we define $\Theta^=$ in terms of $\Theta^\subseteq$ and $\Theta$, under the assumption that $\Theta$ satisfies the ``Determinism Condition''.
Define $\Theta^=(y,y',Y',\tup{P})$ by the \msol-formula
\begin{eqnarray*} \label{eqn:visit}
 & y,y' \in Y' \wedge \\
 & \forall x \in Y'. \exists u,v \in Y'. \exists W \subseteq Y'.\\
 & x \in W \wedge \Theta^\subseteq(y,u,Y',\tup{P}) \wedge   \Theta(u,v,W,\tup{P}) \wedge   \Theta^\subseteq(v,y',Y',\tup{P}).
\end{eqnarray*}

In words, this says that for any intermediate vertex $x$, there is a $\Theta$-path (that stays in $Y'$) from the source $y$ to some vertex $u$, 
a $\Theta$-edge from $u$ to some vertex $v$ that visits some set $W \subseteq Y'$ containing $x$, and a $\Theta$-path from $v$ to the target $y'$ (that stays in $Y'$).

To see that our definition $\Theta^=$ is correct argue as follows.
Clearly if there is a $\Theta$-path from $y$ to $y'$ that visits $Y'$ then $\Theta^=(y,y',Y',\tup{P})$  holds.

For the other direction, suppose the formula $\Theta^=(y,y',Y',\tup{P})$ holds. This means that for every $x \in Y'$ there is a $\Theta$-path $P_x$ from $y$ to $y'$ that visits some set $Y_x$ with $x \in Y_x \subseteq Y'$, i.e., $\Theta^{\subseteq}(y,y',Y_x,\tup{P})$ holds. Since we assumed $\Theta$ satisfies the Determinism Condition, 
we have that for all $a,b \in Y'$, either $Y_a \subseteq Y_b$ or $Y_b \subseteq Y_a$. Therefore there exists a $\subseteq$-maximal element $Y_c = Y'$. Then $P_c$ is a $\Theta$-path from $y$ to $y'$ that visits $Y'$, as required.
This completes the proof of the proposition.
\qed
\end{proof}

We use this proposition in the proof of Lemma~\ref{lem:ext:zeta}, where we will instantiate $\Theta$ to be a formula
like $\xi$ from the proof of Lemma~\ref{lem:zeta} and use the fact that the robots and graphs are deterministic to deduce that the Determinism Condition holds.

% We now supply the extension of Lemma~\ref{lem:zeta}.
\begin{lemma}[extension: finitely-many non-publishing steps: single robot] \label{lem:ext:zeta}
Fix a deterministic $k$-robot ensemble $\tup{R}$, tuples of states $\tup{p},\tup{p}' \in \prod_j Q_j$, tuples of publishing states
$\tup{b},\tup{b}' \in \prod_j B_j$, and a robot index $i \in [k]$.

One can build an \msol formula $\xi(\tup{x},\tup{x}',\tup{z},\tup{z}',\tup{X},\tup{X}',\tup{Z},\tup{Z}')$ (that may depend on $\tup{R}, \tup{p}, \tup{p}', \tup{b}, \tup{b}', i$) such that for all deterministic $\Sigma$-graphs $G$ and all valuations $\nu$ of the free variables we have that:
\[
 (G,\nu) \models \xi \mbox{ iff there exists a partial run
$\onestep{c_0}{c_1}{\{i\}}\onestep{}{c_2}{\{i\}} \cdots \onestep{}{c_N}{\{i\}}$}
\]
for some $N \geq 0$ such that for all $j \in [k]$:
 \begin{enumerate}
 \item $c_0(j) = 	(\nu(x_j),	\nu(z_j),		p_j,		b_j,\nu(X_j),\nu(Z_j))$,
 \item $c_N(j) =	(\nu(x'_j),	\nu(z'_j),	p'_j,	b'_j, \nu(X'_j),\nu(Z'_j))$,
 \end{enumerate}
 and
 $st_i(c_n) \not \in B_i$ for $1 \leq n \leq N$ (which says that robot $i$ does not move into a publishing state anywhere along the run).

I.e., all robots $j \neq i$ are idle and robot $i$ can go in zero or more steps from state $p_i$ and vertex $x_i$ to state $p'_i$ and vertex $x'_i$ while not entering a publishing state, and accumulating states in $X'_j$.
% while resolving tests using states $\tup{b}$ and positions $\tup{z}$.
\end{lemma}


% \begin{lemma}[extension: finitely-many non-publishing steps: multiple robots] \label{lem:ext:steps}
% Fix a $k$-robot ensemble $\tup{R}$, tuples of states $\tup{p},\tup{p}' \in \prod_j Q_j$, and tuples of publishing states
% $\tup{b},\tup{b}' \in \prod_j B_j$.
%
% One can build an $\msol$ formula $\extsteps$ (that depends on $\tup{R},\tup{p},\tup{p}',\tup{b},\tup{b}'$) with free variables
% $\tup{x}, \tup{x}', \tup{z}, \tup{z}',\tup{Z},\tup{X}'$, such that
% for all $\Sigma$-graphs $G$ and all valuations $\nu$ of the free variables:
% \[
%  G \models \extsteps(\tup{x}, \tup{x}', \tup{z},\tup{z}',\tup{Z},\tup{X}') \mbox{ iff there is a partial run }
%  \onestep{c_0}{c_1}{K_0}\onestep{}{c_2}{K_1} \cdots \onestep{}{c_N}{K_{N-1}}
% \]
% for some $N \geq 0$ and some $K_n$ (for $0 \leq n \leq N-1$) such that for all $j \in [k]$:
%  \begin{enumerate}
%  \item $c_0(j) = 	(\nu(x_j),	\nu(z_j),		p_j,		b_j,\nu(Z_j))$ and $c_N(j) =	(\nu(x'_j),	\nu(z'_j),	p'_j,	b'_j,\nu(Z_j))$,
%  \item $st_j(c_n) \not \in B_i$ for $1 \leq n \leq N$ (which says that no robot moves into a publishing state anywhere along the run),
%  \item $X'_i = POS_i(c_j)$.
%  \end{enumerate}
% % The formula $\steps$ may be written $\steps_{\tup{p},\tup{p}',\tup{b},\tup{b}'}$ or $\steps^{\tup{R}}_{\tup{p},\tup{p}',\tup{b},\tup{b}'}$ to stress the parameters it depends on.
% \end{lemma}

% \begin{lemma}[\ref{lem:steps}']
% Fix a $k$-robot ensemble of \emph{deterministic robots} $\tup{R}$, tuples of states $\tup{p},\tup{q} \in \prod_i Q_i$, and
% publishing states $\tup{b} \in \prod_i B_i$ such that $p_i \in B_i$ implies $b_i = p_i$.
% One can build an $\msol$ formula $\steps$ (that depends on $\tup{R},\tup{p},\tup{p}'$ and $\tup{b}$) with free variables
% $\tup{x}, \tup{y}, \tup{z},\tup{Z},\tup{X}'$, such that
% for all {\em deterministic} $\Sigma$-graphs $G$: $G \models \steps(\tup{x}, \tup{x}', \tup{z},\tup{Z},\tup{X}')$ iff there exists $N \in \nat$ and a partial run
% $\onestep{c_0}{c_1}{K_0}\onestep{}{c_2}{K_1} \cdots \onestep{}{c_N}{K_{N-1}}$ with the following properties for all $i \in [k]$:
% \begin{enumerate}
%  \item $x_i = pos_i(c_0)$, $x'_i = pos_i(c_N)$,
%  \item $p_i = st_i(c_0)$, $p'_i = st_i(c_N)$,
%  \item $b_i = \bcst_i(c_0) = \dots = \bcst_i(c_N)$,
%  \item $z_i = \bcpos_i(c_0) = \dots = \bcpos_i(c_N)$,
%  \item $Z_i = \bcPOS_i(c_0) = \dots = \bcPOS_i(c_N)$, and
%  \item $X'_i = \bigcup_{0 \leq j \leq N} pos_i(c_j)$.
%  \end{enumerate}
% The formula $\steps$ may be written $\steps_{\tup{p},\tup{q},\tup{b}}$ or $\steps^{\tup{R}}_{\tup{p},\tup{q},\tup{b}}$ to stress the parameters it depends on.
% \end{lemma}

\begin{proof}
We only supply the difference from the Proof of Lemma~\ref{lem:zeta}.
The published sets $\tup{Z}$ are treated exactly like the published positions $\tup{z}$. To ease reading, write $\tup{w}$ for the tuple of variables
$(\tup{x},\tup{x}',\tup{z},\tup{z}',\tup{X},\tup{X}',\tup{Z},\tup{Z}')$.
% Moreover, for the rest of this proof we may not
% write $\tup{Z}$ or $\tup{z}$, in order to focus on the important differences.

% As before it is sufficient to construct $\psi_i(x_i,x'_i)$. To construct $\psi_i$
We construct $\xi^{s,t}(\tup{w})$ that holds on a graph $G$ iff
 robot $i$ has a run that goes from state $s$, vertex $x_i$ and accumulated states $X_i$ to state $t$, vertex $x_i'$ and accumulated states $X'_i$.
As before, we induct on $m$.
However, this time we use the \msol-formula from Proposition~\ref{prop:ext:TC} instead of the transitive-closure formula from Proposition~\ref{prop:TC}.
% define formulas $\xi_{m}^{s,t}(\tup{w})$ which express that robot $i$ has such an $m$-path whose vertices are \emph{exactly those of $X'_i$}.

Consider the case $m = 0$. This case is similar, except one must update the published sets. To do this we use $IDLE$ instead of $idle$, where $IDLE$ is
$idle \wedge (Z_i = Z'_i \wedge X_i = X'_i \wedge \bigwedge_{j \neq i} X_j = X'_j)$, and use $\extfire$ from Lemma~\ref{lem:ext:istep} instead of $\fire$.



% The required formula is the conjunction of the \msol-formulas
% $X' = \{x,x'\}$  and $(A \vee (s = t \wedge B))$ where $A,B$ are as before (with the aformentioned addition to handle $\tup{Z}$).


% %  \begin{eqnarray*}
% % A  := & \bigvee_{(s,\gc{\tau}{\kappa},t) \in \delta'_i} \hat{\tau}(\tup{z},x_i,\tup{Z}) \wedge \hat{\kappa}(x_i,x'_i,z_i,z'_i,Z_i,Z'_i)\\
% % B  := & x_i = x'_i \wedge z_i = z'_i \wedge p_i = p'_i \wedge b_i = b'_i \wedge Z_i = Z'_i\\
% % \end{eqnarray*}
% % where the hat-transformations are as in Lemma\ref{lem:hat-transforms}
% % (an empty disjunction evaluates to $\false$).


Now consider the case $m > 0$.  The overall principle is the same, i.e., define $\xi_m^{s,t}(\tup{w})$ to be
\[
\xi_{m-1}^{s,t}(\tup{w}) \vee
\left[\exists y,y', Y, Y',Y''. \, D \wedge E^{=} \wedge F \wedge (X_i \subseteq Y \subseteq Y' \subseteq Y'')\right]
% \left[\exists y,y', Y, Y',Y''. \, D \wedge E^{=} \wedge F \wedge (X' = Y \cup Y' \cup Y'')\right]
\]
where
% \begin{eqnarray*}
% D := & \xi_{m-1}^{s,q_m}(x,y,Y)\\
% E := & \xi_{m-1}^{q_m,q_m}(y,y',Y')\\
% F := & \xi_{m-1}^{q_m,t}(y',x',Y'').
% \end{eqnarray*}
%
\begin{eqnarray*}
D := & \xi_{m-1}^{s,q_m}(\tup{x},\tup{x}[i \leftarrow y],\tup{z},\tup{z}',\tup{X},\tup{X}[i \leftarrow Y],\tup{Z},\tup{Z}')\\
E := & \xi_{m-1}^{q_m,q_m}(\tup{x}[i \leftarrow y],\tup{x}[i \leftarrow y'],\tup{z},\tup{z}',\tup{X}[i \leftarrow Y],\tup{X}[i \leftarrow Y'],\tup{Z},\tup{Z}')\\
F := & \xi_{m-1}^{q_m,t}(\tup{x}[i \leftarrow y'],\tup{x}',\tup{z},\tup{z}',\tup{X}[i \leftarrow Y'],\tup{X}'[i \leftarrow Y''],\tup{Z},\tup{Z}'),
\end{eqnarray*}
where $E^=$ is defined as follows. Let $\Theta(y,y',A',\tup{P})$ be the \msol formula $E \wedge A' = Y'$ where
$\tup{P}$ collects all the free variables from $E$.
% \todo{the free variables are shuffled around. ok?}
By Lemma~\ref{lem:deterministic}, $\Theta$ satisfies the Determinism Condition (Definition~\ref{def:DC}). Thus,
by Proposition~\ref{prop:ext:TC}, there is a formula $\Theta^=(y,y',A',\tup{P})$ expressing that there is a $\Theta$-path from $y$ to $y'$ that visits $A'$. Since
$A'$ is the set of vertices visited on this $\Theta$-path we must union it with the set of vertices $Y'$ accumulated after the path corresponding to $D$.
Thus, define $E^=$ as the formula $\exists A'. \Theta^=(y,y',A',\tup{P}) \wedge Y' = Y \cup A'$.
\qed
\end{proof}

As mentioned earlier, Lemmas~\ref{lem:steps} and \ref{lem:boundedly-many-publishing-points} are also extended
by introducing starting and ending variables $\tup{Z},\tup{Z}'$ for the published sets,
and $\tup{X}, \tup{X}'$ for the accumulated sets. The statements and proofs of these extended versions are a straightforward adaptation obtained by using Lemmas~\ref{lem:ext:istep} and~\ref{lem:ext:zeta} instead of Lemmas~\ref{lem:istep} and~\ref{lem:zeta}, respectively; and doing the necessary bookkeeping concerning the new set variables.


%For convenience, we include below the statement of the extension of Lemma~\ref{lem:boundedly-many-publishing-points}.
%
%
%\begin{lemma}[extension: at most $E$ publishing points] \label{lem:ext:boundedly-many-publishing-points}
%Fix a $k$-robot ensemble $\tup{R}$, tuples of states $\tup{p},\tup{p}' \in \prod_i Q_i$ and tuples of publishing states
%$\tup{b},\tup{b}' \in \prod_i B_i$, and a bound $E \in \nat$.
%One can build an $\msol$ formula $\extpub$ (that depends on $\tup{R},\tup{p},\tup{p}',\tup{b},\tup{b}'$ and $E$) with free variables
%$\tup{x}, \tup{x}', \tup{z},\tup{z}',\tup{X},\tup{X}',\tup{Z},\tup{Z}'$, such that for every graph $G$ and every valuation $\nu$ of the free variables:
%\[
%(G,\nu) \models \extpub \mbox{ iff there exists a partial run $\onestep{c_0}{c_1}{K_0}\onestep{}{c_2}{K_1} \cdots \onestep{}{c_N}{K_{N-1}}$}
%\]
%for some $N \geq 0$, with at most $E$ publishing points, such that for all $j \in [k]$, $c_0(j) = 	(\nu(x_j),	\nu(z_j),		p_j,		b_j,\nu(X_j),\nu(Z_j))$
%and $c_N(j) =	(\nu(x'_j),	\nu(z'_j),	p'_j,	b'_j,\nu(X'_j),\nu(Z'_j))$.
%
%The formula $\pub$ may be written $\pub_{\tup{p},\tup{p}',\tup{b},\tup{b}',E}$ or $\pub^{\tup{R}}_{\tup{p},\tup{p}',\tup{b},\tup{b}',E}$ to stress the parameters it depends on.
%\end{lemma}
%
% \begin{lemma}[extension: at most $E$ publishing points]
% Fix a $k$-robot ensemble $\tup{R}$, tuples of states $\tup{p},\tup{p}' \in \prod_i Q_i$ and tuples of publishing states
% $\tup{b},\tup{b}' \in \prod_i B_i$, and a bound $E \in \nat$.
% One can build an $\msol$ formula $\pub$ (that depends on $\tup{R},\tup{p},\tup{p}',\tup{b},\tup{b}'$ and $E$) with free variables
% $\tup{x}, \tup{x}', \tup{z},\tup{z}',\tup{X}$, such that
% $G \models \pub(\tup{x}, \tup{x}', \tup{z},\tup{z}',\tup{X})$ iff there exists $N \in \nat$ and a finite partial run $\onestep{c_0}{c_1}{K_0}\onestep{}{c_2}{K_1} \cdots \onestep{}{c_N}{K_{N-1}}$ with at most $E$ publishing points with the following properties for all $i \leq k$:
% \begin{enumerate}
%  \item $x_i = pos_i(c_0)$, $x'_i = pos_i(c_N)$,
%  \item $p_i = st_i(c_0)$, $p'_i = st_i(c_N)$,
%  \item $b_i = \bcst_i(c_0)$, $b'_i = \bcst_i(c_N)$,
%  \item $z_i = \bcpos_i(c_0)$, $z'_i = \bcpos_i(c_N)$,
%  \item $Z_i = \bcPOS_i(c_0)$, $Z'_i = \bcPOS_i(c_N)$,
%  \item $X'_i = \bigcup_{0 \leq j \leq N} pos_i(c_j)$.
% \end{enumerate}
% \end{lemma}


Finally, the reduction of \RLTL with these new tests to \msol-satisfiability goes through, i.e., Theorem~\ref{thm:reduction} holds for the expanded framework, and the proof is again a straightforward adaptation. Thus:
\begin{theorem} \label{thm:PVPdec-exploration}
$\BPVP({\gclass,k,\rclass,\tclass})$ is decidable where $\gclass$ is a set of graphs with decidable \msol-satisfiability problem,
$k \in \nat$, $\rclass$ is the set of all $k$-robot ensembles with tests $\tau^+_k(\Sigma)$, and $\tclass$ is the set of all \RLTLV formulas.
\end{theorem}






