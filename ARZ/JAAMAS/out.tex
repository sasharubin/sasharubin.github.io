
\section{Decidability of Multi-Robot Systems} \label{sec:dec}
The previous section shows that decidability cannot be achieved if the robots have unlimited publishing abilities, even in very restricted situations, i.e., lines, local testing, and safety tasks. Thus, in order to achieve decidability we make the following restrictions:
\begin{enumerate}
 \item We restrict to classes of graphs $\gclass$ with decidable \msol-satisfiability (e.g., lines, trees, etc.).

 \item We consider runs in which the total number of times a robot broadcasts is bounded.
\end{enumerate}

We discuss each restriction in turn. 

\begin{definition} 
 Let $\gclass$ be a set of $\Sigma$-graphs. The {\em \msol-satisfiability problem of $\gclass$} is to decide, given an $\msol(\Sigma)$-sentence $\phi$, whether there exists $G \in \gclass$ such that $G \models \phi$. 
\end{definition}

Unfortunately, the $\msol$-satisfiability problem for the set $\gclass$ of all $\Sigma$-graphs is undecidable (this is already true for first-order logic~\cite{EbFl95}). However, there are classes of graphs for which it is decidable, notably the context-free sets of graphs. 
Context-free sets of graphs are the analogue of context-free sets of strings, and can be described by graph grammars, or equations using certain graph operations, or $\msol$-transductions of the set of trees. Perhaps the simplest definition is this: $\gclass$ is {\em context-free} if it is $\msol$-definable and of bounded clique-width \cite{CE12}. Examples include the set of lines, rings, $\Delta$-ary trees, series-parallel graphs, cliques, but not the set of  grids. 
For an elaboration on the definition and properties of context-free sets of graphs the reader may consult \cite{CE12}. We only need the following fundamental theorem:

\begin{theorem}[Courcelle \cite{CE12}] \label{thm:courcelle}
If $\gclass$ is a context-free set of graphs, then the \msol-satisfiability problem of $\gclass$ is decidable.
\end{theorem}


We now turn to the second restriction and refine the definition of PVP to only consider runs in which there are at most $B$ publishing points.
A \emph{publishing point} in a run or partial run $\trans{c_0}{c_1}{K_0} \trans{}{c_2}{K_1} \cdots$ 
is an index $n \in \nat$ such that there exists $i \in K_n$ with 
$st_i(c_{n+1}) \in B_i$.

\begin{definition}
Fix a set of $\Sigma$-graphs $\gclass$, positive integer $k$, a set  $\rclass$ of $k$-robot ensembles, and a set $\tclass$ of $\RLTL_k(\Sigma)$ formulas. 
The \textbf{bounded parameterised verification problem} $\BPVP(\gclass,k,\rclass,\tclass)$ is the following decision problem:
given a $k$-robot ensemble $\tup{R} \in \rclass$, a $k$-tuple of \RLTL formulas $\tpl{\varphi_1,\cdots,\varphi_k}$ from $\tclass$, and a bound $B \in \nat$, decide whether for every graph $G \in \gclass$ it holds that every run $\pi$ of $\runs(G,\tup{R})$ in which there are at most $B$ publishing points satisfies that
$proj_i(\pi) \models_i \varphi_i$ for every $i \leq k$.
\end{definition}

\todo{question: is BPVP on grids decidable? it is not hard to see that msol-satisfiability on grids is undecidable.}
%
%\subsection{Reducing parameterised verification to logical validity} \label{subsec:reduc}
%As expected, there is a tradeoff between the various modeling choices.
%This will involve restricting the combination of $k, \T,\gclass$ and (the testing abilities of robots in) $\rclass$.

The following theorem, which we prove in this section, is the main technical result of this section.
 \begin{theorem}[Reduction] \label{thm:reduction} 
  For every $k$-robot ensemble $\tup{R}$, robot index $i \in [k]$, bound $B \in \nat$ and $\RLTL$ formula $\varphi$, there is an \msol-sentence $\psi$ such that for every graph $G$: 
  $G \models \psi$ iff there is a run $\pi \in \runs(G,\tup{R})$ in which there are at most $B$ publishing points satisfying $proj_i(\pi) \models_i \varphi$. 
  Moreover, the sentence $\psi$ can be built effectively from $\tup{R},i,B$ and $\varphi$.
 \end{theorem}

Combining Theorem~\ref{thm:courcelle} and Theorem~\ref{thm:reduction} we get the main result of this section:
\begin{theorem} \label{thm:PVPdec}
$\BPVP({\gclass,k,\rclass,\tclass})$ is decidable where $\gclass$ is a context-free set of graphs, $k \in \nat$, $\rclass$ is the set of all $k$-robot ensembles, and $\tclass$ is the set of all \RLTL formulas.
\end{theorem}
%
\begin{proof}
The algorithm proceeds as follows. Given $B \in \nat,\tup{R} \in \rclass$ and $\tpl{\varphi_1,\cdots,\varphi_k}$ from $\Phi$, build, for each $i$, the sentence $\psi_i$ from Theorem~\ref{thm:reduction} applied to $\neg \varphi_i$. Apply the decision procedure from Theorem~\ref{thm:courcelle} to decide whether or not there exists $G \in \gclass$ such that 
$G \models \bigvee_{i \leq k} \psi_i$. The answer to this problem is ``no'' iff for every $G \in \gclass$, for every $i \leq k$, 
$G \not \models \psi_i$,  iff (by Theorem~\ref{thm:reduction}) for every $G \in \gclass$, for every $i \leq k$, there is no run $\pi \in \runs(G,\tup{R})$ with at most $B$ publishing points satisfying $proj_i(\pi) \models_i \neg \varphi_i$, i.e., 
iff for every $G \in \gclass$ and $i \in [k]$, $(G,\tup{R}) \models_i \varphi_i$.
\qed
%Indeed, he answer to this problem is ``no'' iff for all $G \in \gclass$ and every run $\pi \in \runs(G,\tup{R}$) in which there are at most $B$ broadcasts, we have that $proj_i(\pi) \models \psi_i$ for each $i \in [k]$. \qed
\end{proof}

\begin{remark}
 Actually more is true. There is a single algorithm that given  a description of a context-free set $\gclass$ of $\Sigma$-graphs (by, e.g.,  a grammar, see Section~\ref{subsec:graphs-MSOL}), a number of robots $k \in \nat$, returns an algorithm for solving 
 $\BPVP({\gclass,k,\rclass,\tclass})$. This holds because the $\msol$-satisfiability problem for context-free sets of graphs $\gclass$ is \emph{uniformly} decidable, 
 i.e., there is an algorithm that given a description of a context-free set of graphs $\gclass$ and an $\msol$-sentence $\phi$ decides if every graph in $\gclass$ satisfies $\phi$ \cite{CE12}. \todo{say what sort of description}
\end{remark}

In case $k = 1$ $\BPVP$ is the same as $\PVP$. Thus:
\begin{corollary} \label{cor:k=1}
 $\PVP({\gclass,1,\rclass,\tclass})$ is decidable where $\gclass$ is a context-free set of graphs, $\rclass$ is the set of all $k$-robot ensembles, and 
 $\tclass$ is the set of all \RLTL formulas.
\end{corollary}


The proof of Theorem~\ref{thm:reduction} will occupy the rest of this section. The idea is to build an \msol-formula that states that
there is a partition of the run into at most $B$ segments such that no robot publishes in a segment except possibly at the start of the segment. 
Inside each segment, i.e., from the second configuration to the last configuration, we can treat the robots as operating independently of each other (indeed, the value of a test inside a segment is the same as the value at the second configuration of the segment). 

We begin with some automata-theory preliminaries, and then provide \msol-formulas over graphs that will be used as building blocks in the reduction.

\subsection{Automata preliminaries}

An {\em nondeterministic finite word automaton (NFW)} is a tuple $M = \tpl{A,Q,I,\Delta,F}$ where $A$ is a finite alphabet, $Q$ a finite set of states, $\delta \subseteq Q \times A \times Q$ a transition relation, $I \subseteq Q$ the initial states and $F \subseteq Q$ the accepting states. A run of $M$ on sequence $a_0 a_1 \cdots a_N$ is a sequence of states $q_0 q_1 \cdots q_{N+1}$ such that $q_0 \in I$ and $\delta(q_i,a_i,q_{i+1})$ for all $i \leq N$. The run is successful if $q_{N+1} \in F$. The set of sequences in $A^*$ that have successful runs is called the language of $M$.

{\em Regular-expressions} over a finite alphabet $A$ are built from the 
sets $\emptyset$, $\{\epsilon\}$, and $\{a\}$ ($a \in A$), and the operations 
union $+$, concatenation $\cdot$, and Kleene-star $\phantom{}^*$.

Kleene's Theorem says that for every $L \subseteq A^*$, $L$ is definable by a regular expression if and only if $L$ is the language of an NFW.
Moreover, there is an effective transformation between these two formalisms.

A \emph{nondeterministic B\"uchi word automaton (NBW)} is a tuple $N = (A,Q,I,\Delta,F)$, just as for finite automata. However, $N$ takes infinite sequences over alphabet $A$ as input. A run of $N$ on an infinite sequence $a_0 a_1 \cdots \in A^\omega$ is an infinite sequence $q_0 q_1 \cdots$ such that $\Delta(q_i,a_i,q_{i+1})$ for all $i \in \nat$. A run is successful if, for some $f \in F$, there are infinitely many $n \in \nat$ such that $q_n = f$. The set of  sequences in $A^\omega$ that have successful runs is the language of $N$.


\begin{theorem}[\cite{jjj}] \label{thm:LTL to NBW}
 For every \LTL formula $\varphi$ over $\AP$ there is an NBW $N_\varphi$ over alphabet $2^{\AP}$ whose language is 
 $\{\alpha \in (2^{\AP})^\omega : \alpha \models_\LTL \varphi\}$.
\end{theorem}

\subsection{\msol Building Blocks}


The {\em $k$-ary transitive-closure operator} is the function that maps a $2k$-ary relation $\phi(\tup{x},\tup{y})$ to the $2k$-ary relation $\phi^*(\tup{x},\tup{y})$ such that, in every graph $G$: $\phi^*(\tup{x},\tup{y})$ holds iff there exists a sequence $\tup{v}_1, \tup{v}_2, \cdots, \tup{v}_m$ such that $\tup{x} = \tup{v}_1$, $\tup{y} = \tup{v}_m$, and $\phi(\tup{v}_i,\tup{v}_{i+1})$ holds for every $i < m$. 
 
 $\msol$ can express the $1$-ary transitive closure operator. That is, if $\phi(x,y)$ is an $\msol_2(\Sigma)$ formula then $\phi^*(x,y)$ can also be expressed in $\msol_2(\Sigma)$. To see this define $\phi^*(x,y) := \forall Z [(closed_{\phi}(Z) \wedge x \in Z) \implies y \in Z]$ where $closed_{\phi}(Z)$ is defined as $\forall a \forall b[(a \in Z \wedge \phi(a,b)) \implies b \in Z]$. 
 
\begin{remark}
 
 For $k > 1$, it is not always the case that if a $k$-ary relation $\phi$ is $\msol$-definable then so is its $k$-ary transitive-closure because, intuitively, this would require having $k$-ary relation variables $Z$. To prove this formally, note that $2$-ary transitive closure on finite words (i.e., labeled lines) can define the non-regular language $\{0^n1^n : n \in \nat\}$; now use the fact that, over finite words, $\msol$ can only define the regular languages, part of a result known as the B\"uchi-Elgot-Trakhtenbrot Theorem, see \cite{Thomas96}.
\end{remark}
 

The next lemma says that there is a formula describing one time-step of an ensemble of robots with starting positions $\tup{x}$, ending positions $\tup{x}'$, starting states $\tup{p}$, ending states $\tup{p}'$, previously published positions $\tup{z}$ and states $\tup{b}$ and subsequently published positions $\tup{z}'$ and states $\tup{b}'$.

\begin{lemma}[one step] \label{lem:onestep}
Fix a $k$-robot ensemble $\tup{R}$. For all tuples of states $\tup{p},\tup{p}' \in \prod_i Q_i$ and tuples of publishing states 
$\tup{b},\tup{b'} \in \prod_i B_i$
one can build an $\msol$ formula $\step$ (that depends on $\tup{R},\tup{p},\tup{p}',\tup{b},\tup{b}'$) with free variables 
$\tup{x}, \tup{x}', \tup{z},\tup{z}'$, such that for every $\Sigma$-graph $G$, 
$G \models \step(\tup{x}, \tup{x}', \tup{z},\tup{z}')$ iff there is a partial run $\trans{c}{c'}{K}$ such that
for all $i \in [k]$: $c_i = \tpl{x_i,z_i,p_i,b_i}$ and $c'_i = \tpl{x'_i,z'_i,p'_i,b'_i}$.

The formula $\step$ may be written $\step_{\tup{p},\tup{p}',\tup{b},\tup{b}'}$ or $\step^{\tup{R}}_{\tup{p},\tup{p}',\tup{b},\tup{b}'}$ to stress the parameters it depends on.
\end{lemma}

\begin{proof}
 Let $\step$ be
 $
  \bigwedge_i \bigvee_{\gc{\tau_i}{\kappa_i}} (\hat{\tau_i} \wedge \hat{\kappa_i}) 
 $
where the conjunction is over $i \in [k]$, the disjunction is over $\gc{\tau_i}{\kappa_i}$ for which
$(p_i,\gc{\tau_i}{\kappa_i},p'_i) \in \delta_i$ and 
$b'_i = \begin{cases}
         p'_i & \mbox{ if } p'_i \in B_i,\\
         b_i & \mbox{ otherwise,}
        \end{cases}
$ 

\todo{Pull hat-transform into separate lemma? I think this would be a bit messy...}
and the hat-transformation of tests is as follows: 
\begin{itemize}
 \item if $\tau_i = \neg \phi$ then $\hat{\tau_i} := \neg \hat{\phi}$;
 \item if $\tau_i = \phi \wedge \psi$ then $\hat{\tau_i} := \hat{\phi} \wedge \hat{\psi}$;
 \item if $\tau_i(pos_1,\cdots,pos_k,cur)$ is a position-test then $\hat{\tau_i} := \tau(z_1,\cdots,z_k,x_i)$;
 \item if $\tau_i$ is a state test, say $st_j = l$, then 
 $\hat{\tau_i} := \begin{cases}
                 \true & \mbox{ if } b_j = l,\\
                 \false & \mbox{ otherwise.}
                \end{cases}
 $
  \end{itemize}              
In other words, the since tests are Boolean combination of position-tests and state-tests, the hat-transformation just needs to evaluate the position-tests using the relevant variables, and evaluate the state-tests using the given parameters. Similarly, 

Similarly, the hat-transformation of commands is as follows: 
\begin{itemize}
  \item if $\kappa_i = move(\sigma)$ then $\hat{\kappa_i} := \lambda(x_i,x'_i) = \sigma$;
  \item if $\kappa_i = stay$ then $\hat{\kappa_i} := x_i = x'_i$.
\end{itemize}
In other words, the transformation expresses in \msol, the movement of robot $i$ using the relevant variables ($x_i$ is the position before the transition is taken and $x'_i$ afterwards). \qed
\end{proof}


Iterating the previous lemma yields any fixed number of steps. However, the next lemma says that there is an \msol-formula that expresses that there is a finite number of steps of a robot ensemble (i.e., no apriori bound on the number of steps) as long as all tests are evaluated wrt \emph{fixed} states $\tup{b}$ (treated as parameters) and positions $\tup{z}$ (treated as free variables).


%It says that, given a $k$-robot ensemble, one can build an $\msol$ formula that expresses that robot $i \in [k]$ can move from position $x_i$ and state $p_i$ to position $y_i$ and state $q_i$ while visiting exactly the vertices $X_i$, assuming that $\tup{z}$ and $\tup{b}$ are the last broadcast positions and states of the robots, and that \emph{no new broadcasts appear in the run}. 

\begin{lemma}[finitely-many steps] \label{lem:steps}
Fix a $k$-robot ensemble $\tup{R}$. For all tuples of states $\tup{p},\tup{q} \in \prod_i Q_i$ and published states $\tup{b} \in \prod_i B_i$,
one can build an $\msol$ formula $\steps$ (that depends on $\tup{R},\tup{p},\tup{p}'$ and $\tup{b}$) with free variables 
$\tup{x}, \tup{y}, \tup{z}$, such that 
for all $\Sigma$-graphs $G$, $G \models \steps(\tup{x}, \tup{x}', \tup{z})$ iff there exists $N \in \nat$ and a partial run  
$\trans{c_0}{c_1}{K_0}\trans{}{c_2}{K_1} \cdots \trans{}{c_N}{K_{N-1}}$ with the following properties for all $i \in [k]$:
\begin{enumerate}
 \item $x_i = pos_i(c_0)$, $x'_i = pos_i(c_N)$,
 \item $p_i = st_i(c_0)$, $p'_i = st_i(c_N)$,
 \item $b_i = \bcst_i(c_0) = \dots = \bcst_i(c_N)$,
 \item $z_i = \bcpos_i(c_0) = \dots = \bcpos_i(c_N)$.
 \end{enumerate}
The formula $\steps$ may be written $\steps_{\tup{p},\tup{q},\tup{b}}$ or $\steps^{\tup{R}}_{\tup{p},\tup{q},\tup{b}}$ to stress the parameters it depends on.
\end{lemma}

\begin{proof}
Consider a partial run $\rho = c_0K_0 \cdots K_{N-1}c_N$ satisfying the listed properties. Recall from the preliminaries that the definition of a robot assumes there are no self-loops on publishing states.
Thus no active robot takes a transition whose target state is a publishing state in $\rho$, and so each test is evaluated on the last published 
positions $\tup{z}$ and states $\tup{b}$ which do not change throughout the partial run. Thus, such a partial run $\rho$ exists iff each robot $i$, independently, has a partial run that goes from state $p_i$ and position $x_i$ to state $p'_i$ and position $x'_i$, and resolves tests using the last published positions $\tup{z}$ and states $\tup{b}$.  
Thus, it is enough to find, for each $i \in [k]$, a formula $\psi_{i}(x_i,y_i,\tup{z})$, that depends on $R_i,p_i,p'_i,\tup{b}$, satisfying the listed conditions with the extra condition that only robot $i$ is scheduled, i.e., $K_l = \{i\}$ for all $l \in [0,N-1]$. Then, the required formula is simply 
\[ 
\steps := \bigwedge_{i \in [k]} \psi_i(x_i,y_i,\tup{z}).
\]

It remains to show how to construct $\psi_i$. \todo{Do we need to restrict to non-broadcast states of the robot?}
The $i$th robot $\tpl{Q_i,B_i,I_i,\delta_i}$ 
can be considered a finite automaton $\tpl{Alpha_i,Q_i,\{p_i\},\delta_i,\{p'_i\}}$ with initial state $p_i$ and final state 
$p'_i$ over a finite alphabet $Alph_i$ of guarded commands. By Kleene's theorem we can build a regular expression $exp$ over alphabet $Alph_i$ for the language of this automaton. We now show how to translate the regular expression into the required \msol formula $\psi_i(x_i,y_i,\tup{z})$.
Proceed by induction on the structure of the regular expression $exp$.

\begin{itemize}

\- Suppose $exp = \emptyset$. Then $\psi_i := \false$.

\- Suppose $exp = \epsilon$. Then $\psi_i := x_i = y_i$,

\- Suppose $exp$ is the alphabet symbol $\gc{\tau_i}{\kappa_i}$. Then $\psi_i := \hat{\tau_i} \wedge \hat{\kappa_i}$ where the hat-transformations are
as in the proof of Lemma~\ref{lem:onestep}.
                


\- Suppose $exp = exp' +  exp''$, $\psi'_i$ is the translation of $exp'$, and $\psi''_i$ is the translation of $exp''$. Then 
$\psi_i := \psi'_i \vee \psi''_i$.

\- Suppose $exp = exp' \cdot exp''$, $\psi'_i$ is the translation of $exp'$, and $\psi''_i$ is the translation of $exp''$. Then 
$
\psi_i := \exists w \left[\psi'_i({x_i},{w},\tup{z}) \wedge \psi''_i({w},{x'_i},\tup{z})\right].
$

 \- Suppose $exp = (exp')^*$ and $\psi'_i$ is the translation of $exp'$. Then 
\[
\psi_i := \forall {Z} [(cl({Z},\tup{z}) \wedge {x_i} \in {Z}) \to {x'_i} \in {Z}]
\]
where $cl({Z},\tup{z})$ is defined as $\forall {a},{b} \left[({a} \in {Z} \wedge \psi'_i({a},{b},\tup{z})) \to {b} \in {Z}\right]$ and means that $Z$ is closed under $\psi'_i$. 
\end{itemize}

That the \msol-formula $\psi_i$ has the desired properties follows from the definitions. E.g., if $exp = \emptyset$ then there is no partial run satisfying
the listed properties, and thus $\psi_i := \false$; if $exp = exp' \cdot exp''$ then there every partial run satisfying the listed properties can be split into two partial runs, etc.
\qed
\end{proof}


\begin{lemma}[at most B-many publishing points]
Fix a $k$-robot ensemble $\tup{R}$, tuples of states $\tup{p},\tup{p}' \in \prod_i Q_i$ and tuples of publishing states 
$\tup{b},\tup{b}' \in \prod_i B_i$, and a bound $B \in \nat$.
One can build an $\msol$ formula $\pub$ (that depends on $\tup{R},\tup{p},\tup{q},\tup{b},\tup{b}'$ and $B$) with free variables 
$\tup{x}, \tup{x}', \tup{z},\tup{z}'$, such that 
$G \models \pub(\tup{x}, \tup{x}', \tup{z},\tup{z}')$ iff there exists $N \in \nat$ and a finite partial run $\trans{c_0}{c_1}{K_0}\trans{}{c_2}{K_1} \cdots \trans{}{c_N}{K_{N-1}}$ with at most $B$ publishing points with the following properties for all $i \leq k$:
\begin{enumerate}
 \item $x_i = pos_i(c_0)$, $x'_i = pos_i(c_N)$,
 \item $p_i = st_i(c_0)$, $p'_i = st_i(c_N)$,
 \item $b_i = \bcst_i(c_0)$, $b'_i = \bcst_i(c_N)$,
 \item $z_i = \bcpos_i(c_0)$, $z'_i = \bcpos_i(c_N)$.
 \end{enumerate}
The formula $\pub$ may be written $\pub_{\tup{p},\tup{p}',\tup{b},\tup{b}',B}$ or $\pub^{\tup{R}}_{\tup{p},\tup{p}',\tup{b},\tup{b}',B}$ to stress the parameters it depends on.

\end{lemma}

\begin{proof}
 Define \msol-formulas $\phi_{\tup{p},\tup{p}',\tup{b},\tup{b'}}^{n}$ (for $n \in \nat$) with free variables $\tup{x},\tup{x}',\tup{z},\tup{z}'$ inductively as follows:
\begin{itemize}
 \item If $n = 0$ then $\phi_{\tup{p},\tup{p}',\tup{b},\tup{b}'}^{n}(\tup{x},\tup{x}',\tup{z},\tup{z}')$ is defined to be 
 \[
  \begin{cases}
   \false & \mbox{ if  $\tup{b} \neq \tup{b}'$ or $\tup{z} \neq \tup{z}'$},\\
   \steps_{\tup{p},\tup{p}',\tup{b}}(\tup{x},\tup{x}',\tup{z}) & \mbox{ otherwise.}
  \end{cases}
 \]
 \item If $n > 0$ then $\phi_{\tup{p},\tup{p}',\tup{b},\tup{b}'}^{n}$ is defined to be the disjunction, over $\tup{q},\tup{q}' \in \prod_i Q_i, 
 \tup{b}^* \in \prod_i B_i$ of 
 \[
  \exists \tup{y} \exists \tup{y}' \exists \tup{z}^*\, 
  \phi_{\tup{p},\tup{q},\tup{b},\tup{b}^*}^{n-1}(\tup{x},\tup{y},\tup{z},\tup{z}^*) 
  \wedge 
  \step_{\tup{q},\tup{q}',\tup{b}^*,\tup{b}'}(\tup{y},\tup{y}',\tup{z}^*,\tup{z}')
  \wedge
  \steps_{\tup{q}',\tup{p}',\tup{b}'}(\tup{y}',\tup{x}',\tup{z}').
 \]
 \end{itemize}

Finally, take $\pub$ to be $\phi_{\tup{p},\tup{p}',\tup{b},\tup{b'}}^{B}$. \qed
\end{proof}

\subsection{Putting the Lemmas together}



We now provide the proof of Theorem~\ref{thm:reduction}.
Fix $\tup{R}$, $i \in [k], B \in \nat$ and $\RLTL$ formula $\varphi$. We are required to build a sentence $\psi$ such that for every graph $G$: 
  $G \models \psi$ iff there exists a run $\pi \in \runs(G,\tup{R})$ in which there are at most $B$ broadcasts and that satisfies $proj_i(\pi) \models_i \varphi$. The idea is to compile $\varphi$ into an NBW $N_\varphi$, and annotate robot $i$ by $N_\varphi$. 
  Then we knit together the formulas resulting from the applying the 
  lemmas to the $k$-robot ensemble in which robot $i$ is replaced by the annotated robot.
  
  
  Let $Alph_\varphi$ be the tests appearing in $\varphi$. Thus we can think of $\varphi$ as an \LTL formula over the alphabet $Alph_\varphi$.
  By Theorem~\ref{thm:LTL to NBW}, translate $\varphi$ into an NBW $N_\varphi = (Alph_\varphi,Q_\varphi, I_\varphi, \Delta_\varphi, A_\varphi)$ 
  that accepts exactly the sequences $\alpha \in (Alph_\varphi)^\omega$ such that $\alpha \models_\LTL \varphi$.

  
  Form the robot $P$ as the synchronous product of the robot $\tpl{Q_i,B_i,I_i,\delta_i}$ with $N_\varphi$, 
  i.e., the states of $P$ are pairs $(q,s) \in Q_i \times Q_\varphi$, and 
  the transitions of $P$ are 
  $((q,s), \gc{\tau_1 \wedge \tau_2}{\kappa},(q',s'))$ if $(q,\gc{\tau_1}{\kappa},q') \in \delta_i$ and $(s,\tau_2,s') \in \Delta_\varphi$. 
  The initial (and broadcast) states are those $(q,s)$ with $q \in I_i$ and $s \in I_\varphi$.
  
  Replace $R_i$ by $P$ to form a new $k$-ensemble $\tup{S}$, i.e., $S_j = R_j$ for $j \neq i$, and $S_i = P$. So, for $j \neq i$, the state set of $S_j$ is $Q_j$; while the state set of $S_i$ is $Q_i \times N_\varphi$. Define the \emph{accepting states} of $S_j$, collectively denoted $A_i$, to be those pairs $(q,s)$ such that $s \in A_\varphi$.
      
  Note that there exists a run $\pi \in \runs(G,\tup{R})$ such that $proj_i(\pi) \models \varphi$ iff there exists a run $\pi \in \runs(G,\tup{S})$ such that the states visited by Robot $i$ in $proj_i(\pi)$ see $A_i$ infinitely often, i.e., if $proj_i(\pi) = c_0 c_1 \cdots$ then there are infinitely many $n \in \nat$ such that $st_i(c_n) \in A_i$. Since $G$ is finite and each robot has finitely many states, this is equivalent to the fact that there exist two partial runs $\rho_1$
  and $\rho_2$ with the following properties:
  \it
  \- $\rho_1$ contains at most $B$ publishing points, starts in an initial configuration, ends in some configuration $c$ with $st_i(c) \in A_i$;
  \- $\rho_2$ contains no publishing points, starts in $c$ and ends in $c$, and all tests are evaluated at $c$.
  \ti 
 
 Thus, define the \msol-sentence $\psi$:
 \[
\bigvee_{\tup{p},\tup{q},\tup{b}} \exists \tup{x} \exists \tup{y} \exists \tup{z}\, 
\pub^{\tup{S}}_{\tup{p},\tup{q},\tup{p},\tup{b},B}(\tup{x},\tup{y},\tup{x},\tup{z}) \wedge 
\steps^{\tup{S}}_{\tup{q},\tup{q},\tup{b}}(\tup{y},\tup{y},\tup{z})
 \]
where the disjunction ranges over $\tup{p} \in \prod_j I_j$, $\tup{b} \in \prod_j B_j$, and $\tup{q} \in \prod_j Q_j$ with $q_i \in A_i$. 
This completes the proof of Theorem~\ref{thm:reduction}.
\subsection{Extensions}

Here is a variation of Lemma~\ref{lem:steps} in which the \msol-formula also captures, for every robot, the set of vertices of the graph that it visits. Technically, the only difference with that statement of Lemma~\ref{lem:steps} is that the new formula has set variables $\tup{X}$ and a new item ($5.$) that says that $X_i$ consists of the vertices visited by robot $i$ during the partial run.

\begin{lemma}[visited vertices] \label{lem:visited}
Fix a $k$-robot ensemble $\tup{R}$. For all tuples of states $\tup{p},\tup{q} \in \prod_i Q_i$ and published states $\tup{b} \in \prod_i B_i$,
one can build an $\msol$ formula $\visited$ (that depends on $\tup{R},\tup{p},\tup{p}'$ and $\tup{b}$) with free variables 
$\tup{x}, \tup{y}, \tup{z}$, and $\tup{X}$, such that 
for all $\Sigma$-graphs $G$, $G \models \visited(\tup{x}, \tup{x}', \tup{z},\tup{X})$ iff there exists $N \in \nat$ and a partial run  
$\trans{c_0}{c_1}{K_0}\trans{}{c_2}{K_1} \cdots \trans{}{c_N}{K_{N-1}}$ with the following properties for all $i \in [k]$:
\begin{enumerate}
 \item $x_i = pos_i(c_0)$, $x'_i = pos_i(c_N)$,
 \item $p_i = st_i(c_0)$, $p'_i = st_i(c_N)$,
 \item $b_i = \bcst_i(c_0) = \dots = \bcst_i(c_N)$,
 \item $z_i = \bcpos_i(c_0) = \dots = \bcpos_i(c_N)$,
 \item $X_i = \{st_i(c_j) : j \in [0,N]\}$.
 \end{enumerate}
The formula $\visited$ may be written $\visited_{\tup{p},\tup{q},\tup{b}}$ or $\visited^{\tup{R}}_{\tup{p},\tup{q},\tup{b}}$ to stress the parameters it depends on.
\end{lemma}

\begin{proof}
\todo{Text from PRIMA}
To ensure that $X_i$ designates the exact set of positions
visited by robot $i$, one needs to modify the construction of $\phi_{i,p_i,q_i,\tup{b}}$ in the proof of the
lemma.\footnote{In~\cite{Rubin15AAMAS} it is wrongly stated that one can transform an
$\msol$ formula that says that there is a run (satisfying some property) that stays within a set
$X$, to one that says that it also visits all of $X$, by simply requiring that $X$ be a minimal
set for which a run satisfying the property exits.} The required modifications are
straightforward except for those to the definition of $\varphi_{r^*}$, which are more
complicated. Recall that $\varphi_{r^*}$ has free variables $X,x,y,\tup{z}$, and its
semantic in this case is that robot $i$ can reach $y$ from $x$ (with the other robots' positions
being $\tup{z}$), visiting exactly $X$, using a concatenation of sub-paths each satisfying
$\varphi_{r}$. Intuitively, $\varphi_{r^*}$ existentially quantifies over
the stitching points of these sub-paths and uses appropriate sub-formulas that are all
satisfied iff one can find sub-paths that can be stitched to lead from $x$ to $y$ and that cover
all the positions in $X$. 
\end{proof}


