\subsection{appendix to undecidability proof}

We now explain why the simulation of each instruction of $\cm$ is done in four
phases. First, observe that
if we dispense with the synchronization robots and allow the counter robots to
move freely then there is no way to ensure that the two counter robots are
coordinated. This may lead to a situation where one counter robot has simulated
more instructions than the other, rendering it impossible for it to simulate a
test for zero of the other counter (since the other counter robot encodes a
value that belongs to another point in time). As we later explain, the fact that
a counter robot cannot do anything unless one of its two synchronizing robots
collides with it, together with the fact that the synchronizing robots seek
their counter robots in alternation, ensures that the counter robots remain
coordinated. But first we argue why the naive approach using only two
synchronizing robots (and thus two phases), with the value of a counter encoded
by the distance of the counter robot from $\zero$ (instead of half that
distance), does not work. Consider for example the case where a single
synchronizing robot collides with a counter robot that proceeds to simulate a
decrement by going one step to the left. Unfortunately, when the synchronizing
robot goes left on its way back to the beginning of the line (to signal the
synchronizing robot of the other counter that it is its turn to move right) it
will again collide with the counter robot, which if scheduled can now go on and
simulate the next instruction of $\cm$, resulting in the same counter robot
simulating two (or even more) instructions without the other counter robot doing
anything, and thus losing coordination with it. By using $4$ synchronization
robots and $4$ phases we avoid this problem as follows.
Each counter robot and each synchronization robot remembers if it is in an odd
or even node (this is maintained by simply toggling an internal flag whenever it
moves one step in any direction). When robot $\counter_0$ is at an odd (resp.
even) node then it can move only if robot $\round_0$ (resp. $\round_2$) is on
the same node (the synchronizing robot on the other hand concludes that it
cannot move, thus ensuring that the counter robot is the only one that can
move), and it will move to an even (resp. odd) position. Similarly, when robot
$\counter_1$ is at an odd (resp. even) node then it can move only if robot
$\round_1$ (resp. $\round_3$) is on the same node, and it will move to an even
(resp. odd) position. Thus, for example, in phase $1$, after $\counter_0$ moves
following a collision with $\round_0$, when $\round_0$ collides with it again on
its way back left this does not enable $\counter_0$ to simulate another
instruction since it is now in an even node from which it can only move when
colliding with $\round_2$ (which will happen only in phase $3$), whereas
$\round_0$ is now able to continue moving since it detects that this collision
is in an even node (and not an odd one as in the first collision when it could
not move until $\counter_0$ moved away).

It is worth noting that the boundary robot $\boundary$ is not needed if one
wishes to allow the robots to detect when they have reached the right end of the
line. Note, however, that one cannot use the collision with the counter robot to
signal to the synchronization robot that it should start moving back left.
Indeed, when they collide the only available move is for the counter robot which
then moves away from the synchronizing robot, thus rendering this collision
effectively invisible from the latter.

We now turn to the role of the helper robots and the reason for stage (1) in
each phase. It is not hard to see that without the helper robots and stage (1)
at the beginning of each phase all the synchronization robots are in node $1$.
Thus, due to this symmetry, there is no way for them to know which phase should
start next (i.e., which of them should start moving to the right towards the counter robots). 
The helper robots allow us to break this symmetry, and their
distribution of the helper and synchronisation robots between nodes $1$ and $2$ serves to encode the current phase.

The way this is done is as follows:
At the start of phase $1$ robots $\round_0,\ldots,\round_3$ are on node $1$ and
robots $\helper_0,\ldots,\helper_3$ are on node $2$. When $\round_0$ sees that
$\round_1,\ldots,\round_3$ are with it then it moves right (from $1$ to $2$).
Then $\helper_0$ detects that all helper robots and $\round_0$ are with it and
it moves left from $2$ to $1$. Observe that at this point
$\helper_0,\round_1,\round_2,\round_3$ are on node $1$ and $\round_0, \helper_1,
\helper_2,\helper_3$ are on node $2$.
Now, detecting this configuration on node $1$, robot $\round_1$ moves right from
$1$ to $2$. This ends stage (1) of the first phase.
At this point $\round_0$ detects that it is on an even node (recall that a
synchronization robot remembers the parity of the node it is on) together with
$\round_1$ and starts stage (2) by moving right.
Recall that phase $1$ ends by $\round_0$ moving back all the way to the left
until it meets $\helper_0$ (and thus to node $1$). Thus, at the end of phase
$1$, robots $\round_0,\helper_0,\round_2,\round_3$ are on node $1$ and robots
$\round_1, \helper_1, \helper_2,\helper_3$ are on node $2$.
Phase $2$ starts by $\helper_0$ detecting this setup of node $1$ and moving
right to $2$, followed by $\helper_1$ detecting that all helper robots and
$\round_0$ are at $2$ and moving from $2$ to $1$.
At this point we have $\round_0,\helper_1,\round_2,\round_3$ are on node $1$ and
robots $\helper_0, \round_1, \helper_2,\helper_3$ are on node $2$. Observe that
this is this is a symmetric situation to the one we were in after the first two
moves in phase $1$, i.e., all the synchronizing robots are in node $1$ except
for the one whose index is the phase number minus $1$, and all the helper robots
are in node $2$ except for the one whose index is the phase number minus $1$.
Thus, we can now continue in a similar fashion as we did in phase $1$, by having
$\round_2$ detecting this configuration of node $1$ and moving right to node
$2$, etc. We continue in this fashion wrapping around in phase $4$, in which
after two steps we have that $\round_0,\round_1,\round_2, \helper_3$ are on node
$1$ and robots $\helper_0, \helper_1,\helper_2, \round_3$ are on node $2$, at
which point $\round_0$ detects this configuration of node $1$ and moves right to
node $2$, etc.


